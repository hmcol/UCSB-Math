\documentclass[12pt]{article}

% Packages
\usepackage[margin=1in]{geometry}
\usepackage{parskip, tabularx}
% \renewcommand{\arraystretch}{1.}

% Document
\begin{document}
\title{Syllabus \\
    \large{FDST 101 Introduction to Bread \\
    Spring 2022}
}
\author{}
\date{}
\maketitle

\textbf{Course} FDST 101 Introduction to Bread

\textbf{Lectures} TR, 11:00am--12:15pm, Phelps Hall 1444, or online on zoom (synchronous). 

\textbf{Labs} Sat, 5:15am--1:00pm, Lot 22

\textbf{Instructor}
Dr. Alvin Yang, alvinyang@ucsb.edu, office hours: Sunday, 6:00am-7:09am

\textbf{TA} 
Harry Coleman, hcoleman@ucsb.edu, office hours: Monday, 3:00pm-4:15pm

\textbf{TA}
Joseph Sullivan, josephsullivan@ucsb.edu, office hours: Monday, 3:00-4:15pm

\textbf{Textbooks}
\begin{enumerate}
    \item Paula Figoni, How Baking Works: Exploring the Fundamentals of Baking Science, 3rd edition, ISBN-13: 978-0470392676.
    \item Ken Forkish, Flour Water Salt Yeast: The Fundamentals of Artisan Bread and Pizza, ISBN-13: 978-1607742739.
\end{enumerate}

\textbf{Prerequisites} A love of bread, PSTAT 120A, or equivalent. 

\textbf{Description} This course is an introduction to the art of bread. It will cover the basics of bread baking, including the basic ingredients, the process of making bread, and the basic tools and techniques for making bread. We will discuss the history of bread, the various types of bread and their cultural impact, and the different types of bread. The course will also cover the current trends in bread baking, including the rise of the modern baker, the rise of the modern baker's kitchen, and the rise of the modern baker's kitchen's oven. 

\textbf{Homework} Assignments will be posted on the Gradescope irregularly. The exact format of the assignments will vary. Some assignments will be group assignments and your group will be chosen for you. All assignments will be due on the due date. Late assignments will not be penalized.

\textbf{Exams} There will be a midterm and a final exam. Both exams will include a written component and a baking component. No make ups will be given for missed exams except in cases of severe physical injury. 

\textbf{Final Project} There will be a final project. The final project will be a creative display incorporating the bread and baking techniques learned in the course. You will work on your display throughout the course---planning the design and layout, developing the materials, and developing the bread. 

\textbf{Grading} Homework 20, Midterm 30, Final 40. Credit will be given for homework and midterm if the student is able to complete the homework and midterm. Completion of the final will be determined by the instructor.

\textbf{Extra Credit} Extra credit is given to students who go above and beyond the expectations of the course. Examples from previous courses: bread pilgrimage to sites of great bread importance, independently farming/gathering all the raw ingredients for bread, inventing a new kind of bread that is both nutritious and delicious, operating a successful bakery for 10+ years, becoming and remaining bread, baking a loaf of bread with without crust, bribery.

\textbf{Academic Honesty} All students must maintain the highest standards of academic honesty in all academic endeavors. Collaboration with other students, professors, and staff is mandatory. The course is not open to students who have previously been involved in any form of plagiarism. Dropping the course without giving a formal warning is considered academic dishonesty. Under the circumstances, the instructor may refuse to allow a student to drop the course.

\textit{Syllabus is subject to change. Check back regularly for updates.} 

\end{document}
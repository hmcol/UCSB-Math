\documentclass[12pt]{article}

% Packages
\usepackage[margin=1in]{geometry}
\usepackage{fancyhdr}
\usepackage{amsmath, amsthm, amssymb, physics}

% Page Style
\fancypagestyle{plain}{
    \fancyhf{}
    \renewcommand{\headrulewidth}{0pt}
    \renewcommand{\footrulewidth}{0pt}
    \fancyfoot[R]{\thepage}
}
\pagestyle{plain}

% Problem Box
\setlength{\fboxsep}{4pt}
\newsavebox{\savefullbox}
\newenvironment{fullbox}{\begin{lrbox}{\savefullbox}\begin{minipage}{\dimexpr\textwidth-2\fboxsep\relax}}{\end{minipage}\end{lrbox}\begin{center}\framebox[\textwidth]{\usebox{\savefullbox}}\end{center}}
\newenvironment{pbox}[1][]{\begin{fullbox}\ifx#1\empty\else\paragraph{#1}\fi}{\end{fullbox}}

% Options
\renewcommand{\thesubsection}{\thesection(\alph{subsection})}
\allowdisplaybreaks
\addtolength{\jot}{4pt}
\theoremstyle{definition}
\setlength{\parindent}{0pt}
\setlength{\parskip}{4pt}

% Default Commands
\newtheorem{proposition}{Proposition}
\newtheorem{lemma}{Lemma}
\newcommand{\ds}{\displaystyle}
\newcommand{\isp}[1]{\quad\text{#1}\quad}
\newcommand{\N}{\mathbb{N}}
\newcommand{\Z}{\mathbb{Z}}
\newcommand{\Q}{\mathbb{Q}}
\newcommand{\R}{\mathbb{R}}
\newcommand{\C}{\mathbb{C}}
\newcommand{\eps}{\varepsilon}
\renewcommand{\phi}{\varphi}
\renewcommand{\emptyset}{\varnothing}
\newcommand{\pfrac}[2]{\left(\frac{#1}{#2}\right)}

% Extra Commands
\newcommand{\RR}{\mathcal{R}}
\newcommand{\conj}{\overline}

% Document Info
\fancypagestyle{title}{
    \renewcommand{\headrulewidth}{0.4pt}
    \setlength{\headheight}{15pt}
    \fancyhead[R]{Harry Coleman}
    \fancyhead[L]{MATH 118C Homework 1}
    \fancyhead[C]{April 9, 2021}
}

% Begin Document
\begin{document}
\thispagestyle{title}

\begin{pbox}[Exercise 8.7]
    If $0 < x < \dfrac{\pi}{2}$, prove that
    \[
        \frac{2}{\pi} < \frac{\sin x}{x} < 1.
    \]
\end{pbox}

\begin{proof}
    We have that $\sin x > 0$ on $(0, \pi/2)$, so $\dv[2]{x} \sin x = -\sin x < 0$ on the same interval. Hence, $\sin x$ is strictly concave on $(0, \pi/2)$, so
    \[
        t = (1-t)\sin 0 + t\sin\frac{\pi}{2}
            < \sin((1-t)0 + t\frac{\pi}{2})
            = \sin(t\frac{\pi}{2}),
    \]
    for $t \in (0, 1)$. Take $x = t\pi/2$, we rewrite this as
    \[
        \frac{2}{\pi}x < \sin x \iff \frac{2}{\pi} < \frac{\sin x}{x},
    \]
    for $x \in (0, \pi/2)$.

    Since $\sin x$ is differentiable on $[0, \pi/2]$, then the mean value theorem tells us that for every $x \in (0, \pi/2)$, there is some $c_x \in (0, x)$ such that
    \[
        \sin x = \sin x - \sin 0 = (x - 0)\cos c_x = x \cos c_x.
    \]
    Since both sine and cosine are positive on $(0, \pi/2)$ and cosine is strictly decreasing, then $x\cos c_x < x$. So we obtain the remaining inequality
    \[
        \sin x < x \iff \frac{\sin x}{x} < 1.
    \]
    

\end{proof}



\newpage
\begin{pbox}[Exercise 8.12]
    Suppose $0 < \delta < \pi$, $f(x) = 1$ if $|x| \leq \delta$, $f(x) = 0$ if $\delta < |x| \leq \pi$, and $f(x + 2\pi) = f(x)$ for all $x$.
\end{pbox}

\begin{pbox}[(a)]
    Compute the Fourier coefficients of $f$.
\end{pbox}

First, we find
\begin{align*}
    c_0 &= \frac{1}{2\pi} \int_{-\pi}^{\pi} f(x)e^{-i0x} \dd{x} \\
        &= \frac{1}{2\pi} \int_{-\delta}^{\delta} \dd{x} \\
        &= \frac{1}{2\pi} \cdot 2\delta \\
        &= \frac{\delta}{\pi}.
\end{align*}


Then for each nonzero $n \in \Z$, we find

\begin{align*}
    c_n &= \frac{1}{2\pi} \int_{-\pi}^{\pi} f(x)e^{-inx} \dd{x} \\
        &= \frac{1}{2\pi} \int_{-\delta}^{\delta} e^{-inx} \dd{x} \\
        &= \frac{1}{2\pi} \eval[\frac{-e^{-inx}}{ i n}|_{-\delta}^{\delta} \\
        &= \frac{e^{in\delta} - e^{-in\delta}}{2\pi in} \\
        &= \frac{\sin(n\delta)}{\pi n}.
\end{align*}

Note that for nonzero $n \in \Z$, we have
\[
    c_{-n}
        = \frac{\sin(-\delta)}{-\pi n}
        = \frac{\sin(\delta)}{\pi n}
        = c_n.
\]


\begin{pbox}[(b)]
    Conclude that
    \[
        \sum_{n=1}^{\infty} \frac{\sin(n\delta)}{n} = \frac{\pi - \delta}{2}, \quad (0 < \delta < \pi).
    \]
\end{pbox}

The above coefficients tell us that
\[
    s_N(f; 0)
        = \sum_{n=-N}^{N} c_n e^{in0}
        = \frac{\delta}{\pi} + 2\sum_{n=1}^{N} \frac{\sin(n\delta)}{\pi n}.
\]


Taking constants $\delta > 0$ and $M = 0$, then for all $t \in (-\delta, \delta)$ we have
\[
    |f(0 + t) - f(0)| = 0 \leq M|t|,
\]
so $s_N(f; 0) \to f(0) = 1$ as $N \to \infty$. Then

\begin{align*}
    1 &= \frac{\delta}{\pi} + 2\sum_{n=1}^{\infty} \frac{\sin(n\delta)}{\pi n} \\
    \pi &= \delta + 2\sum_{n=1}^{N} \frac{\sin(n\delta)}{n} \\
    \frac{\pi - \delta}{2} &= \sum_{n=1}^{\infty} \frac{\sin(n\delta)}{n}.
\end{align*}




\begin{pbox}[(c)]
    Deduce from Parseval's theorem that
    \[
        \sum_{n=1}^{\infty} \frac{\sin^2(n\delta)}{n^2\delta} = \frac{\pi - \delta}{2}.
    \]
\end{pbox}

Parseval's theorem gives us

\begin{align*}
    \frac{1}{2\pi} \int_{-\pi}^{\pi} |f(x)|^2 \dd{x} &= \sum_{n=-\infty}^{\infty} |c_n|^2 \\
    \frac{1}{2\pi} \int_{-\delta}^{\delta} \dd{x} &= \frac{\delta^2}{\pi^2} + 2\sum_{n=1}^{\infty} \frac{\sin^2(n\delta)}{\pi^2 n^2} \\  
    \frac{\pi}{2} \cdot 2\delta &= \delta^2 + 2\sum_{n=1}^{\infty} \frac{\sin^2(n\delta)}{n^2} \\  
    \pi &= \delta + 2\sum_{n=1}^{\infty} \frac{\sin^2(n\delta)}{n^2\delta} \\  
    \frac{\pi - \delta}{2} &= \sum_{n=1}^{\infty} \frac{\sin^2(n\delta)}{n^2\delta}.
\end{align*}


\newpage
\begin{pbox}[(d)]
    Let $\delta \to 0$ and prove that
    \[
        \int_{0}^{\infty} \pfrac{\sin x}{x}^2 \dd{x} = \frac{\pi}{2}.
    \]
\end{pbox}

\begin{proof}
    Note that
    \[
        \lim_{x \to 0} \frac{\sin x}{x} = 1,
    \]
    so the function is integrable on $[0, b]$, for any $b > 0$. In particular, the integral over $[0, 1]$ converges, so we will show that the integral over $[1, +\infty)$ converges. If it exists, the improper integral is given by
    \[
        \int_{1}^{\infty} \pfrac{\sin x}{x}^2 \dd{x} = \lim_{b \to \infty} \int_{1}^{b} \pfrac{\sin x}{x}^2 \dd{x}.
    \]
    First, we bound the improper integral
    \[
        \int_{1}^{\infty} \frac{\dd{x}}{x^2} \leq \sum_{n=1}^{\infty} \frac{1}{n^2}. 
    \]
    The series on the right converges since the power of $n$ is greater than $1$, implying that the integral converges. Then for any $b > 1$, we have
    \[
        0 < \int_{1}^{b} \pfrac{\sin x}{x}^2 \dd{x} \leq \int_{1}^{b} \frac{\dd{x}}{x^2} < \int_{1}^{\infty} \frac{\dd{x}}{x^2}. 
    \]
    The right side is a finite constant, so letting $b \to \infty$ gives us convergence of the integral over $[1, +\infty)$. Hence, the improper integral
    \[
        \int_{0}^{\infty} \pfrac{\sin x}{x}^2 \dd{x} 
    \]
    converges. Then for any sequence of partitions with diameter tending to zero, the sequence of Riemann sums converges to the integral. In particular, for $k \in \N$, let $\delta_k = 1/k$. Define the partition of $[0, +\infty)$ by $x_n = n\delta_k$. Then

    \begin{align*}
        \int_{0}^{\infty} \pfrac{\sin x}{x}^2 \dd{x}
            &= \lim_{k \to \infty} \sum_{n=1}^{\infty} \pfrac{\sin x_n}{x_n}^2 \delta_k \\
            &= \lim_{k \to \infty} \sum_{n=1}^{\infty} \frac{\sin^2(n\delta_k)}{n^2\delta_k} \\
            &= \lim_{k \to \infty} \frac{\pi - \delta_k}{2} \\
            &= \frac{\pi}{2}.
    \end{align*}


\end{proof}

\begin{pbox}[(e)]
    Put $\delta = \pi/2$ in (c). What do you get?
\end{pbox}

\begin{align*}
    \sum_{n=1}^{\infty} \frac{\sin^2(n\pi/2)}{n^2\pi/2} &= \frac{\pi - \pi/2}{2} \\
    \sum_{n=1}^{\infty} \frac{\sin^2(n\pi/2)}{n^2} &= \frac{1}{2}.
\end{align*}

Note that $\sin(\pi k) = 0$ for all $k \in \N$, so the even terms in the above series are all zero, giving us

\[
    \sum_{n=0}^{\infty} \frac{\sin^2((2n + 1)\pi/2)}{(2n + 1)^2} = \frac{1}{2}.
\]

Then $\sin((2n + 1)\pi/2) = \pm1$, so its square is $1$, and we obtain

\[
    \sum_{n=0}^{\infty} \frac{1}{(2n + 1)^2} = \frac{1}{2}.
\]



\newpage
\begin{pbox}[Exercise 8.13]
    Put $f(x) = x$ if $0 \leq x <2\pi$, and apply Parseval's theorem to conclude that
    \[
        \sum_{n=1}^{\infty} \frac{1}{n^2} = \frac{\pi^2}{6}.
    \]
\end{pbox}

For this problem, we will integrate over $[0, 2\pi]$ instead of $[-\pi, \pi]$, so that $f(x) = x$. First,
\begin{align*}
    c_0 
        &= \frac{1}{2\pi} \int_{0}^{2\pi} f(x)e^{-i0x} \dd{x} \\
        &= \frac{1}{2\pi} \int_{0}^{2\pi} x \dd{x} \\
        &= \frac{1}{2\pi} \cdot \frac{4\pi^2}{2} \\
        &= \pi.
\end{align*}

Then for nonzero $n \in \Z$, we compute
\[
    c_n = \frac{1}{2\pi} \int_{0}^{2\pi} xe^{-inx} \dd{x}.
\]
For integration by parts, take
\[
    u(x) = x \isp{and} v(x) = \frac{-1}{in}e^{-inx},
\]
so $u'(x) = 1$ and $v'(x) = e^{-inx}$. Then
\begin{align*}
    \int_{0}^{2\pi} xe^{-inx} \dd{x}
        &= \eval[\frac{-xe^{-inx}}{in}|_{0}^{2\pi} - \int_{0}^{2\pi} e^{-inx} \dd{x} \\ 
        &= \frac{-2\pi e^{-2\pi in}}{in} + \frac{0 e^{-in0}}{in} - 0 \\ 
        &= \frac{2\pi i}{n}.
\end{align*}
So $c_n = i/n$. Then by Parseval's theorem,
\begin{align*}
    \sum_{n=-\infty}^{\infty} |c_n|^2 &= \frac{1}{2\pi} \int_{0}^{2\pi} |f(x)|^2 \dd{x} \\
    \pi^2 + 2\sum_{n=1}^{\infty} \frac{1}{n^2} &= \frac{1}{2\pi} \int_{0}^{2\pi} x^2 \dd{x} \\
    \sum_{n=1}^{\infty} \frac{1}{n^2} &= \frac{1}{4\pi} \cdot \frac{8\pi^3}{3} - \frac{\pi^2}{2} \\
    \sum_{n=1}^{\infty} \frac{1}{n^2} &= \frac{\pi^2}{6}.
\end{align*}






\begin{pbox}[Exercise 8.14]
    If $f(x) = (\pi - |x|)^2$ on $[-\pi, \pi]$, prove that
    \[
        f(x) = \frac{\pi^2}{3} + \sum_{n=1}^{\infty} \frac{4}{n^2} \cos nx
    \]
    and deduce that
    \[
        \sum_{n=1}^{\infty} \frac{1}{n^2} = \frac{\pi^2}{6}, \quad \sum_{n=1}^{\infty} \frac{1}{n^4} = \frac{\pi^4}{90}.
    \]
\end{pbox}

First,
\begin{align*}
    c_0 
        &= \frac{1}{2\pi} \int_{-\pi}^{\pi} f(x)e^{-i0x} \dd{x} \\
        &= \frac{1}{2\pi} \int_{-\pi}^{\pi} (\pi - |x|)^2 \dd{x} \\
        &= \frac{1}{\pi} \int_{0}^{\pi} (\pi - x)^2 \dd{x}.
\end{align*}
For change of variables, take $t = \pi - x$, so
\[
    c_0 = \frac{1}{\pi} \int_{0}^{\pi} t^2 \dd{t}
        = \frac{1}{\pi} \cdot \frac{\pi^3}{3}
        = \frac{\pi^2}{3}.
\]

Now for nonzero $n \in \Z$, we find
\begin{align*}
    c_n 
        &= \frac{1}{2\pi} \int_{-\pi}^{\pi} (\pi - |x|)^2e^{-inx} \dd{x} \\ 
        &= \frac{1}{2\pi} \qty(\int_{-\pi}^{0} (\pi + x)^2e^{-inx} \dd{x} + \int_{0}^{\pi} (\pi - x)^2e^{-inx} \dd{x}) \\ 
        &= \frac{1}{2\pi} \qty(\int_{0}^{\pi} (\pi - x)^2e^{inx} \dd{x} + \int_{0}^{\pi} (\pi - x)^2e^{-inx} \dd{x}) \\ 
        &= \frac{1}{2\pi} \int_{0}^{\pi} (\pi - x)^2\qty(e^{inx} + e^{-inx}) \dd{x} \\ 
        &= \frac{1}{\pi} \int_{0}^{\pi} (\pi - x)^2 \cos(nx) \dd{x} \\ 
        &= \pi\int_{0}^{\pi} \cos(nx) \dd{x} - 2\int_{0}^{\pi} x\cos(nx) \dd{x} + \frac{1}{\pi}\int_{0}^{\pi} x^2\cos(nx) \dd{x} \\
        &= 0 - 2\int_{0}^{\pi} x\cos(nx) \dd{x} + \frac{1}{\pi}\int_{0}^{\pi} x^2\cos(nx) \dd{x}.
\end{align*}
For integration by parts, take
\[
    u(x) = x \isp{and} v(x) = \frac{\sin(nx)}{n},
\]
so $u'(x) = 1$ and $v'(x) = \cos(nx)$. Then
\begin{align*}
    \int_{0}^{\pi} x\cos(nx) \dd{x}
        &= \eval[\frac{x\sin(nx)}{n}|_{0}^{\pi} - \frac{1}{n} \int_{0}^{\pi} \sin(nx) \dd{x} \\
        &= \frac{\pi \sin(n\pi)}{n} - \frac{1}{n} \eval[\frac{-\cos(nx)}{n}|_{0}^{\pi} \\
        &= 0 + \frac{\cos(n\pi) - \cos(n0)}{n^2} \\
        &= \frac{\cos(n\pi) - 1}{n^2}.
\end{align*}

For integration by parts, take
\[
    u(x) = x^2 \isp{and} v(x) = \frac{\sin(nx)}{n},
\]
so $u'(x) = 2x$ and $v'(x) = \cos(nx)$. Then
\begin{align*}
    \int_{0}^{\pi} x^2\cos(nx) \dd{x}
        &= \eval[\frac{x^2\sin(nx)}{n}|_{0}^{\pi} - \frac{2}{n} \int_{0}^{\pi} x\sin(nx) \dd{x} \\
        &= \frac{\pi^2\sin(n\pi)}{n} - \frac{2}{n} \int_{0}^{\pi} x\sin(nx) \dd{x} \\ 
        &= 0 - \frac{2}{n} \int_{0}^{\pi} x\sin(nx) \dd{x}.
\end{align*}

For integration by parts, take
\[
    u(x) = x \isp{and} v(x) = \frac{-\cos(nx)}{n},
\]
so $u'(x) = 1$ and $v'(x) = \sin(nx)$. Then
\begin{align*}
    \int_{0}^{\pi} x^2\cos(nx) \dd{x}
        &= \frac{-2}{n} \qty(\eval[\frac{-x\cos(nx)}{n}|_{0}^{\pi} + \frac{1}{n}\int_{0}^{\pi} \cos(nx) \dd{x}) \\
        &= \frac{-2}{n} \qty(\frac{-\pi\cos(n\pi)}{n} + 0) \\
        &= \frac{2\pi\cos(nx)}{n^2}.
\end{align*}

Now, we have
\[
    c_n
        = -2 \cdot \frac{\cos(n\pi) - 1}{n^2} + \frac{1}{\pi} \cdot \frac{2\pi\cos(nx)}{n^2}
        = \frac{2}{n^2}.
\]
In particular, we can see that $c_n = c_{-n}$. So we have the Fourier expansion
\[
    \sum_{n = - \infty}^{\infty} c_n e^{inx} 
        = \frac{\pi^2}{3} + \sum_{n=1}^{\infty}\frac{2}{n^2}\qty(e^{inx} + e^{-inx})
        = \frac{\pi^2}{3} + \sum_{n=1}^{\infty}\frac{4}{n^2}\cos(nx).
\]



We now show that $f(x)$ is equal to its Fourier expansion on $[-\pi, \pi]$. First, $f$ is differentiable for $0 < |x| < \pi$, with $|f'(x)| \leq 2\pi$. Therefore, $f$ is Lipschitz on both $(-\pi, 0)$ and $(0, \pi)$,  implying that $f(x)$ equals its Fourier expansion at these points. It remains to show that the conditions of Theorem 8.14 are satisfied at the points $0, \pm\pi$. Note that $f$ is at least continuous at these points, but not necessarily differentiable. For $0$ and any nonzero $t \in (-\pi, \pi)$, the interval from $0$ to $t$ is contained in either $[-\pi, 0]$ or $[0, \pi]$. In either case, we use the continuity of $f$ on these intervals and the differentiability on the open subintervals to select a point $c$ between $0$ and $t$ such that
\[
    f(t) - f(0) = f'(c)(t - 0).
\]
Taking the absolute value, we obtain
\[
    |f(0 + t) - f(0)| = |f'(c)||t| \leq 2\pi|t|.
\]
Hence, $f(0)$ is equal to its Fourier expansion. By similar argument, $f$ is continuous at $\pm\pi$, and differentiable in a radius of $\pi$ around each point, with derivative bounded by $2\pi$. Thus,
\[
    f(x) = \frac{\pi^2}{3} + \sum_{n=1}^{\infty}\frac{4}{n^2}\cos(nx), \qquad x \in [-\pi, \pi].
\]

Evaluating at $x = 0$, we find
\begin{align*}
    f(0) &= \frac{\pi^2}{3} + \sum_{n=1}^{\infty}\frac{4}{n^2}\cos(n0) \\
    \pi^2 &= \frac{\pi^2}{3} + 4\sum_{n=1}^{\infty}\frac{1}{n^2} \\
    \frac{\pi^2}{6} &= \sum_{n=1}^{\infty}\frac{1}{n^2}.
\end{align*}


Applying Parseval's theorem, we find
\begin{align*}
    \frac{1}{2\pi} \int_{-\pi}^{\pi} |f(x)|^2 \dd{x} &= \sum_{n=-\infty}^{\infty} |c_n|^2 \\
    \frac{1}{\pi} \int_{0}^{\pi} (\pi - x)^4 \dd{x} &= \frac{\pi^4}{9} + 2\sum_{n=1}^{\infty} \frac{2^2}{n^4} \\
    \frac{1}{\pi} \eval[\frac{-(\pi - x)^5}{5}|_{0}^{\pi} - \frac{\pi^4}{9} &= 8\sum_{n=1}^{\infty} \frac{1}{n^4} \\
    \frac{1}{\pi} \cdot \frac{\pi^5}{5} - \frac{\pi^4}{9} &= 8\sum_{n=1}^{\infty} \frac{1}{n^4} \\
    \frac{\pi^4}{90} &= \sum_{n=1}^{\infty} \frac{1}{n^4}.
\end{align*}





\newpage
\begin{pbox}[Exercise 8.15]
    With $D_n$ as defined in (77), put
    \[
        K_N(x) = \frac{1}{N + 1} \sum_{n=0}^{N} D_n(x).
    \]
    Prove that
    \[
        K_N(x) = \frac{1}{N + 1} \cdot \frac{1 - \cos(N + 1)x}{1 - \cos x}
    \]
    and that
\end{pbox}

Let $z = e^{ix}$, then the Dirichlet kernel is given by
\[
    D_n(x)
        = \frac{\sin(n + 1/2)x}{\sin(x/2)}
        = \frac{z^{n+1/2} - z^{-n - 1/2}}{z^{1/2} - z^{-1/2}}
        = \frac{z^{-n} - z^{n + 1}}{1 - z}.
\]
Then
\begin{align*}
    (N+1)K_N(x)
        &= \sum_{n=0}^{N} \frac{z^{-n} - z^{n + 1}}{1 - z} \\
        &= \frac{1}{1 - z} \qty(\sum_{n=0}^{N} z^{-n} - z\sum_{n=0}^{N} z^n) \\
        &= \frac{1}{1 - z} \qty(\frac{1 - z^{-(N + 1)}}{1 - z^{-1}} - z \cdot \frac{1 - z^{N + 1}}{1 - z}) \\
        &= \frac{-z}{1 - z} \qty(\frac{1 - z^{-(N + 1)}}{1 - z} + \frac{1 - z^{N + 1}}{1 - z}) \\
        &= \frac{1}{1 - z^{-1}} \cdot \frac{2 - \qty(z^{N+1} + z^{-(N+1)})}{1 - z} \\
        &= \frac{2 - 2\cos(N + 1)x}{2 - \qty(z + z^{-1})} \\
        &= \frac{1 - \cos(N + 1)x}{1 - \cos x}.
\end{align*}
Hence,
\[
    K_N(x) = \frac{1}{N + 1} \cdot \frac{1 - \cos(N + 1)x}{1 - \cos x}.
\]



\begin{pbox}[(a)]
    $K_N \geq 0$
\end{pbox}

Since $\cos x \leq 1$ for all $x \in \R$, then we must also have $1 - \cos x \geq 0$. So in fact
\[
    K_N(x) = \frac{1}{N + 1} \cdot \frac{1 - \cos(N + 1)x}{1 - \cos x} \geq 0.
\]


\newpage
\begin{pbox}[(b)]
    \[
        \frac{1}{2\pi} \int_{-\pi}^{\pi} K_N(x) \dd{x} = 1,
    \]
\end{pbox}

First, we see that
\[
    \frac{1}{2\pi} \int_{-\pi}^{\pi} D_n(x) \dd{x} 
        = \sum_{k=-n}^{n} \frac{1}{2\pi} \int_{-\pi}^{\pi} e^{ikx} \dd{x}
        = 1.
\]
So
\begin{align*}
    \frac{1}{2\pi} \int_{-\pi}^{\pi} K_N(x) \dd{x}
        = \frac{1}{N+1} \sum_{n=0}^{N} \frac{1}{2\pi} \int_{-\pi}^{\pi} D_n(x) \dd{x}
        = \frac{1}{N+1} \sum_{n=0}^{N} 1 
        = 1.
\end{align*}


\begin{pbox}[(c)]
    \[
        K_N(x) \leq \frac{1}{N + 1} \cdot \frac{2}{1 - \cos \delta} \quad \text{if } 0 < \delta \leq |x| \leq \pi.
    \]
\end{pbox}

Since $\cos x$ is decreasing on $[0, \pi]$, then $0 < \delta \leq |x| \leq \pi$ implies $\cos x \leq \cos\delta$. Additionally, since $\cos x \geq - 1$ for all $x \in \R$, then $1 - \cos x \leq 2$. So
\[
    K_N(x)
        = \frac{1}{N+1} \cdot \frac{1 - \cos(N + 1)x}{1 - \cos x}
        \leq \frac{1}{N+1} \cdot \frac{2}{1 - \cos \delta}.
\]

\begin{pbox}
    If $s_N = s_N(f; x)$ is the $N$th partial sum of the Fourier series of $f$, consider the arithmetic means
    \[
        \sigma_N = \frac{s_0 + s_1 + \dots + s_N}{N + 1}.
    \]
    Prove that
    \[
        \sigma_N(f; x) = \frac{1}{2\pi} \int_{-\pi}^{\pi} f(x-t)K_N(t) \dd{t},
    \]
\end{pbox}

We have that
\[
    s_n(f; x) = \frac{1}{2\pi} \int_{-\pi}^{\pi} f(x - t) D_n(t) \dd{t}, 
\]
so
\begin{align*}
    \sigma_N(f; x)
        &= \frac{1}{N+1} \sum_{n=0}^{N} s_n(f; x) \\
        &= \frac{1}{N+1} \sum_{n=0}^{N} \qty(\frac{1}{2\pi} \int_{-\pi}^{\pi} f(x - t) D_n(t) \dd{t}) \\
        &= \frac{1}{2\pi} \int_{-\pi}^{\pi} f(x - t) \qty(\frac{1}{N+1} \sum_{n=0}^{N} D_n(t)) \dd{t} \\
        &= \frac{1}{2\pi} \int_{-\pi}^{\pi} f(x - t) K_N(t) \dd{t}.
\end{align*}


\newpage
\begin{pbox}[]
    and hence prove Fej\'er's theorem: If $f$ is continuous, with period $2\pi$, then $\sigma_N(f; x) \to f(x)$ uniformly on $[-\pi, \pi]$. Hint: use properties (a), (b), (c), to proceed as in Theorem 7.26.
\end{pbox}

\begin{proof}
    Since $f$ is continuous on the compact interval $[-\pi, \pi]$, then it is uniformly continuous. Additionally $f$ is bounded; let $M > 0$ such that $|f(x)| \leq M$ for all $x \in [-\pi, \pi]$. 

    Let $\eps > 0$ be given. To show uniform convergence, we want to find some $N \in \N$ such that
    \[
        n \geq N, \implies |\sigma_n(f; x) - f(x)| < \eps,
    \]
    for all $x \in [-\pi, \pi]$. Using (a) and (b), we find
    \begin{align*}
        |\sigma_n(f; x) - f(x)|
            &= \abs{\frac{1}{2\pi} \int_{-\pi}^{\pi} f(x - t) K_n(t) \dd{t} - f(x) \qty(\frac{1}{2\pi} \int_{-\pi}^{\pi} K_n(t) \dd{t})} \\
            &= \abs{\frac{1}{2\pi} \int_{-\pi}^{\pi} \qty\big[f(x - t) - f(x)] K_n(t) \dd{t}} \\
            &\leq \frac{1}{2\pi} \int_{-\pi}^{\pi} \abs\big{f(x - t) - f(x)} K_n(t) \dd{t}.
    \end{align*}
    By the uniform continuity of $f$, let $\delta > 0$ such that 
    \[
        |x - y| < \delta \implies |f(x) - f(y)| < \frac{\eps}{2}.
    \]
    Additionally, assume $\delta < \pi$. We will estimate the above integral in two parts: first for $|t| \leq \delta$ and second for $\delta \leq |t| \leq \pi$. Using (a) and the choice of $\delta$, we find
    \begin{align*}
        \frac{1}{2\pi} \int_{-\delta}^{\delta} \abs\big{f(x - t) - f(x)} K_n(t) \dd{t}
            &\leq \frac{1}{2\pi} \int_{-\delta}^{\delta} \frac{\eps}{2} K_n(t) \dd{t} \\
            &\leq \frac{\eps}{2} \cdot \frac{1}{2\pi} \int_{-\pi}^{\pi}K_n(t) \dd{t} \\
            &= \frac{\eps}{2}.
    \end{align*}
    Now using (c) and the choice of $M$, we find
    \begin{align*}
        \frac{1}{2\pi} \int_{\delta \leq |t| \leq \pi} \big|f(x - t) - f(x)\big| K_n(t) \dd{t}
            &\leq \frac{1}{2\pi} \int_{\delta \leq |t| \leq \pi} 2M \cdot \frac{1}{n + 1} \cdot \frac{2}{1 - \cos \delta} \dd{t} \\
            &\leq \frac{1}{n + 1} \cdot \frac{4M}{1-\cos \delta} \cdot \frac{1}{2\pi} \int_{-\pi}^{\pi} \dd{t} \\
            &= \frac{1}{n + 1} \cdot \frac{4M}{1-\cos \delta}.
    \end{align*}
    Then we can choose $N \in \N$ such that
    \[
        \frac{1}{N} < \frac{\eps}{2} \cdot \frac{1 - \cos\delta}{4M}.
    \]
    In which case, for any $n \geq N$ and $x \in [-\pi, \pi]$ we have
    \[
        |\sigma_n(f; x) - f(x)|
            \leq \frac{\eps}{2} + \frac{1}{n + 1} \cdot \frac{4M}{1-\cos \delta}
            < \frac{\eps}{2} + \frac{\eps}{2}
            = \eps.
    \]
    Thus, $\sigma_n(f; x) \to f(x)$ uniformly on $[-\pi, \pi]$.
    
\end{proof}

\newpage
\begin{pbox}[Exercise 8.17]
    Assume $f$ is bounded and monotonic on $[-\pi, \pi)$, with Fourier coefficients $c_n$, as given by (62).
\end{pbox}

\begin{pbox}[(a)]
    Use Exercise 17 of Chap. 6 to prove that $\{nc_n\}$ is a bounded sequence.
\end{pbox}


By definition, we have
\[
    c_n = \frac{1}{2\pi} \int_{-\pi}^{\pi} f(x)e^{-inx} \dd{x}. 
\]
We take $g(x) = e^{-inx}$ and $G(x) = ie^{-inx}/n$ for Exercise 6.17, giving us
\begin{align*}
    \abs{\int_{-\pi}^{\pi} f(x)e^{-inx} \dd{x}}
        &= \abs{f(\pi)G(\pi) - f(-\pi)G(-\pi) - \int_{-\pi}^{\pi} G \dd{f}} \\
        &\leq |f(\pi)G(-\pi)| + |f(-\pi)G(\pi)| + \int_{-\pi}^{\pi} |G| |{\dd{f}}|.
\end{align*}
Note that $|G(x)| = 1/n$ for all real $x$, so
\begin{align*}
    \abs{\int_{-\pi}^{\pi} f(x)e^{-inx} \dd{x}}
        &\leq \frac{1}{n}|f(\pi)| + \frac{1}{n}|f(-\pi)| + \frac{1}{n}|f(\pi) - f(-\pi)| \\
        &\leq \frac{2|f(\pi)| + 2|f(-\pi)|}{n}.
\end{align*}
Hence,
\[
    |nc_n|
        \leq n \cdot \frac{1}{2\pi} \cdot \frac{2|f(\pi)| + 2|f(-\pi)|}{n}
        = \frac{2|f(\pi)| + 2|f(-\pi)|}{\pi},
\]
which is a uniform bound of the terms by a constant.



\begin{pbox}[(b)]
    Combine (a) with Exercise 16 and with Exercise 14(e) of Chap. 3, to conclude that
    \[
        \lim_{N \to \infty} s_N(f; x) = \tfrac12 [f(x+) + f(x-)]
    \]
    for every $x$.
\end{pbox}

Given $x$, we take $a_n = c_ne^{inx}$. Then $|na_n| = |c_n|$ is bounded as shown in (a). Let $s_N = s_N(f; x)$ and $\sigma_N = \sigma_N(f; x)$. Then from Exercises 3.14(e) and 8.16, we know that
\[
    \lim_{N \to \infty} s_N(f; x)
        = \lim_{N \to \infty} \sigma_N(f; x)
        = \tfrac12[f(x+) + f(x-)].
\]


\newpage
\begin{pbox}[(c)]
    Assume only that $f \in \RR$ on $[-\pi, \pi]$ and that $f$ is monotonic in some segment $(\alpha, \beta) \subset [-\pi, \pi]$. Prove that the conclusion of (b) holds for every $x \in (\alpha, \beta)$. (This is an application of the localization theorem.)
\end{pbox}

Since $f \in \RR[-\pi, \pi]$, then $f$ is bounded on $[-\pi, \pi]$. In particular, $f$ is bounded on $(\alpha, \beta)$; define $a = \inf_{(\alpha, \beta)} f$ and $b = \sup_{(\alpha, \beta)} f$. Then define the function
\[
    g(x) = \begin{cases}
        a & x \in [-\pi, \alpha], \\
        f(x) & x \in (\alpha, \beta), \\
        b & x \in [\alpha, \pi].
    \end{cases}
\]
Then  $g(x) = f(x)$ for all $x \in (\alpha, \beta)$, so the localization theorem tells us that
\[
    \lim_{N \to \infty} s_N(f; x) = \lim_{N \to \infty} s_N(g; x), \quad x \in (\alpha, \beta).
\]
Additionally, $g$ is monotonic on $[-\pi, \pi]$, so for any $x \in (\alpha. \beta)$ we find
\begin{align*}
    \lim_{N \to \infty} s_N(f; x)
        &= \lim_{N \to \infty} s_N(g; x) \\
        &= \tfrac12[g(x+) + g(x-)] \\
        &= \tfrac12[f(x+) + f(x-)].
\end{align*}





\end{document}
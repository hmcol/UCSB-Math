\documentclass[12pt]{article}

% Packages
\usepackage[margin=1in]{geometry}
\usepackage{parskip}
\usepackage{amsmath, amsthm, amssymb}
\usepackage{tikz, tikz-cd}
\usepackage[shortlabels]{enumitem}

\usepackage{bbm}

\usepackage{suffix}
\usetikzlibrary{decorations.pathmorphing}
\usetikzlibrary{decorations.markings, arrows.meta}

\tikzset{
    marrow/.style={decoration={
        markings,
        mark=at position 0.5 with {\arrow{>}}}, postaction=decorate,
    }
}


\makeatletter
\tikzset{nomorepostaction/.code=\let\tikz@postactions\pgfutil@empty}
\makeatother

% Problem Box
\setlength{\fboxsep}{4pt}
\newlength{\myparskip}
\setlength{\myparskip}{\parskip}
\newsavebox{\savefullbox}
\newenvironment{fullbox}{\begin{lrbox}{\savefullbox}\begin{minipage}{\dimexpr\textwidth-2\fboxsep\relax}\setlength{\parskip}{\myparskip}}{\end{minipage}\end{lrbox}\framebox[\textwidth]{\usebox{\savefullbox}}}

% Environments
\setlist[enumerate]{nosep}
\newcommand{\keyword}[1]{\textbf{#1}}
\newcommand{\sepline}{\rule{\textwidth}{0.4pt}}

% Tikz Environments
\newenvironment{drawing}{\begin{center}\begin{tikzpicture}}{\end{tikzpicture}\end{center}}
% \tikzcdset{row sep/normal=0pt}
\newenvironment{cd}{\begin{center}\begin{tikzcd}}{\end{tikzcd}\end{center}}


% Document Formatting
\newtheoremstyle{mythmstyle}% name of the style to be used
{ }% measure of space to leave above the theorem. E.g.: 3pt
{ }% measure of space to leave below the theorem. E.g.: 3pt
{ }% name of font to use in the body of the theorem
{ }% measure of space to indent
{\scshape}% name of head font
{.}% punctuation between head and body
{ }% space after theorem head; " " = normal interword space
{\thmname{#1}\thmnumber{ #2}\thmnote{ (#3)}}% Manually specify head

\theoremstyle{definition}
\newtheorem{theorem}{Theorem}
\newtheorem{corollary}{Corollary}
\newtheorem{lemma}{Lemma}
\newtheorem{proposition}{Proposition}


% Math Formatting
\newcommand{\isp}[1]{\quad\text{#1}\quad}

% mathbb
\newcommand{\N}{\mathbb{N}}
\newcommand{\Z}{\mathbb{Z}}
\newcommand{\Q}{\mathbb{Q}}
\newcommand{\R}{\mathbb{R}}
\newcommand{\C}{\mathbb{C}}

% mathcal
\renewcommand{\AA}{\mathcal{A}}
\newcommand{\BB}{\mathcal{B}}
\newcommand{\CC}{\mathcal{C}}
\newcommand{\DD}{\mathcal{D}}
\newcommand{\OO}{\mathcal{O}}

% Symbols

\newcommand{\eps}{\varepsilon}
\renewcommand{\phi}{\varphi}
\renewcommand{\emptyset}{\varnothing}

% Delimiters
\newcommand{\<}{\left\langle}
\renewcommand{\>}{\right\rangle}

% Relations
\newcommand{\isom}{\cong}
\newcommand{\seq}{\subseteq}

\newcommand{\inc}{\hookrightarrow}
\newcommand{\To}{\longrightarrow}
\newcommand{\Mapsto}{\longmapsto}



% Math Roman
\newcommand{\source}{\mathrm{source}}
\newcommand{\target}{\mathrm{target}}


% Math Operators
\DeclareMathOperator{\Aut}{Aut}
\DeclareMathOperator{\Cay}{Cay}



% Other
\newcommand{\eqc}{\overline}
\newcommand{\udl}{\underline}

\newcommand{\adj}{\leftrightarrow}




\title{Graph \\
    \large 
}
\author{}
\date{}


\begin{document}

A \keyword{graph} $\Gamma$ is characterized by the following information:
\begin{itemize}
    \item a collection $V$ (or $\Gamma_0$) of \keyword{vertices} (sing. vertex) (or \keyword{nodes} or \keyword{points});
    \item a collection $E$ (or $\Gamma_1$) of \keyword{edges} (or \keyword{arcs} or \keyword{lines});
    \item a rule that associates each edge with two vertices (not necessarily distinct) called its \keyword{endpoints};
\end{itemize}

A \keyword{loop} is an edge whose endpoints are the same vertex.

Two different edges are \keyword{parallel} is they have the same endpoints.

A graph is \keyword{simple} if it has no loops or parallel edges.

When two vertices are the endpoints of an edge, they are said to be \keyword{adjacent} or \keyword{neighbors}.

A graph is \keyword{finite} if it has finitely many vertices and edges.

The \keyword{null graph} is the graph with no vertices and no edges.
It is stupid and we ignore it, I think.

\sepline

Let $\mathbbm{2} = \{0, 1\}$ be a two-element set with distinguished elements.

\sepline

A \keyword{simple graph} is implemented with the following data:
\begin{itemize}
    \item a set $V$;
    \item an anti-reflexive symmetric relation $\adj$ on $V$ called the \keyword{adjacency} relation, i.e.,
    \begin{itemize}
        \item $v \not\adj v$ for all $v \in V$,
        \item $u \adj v$ iff $v \adj u$ for all $u, v \in V$;
    \end{itemize}
\end{itemize}

A simple graph is an implementation of a graph as follows:
\begin{itemize}
    \item vertices $V$;
    \item edges $E = \{\{u, v\} \seq V \mid u \adj v\}$;
    \item endpoints of $e = \{u, v\} \in E$ are its elements, $u$ and $v$.
\end{itemize}

By this implementation, a simple graph is finite if and only if both its vertex set $V$ and edge set $E$ have finite cardinality.

Because the adjacency relation is anti-reflexive, each edge always has two distinct vertices.
Moreover, if two edges $e_1, e_2 \in E$ have the same endpoints if and only if they are the same set---hence there are no parallel edges.
Therefore, a ``simple graph'' is indeed a graph which is simple in the above sense.



\sepline

A \keyword{graph} $\Gamma$ is given by the following data:
\begin{itemize}
    \item a collection of vertices $V$;
    \item a relation $\sim$ on $V$ called the \keyword{adjacency} relation.
\end{itemize}

\sepline

A \keyword{directed graph} is a type of graph in which edges have a certain orientation/direction.
We might think of the edges as arrows.

Given by the following data:
\begin{itemize}
    \item a collection of vertices $V$;
    \item a collection of edges $E$;
    \item for each edge $e \in E$ two vertices $\source(e), \target(e) \in V$.
\end{itemize}

\sepline



\begin{itemize}
    \item simple graph: injective function $E \to \binom{V}{2}$; 
    (or symmetric irreflexive relation on $V$)
    \item multigraph: arbitrary function $E \to \binom{V}{2}$;
    \item loop graph: injective function $E \to \binom{V}{2} \cup \binom{V}{1}$;
    (or symmetric relation on $V$)
    \item pseudograph: arbitrary function $E \to \binom{V}{2} \cup \binom{V}{1}$;
\end{itemize}


directed variants
\begin{itemize}
    \item directed graph: injective function $E \to V^2 \setminus \Delta_V$;
    (or irreflexive relation on $V$)
    \item directed multigraph: arbitrary function $E \to V^2 \setminus \Delta_V$;
    \item directed loop graph: injective function $E \to V^2$;
    (or any relation on $V$)
    \item directed pseudograph (quiver): arbitrary function $E \to V^2$;
\end{itemize}

not commutative:
\begin{cd}[math mode=false, arrows={shift right=1.25, math mode=true}]
    di(simple)
        \ar[rrr, hook]
        \ar[ddd, hook]
        \drar[two heads, "\pi"']
    &&& dimulti
        \ar[lll, two heads, "\pi"'] 
        \ar[ddd, hook]
        \dlar[two heads, "\pi"']
    \\
    & simple
        \rar[hook]
        \dar[hook]
        \ular["\delta"']
    & multi
        \lar[two heads, "\pi"'] 
        \dar[hook]
        \urar["\delta"']
    \\
    & loop
        \uar[two heads, "\lambda"']
        \rar[hook]
        \dlar["\delta"']
    & pseudo
        \lar[two heads, "\pi"']
        \uar[two heads, "\lambda"']
        \drar["\delta"']
    \\
    diloop
        \ar[uuu, two heads, "\lambda"']
        \ar[rrr, hook]
        \urar[two heads, "\pi"']
    &&& dipseudo
        \ar[lll, two heads, "\pi"']
        \ar[uuu, two heads, "\lambda"']
        \ular[two heads, "\pi"']
\end{cd}

\sepline

An \keyword{isomorphism} of graphs $f : \Gamma \to \Gamma'$ is consists of the following data:
\begin{itemize}
    \item a bijection $f : V \to V'$;
    \item a bijection $f : E \to E'$;
\end{itemize}
such that
\begin{itemize}
    \item $\source(f(e)) = f(\source(e))$ for all $e \in E$;
    \item $\target(f(e)) = f(\target(e))$ for all $e \in E$;
    \item equiv $d(e) = (u, v)$ implies $d(f(e)) = (f(u), f(v))$ for all $e \in E$.
    \item map $f_{u, v} : \Gamma(u, v) \to \Gamma'(f(u), f(v))$ for all $u, v \in V$.
\end{itemize}

\sepline

An \keyword{automorphism} of a graph $\Gamma$ is an isomorphism $\Gamma \to \Gamma$.

Denote the set of such automorphisms by $\Aut(\Gamma)$.

Note that $\Aut(\Gamma)$ is ``naturally'' a group under composition.

\sepline

Let $G$ be a group and $\Gamma$ be a graph.

An \keyword{acton} of $G$ on $\Gamma$ consists of the following data:
\begin{itemize}
    \item a group homomorphism $G \to \Aut(\Gamma)$, denoted $g \mapsto (g \cdot -)$.
\end{itemize}

Equivalently, an action consists of a function $A : G \times \Gamma \to \Gamma$, denoted $(g, x) \mapsto g \cdot x$ for $x$ in $V$ or $E$, such that
\begin{itemize}
    \item $1_G \cdot x = x$ for all $x \in \Gamma$;
    \item $g \cdot (h \cdot x) = (gh) \cdot x$ for all $x \in \Gamma$;
    \item $g \cdot (e : u \to v) = g \cdot e : g \cdot u \to g \cdot v$ for all $e \in E$;
\end{itemize}

\sepline

Let $G$ be a group generated by $S \seq G$.

We construct a directed graph $\Cay(G, S)$, called the \keyword{Cayley graph} of $G$ with respect to $S$, as follows:
\begin{itemize}
    \item vertices of $\Cay(G, S)$ are the elements of $G$;
    \item edges of $\Cay(G, S)$ are $g \to gs$ for all $g \in G$ and $s \in S$.
\end{itemize}

There is a path $(s_1, \dots, s_n) : g \mapsto h$ in $\Cay(G, S)$ if and only $gs_1 \cdots s_n = h \in G$.

\sepline

Let $D_n = \<s, r \mid s^2 = r^n = (sr)^2 = 1\>$

Then $\Cay(D_4, \{s, r\})$ is as follows:
\begin{drawing}[decoration={
        markings,
        mark=at position 0.5 with {\arrow{>}}
    }]
    \node[inner sep=0pt] (e) at (0, 0) {};
    \node[inner sep=0pt] (r) at (1, 0) {};
    \node[inner sep=0pt] (r2) at (1, 1) {};
    \node[inner sep=0pt] (r3) at (0, 1) {};
    \node[inner sep=0pt] (s) at (-1, -1) {};
    \node[inner sep=0pt] (sr) at (-1, 2) {};
    \node[inner sep=0pt] (sr2) at (2, 2) {};
    \node[inner sep=0pt] (sr3) at (2, -1) {};

    \foreach \p in {e, r, r2, r3, s, sr, sr2, sr3} {
        \fill (\p) circle (2pt);
    }

    \begin{scope}[every path/.style={postaction={nomorepostaction, decorate}}]
        \draw (e) -- (r) node[midway, below]{$r$};
        \draw (r) -- (r2) node[midway, right]{$r$};
        \draw (r2) -- (r3) node[midway, above]{$r$};
        \draw (r3) -- (e) node[midway, left]{$r$};

        \draw (s) -- (sr) node[midway, left]{$r$};
        \draw (sr) -- (sr2) node[midway, above]{$r$};
        \draw (sr2) -- (sr3) node[midway, right]{$r$};
        \draw (sr3) -- (s) node[midway, below]{$r$};

        \draw (e) arc (90 : 180 : 1);
        \draw (s) arc (270 : 360 : 1);

        \draw (r) arc (180 : 270 : 1);
        \draw (sr3) arc (0 : 90 : 1);

        \draw (r2) arc (270 : 360 : 1);
        \draw (sr2) arc (90 : 180 : 1);

        \draw (r3) arc (0 : 90 : 1);
        \draw (sr) arc (180 : 270 : 1);

    \end{scope}
    \node at (-0.5, -0.5) {$s$};
    \node at (1.5, -0.5) {$s$};
    \node at (1.5, 1.5) {$s$};
    \node at (-0.5, 1.5) {$s$};


    
\end{drawing}

\sepline

There is a natural action of $G$ on $\Cay(G, S)$ by $g \mapsto \phi_g \in \Aut(\Cay(G, S))$, where
\begin{cd}[column sep=0pt, row sep=tiny]
    \Cay(G, S) \ar[rr, "\phi_g"] && \Cay(G, S) \\
    v \ar[dd, -, marrow, "s"'] & \longmapsto & gv \ar[dd, -, marrow, "s"] \\
    & \longmapsto & \\
    vs & \longmapsto & gvs \\
\end{cd}

The map $\phi : G \to \Aut(\Cay(G, S))$ is an isomorphism of groups.

\end{document}
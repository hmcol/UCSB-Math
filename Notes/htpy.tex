\documentclass[12pt]{article}

% Packages
\usepackage[margin=1in]{geometry}
\usepackage{fancyhdr, parskip}
\usepackage{amsmath, amsthm, amssymb}
\usepackage{tikz, tikz-cd}
\usepackage[shortlabels]{enumitem}

% Page Style
\makeatletter
\fancypagestyle{title}{
    \renewcommand{\headrulewidth}{0.4pt}
    \setlength{\headheight}{15pt}
    \fancyhead[R]{\@author}
    \fancyhead[L]{\@title}
    \fancyhead[C]{\@date}
}
\makeatother
\renewcommand{\maketitle}{\thispagestyle{title}}
\fancypagestyle{plain}{
    \fancyhf{}
    \renewcommand{\headrulewidth}{0pt}
    \renewcommand{\footrulewidth}{0pt}
    \fancyfoot[R]{\thepage}
}
\pagestyle{plain}

% Notes Stuff
\newcommand{\keyword}[1]{\textbf{#1}}
\newcommand{\sepline}{\rule{\textwidth}{0.4pt}}

% Problem Box
\setlength{\fboxsep}{4pt}
\newlength{\myparskip}
\setlength{\myparskip}{\parskip}
\newsavebox{\savefullbox}
\newenvironment{fullbox}{\begin{lrbox}{\savefullbox}\begin{minipage}{\dimexpr\textwidth-2\fboxsep\relax}\setlength{\parskip}{\myparskip}}{\end{minipage}\end{lrbox}\framebox[\textwidth]{\usebox{\savefullbox}}}
\newenvironment{pbox}[1][]{\begin{fullbox}\ifx#1\empty\else\paragraph{#1}\phantom{}\fi}{\end{fullbox}}

% Theorem Environments
\theoremstyle{definition}
\newtheorem{lemma}{Lemma}

% Tikz Environments
\newenvironment{drawing}{\begin{center}\begin{tikzpicture}}{\end{tikzpicture}\end{center}}
\tikzcdset{row sep/normal=0pt}
\newenvironment{cd}{\begin{center}\begin{tikzcd}}{\end{tikzcd}\end{center}}

% Default Commands
\newcommand{\isp}[1]{\quad\text{#1}\quad}
\newcommand{\N}{\mathbb{N}} 
\newcommand{\Z}{\mathbb{Z}}
\newcommand{\Q}{\mathbb{Q}}
\newcommand{\R}{\mathbb{R}}
\newcommand{\C}{\mathbb{C}}
\newcommand{\A}{\mathbb{A}}
\renewcommand{\P}{\mathbb{P}}
\newcommand{\eps}{\varepsilon}
\renewcommand{\phi}{\varphi}
\renewcommand{\emptyset}{\varnothing}
\newcommand{\<}{\langle}
\renewcommand{\>}{\rangle}
\newcommand{\isom}{\cong}
\newcommand{\eqc}{\overline}
\newcommand{\clo}{\overline}
\newcommand{\teq}{\trianglelefteq}
\DeclareMathOperator{\id}{\mathbf{1}}
\DeclareMathOperator{\im}{im}

% Extra Commands
\newcommand{\htpy}{\simeq}
\DeclareMathOperator{\rel}{rel}

% Document
\begin{document}
\title{MATH 2 Homework }
\author{Harry Coleman}
\date{, 2022}
\maketitle

Notation

Denote $I = [0, 1] \subseteq \R$ with the usual topology.

\sepline

Terminology

Let $X$ and $Y$ be (topological) spaces.

A \keyword{function} $X \to Y$ (from $X$ to $Y$) is used to mean the most general sort of function of the underlying sets.

A \keyword{map} $X \to Y$ (from $X$ to $Y$) is used to mean a continuous function. 



\sepline

Let $X$ and $Y$ be (topological) spaces.

A \keyword{homotopy} from $X$ to $Y$ is a map $F : X \times I \to Y$.

A \keyword{homotopy} from $X$ to $Y$ is a family of functions $\{f_t : X \to Y\}_{t \in I}$ such that the associated function $X \times I \to Y$ sending $(x, t) \mapsto f_t(x)$ is continuous.
(Footnote: In particular, each $f_t$ will be continuous.)

We say that the maps $f_0$ and $f_1$ are \keyword{homotopic}, written $f_0 \htpy f_1$.

\sepline

Let $A \subseteq X$ be a subspace.

For a map $f : A \to Y$, an \keyword{extension} of $f$ to $X$ is a map $F : X \to Y$ such that $F|_A = f$.

A \keyword{retraction} of $X$ onto $A$ is a map $r : X \to X$ such that $r(X) = A$ and $r|_A = \id_A$.

A \keyword{retraction} of $X$ onto $A$ is a map $r : X \to A$ such that $r|_A = \id_A$.

A \keyword{retraction} of $X$ onto $A$ is an extension of $\id_A$ to $X$.

When such a retraction exists, we say $A$ is a \keyword{retract} of $X$.

A \keyword{retraction} of $X$ is a map $r : X \to X$ such that $r^2 = r$. (Then $r(X)$ is the retract.)

\sepline

A \keyword{deformation retraction} of $X$ onto $A$ is a homotopy $f_t : X \to X$ such that $f_0 = \id_X$, $f_t|_A = \id_A$ for all $t \in I$, and $f_1(X) = A$.
(Footnote: $f_1$ is a retraction of $X$ onto $A$.)

In which case, say $A$ is a \keyword{deformation retract} of $X$.

\sepline

Given a homotopy $f_t : X \to Y$ such that $f_t|_A = f_0|_A$ for all $t \in I$ is called a \keyword{homotopy relative} to $A$, or a homotopy $\rel A$.

A \keyword{deformation retraction} of $X$ onto $A$ is a homotopy rel $A$ from $\id_X$ to a retraction of $X$ onto $A$.


\sepline

Given subspace $A \subseteq X$ and map $f : A \to Y$.
Construct


\[
    X \sqcup_f Y = X \sqcup Y / \{a \sim f(a) : a \in A\}
\]
\[
    X \sqcup_f Y = \frac{X \sqcup Y}{a \sim f(a) : a \in A}
\]
\[
    X \sqcup_f Y = X \sqcup Y / \Gamma(f)
    \isp{where}
    \Gamma(f) = \{(x, f(x)) : x \in \operatorname{dom} f\}
\]



\end{document}
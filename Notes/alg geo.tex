\documentclass[12pt]{article}

% Packages
\usepackage[margin=1in]{geometry}
\usepackage{fancyhdr, parskip}
\usepackage{amsmath, amsthm, amssymb, mathrsfs, tikz-cd}

% Page Style
\fancypagestyle{plain}{
    \fancyhf{}
    \renewcommand{\headrulewidth}{0pt}
    \renewcommand{\footrulewidth}{0pt}
    \fancyfoot[R]{\thepage}
}
\pagestyle{plain}

% Problem Box
\setlength{\fboxsep}{4pt}
\newsavebox{\savefullbox}
\newenvironment{fullbox}{\begin{lrbox}{\savefullbox}\begin{minipage}{\dimexpr\textwidth-2\fboxsep\relax}}{\end{minipage}\end{lrbox}\begin{center}\framebox[\textwidth]{\usebox{\savefullbox}}\end{center}}
\newenvironment{pbox}[1][]{\begin{fullbox}\ifx#1\empty\else\paragraph{#1}\fi}{\end{fullbox}}

% Theorem Environments
\theoremstyle{definition}
\newtheorem{theorem}{Theorem}
\newtheorem{lemma}{Lemma}
\newtheorem{corollary}{Corollary}
\newtheorem{proposition}{Proposition}

\newtheorem{definition}{Definition}
\newtheorem{construction}{Construction}
\newtheorem{notation}{Notation}

\newenvironment{btheorem}{\begin{fullbox}\begin{theorem}}{\end{theorem}\end{fullbox}}
\newenvironment{blemma}{\begin{fullbox}\begin{lemma}}{\end{lemma}\end{fullbox}}
\newenvironment{bproposition}{\begin{fullbox}\begin{lemma}}{\end{lemma}\end{fullbox}}

% Options
%\allowdisplaybreaks
%\addtolength{\jot}{4pt}

% Default Commands
\newcommand{\isp}[1]{\qquad\text{#1}\qquad}
\newcommand{\N}{\mathbb{N}} 
\newcommand{\Z}{\mathbb{Z}}
\newcommand{\Q}{\mathbb{Q}}
\newcommand{\R}{\mathbb{R}}
\newcommand{\C}{\mathbb{C}}
\newcommand{\eps}{\varepsilon}
\renewcommand{\phi}{\varphi}
\renewcommand{\emptyset}{\varnothing}
\newcommand{\<}{\langle}
\renewcommand{\>}{\rangle}
\newcommand{\isom}{\cong}
\newcommand{\eqc}{\overline}
\newcommand{\clo}{\overline}

% Extra Commands
\newcommand{\A}{\mathbb{A}}
\renewcommand{\P}{\mathbb{P}}
\renewcommand{\O}{\mathscr{O}}

\newcommand{\teq}{\trianglelefteq}
\newcommand{\inc}{\hookrightarrow}
\newcommand{\blow}{\widetilde}

\newcommand{\Vp}{V_{\mathrm{p}}}
\newcommand{\Va}{V_{\mathrm{a}}}

\DeclareMathOperator{\codim}{codim}
\DeclareMathOperator{\Hom}{Hom}
\renewcommand{\Im}{\operatorname{Im}}

\newcommand{\longlimit}[2]{
    {
        \def\arraystretch{0}
        \begin{array}[t]{@{}l@{}}
            \displaystyle{#2} \\[8.5pt] \scriptstyle{#1}
        \end{array}
    }
}

\newcommand{\longprod}[2]{\longlimit{#1}{\prod #2}}
\newcommand{\longsum}[2]{\longlimit{#1}{\sum #2}}

\DeclareMathOperator{\LP}{LP}

\newcommand{\KK}{\mathbf{K}}

\DeclareMathOperator{\dom}{dom}
\DeclareMathOperator{\im}{im}

\newcommand{\dto}[1][]{\overset{#1}{\longrightarrow}}
\newcommand{\mto}{\longmapsto}

% Document Info
\fancypagestyle{title}{
    \renewcommand{\headrulewidth}{0.4pt}
    \setlength{\headheight}{15pt}
    \fancyhead[R]{Harry Coleman}
    \fancyhead[L]{MATH CS 120AG Notes}
    \fancyhead[C]{date}
}

% Begin Document
\begin{document}
\thispagestyle{title}

\section*{Notation}

Let $K$ denote an algebraically closed ground field.

Let $K[x_1, \dots, x_n]$ to be the $K$-alegbra of polynomials, graded by degree. We ill mostly focus on $K[x, y]$.

For $n \in \N$, we call $\A^n = \A_K^n$ the \textbf{affine $n$-space} over $K$.

For $S \subseteq K[x_1, \dots, x_n]$, call
\[
    V(S) = \{x \in \A^n : f(x) = 0 \text{ for all } f \in S\}
\]
the (affine) \textbf{zero locus} of $S$. Subsets of $\A^n$ of this form are called \textbf{affine varieties}.




\begin{definition}(Affine Curves)
    \begin{itemize}
        \item[(a)] An \textbf{(affine plane algebraic) curve} is a nonconstant polynomial $F \in K[x, y]$ modulo units, i.e., modulo the equivalence relation $F \sim G$ if $F = \lambda G$ for some $\lambda \in K^\times$. 

        Call $V(F) = \{P \in \A^2 : F(P) = 0\}$ the \textbf{set of points} of $F$.
        
        \item[(b)] The \textbf{degree} of a curve is its degree as a polynomial, denoted $\deg F$.
        
        \item[(c)] A curve $F$ is called \textbf{irreducible} if it is as a polynomial, and \textbf{reducible} otherwise. Similarly, if $F = F_1^{d_1} \cdots F_k^{d_k}$ is the irreducible decomposition of $F$ as a polynomial, we will also call this the \textbf{irreducible decomposition} of the curve $F$. The curves $F_1, \dots, F_k$ are called the \textbf{(irreducible) components} of $F$ and $d_1, \dots, d_k$ their multiplicities.
    \end{itemize}
    
\end{definition}


\begin{lemma}
    Let $F$ be an affine curve.
    \begin{itemize}
        \item[(a)] If $K$ is algebraically closed then $V(F)$ is infinite.
        \item[(b)] If $K$ is infinite then $\A_K^2 \setminus V(F)$ is infinite.
    \end{itemize}
\end{lemma}


\begin{proposition}
    If two curves $F$ and $G$ have no common component then their intersection $V(F, G)$ is finite.
\end{proposition}

\begin{corollary}
    Let $F$ be a curve over an algebraically closed field. THe for any irreducible curve $G$ we have $G \mid F$ if and only if $V(G) \subseteq V(F)$.

    In particular, the irreducible components of $F$ (but not their multiplicities) can be recovered from $V(F)$.
\end{corollary}


\begin{notation}
    Due to the above correspondence between a curve $F$ and its set of points $V(F)$, we will sometimes write
    \begin{itemize}
        \item[(a)] $P \in F$ instead of $P \in V(F)$, i.e., $F(P) = 0$;
        \item[(b)] $F \cap G$ instead of $V(F, G)$ for the points that lie on both $F$ and $G$;
        \item[(c)] $F \cup G$ for the curve $FG$;
        \item[(d)] $G \subseteq F$ instead of $G \mid F$.
    \end{itemize}
\end{notation}


\begin{definition}
    Let $a \in \A^2$ be a point.
    \begin{itemize}
        \item[(a)] The \textbf{local ring} of $\A^2$ at $P$ is defined as
        \[
            \O_a 
                = \O_{\A^2, a} 
                = \left\{\frac{g}{f} : f, g \in K[x, y] \text{ with } f(a) \ne 0 \right\}
                \subseteq K(x, y)
        \]
        \item[(b)] It admits a well-defined ring homomorphism
        \[
            \O_a \to K, \frac{g}{f} \mapsto \frac{g(a)}{f(a)}
        \] 
        which we call the \textbf{evaluation map}. Its kernel will be denoted by
        \[
            I_a = I_{\A^2, a} = \{\phi \in \O_a \mid \phi(a) = 0\}
        \]
        which is the unique maximal ideal in $\O_a$.
    \end{itemize}
\end{definition}


\begin{definition}
    For a point $a \in \A^2$ and two curves $F$ and $G$ we define the \textbf{intersection multiplicity} of $F$ and $G$ at $a$ to be
    \[
        \mu_a(F, G) = \dim \O_a / \<F, G\> \in \N \cup \{\infty\},
    \]
    where $\dim$ denotes the dimension as a vector space over $K$.
\end{definition}

\begin{lemma}
    Let $a \in \A^2$ and let $F$ and $G$ be two curves. We have
    \begin{itemize}
        \item[(a)] $\mu_a(F, G) \geq 1$ if and only if $a \in F \cap G$;
        \item[(b)] $\mu_a(F, G) = 1$ if and only if $\<F, G\> = I_a$ in $\O_a$.
    \end{itemize}
\end{lemma}

\begin{notation}
    For a polynomial $F \in K[x, y]$ of degree $d$ and $i = 0, \dots, d$, we define the \textbf{degree-$i$ part} of $F$ to be the sum of all terms of $F$ of degree $i$. Hence all $F_i$ are homogeneous, and we have $F = F_0 + \cdots + F_d$. We call $F_0$ the \textbf{constant} part, $F_1$ the \textbf{linear} part, and $F_d$ the \textbf{leading} part of $F$.
\end{notation}


\begin{proposition}
    Let $F$ and $G$ be two curves through the origin. Then $\mu_0(F, G) = 1$ if and only if the linear parts of $F$ and $G$ are linearly independent.
\end{proposition}


\section{projective}

\begin{definition}
    For $n \in \N$, we define the \textbf{projective $n$-space} over $K$ as the set of all $1$-dimensional linear subspaces of $K^{n+1}$. It is denoted by $\P_K^n$ or simply $\P^n$.

    In other words, we have
    \[
        \P^n = (K^{n+1} \setminus \{0\}) / \sim
    \]
    with the equivalence relation $x \sim y$ if and only if $x = \lambda y$ for some $\lambda \in K^\times$. We denote the equivalence class of $(x_0, \dots, x_n)$ by $[x_0 : \cdots : x_n] \in \P^n$. Call $x_0, \dots, x_n$ the \textbf{homogeneous} or \textbf{projective coordinate} of the point $[x_0 : \cdots : x_n]$.

    For a subset $S \subseteq K[x_0, \dots, x_n]$ of homogeneous polynomials we call
    \[
        V(S) = \{P \in \P^n : f(P) = 0 \text{ for all } f \in S\} \subseteq \P^n
    \]
    the projective \textbf{zero locus} of $S$. Subsets of $\P^n$ of this form are called \textbf{projective varieties}.
\end{definition}


\begin{definition}(Projective curves)
    A \textbf{(projective plane algebraic) curve} is a nonconstant homogeneous polynomial $F \in K[x, y, z]$ modulo units. We call $V(F) = \{P \in \P^2 : F(P) = 0\}$ is \textbf{set of points}.

    The \textbf{degree} of a projective curve is its degree as a polynomial.

    The notions of irreducible/reducible/reduced curves, as well as irreducible components and their multiplicities, are defined in the same way as for affine curves. 
\end{definition}


\begin{construction}
    For $P \in \P^2$ we define the \textbf{local ring} of $\P^2$ at $P$ as
    \[
        \O_P = \O_{\P^2, P} = \{\frac{g}{f} : f, g \in K[x, y, z] \text{ homogeneous of the same degree with } f(P) \ne 0\}
    \]
    and the unique maximal ideal
    \[
        I_P = I_{\P^2, P} = \{\phi \in \O_P : f(P) = 0\}.
    \]
    There is an isomorphism $\O_{\P^2, [x_0 : y_0 : 1]} \xrightarrow{\sim} \O_{\A^2, (x_0, y_0)}$ given by $\phi \mapsto \phi^{\mathrm{i}}$, which then restricts to $I_{\P^2, [x_0 : y_0 : 1]} \xrightarrow{\sim} I_{\A^2, (x_0, y_0)}$.
\end{construction}




\begin{construction}
    Note that the local ring $\O_{\P^2, P}$ does not contain $K[x, y, z]$ as a subring. But for $F_1, \dots, F_k$ homogeneous there is still a generated ideal
    \[
        \<F_1, \dots, F_k\> = \left\{\frac{a_1}{b_1}F_1 + \cdots + \frac{a_k}{b_k}F_k : a_i = 0 \text{ or $a_ib_i$ homogeneous with $\deg(a_iF_i) = \deg b_i$ for all $i$}\right\}
    \]
    in $\O_P$. As in the affine case we can therefore define \textbf{intersection multiplicity} of two curves $F$ and $G$ at a point $P \in \P^2$ as
    \[
        \mu_P(F, G) = \dim \O_P/\<F, G\>.
    \]
\end{construction}

\end{document}
\documentclass[12pt]{article}

% Packages
\usepackage[margin=1in]{geometry}
\usepackage{amsmath, amsthm, amssymb, physics}
\usepackage[shortlabels]{enumitem}
\usepackage{tikz-cd}

% Problem Box
\setlength{\fboxsep}{4pt}
\newsavebox{\savefullbox}
\newenvironment{fullbox}{\begin{lrbox}{\savefullbox}\begin{minipage}{\dimexpr\textwidth-2\fboxsep\relax}\setlength{\parskip}{8pt}}{\end{minipage}\end{lrbox}\begin{center}\framebox[\textwidth]{\usebox{\savefullbox}}\end{center}}

% Environments
\newenvironment{definition}{\begin{fullbox}}{\end{fullbox}}
\newcommand{\keyword}[1]{\textbf{#1}}
\setlength{\parindent}{0pt}
\setlength{\parskip}{6pt}
\setlist[enumerate]{nosep}
\def\[#1\]{\begin{align*}#1\end{align*}}

\newcommand{\sepline}{\rule{\textwidth}{0.4pt}}


% Document Formatting
\theoremstyle{definition}
\newtheorem{theorem}{Theorem}
\newtheorem{corollary}{Corollary}
\newtheorem{lemma}{Lemma}
\newtheorem{proposition}{Proposition}

% Math Formatting
\newcommand{\ds}{\displaystyle}
\newcommand{\isp}[1]{\quad\text{#1}\quad}
\newcommand{\tc}[1]{, \qquad \text{#1}}
\newcommand{\mc}[1]{, \qquad #1}
\newcommand{\cfa}[1]{, \qquad \text{for all $#1$}}

% Symbols
\newcommand{\N}{\mathbb{N}}
\newcommand{\Z}{\mathbb{Z}}
\newcommand{\Zpos}{\mathbb{Z}_{\geq0}}
\newcommand{\Q}{\mathbb{Q}}
\newcommand{\R}{\mathbb{R}}
\newcommand{\C}{\mathbb{C}}
\newcommand{\eps}{\varepsilon}
\renewcommand{\phi}{\varphi}
\renewcommand{\emptyset}{\varnothing}
\newcommand{\defeq}{\overset{\text{def}}{=}}

% Delimiters
\newcommand{\<}{\left\langle}
\renewcommand{\>}{\right\rangle}

% Relations


% Math Roman
\DeclareMathOperator{\dom}{dom}



% Begin Document
\begin{document}

Insert ZF(C) axioms and such here.

\sepline

An ordered pair can be thought of as something like a set containing two elements, but with an additional distinction between the two elements. In particular, one of the two is distinguished as the first element and the other as the second element.

Given any two things $a$ and $b$, the \keyword{ordered pair} of $a$ and $b$ is denoted by $(a, b)$, and we say that $a$ is the \keyword{first component} and $b$ is the \keyword{second component} of the ordered pair.

The distinction between components is part of the identity of ordered pairs, i.e., $(a, b) = (c, d)$ if and only if $a = c$ and $b = d$.

Contrasting this with sets, it is always the case that $\{a, b\} = \{b, a\}$. This is because sets only contain information about membership, meaning that both the notations `$\{a, b\}$' and `$\{b, a\}$' actually refer to precisely the same set, namely the set containing only $a$ and $b$. On the other hand, $(a, b)$ may not be the same ordered pair as $(b, a)$. In particular, if $a \ne b$, then $(a, b) \ne (b, a)$.

A set-theoretic implementation of ordered pairs is given by
\[
    (a, b) \defeq \{\{a\}, \{a, b\}\},
\]
and one can check that this satisfies the necessary properties.

\sepline


In general, there can be many different names for mathematical \emph{things}, depending on the context. The most general type of thing we talk about are mathematical \emph{objects} or \emph{structures}. There is no strict distinction between these terms but, loosely, we typically consider objects to be smaller things which are being moved around or acted upon by some larger, more complicated thing. Those larger things which have more complicated behavior or rules associated to them we will refer to as structures. However, this is heavily dependent on context; something which is an object in one context may be a structure in another context, depending on how \emph{zoomed in/out} we are from the things in question.

For example, we might consider the numbers $1$, $2$, $3$, etc., and suppose that we know how to add any two numbers together and nothing else. For example, we would know that `$1 + 1 = 2$', but not that `$2 \times 3 = 6$'. Of course, there is a sense in which `$2 \times 3$' is simply a bit of fancy notation for the sum `$3 + 3$' or `$2 + 2 + 2$', but we are going to ignore that and assume that multiplication is a totally opaque concept to us for a moment. In this case, we would consider the numbers $1$, $3$,  now In this case, the numbers would be our objects and the rules of addition would be our structure, which contains information about how those objects behave. Alternately, we might consider the structure of multiplication or a structure where we understand what it means to say that $1$ is less than $2$. Without any of these additional structures on the numbers,  





\newpage

Given $a$ and $b$, we construct the \keyword{ordered pair} of $a$ and $b$ is the set
\[
    (a, b) \defeq \{\{a\}, \{a, b\}\},
\]
where we say that $a$ is the \keyword{first component} and $b$ is the \keyword{second component}.

One can check that $(a, b) = (c, d)$ if and only if $a = c$ and $b = d$.

Given $a_1, \dots, a_n$ with $n \geq 2$, we construct the $n$-\keyword{tuple} of $a_1, \dots, a_n$ recursively. A $2$-tuple is simply an ordered pair, and an $n$-tuple for $n > 2$ is the ordered pair
\[
    (a_1, \dots, a_n) \defeq (a_1, (a_2, \dots, a_n)),
\]
where $(a_2, \dots, a_n)$ is an $(n-1)$-tuple.



Given two sets $A$ and $B$, the \keyword{Cartesian product} of $A$ and $B$ is the set
\[
    A \times B \defeq \{(a, b) \mid a \in A,\; b \in B\}.
\]

\sepline

A function

\sepline

A set $f$ is a \keyword{function} if all of its elements are ordered pairs and
\[
    (x, y_1), (x, y_2) \in f \implies y_1 = y_2.
\]

The \keyword{domain} of $f$ is the set
\[
    \dom f \defeq \{x \mid (x, y) \in f \text{ for some } y\}.
\]

For all $x \in \dom f$, the element $y \in Y$ such that $(x, y) \in f$ is uniquely determined by $x$$f(x)$ to denote to the element of $Y$ such that $(x, f(x)) \in f$. Which means that
\[
    f = \{(x, f(x)) \mid x \in \dom f\}.
\]

\[
    f(x) \defeq y \iff f(\{x\}) = \{y\}
\]


Given a subset $A \subseteq \dom f$, the \keyword{image} of $A$ under $f$ is the set
\[
    f(A) \defeq \{y \mid (a, y) \in f \text{ for some } a \in A\}.
\]

Given two sets $X$ and $Y$.
a function from $X$ to $Y$, written  $f : X \to Y$.

We sometimes write $Y^X$ to denote the set of functions from $X$ to $Y$.

\sepline

Let $f : X \to Y$.

We say that $X$ is the \keyword{domain} of $f$ and $Y$ is the \keyword{codomain} of $f$. 

Given a subset $A \subseteq X$, the \keyword{image} of $A$ under $f$ is the set
\[
    f(A) \defeq \{y \in Y \mid (a, y) \in f \text{ for some } a \in A\}.
\]
For any $x \in X$, we know that $f(\{x\}) = \{y\}$ for some $y \in Y$, meaning that $f(x) \defeq y$.

Then the image of $A$ under $f$ has the equivalent form
\[
    f(A) = \{f(a) \mid a \in A\}.
\]

\sepline


For each $x \in X$, the \keyword{image} of $x$ under $f$ is the element $f(x) \in Y$ such that $(x, f(x)) \in f$. We know that that $f(x)$ always specifies exactly one element of $Y$ by the uniqueness requirement in the definition of a function.



Sometimes, we write $\dom f$ to refer to the domain of $f$, i.e., we would have $X = \dom f$. Similar notation for the codomain is uncommon.

For each $x \in X$, the \keyword{image} of $x$ under $f$ is the unique element $f(x) \in Y$ such that $(x, f(x)) \in f$.

$f(x)$ to refer to the unique element of $Y$ such that $(x, f(x)) \in f$.

the \keyword{image} of $x$ under $f$ is the unique element $y \in Y$ such that $(x, y) \in f$.

For each $x \in X$, $f(x) \defeq y$.

The uniqueness can be expressed as
\[
    (x, y_1), (x, y_2) \in f \implies y_1 = y_2.
\]

Then the unique



\sepline

A \keyword{binary relation} over $X$ and $Y$ is a subset $R \subseteq X \times Y$. We say $X$ is the \keyword{domain} of $R$ and $Y$ is the \keyword{codomain} of $R$. For a pair $(x, y) \in R$, we write $x \mathrel{R} y$.

We say $R$ is \keyword{homogeneous} if $X = Y$, and \keyword{heterogeneous} otherwise.

\sepline

We can equivalently define a binary relation on $X$ and $Y$ to be a function from the cartesian product to the set of truth values, i.e.,
\[
    R : X \times Y \to \{\text{True, False}\}.
\]
Then $x \mathrel{R} y$ if and only if $R(x, y) = \text{True}$.

Under either definition, the statement that $2^{X \times Y}$ is the set of binary relations on $X$ and $Y$ makes sense under both interpretations of the notation 


\end{document}
\documentclass[12pt]{article}

% Packages
\usepackage[margin=1in]{geometry}
\usepackage{fancyhdr, parskip}
\usepackage{amsmath, amsthm, amssymb}
\usepackage{tikz, tikz-cd}

% \usepackage{newpxtext, euler}

% Page Style
\makeatletter
\fancypagestyle{title}{
    \renewcommand{\headrulewidth}{0.4pt}
    \setlength{\headheight}{15pt}
    \fancyhead[R]{\@author}
    \fancyhead[L]{\@title}
    \fancyhead[C]{\@date}
}
\makeatother
\renewcommand{\maketitle}{\thispagestyle{title}}
\fancypagestyle{plain}{
    \fancyhf{}
    \renewcommand{\headrulewidth}{0pt}
    \renewcommand{\footrulewidth}{0pt}
    \fancyfoot[R]{\thepage}
}
\pagestyle{plain}

% Problem Box
\setlength{\fboxsep}{4pt}
\newlength{\myparskip}
\setlength{\myparskip}{\parskip}
\newsavebox{\savefullbox}
\newenvironment{fullbox}{\begin{lrbox}{\savefullbox}\begin{minipage}{\dimexpr\textwidth-2\fboxsep\relax}\setlength{\parskip}{\myparskip}}{\end{minipage}\end{lrbox}\framebox[\textwidth]{\usebox{\savefullbox}}}
\newenvironment{pbox}[1][]{\begin{fullbox}\ifx#1\empty\else\paragraph{#1}\phantom{}\fi}{\end{fullbox}}

% Theorem Environments
\theoremstyle{definition}
\newtheorem{lemma}{Lemma}

% Tikz Environments
\newenvironment{drawing}{\begin{center}\begin{tikzpicture}}{\end{tikzpicture}\end{center}}
\tikzcdset{row sep/normal=0pt}
\newenvironment{cd}{\begin{center}\begin{tikzcd}}{\end{tikzcd}\end{center}}

% Default Commands
\newcommand{\isp}[1]{\quad\text{#1}\quad}
\newcommand{\N}{\mathbb{N}} 
\newcommand{\Z}{\mathbb{Z}}
\newcommand{\Q}{\mathbb{Q}}
\newcommand{\R}{\mathbb{R}}
\newcommand{\C}{\mathbb{C}}
\newcommand{\A}{\mathbb{A}}
\renewcommand{\P}{\mathbb{P}}
\newcommand{\eps}{\varepsilon}
\renewcommand{\phi}{\varphi}
\renewcommand{\emptyset}{\varnothing}
\newcommand{\<}{\langle}
\renewcommand{\>}{\rangle}
\newcommand{\isom}{\cong}
\newcommand{\eqc}{\overline}
\newcommand{\clo}{\overline}
\newcommand{\teq}{\trianglelefteq}
\DeclareMathOperator{\id}{id}
\DeclareMathOperator{\im}{im}

% Extra Commands
\newcommand{\mat}[1]{\begin{bmatrix}#1\end{bmatrix}}
\DeclareMathOperator{\GL}{GL}

% Document
\begin{document}
\title{MATH 220A Homework 7}
\author{Harry Coleman\makebox[0pt][r]{\raisebox{-0.25in}[0pt][0pt]{(worked with Joseph Sullivan and Gahl Shemy)}}}
\date{November 23, 2021}
\maketitle

\begin{pbox}
    All the problems in this problem set refer to the group $G$ with the identity element $1 \in G$ and presentation:
    \[
        G = \<-1, x, y \mid (-1)^2 = 1, x^2 = y^2 = (xy)^2 = -1\>.
    \]
\end{pbox}

\begin{pbox}[1]
    Show that $G$ has order $8$, and show that $x$ generates a normal subgroup of $G$.    
\end{pbox}

\begin{proof}
    We first see that $-1$ commutes with both $x$ and $y$:
    \[
        (-1)x= x^2x = xx^2 = x(-1) \isp{and} (-1)y = y^2y = yy^2 = y(-1).
    \]
    Therefore, it makes sense to write $-x = (-1)x = x(-1)$ and $-y = (-1)y = y(-1)$.
    Then we find the inverses of $x$ and $y$:
    \[
        x^4 = (-1)^2 = 1 \isp{and} y^4 = (-1)^2 = 1,
    \]
    so $x^{-1} = x^3 = -x$ and $y^{-1} = y^3 -y$.
    We compute $yx$:
    \[
        -1 = xyxy \implies yx = (-x)(-1)(-y) = -xy.
    \]
    Hence, we know how to commute all pairs of generators so every element of $G$ can be written in the form $(-1)^ax^by^c$ where $a, b, c \in \{0, 1\}$.
    In particular, we deduce that $|G| \leq 8$.

    Define the set of $8$ elements $H = \{(-1)^ax^by^c \mid a, b, c \in \{0, 1\} \}$.
    One can define the multiplication $(-1)^ax^by^c(-1)^\alpha x^\beta y^\gamma$ in $H$ in a way that agrees with the relations on $G$.
    Under this multiplication, $H$ is a group of order $8$ and there is an obvious surjective group homomorphism $G \to H$.
    This implies $|G| \geq 8$.

    Since the order of $x$ is $4$, the subgroup $\<x\>$ is index $2$ in $G$ which implies $\<x\> \teq G$.
\end{proof}

\begin{pbox}[2]
    Calculate $xyx^{-1}y^{-1}$, and find the center of $G$.
\end{pbox}

We compute
\[
    xyx^{-1}y^{-1}
        = xy(-x)(-y)
        = (-1)^2xyxy
        = (xy)^2
        = -1.
\]
This implies $x, y \notin Z(G)$.
Similarly,
\[
    (-x)(-y)(-x)^{-1}(-y)^{-1}
        = (-1)^2xyxy
        = -1,
\]
so $-x, -y \notin Z(G)$.
Then
\[
    (xy)y(xy)^{-1}y^{-1}
        = xy^2y^{-1}x^{-1}(-y)
        = x(-1)(-y)(-x)(-y)
        = xyxy
        = -1,
\]
so $\pm xy \notin Z(G)$.
Hence, $Z(G) = \{\pm 1\}$.

\newpage
\begin{pbox}[3]
    Find four distinct one-dimensional representations of $G$, and one irreducible two-dimensional representation.
\end{pbox}

We find the conjugacy classes of $G$; let $\eqc{a}$ denote the conjugacy class of $a \in G$.
Clearly, $\eqc{1} = \{1\}$.
We conjugate $-1$:
\[
    x(-1)x^{-1} = (-1)^2x^2 = -1 \isp{and} y(-1)y^{-1} = (-1)^2y^2 = -1,
\]
hence $\eqc{-1} = \{-1\}$.
Then
\[
    (-1)x(-1)^{-1} = (-1)^2x = x \isp{and} yxy^{-1} = -xy(-y) = xy^2 = -x,
\]
so $\eqc{x} = \{\pm x\}$.
It is the same to see $[y] = \{\pm y\}$.
Then
\[
    x(xy)x^{-1} = -y(-x) = yx = -xy,
\]
so $\eqc{xy} = \{\pm xy\}$.

Hence, we have five conjugacy classes: $\eqc{1}, \eqc{-1}, \eqc{x}, \eqc{y}, \eqc{xy}$.

There is always the trivial representation which maps all elements to $[1] \in \GL_1(\C)$.

Note that $\<y\>$ and $\<xy\>$ are normal subgroups of order $4$ by the same argument as $\<x\>$.
So for $a = x, y, xy$ we have $G/\<a\> \isom \Z/2\Z$ and there is a representation sending $\<a\> \mapsto [1]$ and all other elements to $[-1]$.

There is an irreducible representation $G \to \GL_2(\C)$ given by
\[
    -1 \mapsto \mat{-1 & 0 \\ 0 & -1}, \quad
    x \mapsto \mat{i & 0 \\ 0 & -i}, \quad
    y \mapsto \mat{0 & 1 \\ -1 & 0}, \quad
    xy \mapsto \mat{0 & i \\ i & 0}, \quad
\]


\begin{pbox}[4]
    Find as much of the character table of $G$ as you can.
\end{pbox}
\[
    \begin{array}{r|rrrrr}
        & \eqc{1} & \eqc{-1} & \eqc{x} & \eqc{y} & \eqc{xy} \\
        \hline
        \text{triv} & 1 & 1 & 1 & 1 & 1 \\
        G/\<x\> & 1 & 1 & 1 & -1 & -1 \\
        G/\<y\> & 1 & 1 & -1 & 1 & -1 \\
        G/\<xy\> & 1 & 1 & -1 & -1 & 1 \\
        \text{2d} & 2 & -2 & 0 & 0 & 0
    \end{array}    
\]

\end{document}
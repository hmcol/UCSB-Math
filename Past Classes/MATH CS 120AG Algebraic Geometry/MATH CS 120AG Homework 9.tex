\documentclass[12pt]{article}

% Packages
\usepackage[margin=1in]{geometry}
\usepackage{fancyhdr, parskip}
\usepackage{amsmath, amsthm, amssymb, mathrsfs, tikz-cd}

% Page Style
\fancypagestyle{plain}{
    \fancyhf{}
    \renewcommand{\headrulewidth}{0pt}
    \renewcommand{\footrulewidth}{0pt}
    \fancyfoot[R]{\thepage}
}
\pagestyle{plain}

% Problem Box
\setlength{\fboxsep}{4pt}
\newsavebox{\savefullbox}
\newenvironment{fullbox}{\begin{lrbox}{\savefullbox}\begin{minipage}{\dimexpr\textwidth-2\fboxsep\relax}}{\end{minipage}\end{lrbox}\begin{center}\framebox[\textwidth]{\usebox{\savefullbox}}\end{center}}
\newenvironment{pbox}[1][]{\begin{fullbox}\ifx#1\empty\else\paragraph{#1}\fi}{\end{fullbox}}

% Theorem Environments
%\theoremstyle{definition}
%\newtheorem{proposition}{Proposition}
%\newtheorem{lemma}{Lemma}

% Options
%\allowdisplaybreaks
%\addtolength{\jot}{4pt}

% Default Commands
\newcommand{\isp}[1]{\quad\text{#1}\quad}
\newcommand{\N}{\mathbb{N}} 
\newcommand{\Z}{\mathbb{Z}}
\newcommand{\Q}{\mathbb{Q}}
\newcommand{\R}{\mathbb{R}}
\newcommand{\C}{\mathbb{C}}
\newcommand{\eps}{\varepsilon}
\renewcommand{\phi}{\varphi}
\renewcommand{\emptyset}{\varnothing}
\newcommand{\<}{\langle}
\renewcommand{\>}{\rangle}
\newcommand{\isom}{\cong}
\newcommand{\eqc}{\overline}
\newcommand{\clo}{\overline}

% Extra Commands
\newcommand{\A}{\mathbb{A}}
\renewcommand{\P}{\mathbb{P}}
\renewcommand{\O}{\mathscr{O}}

\newcommand{\teq}{\trianglelefteq}
\newcommand{\inc}{\hookrightarrow}
\newcommand{\blow}{\widetilde}

\newcommand{\Vp}{V_{\mathrm{p}}}
\newcommand{\Va}{V_{\mathrm{a}}}

\DeclareMathOperator{\codim}{codim}
\DeclareMathOperator{\Hom}{Hom}
\renewcommand{\Im}{\operatorname{Im}}


% Document Info
\fancypagestyle{title}{
    \renewcommand{\headrulewidth}{0.4pt}
    \setlength{\headheight}{15pt}
    \fancyhead[R]{Harry Coleman}
    \fancyhead[L]{MATH 120AG Homework 9}
    \fancyhead[C]{May 31, 2021}
}

% Begin Document
\begin{document}
\thispagestyle{title}

\begin{pbox}[Exercise 10.6]
    Let $f : X \to Y$ be a morphism of varieties, and let $a \in X$. Show that $f$ induces a linear map $T_aX \to T_{f(a)}Y$ between tangent spaces.
\end{pbox}

\textit{Note: While there are likely countless linear maps that may possibly be constructed between the tangent spaces, it could be argued that the following construction (while leading to some rather inelegant notation) makes no ``arbitrary'' choices, and that each step is considered ``natural,'' in some sense.}


For every open subset $V \subseteq Y$, the pullback of $f$ is a ring homomorphism
\begin{align*}
    f^* : \O_Y(V) &\to \O_X(f^{-1}(V)), \\
        \phi &\mapsto f^*\phi = \phi \circ f.
\end{align*}
In particular, this holds for every open neighborhood $V$ of $f(a)$ and, in which case, $f^{-1}(V)$ is an open neighborhood of $a$. Moreover, the pullback is compatible with restrictions, i.e., the pullbacks on any pair of open subsets agree on the intersection of those subsets. This means that $f^*$ naturally induces well-defined ring homomorphism between stalks
\begin{align*}
    f^*_a : \O_{Y, f(a)} &\to \O_{X, a}, \\
        \eqc{\phi} &\mapsto \eqc{f^*\phi}.
\end{align*}
Hereafter, we will not denote germs of stalks as equivalence classes, and understanding that evaluation is only well-defined at the point where the stalk is taken.

Next, we restrict this ring homomorphism to the maximal idea $I_{f(a)} \teq \O_{Y, f(a)}$, and claim that the image lies in the maximal ideal $I_a \teq \O_{X, a}$. By definition of the maximal ideal, we have $\phi(f(a)) = 0$ for all $\phi \in I_{f(a)}$. Evaluating the image of $\phi$ under $f^*_a$ at $a$, we find
\[
    f^*_a(\phi)(a) = f^*\phi(a) = \phi(f(a)) = 0,
\]
which implies $f^*_a(\phi) \in I_a$. Hence, the restriction
\[
    f^*_a|_{I_{f(a)}} : I_{f(a)} \to I_a
\]
is well-defined. (This result is described in Remark 12.23.)

Define the projection of $I_a$ onto the quotient ring $I_a/I_a^2$
\begin{align*}
    \pi : I_a &\to I_a/I_a^2, \\
        \phi &\mapsto \eqc{\phi}.
\end{align*}
Then there is a ring homomorphism, given by the composition 
\begin{align*}
    \pi \circ f^*_a|_{I_{f(a)}} : I_{f(a)} &\to I_a/I_a^2, \\
        \phi &\mapsto \eqc{f^*\phi}.
\end{align*}
Moreover, it can be seen that this ring homomorphism is well-defined on equivalence classes in the quotient $I_{f(a)}/I_{f(a)}^2$. By definition, we have the ideal
\[
    I_{f(a)}^2 = \<\phi_1\phi_2 \mid \phi_1, \phi_2 \in I_{f(a)}\> \teq I_{f(a)}.
\]
For any pair $\phi_1, \phi_2 \in I_{f(a)}$, we have already seen that their images $f^*\phi_1, f^*\phi_2 \in I_a$, implying
\[
    f^*(\phi_1\phi_2) = (f^*\phi)(f^*\phi) \in I_a^2.
\]
So the ideal generated by such elements in $I_a$ must be at least contained in $I_a^2$, i.e.,
\[
    I_{f(a)}^2 \subseteq f^*_a|_{I_{f(a)}}^{-1}(I_a^2) = \ker(\pi \circ f^*_a|_{I_{f(a)}}).
\]
Hence, $\pi \circ f^*_a|_{I_{f(a)}}$ induces a ring homomorphism 
\begin{align*}
    F : I_{f(a)}/I_{f(a)}^2 &\to I_a/I_a^2, \\
        \eqc{\phi} &\mapsto \eqc{f^*\phi}.
\end{align*}
(This $F$ is the unique map given by the fundamental theorem of (ring) homomorphisms, and could also be obtained by applying the first and third isomorphism theorems for rings.)

In particular, $F$ is a homomorphism between the $K$-algebras, which is a linear map between the $K$-vector spaces. By Corollary 10.5, there are natural vector space isomorphisms
\[
    \phi_Y : I_{f(a)}/I_{f(a)}^2 \xrightarrow{\sim} \Hom_K(T_{f(a)}Y, K) = (T_{f(a)}Y)^*
\]
and
\[
    \phi_X :  I_a/I_a^2 \xrightarrow{\sim} \Hom_K(T_aX, K) = (T_aX)^*.
\]
Then we have a linear map between the dual spaces, given by the composition
\[
    \phi_X \circ F \circ \phi_Y^{-1} : (T_{f(a)}Y)^* \to (T_aX)^*.
\]
The pullback of this is a linear map between the double dual spaces,
\[
    (\phi_X \circ F \circ \phi_Y^{-1})^* : (T_aX)^{**} \to (T_{f(a)}Y)^{**}.
\]
There are natural isomorphisms
\[
    \psi_X : T_aX \xrightarrow{\sim} (T_aX)^{**}
    \isp{and}
    \psi_Y : T_{f(a)}Y \xrightarrow{\sim} (T_{f(a)}Y)^{**},
\]
giving us a linear map between the tangent spaces
\[
    \psi_Y^{-1} \circ (\phi_X \circ F \circ \phi_Y^{-1})^* \circ \psi_X : T_aX \to T_{f(a)}Y.
\]





\newpage
\begin{pbox}[Exercise 10.18]
    Let $X \subseteq \P^3$ be the degree $3$ Veronese embedding of $\P^1$ as in Exercise 7.30. Of course, $X$ must be smooth since it is isomorphic to $\P^1$. Verify this directly using the projective Jacobi criterion of Exercise 10.13(b). 
\end{pbox}

\begin{proof}
    Denote the Veronese embedding of $\P^1$ into $\P^3$ by
    \begin{align*}
        f : \P^1 &\xrightarrow{\sim} X \subseteq \P^3, \\
            [x_0 : x_1] &\mapsto [y_0 : y_1 : y_2 : y_3] = [x_0^3 : x_0^2x_1 : x_0x_1^2 : x_1^3].
    \end{align*}
    We aim to write $X$ as the zero locus of homogeneous polynomials in $K[y_0, y_1, y_2, y_3]$. First, we characterize $X \cap U_0$, with $U_0 \subseteq \P^3$ being the subset where $y_0 \ne 0$. Assuming $y_0 = 1$, we want to impose the conditions
    \[
        y_2 = y_1^2 \isp{and} y_3 = y_1^3.
    \]
    Homogenizing these equations, we claim that
    \[
        X \cap U_0 = \Vp(y_0y_2 - y_1^2,\; y_0^2y_3 - y_1^3) \cap U_0.
    \]

    For any $y \in X \cap U_0$, we have $y = f(x)$ for some $x \in \P^1$, with $y_0 = x_0^3 \ne 0$. Assuming $x_0 = 1$,
    \[
        [y_0 : y_1 : y_2 : y_3] = [1 : x_1 : x_1^2 : x_1^3].
    \]
    Then
    \[
        y_0y_2 - y_1^2 = x_1^2 - x_1^2 = 0 \isp{and} y_0^2y_3 - y_1^3 = x_1^3 - x_1^3 = 0,
    \]
    implying $y \in \Vp(y_0y_2 - y_1^2,\; y_0^2y_3 - y_1^3) \cap U_0$.

    On the other hand, if $y \in \Vp(y_0y_2 - y_1^2,\; y_0^2y_3 - y_1^3) \cap U_0$, then $y_0 \ne 0$. Assuming $y_0 = 1$, then we know $y_2 = y_1^2$ and $y_3 = y_1^3$, i.e.,
    \[
        y = [1 : y_1 : y_1^2 : y_1^3].
    \]
    Since $x = [1 : y_1] \in \P^1$, then $y = f(x) \in X \cap U_0$. Hence,
    \[
        X \cap U_0 = \Vp(y_0y_2 - y_1^2,\; y_0^2y_3 - y_1^3) \cap U_0.
    \]

    Applying the same argument to $U_3 \subseteq \P^3$, where $y_3 \ne 0$, we deduce that
    \[
        X \cap U_3 = \Vp(y_1y_3 - y_2^2,\; y_0y_3^2 - y_2^3) \cap U_3.
    \]
    For all $x \in \P^1$, either $x_0 \ne 0$ or $x_1 \ne 0$, implying that either $f(x) \in U_0$ or $f(x) \in U_3$. This means that $X \subseteq U_0 \cup U_3$, so
    \[
        X = \Vp(y_0y_2 - y_1^2,\; y_0^2y_3 - y_1^3,\; y_1y_3 - y_2^2,\; y_0y_3^2 - y_2^3) \cap (U_0 \cup U_3).
    \]
    We claim that this zero locus is contained in $U_0 \cup U_3$, in which case it fully characterizes $X$. Suppose $y$ is in the zero locus, but $y \notin U_3$, i.e., $y_3 = 0$. Then the second and third equations imply
    \[
        y_1^3 = y_0^2y_3 = 0 \isp{and} y_2^2 = y_1y_2 = 0,
    \]
    meaning $y_1 = y_2 = 0$. Then since
    \[
        y = [y_0 : 0 : 0 : 0] \in \P^3,
    \]
    we must have $y_0 \ne 0$, so $y \in U_0$. Therefore, $X$ is indeed cut off by the claimed equations.

    While the equivalency in the projective Jacobi criterion of Exercise 10.13(b) requires generators for the ideal of $X$, it follows from Corollary 10.14 that in order to prove $X$ is smooth, it suffices to check the functions which characterize $X$ as a zero locus. The Jacobian matrix of the above four polynomials is
    \[
        \begin{bmatrix}
            y_2 & -2y_1 & y_0 & 0 \\
            2y_0y_3 & -3y_1^2 & 0 & y_0^2 \\
            0 & y_3 & -2y_2 & y_1 \\
            y_3^2 & 0 & -3y_2^2 & 2y_07_3
        \end{bmatrix}.
    \]
    Then for any point $a \in X \subseteq U_0 \cup U_3$, either $a_0 = 1$ or $a_3 = 1$, and the matrix is either
    \[
        \begin{bmatrix}
            * & * & 1 & 0 \\
            * & * & 0 & 1 \\
            * & * & * & * \\
            * & * & * & *
        \end{bmatrix}
        \isp{or}
        \begin{bmatrix}
            * & * & * & * \\
            * & * & * & * \\
            0 & 1 & * & * \\
            1 & 0 & * & *
        \end{bmatrix}.
    \]
    Since $\P^1 \isom X$, then $\dim X = 1$. So in order for $X$ to be smooth at $a$, we must have the rank of the above matrix to be at least $3 - \codim_X{a} = 2$. As this is always the case, we conclude that $X$ is smooth.

\end{proof}

\end{document}
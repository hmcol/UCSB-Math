\documentclass[12pt]{article}

% Packages
\usepackage[margin=1in]{geometry}
\usepackage{fancyhdr, parskip}
\usepackage{amsmath, amsthm, amssymb}

% Page Style
\fancypagestyle{plain}{
    \fancyhf{}
    \renewcommand{\headrulewidth}{0pt}
    \renewcommand{\footrulewidth}{0pt}
    \fancyfoot[R]{\thepage}
}
\pagestyle{plain}

% Problem Box
\setlength{\fboxsep}{4pt}
\newsavebox{\savefullbox}
\newenvironment{fullbox}{\begin{lrbox}{\savefullbox}\begin{minipage}{\dimexpr\textwidth-2\fboxsep\relax}}{\end{minipage}\end{lrbox}\begin{center}\framebox[\textwidth]{\usebox{\savefullbox}}\end{center}}
\newenvironment{pbox}[1][]{\begin{fullbox}\ifx#1\empty\else\paragraph{#1}\fi}{\end{fullbox}}

% Theorem Environments
%\theoremstyle{definition}
%\newtheorem{proposition}{Proposition}
%\newtheorem{lemma}{Lemma}

% Options
%\allowdisplaybreaks
%\addtolength{\jot}{4pt}

% Default Commands
\newcommand{\isp}[1]{\quad\text{#1}\quad}
\newcommand{\N}{\mathbb{N}} 
\newcommand{\Z}{\mathbb{Z}}
\newcommand{\Q}{\mathbb{Q}}
\newcommand{\R}{\mathbb{R}}
\newcommand{\C}{\mathbb{C}}
\newcommand{\eps}{\varepsilon}
\renewcommand{\phi}{\varphi}
\renewcommand{\emptyset}{\varnothing}
\newcommand{\<}{\langle}
\renewcommand{\>}{\rangle}
\newcommand{\isom}{\cong}
\newcommand{\eqc}{\overline}
\newcommand{\clo}{\overline}

% Extra Commands
\newcommand{\A}{\mathbb{A}}
\renewcommand{\P}{\mathbb{P}}
\newcommand{\teq}{\trianglelefteq}
\newcommand{\inc}{\hookrightarrow}
\newcommand{\Ip}{I_{\mathrm{p}}}
\newcommand{\Vp}{V_{\mathrm{p}}}
\newcommand{\Va}{V_{\mathrm{a}}}
\newcommand{\rad}{\sqrt}
\DeclareMathOperator{\codim}{codim}
\newcommand{\Kx}[2]{K[x_{#1}, \dots, x_{#2}]}
\newcommand{\sepline}{\rule{\textwidth}{0.4pt}}

% Document Info
\fancypagestyle{title}{
    \renewcommand{\headrulewidth}{0.4pt}
    \setlength{\headheight}{15pt}
    \fancyhead[R]{Harry Coleman}
    \fancyhead[L]{MATH 120AG Homework 6}
    \fancyhead[C]{May 9, 2021}
}

% Begin Document
\begin{document}
\thispagestyle{title}


\begin{pbox}[Exercise 6.28]
    Let $L_1, L_2 \subset \P^3$ be two disjoint lines (i.e. $1$-dimensional linear subspaces in the sense of Example 6.12(b)), and let $a \in \P^3 \setminus (L_1 \cup L_2)$. Show that there is a unique line $L \subset \P^3$ through $a$ that intersects both $L_1$ and $L_2$.
\end{pbox}

\paragraph{Construction.} Let $L \subset \P^3$ be a line not containing $a$. Taking for granted that the results of dimension carry over from affine varieties, we can write $L = \Vp(f_1, f_2)$ for some homogenous linear polynomials $f_1, f_2 \in \Kx{0}{3}$. Moreover, $f_1$ and $f_2$ are linearly independent; otherwise their ideals would be equal and $L$ would be the zero locus of a single polynomial, which is not the case. We consider the cone over $L$, given by
\[
    C(L) = C(\Vp(f_1, f_2)) = \Va(f_1, f_2) \subset \A^4.
\]
As a subset of the vector space $K^4$, we see that $C(L)$ is a $2$-dimensional linear subspace. Let $A \in M_{2 \times 4}(K)$ be the $2 \times 4$ matrix with entries in $K$, such that the entry $a_{ij}$ is the coefficient in $f_i$ of $x_{j-1}$. Then for all $v \in K^4$, we have $f_1(v) = f_2(v) = 0$ if and only if $Av = 0$. That is, $C(L)$ is precisely $\ker A$, which is a $2$-dimensional linear subspace of $K^4$ since the linear independence of $f_1$ and $f_2$ means  that the rank of $A$ is $2$.

By construction of the projective space $\P^3$, it is natural to interpret $C(a)$ as a $1$-dimensional linear subspace of $K^4$. And since $a \notin L = \Vp(f_1, f_2)$, then $Av \ne 0$ for all nonzero $v \in C(a)$, which means $C(a) \cap C(L) = \{0\}$. This implies that the linear subspace $H = C(a) + C(L)$ of $K^4$ has
\[
    \dim_K H = \dim_K C(a) + \dim_K C(L) = 3.
\]
In particular, $H$ is a hyperplane in $K^4$.

\sepline

\begin{proof}
    Let $H_1, H_2 \subset K^4$ be the hyperplanes obtained from applying the above construction to $L_1$, $L_2$, respectively. We interpret $C(L_1)$ and $C(L_2)$ as $2$-dimensional linear subspaces of $K^4$, then their intersection is also a linear subspace of $K^4$. Since $L_1$ and $L_2$ are disjoint, then we must have $C(L_1) \cap C(L_2) = \{0\}$; otherwise their intersection would contain a $1$-dimensional linear subspace of $K^4$, corresponding (under projectivization) to a point in the intersection of $L_1$ and $L_2$.
    
    Therefore,
    \[
        H_1 + H_2 \supset C(L_1) + C(L_2) = K^4,
    \]
    which implies $\dim_K(H_1 + H_2) = 4$. Let $W = H_1 \cap H_2$, which is a linear subspace of $K^4$ with
    \[
        \dim_K W = \dim_K H_1 + \dim_K H_2 - \dim_K(H_1 + H_2) = 2.
    \]
    By the constructions of $H_1$ and $H_2$, we know $C(a) \subset W$, implying that $a \in \P(W)$. Since $\dim_K C(L_1) = 2$ and $\dim_K H_2 = 3$, then $\dim_K (C(L_1) \cap H_2) \geq 1$. And since
    \[
        C(L_1) \cap H_2 \subset C(L_1) \cap H_2 = W,
    \]
    then the intersection of $C(L_1)$ and $W$ is at least dimension $1$, and the same can be said of $C(L_2)$ and $W$. That is, the projectivization $\P(W)$ intersects both $L_1$ and $L_2$.
    
    As a $2$-dimensional linear subspace of $K^4$, $W$ can be written as $W = \ker A$ for some $A \in M_{2 \times 4}(K)$. Then we define the homogeneous linear polynomials $f_1, f_2 \in \Kx{0}{3}$, where the coefficient in $f_i$ of $x_{j-1}$ is the entry $a_{ij}$ of $A$, so
    \[
        W = \ker A = \Va(f_1, f_2).
    \]
    Let $L = \P(W)$, so
    \[
        L = \P(\Va(f_1, f_2)) = \Vp(f_1, f_2).
    \]
    Since $f_1, f_2$ are homogeneous degree $1$, then $L$ is a linear subspace of $\P^3$. And since $W$ is a cone in $K^4$ of dimension $2$, then $L$ has dimension $1$ in $\P^3$. That is, $L$ is a line in $\P^3$ through $a$ and intersecting both $L_1$ and $L_2$.

    To show that $L$ is the unique such line, suppose that $L' \subset \P^3$ is a line through $a$ and intersecting both $L_1$ and $L_2$. We consider the cone $C(L')$ as a subset of $K^4$, in fact a $2$-dimensional linear subspace. Since $a \in L'$, then we know $C(a) \subset C(L')$. Similarly, since $L_1 \cap L' \ne \emptyset$, then $C(L_1)$ and $C(L')$ have nontrivial intersection. Any nonzero vector in $C(a)$ is linearly independent with any vector in $C(L_1)$. Choosing a vector from $C(a)$ and one from $C(L_1) \cap C(L')$, we obtain two linearly independent vectors in $C(L')$. Since $\dim_K C(L') = 2$, then these form a basis for $C(L')$, implying that
    \[
        C(L') \subset C(a) + C(L_1) = H_1.
    \]
    By the same argument, $C(L') \subset H_2$. Therefore, $C(L') \subset H_1 \cap H_2 = W$. Since both $C(L')$ and $W$ are $2$-dimensional linear subspaces of $K^4$, then they must be equal. Hence, $C(L') = W = C(L)$, so in fact $L' = L$.
    

\end{proof}


\begin{pbox}
    Is the corresponding statement for lines and points in $\A^3$ true as well?
\end{pbox}

No, in $\A^3$, we could choose $L_1$ and $L_2$ to be a pair of parallel lines and $a$ to be a point not coplanar with those lines. Then no single line in $\A^3$ could pass through all three items.




\newpage
\begin{pbox}[Exercise 7.13]
    Let $X \subset \P^2$ be a cubic curve. Moreover, let $U \subset X \times X$ be the set of all $(a, b) \in X \times X$ such that $a \ne b$ and the unique line through the two points $a$ and $b$ meets $X$ in exactly three distinct points; we will denote the third one by $f(a, b) \in X$.

    Show that $U \subset X \times X$ is open, and that $f : U \to X$ is a morphism.
\end{pbox}

\begin{proof}
    Given a pair $(a, b) \in X \times X$, we parameterize the line passing through $a$ and $b$ by
    \[
        L = \{[p_0(t), p_1(t), p_2(t)] : t \in K\} \subset \P^3,
    \]
    where $p_j(t) = (1 - t)a_j + tb_j$ for $j = 0, 1, 2$. (This misses a certain point at infinity, depending on the particular representative coordinates of $a$ and $b$. I think this can be fixed by considering two parameterizations, simultaneously, and adjusting the following argument, accordingly.) Suppose that $X = \Vp(g)$, where $g \in K[x_1, x_2, x_3]$ is a homogeneous cubic polynomial. Then the composition
    \[
        h(t) = g(p_0(t), p_1(t), p_2(t)) \in K[t]
    \]
    is an inhomogeneous cubic polynomial, whose coefficients are determined by $a$ and $b$. We know that $h$ has roots at $t = 0, 1$, which parameterize the points $a$ and $b$, respectively, in $L$. And $h$ has a third root, distinct from $0$ and $1$, if and only if $L$ intersects $X$ at some third point distinct from $a$ and $b$. Let $c_0, \dots, c_3 \in K$ be the coefficients of $h$, i.e.,
    \[
        h(t) = c_0 + c_1t + c_2t^2 + c_3t^3.
    \]
    The discriminant of $h$ is given by
    \[
        \operatorname{Disc}_t(h) = c_1^2c_2^2 - 4c_1^3c_3 - 4c_0c_2^3 - 27c_0^2c_3^2 + 18c_0c_1c_2c_3,
    \]
    and has the property of being equal to zero if and only if $h$ has a multiple root. Notice that we can consider the discriminant as a function of the coefficients of $h$; in particular, define the polynomial $d \in K[x_0, x_1, x_2, x_3]$ such that $\operatorname{Disc}_t(h) = d(c_0, c_1, c_2, c_3)$. Moreover, we see that $d$ is in fact a homogenous polynomial of degree $4$.

    Recall that $p_j(t) = (1 - t)a_j + tb_j$, which can be considered as a homogeneous linear polynomial in terms of $a_j$ and $b_j$. Explicitly, for a fixed $t$, we define
    \[
        q_j = (1-t)x_j + ty_j \in K[x_0, x_1, x_2, y_0, y_1, y_2],
    \]
    for $j = 0, 1, 2$. Then the composition
    \[
        g(q_0, q_1, q_2) \in K[x_0, x_1, x_2, y_0, y_1, y_2]
    \]
    is a homogeneous polynomial of degree $4$. Separating the terms by $t$, we find that the coefficients of $h$ are each homogeneous polynomials of degree $4$ in the same ring. Lastly, composing with $d$, we obtain a homogeneous polynomial of degree $8$. Therefore, we conclude that the discriminant of $h$ is a homogenous polynomial on $X \times X$, denote it by $u \in K[x_0, x_1, x_2, y_0, y_1, y_2]$.

    Then for all $(a, b) \in X \times X$ with $a \ne b$, we know that the unique line through $a$ and $b$ intersects $X$ in exactly three distinct points if and only if $u(a, b) \ne 0$. Additionally, our construction ensures that if $a = b$, then we still have $u(a, b) = 0$. Thus,
    \[
        U = (X \times X) \setminus  \Vp(u),
    \]
    which is an open subset of $X \times X$.

\end{proof}

(Not sure if the following would be the correct route, but it feels close to working.)

To show that $f : U \to X$ is a morphism, we show that its components are locally fractions of homogeneous polynomials of the same degree. For $(a, b) \in U$, we know that the discriminant of the related $h$ is $u(a, b) \ne 0$. That is, $h$ has exactly the roots $0, 1, t_c$, where $t_c \in K$ depends on $a$ and $b$ such that
\[
    f(a, b) = [p_0(t_c), p_1(t_c), p_2(t_c)].
\]
We can (maybe) find $t_c$ in terms of the determinant by
\[
    u(a, b) = c_3^4(0 - 1^2)(1 - t_c)^2(t_c - 0)^2
        = c_3^4(1 - t_c)^2t_c^2.
\]
Recall that $u$ is a homogenous polynomial of degree $8$ and $c_3^4$ is a homogeneous polynomial of degree of $4$. The last thing necessary would be a way to solve for $t_c$ while keeping homogeneity, then composing with the homogenous linear polynomials $p_j$ to obtain $f$ as being a coordinate-wise regular function on $U$.


\end{document}
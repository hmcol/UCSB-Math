\documentclass[12pt]{article}

% Packages
\usepackage[margin=1in]{geometry}
\usepackage{fancyhdr, parskip}
\usepackage{amsmath, amsthm, amssymb}
\usepackage{tikz, tikz-cd}

% Page Style
\makeatletter
\fancypagestyle{title}{
    \renewcommand{\headrulewidth}{0.4pt}
    \setlength{\headheight}{15pt}
    \fancyhead[R]{\@author}
    \fancyhead[L]{\@title}
    \fancyhead[C]{\@date}
}
\makeatother
\renewcommand{\maketitle}{\thispagestyle{title}}
\fancypagestyle{plain}{
    \fancyhf{}
    \renewcommand{\headrulewidth}{0pt}
    \renewcommand{\footrulewidth}{0pt}
    \fancyfoot[R]{\thepage}
}
\pagestyle{plain}

% Problem Box
\setlength{\fboxsep}{4pt}
\newlength{\myparskip}
\setlength{\myparskip}{\parskip}
\newsavebox{\savefullbox}
\newenvironment{fullbox}{\begin{lrbox}{\savefullbox}\begin{minipage}{\dimexpr\textwidth-2\fboxsep\relax}\setlength{\parskip}{\myparskip}}{\end{minipage}\end{lrbox}\framebox[\textwidth]{\usebox{\savefullbox}}}
\newenvironment{pbox}[1][]{\begin{fullbox}\ifx#1\empty\else\paragraph{#1}\phantom{}\fi}{\end{fullbox}}

% Theorem Environments
\theoremstyle{definition}
\newtheorem{lemma}{Lemma}

% Tikz Environments
\newenvironment{drawing}{\begin{center}\begin{tikzpicture}}{\end{tikzpicture}\end{center}}
\tikzcdset{row sep/normal=0pt}
\newenvironment{cd}{\begin{center}\begin{tikzcd}}{\end{tikzcd}\end{center}}

% Default Commands
\newcommand{\isp}[1]{\quad\text{#1}\quad}
\newcommand{\N}{\mathbb{N}} 
\newcommand{\Z}{\mathbb{Z}}
\newcommand{\Q}{\mathbb{Q}}
\newcommand{\R}{\mathbb{R}}
\newcommand{\C}{\mathbb{C}}
\newcommand{\A}{\mathbb{A}}
\renewcommand{\P}{\mathbb{P}}
\newcommand{\eps}{\varepsilon}
\renewcommand{\phi}{\varphi}
\renewcommand{\emptyset}{\varnothing}
\newcommand{\<}{\langle}
\renewcommand{\>}{\rangle}
\newcommand{\isom}{\cong}
\newcommand{\eqc}{\overline}
\newcommand{\clo}{\overline}
\newcommand{\teq}{\trianglelefteq}
\DeclareMathOperator{\id}{id}
\DeclareMathOperator{\im}{im}

% Extra Commands
\renewcommand{\AA}{\mathcal{A}}
\newcommand{\UU}{\mathcal{U}}

% Document
\begin{document}
\title{MATH 221A Final}
\author{Harry Coleman}
\date{December 10, 2021}
\maketitle


\begin{pbox}[3]
    Let $X$ be a compact metric space and $f : X \to X$ an isometric embedding.
    Show that $f$ is surjective.

    \textbf{Hint:} Suppose not, find a point $x_0$ not in the image, and consider the sequence $x_0, f(x_0), f(f(x_0)), \dots$. 
\end{pbox}

\begin{proof}
    Let $x_0 \in X$ and inductively define a sequence $x_n = f(x_{n-1})$ for all $n \geq 1$.
    In other words, $x_n = f^n(x_0)$, where $f^n$ denotes $f$ composed with itself $n$ times.
    Note that for $n, m \in \N$ with $n \leq m$ we have
    \[
        d(x_n, x_m)
            = d(f^n(x_0), f^m(x_0))
            = d(x_0, f^{m-n}(x_0)).
    \]

    Since $X$ is a compact metric space, it is sequentially compact; let $\{x_{n_k}\}_{k \in \N}$ be a convergent subsequence.
    In particular, this sequence is Cauchy.
    Given $\eps > 0$ there exists $N \in \N$ such that for all $k, \ell \geq N$ we have $d(x_{n_k}, x_{n_\ell}) < \eps$.
    That is, if $k < \ell$ (so $n_\ell - n_k \geq 1$) then
    \[
        d(x_0, f^{n_\ell - n_k}(x_0)) = d(x_k, x_\ell) < \eps.
    \]
    In particular, we have found the point $x_{n_\ell - n_k} \in B_{\eps}(x_0) \cap f(X)$.
    Since we can find such a point for all $\eps > 0$, this means $x_0$ is a limit point of $f(X)$.
    Since $f(X)$ is compact, this implies $x_0 \in f(X)$, hence $f$ is surjective.
\end{proof}


\newpage
\begin{pbox}[4]
    Define $\R^\infty = \bigcup_{n=1}^{\infty} \R^n$, where the points 
    \[
        (x_1, \dots, x_n) \isp{and} (x_1, \dots, x_n, 0, \dots, 0)
    \]
    are identified, for any number of zeros.
    We topologize $\R^\infty$ as follow: a set in $\R^\infty$ is open iff its intersection with each $\R^n$ is open.

    There is an obvious injection $f : \R^\infty \to \ell^\infty$, where $\ell^\infty = C_B(\N)$ is the set of bounded sequences of real numbers with the sup norm. 
\end{pbox}

For $a \in \R$ and $r > 0$ denote the interval $I_r(a) = (a - r, a + r) \subseteq \R$.




\begin{pbox}[(a)]
    Show $f$ is continuous.
\end{pbox}

\begin{proof}
    Note that $\ell^\infty$ has a basis of open balls $B_r(x) = \prod_{k=1}^{\infty} I_r(x_k)$ for $x \in \ell^\infty$ and $r > 0$.
    It suffices to check the continuity of $f$ on these basis sets.

    In order for a point $a \in \R^n$ to be in the preimage $f^{-1}(B_r(x))$, we must have $|a_k - x_k| < r$ for $k = 1, \dots, n$ and $|x_k| < r$ for all $k > n$.
    So if it is the case that $|x_k| \geq r$ for some $k > n$, then we know $f^{-1}(B_r(x)) \cap \R^n = \emptyset$, which is open in $\R^n$.

    If $|x_k| < r$ for all $i > n$ then
    \[\textstyle
        f^{-1}(B_r(x)) \cap \R^n = \prod_{k=1}^{n} I_r(x_k).
    \]
    If we consider $\R^n \subseteq \R^\infty$ with the sup norm (which generates the usual topology), this is simply the open ball of radius $r$ centered at $(x_1, \dots, x_n) \in \R^n$.

    We conclude that $f^{-1}(B_r(x)) \cap \R^n$ is open in $\R^n$ for all $n$.
    So in fact $f^{-1}(B_r(x))$ is open in the topology on $\R^\infty$, hence $f$ is continuous.
\end{proof}

\begin{pbox}[(b)]
    Is $f$ an embedding?
    That is, is the subspace $f(\R^\infty) \subseteq \ell^\infty$ homeomorphic to $\R^\infty$?
    Justify your answer.
\end{pbox}

No.

\begin{proof}
    Consider the set $U = \prod_{k=1}^{\infty} I_{1/k}(0) \subseteq \R^\infty$.
    Since $U \cap \R^n = \prod_{k=1}^{n} I_{1/k}(0)$ is open in $\R^n$ for all $n \in \N$, we know that $U$ is open in $\R^\infty$.
    However, we will show that $f(U)$ is not open in $f(\R^\infty)$.

    Assume for contradiction that $f(U)$ is open in the subspace $f(\R^\infty) \subseteq \ell^\infty$, i.e., there is some open set $V \subseteq \ell^\infty$ such that $f(U) = V \cap f(\R^\infty)$.
    Then $V$ is an open neighborhood of $0 \in \ell^\infty$, so there is some $\eps > 0$ such that $B_\eps(0) \subseteq V$, implying $B_\eps(0) \cap f(\R^\infty) \subseteq f(U)$.
    
    Choose $n \in \N$ such that $1/n < \eps/2$, and let $x = (0, \dots, 0, \eps/2) \in \R^n$.
    Then by construction $x \notin U$ since $x_n \notin I_{1/n}(0)$, so $f(x) \notin f(U)$. On the other hand, we have $f(x) \in B_\eps(0) \cap f(\R^\infty)$, which is a contradiction.

    In particular, we conclude that $f$ is not an open map and therefore not a homeomorphism to its image.
\end{proof}


\newpage
\begin{pbox}[5 (a)]
    Show that a set of open subsets of a topological space $X$ is a basis if and only if it contains a neighborhood basis for every point $x \in X$.
\end{pbox}

\begin{proof}
    Let $\UU$ be a collection of open subsets of $X$.

    Suppose $\UU$ is a basis and $x \in X$ is any point.
    We claim that $\UU_x = \{U \in \UU : x \in U\}$ is a neighborhood basis for $x$.
    If $N \subseteq X$ is a neighborhood of $x$ it contains an open neighborhood $V \subseteq N$ of $x$.
    As $\UU$ is a basis, we can write $V = \bigcup_{\alpha \in I} U_\alpha$ for some $U_\alpha \in \UU$.
    Since $x \in V$, we must have $x \in U_\alpha$ for some $\alpha \in I$.
    Then $U_\alpha \in \UU_x$ with $x \in U_\alpha \subseteq V \subseteq N$, hence $\UU_x$ is a neighborhood basis for $x$.

    Suppose $\UU$ contains a neighborhood basis for every point $x \in X$.
    Let $V \subseteq X$ be an open subset. For each point $x \in V$ we can choose some $U_x \in \UU$ such that $x \in U_x \subseteq V$.
    Then we can write $V = \bigcup_{x \in V} U_x$, hence $\UU$ is a basis.
\end{proof}

\begin{pbox}[(b)]
    Show that every compact totally separated space has a basis of clopen sets.
\end{pbox}

\begin{proof}
    Let $X$ be a compact totally separated space.
    For each pair of distinct points $a, b \in X$ choose some clopen set $A_{a, b} \subseteq X$ such that $a \in A_{a, b}$ and $b \in A_{a, b}^c$.
    Let $\AA$ be the collection of all finite intersections of such clopen sets; we claim that $\AA$ is a basis.

    As per part (a), it suffices to show $\AA$ contains a neighborhood basis for each point.
    Let $x \in X$ and $U \subseteq X$ be an open neighborhood of $x$.
    The collection $\{A_{x, y}^c\}_{y \in U^c}$ forms a clopen cover of $U^c$.
    As a closed subset of a compact space, $U^c$ is compact, so we can choose a finite subcover $\{A_i^c\}_{i=1}^{n}$.
    Then $U^c \subseteq \bigcup_{i=1}^{n} A_i^c$, which implies
    \[\textstyle
        A
            = \bigcap_{i=1}^{n} A_i
            \subseteq \left(\bigcup_{i=1}^{n} A_i^c\right)^c
            \subseteq U.
    \]
    And since $x \in A_i$ for $i = 1, \dots, n$, we know that $x \in A$.
    With $A \in \AA$, we conclude that $\AA$ contains a neighborhood basis for $x$ and is therefore a basis (of clopen sets).
\end{proof}


\newpage
\begin{pbox}[6]
    Prove or disprove:
\end{pbox}

\begin{pbox}[(a)]
    The arbitrary product of path-connected spaces is path-connected.
\end{pbox}

Yes.

\begin{proof}
    Let $\{X_\lambda\}_{\lambda \in \Lambda}$ be a collection of path-connected spaces and $X = \prod_{\lambda \in \Lambda} X_\lambda$ have the product topology with natural projections $\pi_\lambda : X \to X_\lambda$.
    (Write $x_\lambda = \pi_\lambda(x)$ for each $x \in X$.)
    Given $x, y \in X$, let $\gamma_\lambda$ be a path from $x_\lambda$ to $y_\lambda$ in $X_\lambda$, i.e., a continuous map $[0, 1] \to X_\lambda$ with $\gamma_\lambda(0) = x_\lambda$ and $\gamma_\lambda(1) = y_\lambda$.
    By the universal property of topological products, there is a continuous map $\gamma : [0, 1] \to X$ such that $\pi_\lambda \circ \gamma = \gamma_\lambda$.
    The fact that $\pi_\lambda(\gamma(0)) = x_\lambda$ and $\pi_\lambda(\gamma(1)) = y_\lambda$ for all $\lambda \in \Lambda$ implies $\gamma(0) = x$ and $\gamma(1) = y$.
    That is, $\gamma$ is a path from $x$ to $y$ in $X$, hence $X$ is path-connected.
\end{proof}

\begin{pbox}[(b)]
    The arbitrary product of locally path-connected spaces is locally path-connected.
\end{pbox}

No.

With the discrete topology, $\{0, 1\}$ is disconnected---and therefore path-disconnected---since each singleton is clopen.
However, $\{0, 1\}$ is locally path-connected since each singleton is path-connected.

Consider $X = \{0, 1\}^\N$ with the product topology.

We claim $X$ is totally path-disconnected, i.e., no two distinct points in $X$ have a path between them.
If $x, y \in X$ are distinct, we must have $x_n \ne y_n$ for some $n \in \N$; assume $x_n = 0$ and $y_n = 1$.
If $\gamma : [0, 1] \to X$ were a path from $x$ to $y$, then $\pi_n \circ \gamma$ would be a path from $0$ to $1$ in $\{0, 1\}$, which is not possible.
Therefore, no path from $x$ to $y$ exists in $X$.

Since no singleton of $X$ is open, every open set must contain at least two points.
Thus, no open subset of $X$ is path-connected, so $X$ is not locally path-connected.


\end{document}
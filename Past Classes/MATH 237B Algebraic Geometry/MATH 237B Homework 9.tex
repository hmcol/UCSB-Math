\documentclass[12pt]{article}

% Packages
\usepackage[margin=1in]{geometry}
\usepackage{fancyhdr, parskip}
\usepackage{amsmath, amsthm, amssymb}
\usepackage{tikz, tikz-cd}

% Page Style
\makeatletter
\fancypagestyle{title}{
    \renewcommand{\headrulewidth}{0.4pt}
    \setlength{\headheight}{15pt}
    \fancyhead[R]{\@author}
    \fancyhead[L]{\@title}
    \fancyhead[C]{\@date}
}
\makeatother
\renewcommand{\maketitle}{\thispagestyle{title}}
\fancypagestyle{plain}{
    \fancyhf{}
    \renewcommand{\headrulewidth}{0pt}
    \renewcommand{\footrulewidth}{0pt}
    \fancyfoot[R]{\thepage}
}
\pagestyle{plain}

% Problem Box
\setlength{\fboxsep}{4pt}
\newlength{\myparskip}
\setlength{\myparskip}{\parskip}
\newsavebox{\savefullbox}
\newenvironment{fullbox}{\begin{lrbox}{\savefullbox}\begin{minipage}{\dimexpr\textwidth-2\fboxsep\relax}\setlength{\parskip}{\myparskip}}{\end{minipage}\end{lrbox}\framebox[\textwidth]{\usebox{\savefullbox}}}
\newenvironment{pbox}[1][]{\begin{fullbox}\ifx#1\empty\else\paragraph{#1}\phantom{}\fi}{\end{fullbox}}

% Theorem Environments
\theoremstyle{definition}
\newtheorem{lemma}{Lemma}

% Tikz Environments
\newenvironment{drawing}{\begin{center}\begin{tikzpicture}}{\end{tikzpicture}\end{center}}
% \tikzcdset{row sep/normal=0pt}
\newenvironment{cd}{\begin{center}\begin{tikzcd}}{\end{tikzcd}\end{center}}

% Default Commands
\newcommand{\isp}[1]{\quad\text{#1}\quad}
\newcommand{\N}{\mathbb{N}} 
\newcommand{\Z}{\mathbb{Z}}
\newcommand{\Q}{\mathbb{Q}}
\newcommand{\R}{\mathbb{R}}
\newcommand{\C}{\mathbb{C}}
\newcommand{\A}{\mathbb{A}}
\renewcommand{\P}{\mathbb{P}}
\newcommand{\eps}{\varepsilon}
\renewcommand{\phi}{\varphi}
\renewcommand{\emptyset}{\varnothing}
\newcommand{\<}{\langle}
\renewcommand{\>}{\rangle}
\newcommand{\isom}{\cong}
\newcommand{\eqc}{\overline}
\newcommand{\clo}{\overline}
\newcommand{\seq}{\subseteq}
\newcommand{\teq}{\trianglelefteq}
\DeclareMathOperator{\id}{id}
\DeclareMathOperator{\im}{im}
\newcommand{\inc}{\hookrightarrow}

% Extra Commands
\newcommand{\OO}{\mathcal{O}}
\newcommand{\FF}{\mathcal{F}}
\newcommand{\GG}{\mathcal{G}}

\newcommand{\tensor}{\otimes}
\newcommand{\dd}{\mathrm{d}}

% Document
\begin{document}
\title{MATH 237B Homework}
\author{Harry Coleman}
\date{2022}
\maketitle

\begin{pbox}[Exercise 8.6.1]
    Let $X$ be a smooth projective curve.
    For any point $P \in X$ consider the exact skyscraper sequence of sheaves on $X$
    \begin{cd}
        0 \rar & \omega_X \rar & \omega_X \tensor \OO_X(P) \rar & k_P \rar & 0.
    \end{cd}
    Show that the induces sequence of global sections is not exact, i.e., the last map $\Gamma(\omega_X \tensor \OO_X(P)) \to \Gamma(k_P)$ is not surjective.
\end{pbox}

Note that as $\dim X = 1$, we have $\omega_X = \omega_X$.

We obtain a long exact sequence of cohomologies
\begin{cd}[row sep=tiny]
    0 \rar & H^0(\omega_X) \rar & H^0(\omega_X \tensor \OO_X(P)) \rar & H^0(k_P) \\
    \rar & H^1(\omega_X) \rar & H^1(\omega_X \tensor \OO_X(P)) \rar & H^1(k_P) \\
    \rar & \cdots
\end{cd}
The $0$th cohomology is precisely the global sections, so we have the exact sequence
\begin{cd}
    \Gamma(\omega_X \tensor \OO_X(P)) \rar & \Gamma(k_P) \rar & H^1(\omega_X) \rar & H^1(\omega_X \tensor \OO_X(P)) \rar & H^1(k_P).
\end{cd}
Fix an affine open cover $\{U_i\}_{i \in I}$ of $X$ (with $I$ ordered).
Without loss of generality, assume $P \in U_{i_0}$ and and $P \notin U_{i_1}$ for all $i_1 \ne i_0$.
(If $U_{i_1}$ does contain $P$, then $U_{i_1} \setminus \clo{\{P\}}$ is an open subset of $X$, i.e., it is an open subscheme.
We can then replace $U_{i_1}$ in the cover of $X$ by an affine open cover of $U_{i_1}$.)
By definition of the skyscraper sheaf, we find that
\[
    C^1(k_P)
    = \prod_{i_0 < i_1} k_P(U_{i_0} \cap U_{1_i})
    = 0,
\]
and it follows that $H^1(k_P) = 0$.
This gives us the exact sequence
\begin{cd}
    \Gamma(\omega_X \tensor \OO_X(P)) \rar & \Gamma(k_P) \rar & H^1(\omega_X) \rar & H^1(\omega_X \tensor \OO_X(P)) \rar & 0.
\end{cd}
Not sure if there is anywhere to go from here in general, so I will simplify to an easy case.



Consider $X = \P^1$ with the fixed affine open cover $U_i = \{x_i \ne 0\}$ for $i = 0, 1$.
Additionally, we assume $P = [0 : 1]$ so that $P \in U_0$ but $P \notin U_1$.
Then
\[
    \Gamma(k_P)
        = H^0(k_P)
        = k_P(U_0) \times k_P(U_1)
        = k \times 0
        = k,
\]
which means in particular that $h^0(k_P) = 1$.
By Example 8.4.3,
\[
    h^1(\omega_X) = h^1(\P^1, \omega_{\P^1}) = 1.
\]
This means that the map $\Gamma(k_P) \to H^1(\omega_X)$ is either the zero map or surjective; we want to show that it is not the zero map, so we need $H^1(\omega_X \tensor \OO_X(P)) = 0$.
By Lemma 7.4.15,
\[
    \omega_X = \omega_{\P^1} = \OO_{\P^1}(-1-1) = \OO_X(-2).
\]
Additionally, since the line bundles on $\P^1$ are precisely the twisting sheaves $\OO_{\P^1}(n)$, then
\[
    \OO_X(P) = \OO_{\P^1}(\deg P) = \OO_{\P^1}(1) = \OO_X(1).
\]
Hence,
\[
    \omega_X \tensor \OO_X(P) = \OO_X(-2) \tensor \OO_X(1) = \OO_X(-1).
\]
Note that $\P^1$ is a genus $g = 0$ curve and $\deg \OO_X(-1) = -1 \geq 2g - 1$.
By the Kodaira vanishing theorem, we conclude that
\[
    H^1(\omega_X \tensor \OO_X(P)) = H^1(\OO_X(-1)) = 0.
\]
Hence, the map $\Gamma(k_P) \to H^1(\omega_X)$ is surjective, with each of dimension $1$, so the kernel is proper.
In other words, the image of the map $\Gamma(\omega_X \tensor \OO_X(P)) \to \Gamma(k_P)$ is proper, hence the map is not surjective.




\newpage
\begin{pbox}[Exercise 8.6.3]
    Compute the cohomology groups $H^i(\P^1 \times \P^1, p^*\OO_{\P^1}(a)) \tensor q^*\OO_{\P^1}(b))$ for all $a, b \in \Z$, where $p$ and $q$ denote the two projection maps from $\P^1 \times \P^1$ to $\P^1$.
\end{pbox}

A result from a paper online gives
\[
    H^i(\P^1 \times \P^1, p^*\OO_{\P^1}(a) \tensor q^*\OO_{\P^1}(b))
        \isom H^i(\P^1, \OO_{\P^1}(a)) \tensor H^i(\P^1, \OO_{\P^1}(b)).
\]
We have computed
\[
    H^i(\P^1, \OO_{\P^1}(d)) = \begin{cases}
        k[x_0, x_1]_d &\text{if } i = 0, \\
        k[x_0, x_1]_{-2-d} &\text{if } i = 1, \\
        0 &\text{otherwise},
    \end{cases}
\]
so
\[
    H^i = \begin{cases}
        k[x_0, x_1]_a \tensor k[y_0, y_1]_b &\text{if } i = 0, \\
        k[x_0, x_1]_{-2-a} \tensor k[y_0, y_1]_{-2-b} &\text{if } i = 1, \\
        0 &\text{otherwise}.
    \end{cases}
\]

\end{document}
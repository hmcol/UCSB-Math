\documentclass[12pt]{article}

% packages
\usepackage{kantlipsum}
\usepackage[margin=1in]{geometry}
\usepackage[labelfont=it]{caption}
\usepackage[table]{xcolor}
\usepackage{subcaption,framed,colortbl,multirow,enumitem}
\usepackage{amsmath,amsthm,amssymb,wasysym,mathrsfs,mathtools,babel}
\usepackage{tikz,graphicx,pgf,pgfplots}
\usetikzlibrary{arrows, angles, quotes, decorations.pathreplacing, math, patterns, calc}
\pgfplotsset{compat=1.16}

% Theorems
\newtheorem{theorem}{Theorem}
\newtheorem{lemma}{Lemma}
\newtheorem{proposition}{Proposition}

% Problem Box
\setlength{\fboxsep}{4pt}
\newsavebox{\mybox}
\newenvironment{problem}
    {\begin{lrbox}{\mybox}\begin{minipage}{\textwidth-10pt}}
    {\end{minipage}\end{lrbox}\framebox[6.5in]{\usebox{\mybox}}}

% Environments
\newenvironment{drawing}{\begin{center}\begin{tikzpicture}}{\end{tikzpicture}\end{center}}
\newenvironment{response}{\paragraph{}}{}

% Formatting
\newcommand{\ds}{\displaystyle}
\newcommand{\isp}[1]{\quad\text{#1}\quad}
\newcommand{\seq}[2]{\left\{#1\right\}_{#2=1}^\infty}
\newcommand{\clo}[1]{\overline{#1}}

% Paired Delimiters
\DeclarePairedDelimiter{\ceil}{\lceil}{\rceil}
\DeclarePairedDelimiter\floor{\lfloor}{\rfloor}
\DeclarePairedDelimiter{\ang}{\langle}{\rangle}

% Sets
\newcommand{\N}{\mathbb{N}}
\newcommand{\Z}{\mathbb{Z}}
\newcommand{\I}{\mathbb{I}}
\newcommand{\R}{\mathbb{R}}
\newcommand{\Q}{\mathbb{Q}}
\newcommand{\C}{\mathbb{C}}
\newcommand{\F}{\mathbb{F}}

% Misc Characters
\newcommand{\powerset}{\raisebox{.15\baselineskip}{\Large\ensuremath{\wp}}}
\let\eps\varepsilon
\let\emptyset\varnothing

% Functions
\newcommand{\id}[1]{\mathsf{id}_{#1}}

% Babel Shorthands
\useshorthands*{"}
\defineshorthand{"-}{\setminus}
\defineshorthand{"d}{\partial}

% Topology
\newcommand{\T}{\mathscr{T}}
\renewcommand{\S}{\mathscr{S}}
\newcommand{\B}{\mathscr{B}}
\renewcommand{\int}{\text{int}}
\newcommand{\diam}{\text{diam}}
 
\begin{document}
 
\title{Homework 1\\
    \large MATH 118A Intro to Real Analysis
}
\author{Harry Coleman}
\date{October 8, 2020}
\maketitle

\section*{Exercise 1}
\begin{problem}
    Prove that $\sqrt{2} + \sqrt{3}$ is irrational.
\end{problem}

\begin{proof}
    Suppose, for contradiction, that $\sqrt{2} + \sqrt{3} = p \in\Q$. Now consider
    \begin{align*}
        p^2 &= (\sqrt{2} + \sqrt{3})^2, \\
        p^2 &= 5 + 2\sqrt{6}, \\
        \frac{p^2-5}2 &= \sqrt{6},
    \end{align*}
    which implies that $\sqrt{6}\in\Q$. Let $r,s\in\Z$ such that $(r,s)=1$ and $\ds\frac{r}{s} = \sqrt6$. Then
    \begin{align*}
        \frac{r^2}{s^2} &= 6, \\
        r^2 &= 6s^2,
    \end{align*}
    which implies that $2|r$. Let $t\in\Z$ such that $r=2t$. Then
    \begin{align*}
        (2t)^2 &= 6s^2, \\
        2t &= 3s^2,
    \end{align*}
    which implies that $2|s$. However, $(r,s)=1$, which is a contradiction.
     
\end{proof}

\newpage
\section*{Exercise 2}
\begin{problem}
    Given nonempty subsets $A$ and $B$ of $\mathbb{R}$, let $C$ denote the set
    \[
     C = \{ x+y:\ x \in A,\ y\in B\}.
    \]
    If each $A$ and $B$ has a supremum, then prove that $C$ has a supremum and 
    \[
     \sup C = \sup A + \sup B.
    \]
\end{problem}

\begin{proof}
    Suppose both $A$ and $B$ have suprema. Let $x+y\in C$ with $x\in A$ and $y\in B$. Then by the definition of supremum,
    \[x+y \leq \sup A + \sup B.\]
    Therefore, $C$ is bounded above by $\sup A + \sup B$. Now suppose that $c<\sup A + \sup B$. By definition of supremum, there exists $x\in A$ such that
    \[c-\sup B < x \leq \sup A,\]
    and there exists some $y\in B$ such that
    \[c-x < y \leq \sup B,\]
    which gives us
    \[c < x + y \leq \sup A + \sup B.\]
    And since $x+y\in C$, we know that $c$ is not an upper bound for $C$. Therefore, $\sup A + \sup B$ is the supremum of $C$.
    
\end{proof}

\section*{Exercise 3}
\begin{problem}
    Given nonempty subsets $S$ and $T$ of $\mathbb{R}$ such that $S \subseteq T$ and $T$ has a supremum. Prove that $S$ has a supremum and $\sup S \le \sup T$. Give an example of two sets $S$ and $T$ in the real line such that $S \subset T$ (strict content), and $\sup S = \sup T$. 
\end{problem}

\begin{proposition}
    $S$ has a supremum and $\sup S \le \sup T$.
\end{proposition}

\begin{proof}
    Since $\sup T$ is an upper bound for $T$, then for any $s\in S\subseteq T$, we have that $s\leq \sup T$, so $\sup T$ is an upper bound for $S$. Since $S$ is bounded above, then by the least upper bound property of $\R$, $\sup S$ exists. Since $\sup T$ is an upper bound for $S$, then by definition of supremum, $\sup S \leq \sup T$.
    
\end{proof}

If $S=[0,1)$ and $T=[0,1]$, then $S\subset T\subseteq\R$ and $\sup S = \sup T=1$.


\section*{Exercise 4}
\begin{problem}
    Let $A$ be a nonempty set of real numbers which is bounded below. Let $-A$ be the set of all numbers $-x$, where $x\in A$. Prove that 
    \[
     \inf A = - \sup (-A).
    \]
\end{problem}

\begin{proof}
    Since $A\subseteq \R$ is bounded below, then by the least upper bound property, $\inf A$ exists. Let $-x\in-A$, that is, $x\in A$. By the definition of infimum, $\inf A \leq x$, equivalently, $-x\leq -\inf A$. Therefore, $-\inf A$ is an upper bound for $-A$. Now suppose $-c< -\inf A$, that is, $\inf A < c$. By definition of infimum, there exists some $x\in A$ such that
    \[\inf A \leq x < c,\]
    which implies that
    \[-c < -x \leq -\inf A.\]
    And since $x\in A$, then $-x\in-A$, so $-c$ is not an upper bound for $-A$. Therefore, $-\inf(A) = \sup(-A)$.
    
\end{proof}




\end{document}
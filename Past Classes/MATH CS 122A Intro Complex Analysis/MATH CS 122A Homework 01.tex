\documentclass[12pt]{article}

% packages
\usepackage{kantlipsum}
\usepackage[margin=1in]{geometry}
\usepackage[labelfont=it]{caption}
\usepackage[table]{xcolor}
\usepackage{subcaption,framed,colortbl,multirow,enumitem}
\usepackage{amsmath,amsthm,amssymb,wasysym,mathrsfs,mathtools,babel}
\usepackage{tikz,graphicx,pgf,pgfplots,tkz-euclide}
\usetikzlibrary{arrows, angles, quotes, decorations.pathreplacing, math, patterns, calc}
\pgfplotsset{compat=1.16}

% Theorems
\newtheorem{theorem}{Theorem}
\newtheorem{lemma}{Lemma}
\newtheorem{proposition}{Proposition}

% Problem Box
\setlength{\fboxsep}{4pt}
\newsavebox{\mybox}
\newenvironment{problem}
    {\begin{lrbox}{\mybox}\begin{minipage}{\textwidth-10pt}}
    {\end{minipage}\end{lrbox}\framebox[6.5in]{\usebox{\mybox}}}

% Environments
\newenvironment{drawing}{\begin{center}\begin{tikzpicture}}{\end{tikzpicture}\end{center}}
\newenvironment{response}{\paragraph{}}{}

% Formatting
\newcommand{\ds}{\displaystyle}
\newcommand{\isp}[1]{\quad\text{#1}\quad}
\newcommand{\seq}[2]{\left\{#1\right\}_{#2=1}^\infty}
\newcommand{\clo}[1]{\overline{#1}}
\newcommand{\conj}[1]{\overline{#1}}

% Paired Delimiters
\DeclarePairedDelimiter{\ceil}{\lceil}{\rceil}
\DeclarePairedDelimiter\floor{\lfloor}{\rfloor}
\DeclarePairedDelimiter{\ang}{\langle}{\rangle}

% Sets
\newcommand{\N}{\mathbb{N}}
\newcommand{\Z}{\mathbb{Z}}
\newcommand{\I}{\mathbb{I}}
\newcommand{\R}{\mathbb{R}}
\newcommand{\Q}{\mathbb{Q}}
\newcommand{\C}{\mathbb{C}}
\newcommand{\F}{\mathbb{F}}

% Misc Characters
\newcommand{\powerset}{\raisebox{.15\baselineskip}{\Large\ensuremath{\wp}}}
\let\eps\varepsilon
\let\emptyset\varnothing

% Functions
\newcommand{\id}[1]{\mathsf{id}_{#1}}

% Babel Shorthands
\useshorthands*{"}
\defineshorthand{"-}{\setminus}
\defineshorthand{"d}{\partial}

% Probability
\newcommand{\FF}{\mathcal{F}}
\renewcommand{\P}{\mathbb{P}}

% Complex Analysis
\renewcommand{\Im}{\text{Im }}
\renewcommand{\Re}{\text{Re }}
\newcommand{\Arg}{\text{Arg }}

 
\begin{document}
 
\title{Homework 1\\
    \large MATH CS 122A Complex Analysis I
}
\author{Harry Coleman}
\date{October 17, 2020}
\maketitle

\section*{Exercise I.1.1}
\begin{problem}
    Identify and sketch the set of points satisfying:
    
\end{problem}
\begin{enumerate}[label=(\alph*)]
    \item $|z-1-i| = 1$
    
    If $z=x+iy$, then
    \begin{align*}
        |x+iy-1-i| &= 1, \\
        |(x-1)+i(y-1)| &= 1, \\
        \sqrt{(x-1)^2 + (y-1)^2} &= 1, \\
        (x-1)^2 + (y-1)^2 = 1.
    \end{align*}
    This is the circle centered at $(1,1)$ with radius $1$:
    \begin{drawing}
        \tkzInit[xmax=3,ymax=2,xmin=-1,ymin=-1]
        \tkzGrid
        \tkzAxeXY
        
        \draw (1,1) circle (1cm);
    \end{drawing}
    
    \item $1<|2z-6|<2$
    
    If $z=x+iy$, then
    \begin{alignat*}{3}
        1 &<& |2z-6| &<& 2, \\
        \frac12 &<& |z-3| &<& 1, \\
        \frac12 &<& \sqrt{(x-3)^2 + y^2} &<& 1, \\
        \left(\frac12\right)^2 &<& (x-3)^2 + y^2 &<& 1.
    \end{alignat*}
    This is the outside of the circle centered at $(3,0)$ with radius $\ds\frac12$ intersected with the inside of the circle centered at $(3,0)$ with radius $1$:
    \begin{drawing}
        \draw[fill=lightgray, dashed] (3,0) circle (1cm);
        \draw[fill=white, dashed] (3,0) circle (0.5cm);
    
        \tkzInit[xmax=5,ymax=1,xmin=-1,ymin=-1]
        \tkzGrid
        \tkzAxeXY
    \end{drawing}
    
    \item $|z-1|^2 + |z+1|^2 < 8$
    
    Note that $|z|^2 = z\conj z$ for all $z\in\C$. Then
    \begin{align*}
        (z-1)(\conj{z-1}) + (z+1)(\conj{z+1}) &< 8, \\
        (z-1)(\conj z -1) + (z+1)(\conj z+1) &< 8, \\
        z\conj z - z - \conj z + 1 + z\conj z  + z + \conj z + 1 &< 8, \\
        2z\conj z + 2 < 8, \\
        z\conj z < 3, \\
        |z|^2 < 3, \\
        |z| < \sqrt{3}.
    \end{align*}
    This is the inside of the circle centered at $(0,0)$ with radius $\sqrt{3}$:
    \begin{drawing}
        \draw[fill=lightgray, dashed] (0,0) circle (1.732cm);
    
        \tkzInit[xmax=2,ymax=2,xmin=-2,ymin=-2]
        \tkzGrid
        \tkzAxeXY
    \end{drawing}
    
    \item $|z-1| + |z+1| \leq 2$
    
    This is the inside of the ``ellipse'' with foci at $(-1,0)$ and $(1,0)$ and constant distance $2$. However, since the distance between the foci is also two, this is simply the line segment between $(-1,0)$ and $(1,0)$:
    
    \begin{drawing}
        \tkzInit[xmax=2,ymax=1,xmin=-2,ymin=-1]
        \tkzGrid
        \tkzAxeXY
        
        \draw[fill=black, ultra thick] (-1,0) circle (2pt) -- (1,0) circle (2pt);
        
    \end{drawing}
    
    
    \item $|z-1|<|z|$
    \begin{align*}
        |z-1| &< |z| \\
        \sqrt{(z-1)\conj{(z-1)}} &< \sqrt{z\conj z} \\
        z\conj z - z - \conj z + 1 &< z\conj z \\
        1 &< z + \conj z \\
        1 &< 2\Re z \\
        \frac12 &< \Re z
    \end{align*}
    
    \begin{drawing}
        \draw[fill=lightgray, line width = 0pt] (0.5,-2) rectangle (2, 2);
        \draw[dashed, ultra thick] (0.5,-2)--(0.5,2);
        
        
        \tkzInit[xmax=2,ymax=2,xmin=-2,ymin=-2]
        \tkzGrid
        \tkzAxeXY
    \end{drawing}
    
    
    \item $0< \Im z < \pi$
    
    \begin{drawing}
        \draw[fill=lightgray, line width = 0pt] (-3,0) rectangle (3, 3.14);
        \draw[dashed, ultra thick] (-3,0)--(3,0) (-3,3.14)--(3,3.14);
        
        
        \tkzInit[xmax=3,ymax=5,xmin=-3,ymin=-1]
        \tkzGrid
        \tkzAxeXY
    \end{drawing}
    
    \item $-\pi < \Re z < \pi$
    
    \begin{drawing}
        \draw[fill=lightgray, line width = 0pt] (-3.14,-2) rectangle (3.14, 2);
        \draw[dashed, ultra thick] (-3.14,-2)--(-3.14,2) (3.14,-2)--(3.14,2);
        
        
        \tkzInit[xmax=4,ymax=2,xmin=-4,ymin=-2]
        \tkzGrid
        \tkzAxeXY
    \end{drawing}
    
    \item $|\Re z| < |z|$
    
    If $z=x+iy$, then
    \begin{align*}
        |\Re z| &< |z|, \\
        \sqrt{x^2} &< \sqrt{x^2 + y^2}, \\
        x^2 &< x^2 + y^2, \\
        0 &<y^2.
    \end{align*}
    This is the set of $z=x+iy\in\C$ such that $y\ne 0$:
    \begin{drawing}
        \draw[fill=lightgray, line width = 0pt] (-2,-2) rectangle (2, 2);
        \draw[dashed, ultra thick] (-2,0)--(2,0);
        
        
        \tkzInit[xmax=2,ymax=2,xmin=-2,ymin=-2]
        \tkzGrid
        \tkzAxeXY
    \end{drawing}
    
    \item $\Re(iz+2)>0$
    
    If $z=x+iy$, then
    \begin{align*}
        \Re(iz+2) &> 0, \\
        \Re(i(x+iy)+2) &> 0,\\
        \Re(ix-y+2) &> 0, \\
        -y+2 &> 0, \\
        2 &> y.
    \end{align*}
    \begin{drawing}
        \draw[fill=lightgray, line width = 0pt] (-3,-1) rectangle (3, 2);
        \draw[dashed, ultra thick] (-3,2)--(3,2);
        
        
        \tkzInit[xmax=3,ymax=3,xmin=-3,ymin=-1]
        \tkzGrid
        \tkzAxeXY
    \end{drawing}
    
    
    \item $|z-i|^2 + |z+i|^2 < 2$
    \begin{align*}
        |z-i|^2 + |z+i|^2 &< 2 \\
        (z-i)\conj{(z-i)} + (z+i)\conj{(z+i)} &<2 \\
        (z-i)(\conj z + i) + (z+i)(\conj z - i) &<2 \\
        z\conj z -iz +i\conj z - i^2 + z\conj z +iz -i\conj z - i^2 &<2 \\
        2z\conj z +2 &<2 \\
        z\conj z &< 0 \\
        |z|^2 &< 0 \\
        |z| &< 0.
    \end{align*}
    This has no solution, so it is the empty set.
    
\end{enumerate}


\section*{Exercise I.2.1}
\begin{problem}
    Express all values of the following expressions in both polar and Cartesian coordinates, and plot them.
\end{problem}

\begin{enumerate}[label=(\alph*)]
    \item $\sqrt{i}$
    
    Note that
    \[|i| = 1 \isp{and} \Arg i = \frac\pi2,\]
    so we have all values of $re^{i\alpha}$ where
    \[r=1^{1/2}=1 \isp{and} \alpha = \frac{\pi/2}2 + \frac{2\pi k}2=\frac\pi4+\pi k,\quad k\in\Z.\]
    The solution set is
    \begin{align*}
        e^{i\pi/4} &=\quad \frac1{\sqrt{2}} + i\frac1{\sqrt{2}}, \\
        e^{i5\pi/4} &=\quad -\frac1{\sqrt{2}} - i\frac1{\sqrt{2}}.
    \end{align*}
    
    \begin{drawing}
        \tkzInit[xmax=2,ymax=2,xmin=-2,ymin=-2]
        \tkzGrid
        \tkzAxeXY
        
        \draw[fill=black] (0.707,0.707) circle (2pt);
        \draw[fill=black] (-0.707,-0.707) circle (2pt);
    \end{drawing}
    
    \item $\sqrt{i-1}$
    
    Note that
    \[|i-1| = \sqrt{2} \isp{and} \Arg(i-1) = \frac{3\pi}4,\]
    so we have all values of $re^{i\alpha}$ where
    \[r=(\sqrt{2})^{1/2}=\sqrt[4]{2} \isp{and} \alpha = \frac{3\pi/4}2 + \frac{2\pi k}2=\frac{3\pi}8+\pi k,\quad k\in\Z.\]
     The solution set is
    \begin{align*}
        \sqrt[4]{2}e^{i3\pi/8} &=\quad \frac{\sqrt{2\sqrt{2}-2}}{2} + i\frac{\sqrt{2\sqrt{2}+2}}{2}, \\
        \sqrt[4]{2}e^{i11\pi/8} &=\quad -\frac{\sqrt{2\sqrt{2}-2}}{2} - i\frac{\sqrt{2\sqrt{2}+2}}{2}.
    \end{align*}
    \begin{drawing}
        \tkzInit[xmax=2,ymax=2,xmin=-2,ymin=-2]
        \tkzGrid
        \tkzAxeXY
        
        \draw[fill=black] (0.455,1.098) circle (2pt);
        \draw[fill=black] (-0.455,-1.098) circle (2pt);
    \end{drawing}
    
    \item $\sqrt[4]{-1}$
    
    Note that
    \[|-1| = 1 \isp{and} \Arg(-1) = \pi,\]
    so we have all values of $re^{i\alpha}$ where
    \[r=1^{1/4} = 1 \isp{and} \alpha = \frac\pi4 + \frac{2\pi k}4=\frac\pi4+\frac\pi2 k,\quad k\in\Z.\]
     The solution set is
     \begin{align*}
        e^{i\pi/4} &= \frac1{\sqrt{2}} + i\frac1{\sqrt{2}}, \\
        e^{i3\pi/4} &= -\frac1{\sqrt{2}} + i\frac1{\sqrt{2}}, \\
        e^{i5\pi/4} &= -\frac1{\sqrt{2}} - i\frac1{\sqrt{2}}, \\
        e^{i7\pi/4} &= \frac1{\sqrt{2}} - i\frac1{\sqrt{2}}.
    \end{align*}
    \begin{drawing}
        \tkzInit[xmax=2,ymax=2,xmin=-2,ymin=-2]
        \tkzGrid
        \tkzAxeXY
        
        \draw[fill=black] (0.707,0.707) circle (2pt);
        \draw[fill=black] (-0.707,-0.707) circle (2pt);
        \draw[fill=black] (-0.707,0.707) circle (2pt);
        \draw[fill=black] (0.707,-0.707) circle (2pt);
    \end{drawing}
    
    \item $\sqrt[4]{i}$
    
    Note that
    \[|i| = 1 \isp{and} \Arg(i) = \frac\pi2,\]
    so we have all values of $re^{i\alpha}$ where
    \[r=1^{1/4} = 1 \isp{and} \alpha = \frac{\pi/2}4 + \frac{2\pi k}4=\frac\pi8+\frac\pi2 k,\quad k\in\Z.\]
     The solution set is
     \begin{align*}
        e^{i\pi/8} &= \frac{\sqrt{2+\sqrt{2}}}{2} + i\frac{\sqrt{2-\sqrt{2}}}{2}, \\
        e^{i5\pi/8} &=  \frac{\sqrt{2-\sqrt{2}}}{2} - i\frac{\sqrt{2+\sqrt{2}}}{2}, \\
        e^{i9\pi/8} &= -\frac{\sqrt{2+\sqrt{2}}}{2} - i\frac{\sqrt{2-\sqrt{2}}}{2}, \\
        e^{i13\pi/8} &= -\frac{\sqrt{2-\sqrt{2}}}{2} + i\frac{\sqrt{2+\sqrt{2}}}{2}.
    \end{align*}
    \begin{drawing}
        \tkzInit[xmax=2,ymax=2,xmin=-2,ymin=-2]
        \tkzGrid
        \tkzAxeXY
        
        \draw[fill=black] (0.923,0.382) circle (2pt);
        \draw[fill=black] (-0.923,-0.382) circle (2pt);
        \draw[fill=black] (-0.382,0.923) circle (2pt);
        \draw[fill=black] (0.382,-0.923) circle (2pt);
    \end{drawing}
    
    \item $(-8)^{1/3}$
    
    Note that
    \[|-8| = 8 \isp{and} \Arg(-8) = \pi,\]
    so we have all values of $re^{i\alpha}$ where
    \[r=8^{1/3} = 2 \isp{and} \alpha = \frac\pi3 + \frac{2\pi k}3=\frac\pi3 + \frac{2\pi}3 k,\quad k\in\Z.\]
     The solution set is
     \begin{align*}
        e^{i\pi/3} &= \frac12 + i\frac{\sqrt{3}}2, \\
        e^{i\pi} &= -1, \\
        e^{i5\pi/3} &= \frac12 - i\frac{\sqrt{3}}2.
    \end{align*}
    \begin{drawing}
        \tkzInit[xmax=2,ymax=2,xmin=-2,ymin=-2]
        \tkzGrid
        \tkzAxeXY
        
        \draw[fill=black] (0.5,0.866) circle (2pt);
        \draw[fill=black] (-1,0) circle (2pt);
        \draw[fill=black] (0.5,-0.866) circle (2pt);
    \end{drawing}
    
    \item $(3-4i)^{1/8}$
    
    Note that
    \[|3-4i| = 5 \isp{and} \Arg(3-4i) = \tan^{-1}(4/3),\]
    so we have all values of $re^{i\alpha}$ where
    \[r=5^{1/8} = \sqrt[8]{5} \isp{and} \alpha = \frac{\tan^{-1}(4/3)}8 + \frac{2\pi k}8=\frac{\tan^{-1}(4/3)}8  + \frac{2\pi}8 k,\quad k\in\Z.\]
    The solution set is
     \begin{drawing}
        \tkzInit[xmax=2,ymax=2,xmin=-2,ymin=-2]
        \tkzGrid
        \tkzAxeXY
        
        \draw[fill=black] (1.214,0.141) circle (2pt);
        \draw[fill=black] (0.758,0.958) circle (2pt);
        \draw[fill=black] (-0.141,1.21) circle (2pt);
        \draw[fill=black] (-0.95,0.75) circle (2pt);
        \draw[fill=black] (-1.2,-0.14) circle (2pt);
        \draw[fill=black] (-0.75,-0.95) circle (2pt);
        \draw[fill=black] (0.14,-1.21) circle (2pt);
        \draw[fill=black] (0.95,-0.75) circle (2pt);
    \end{drawing}
    
    \item $(1+i)^8$
    
    Note that
    \[|1+i| = \sqrt{2} \isp{and} \Arg(1+i) = \pi/4,\]
    so 
    \[(1+i)^8 = \sqrt{2}^8 e^{i8(\pi/4)} = 16e^{i2\pi} = 16.\]
    
    \item $\ds\left(\frac{1+i}{\sqrt{2}}\right)^{25}$
    
    Note that
    \[|\frac{1+i}{\sqrt{2}}| = 1 \isp{and} \Arg(1+i) = \pi/4,\]
    so
    \[\left(\frac{1+i}{\sqrt{2}}\right)^{25} = 1^{25}e^{i25\pi/4} = e^{i\pi/4} = \frac{1+i}{\sqrt{2}}.\]
    
\end{enumerate}


\section*{Exercise I.2.2}
\begin{problem}
    Sketch the following sets:
\end{problem}

\begin{enumerate}[label=(\alph*)]
    \item $|\arg z| < \pi/4$
    
    \begin{drawing}
        \draw[dashed, ultra thick, fill=lightgray] (2,2)--(0,0)--(2,-2);
        
        
        \tkzInit[xmax=2,ymax=2,xmin=-2,ymin=-2]
        \tkzGrid
        \tkzAxeXY
    \end{drawing}
    
    \item $0 < \arg(z-1-i) < \pi/3$
    
    \begin{drawing}
        \draw[dashed, ultra thick, fill=lightgray] (1.155,2)--(-1,-1)--(2,-1);
        \draw[line width=0pt, fill=lightgray] (1.155,2)--(2,2)--(2,-1);
        
        
        \tkzInit[xmax=2,ymax=2,xmin=-2,ymin=-2]
        \tkzGrid
        \tkzAxeXY
    \end{drawing}
    
    \item $|z| = \arg z$
    
    \begin{drawing}
        \draw [domain=0:3.14,variable=\t,smooth,samples=75]
        plot ({\t r}: {\t});
        
        
        \tkzInit[xmax=2,ymax=2,xmin=-4,ymin=-2]
        \tkzGrid
        \tkzAxeXY
    \end{drawing}
    
    \item $\log|z| = -2\arg z$
    
    \begin{drawing}
        \draw [domain=-0.6:5,variable=\t,smooth,samples=75]
        plot ({\t r}: {exp(-2*\t)});
        
        
        \tkzInit[xmax=4,ymax=1,xmin=-2,ymin=-2]
        \tkzGrid
        \tkzAxeXY
    \end{drawing}
    
\end{enumerate}


\section*{Exercise I.2.5}
\begin{problem}
    For $n\geq 1$, show that
\end{problem}

\begin{enumerate}[label=(\alph*)]
    \item $\ds1+z+z^2+\cdots+z^n = \frac{1-z^{n+1}}{1-z}, \quad z\ne1$,
    Let
    \[S_n = 1+z + z^2 + \cdots + z^n.\]
    Then
    \[zS_n = z + z^2 + \cdots + z^n + z^{n=1},\]
    And we find
    \[S_n-zS_n = 1-z^{n+1} \implies S_n = \frac{1-z^{n+1}}{1-z}.\]
    
    
    
    \item $1+\cos\theta + \cos2\theta + \cdots + \cos n\theta = \frac12 + \frac{\sin(n+\frac12)}{2\sin(\theta/2)}$.
    
    \begin{align*}
        1+\cos\theta + \cos2\theta + \cdots + \cos n\theta
            &= 1 + \frac12(e^{i\theta} + e^{-i\theta}) + \frac12(e^{i2\theta} + e^{-i2\theta}) + \cdots + \frac12(e^{in\theta} + e^{-in\theta}) \\
            &= \frac12\left[(1 + e^{i\theta} + e^{i2\theta} + \cdots + e^{in\theta}) + (1 + e^{-i\theta} + e^{-i2\theta} + \cdots + e^{-in\theta})\right] \\
            &= \frac12\left[\frac{1-e^{i(n+1)\theta}}{1-e^{i\theta}} + \frac{1-e^{-i(n+1)\theta}}{1-e^{-i\theta}}\right] \\
            &= \frac12\frac{(1-e^{i(n+1)\theta})(1-e^{-i\theta}) + (1-e^{-i(n+1)\theta})(1-e^{i\theta})}{(1-e^{i\theta})(1-e^{-i\theta})} \\
            &= \frac12\frac{1-e^{-i\theta} - e^{i(n+1)\theta} + e^{in\theta} + 1 - e^{i\theta} - e^{-i(n+1)\theta} + e^{-in\theta}}{1 - e^{-i\theta} - e^{i\theta} + e^0} \\
            &= \frac12\frac{2 - (e^{i\theta} + e^{-i\theta}) + (e^{in\theta} + e^{-in\theta}) - (e^{i(n+1)\theta} + e^{-i(n+1)\theta})}{1-(e^{i\theta} + e^{-i\theta})} \\
            &= \frac12\frac{2 - 2\cos\theta + 2\cos n\theta - 2\cos(n+1)\theta}{2-2\cos\theta} \\
            &= \frac12 + \frac12\frac{\cos n\theta - \cos(n+1)\theta}{1-\cos\theta}
    \end{align*}
    I was unable to complete the proof.
    
\end{enumerate}


\section*{Exercise I.5.1}
\begin{problem}
    Calculate and plot $e^z$ for the following points $z$:
\end{problem}

\begin{enumerate}[label=(\alph*)]
    \item $0$
    
    \[e^0 = 1.\]
    \begin{drawing}
        \draw[fill=black] (1,0) circle (2pt);
        
        
        \tkzInit[xmax=2,ymax=2,xmin=-2,ymin=-2]
        \tkzGrid
        \tkzAxeXY
    \end{drawing}
    
    
    \item $\pi i + 1$
    
    \[e^{\pi i +1} = e^{i\pi}e^1 = -e.\]
    \begin{drawing}
        \draw[fill=black] (-2.718,0) circle (2pt);
        
        
        \tkzInit[xmax=2,ymax=2,xmin=-3,ymin=-2]
        \tkzGrid
        \tkzAxeXY
    \end{drawing}
    
    \item $\pi(i-1)/3$
    
    \[e^{\pi(i-1)/3} = e^{i\pi}e^{-\pi/3} = -e^{-\pi/3}.\]
    \begin{drawing}
        \draw[fill=black] (-0.350,0) circle (2pt);
        
        
        \tkzInit[xmax=2,ymax=2,xmin=-2,ymin=-2]
        \tkzGrid
        \tkzAxeXY
    \end{drawing}
    
    \item $37\pi i$
    
    \[e^{i37\pi} = e^{i\pi} = -1.\]
    \begin{drawing}
        \draw[fill=black] (-1,0) circle (2pt);
        
        
        \tkzInit[xmax=2,ymax=2,xmin=-2,ymin=-2]
        \tkzGrid
        \tkzAxeXY
    \end{drawing}
    
    \item $\pi i/m, \quad m=1,2,3,\dots$
    
    \[e^{i\pi/m}.\]
    \begin{drawing}
        \draw[fill=black] (-1,0) circle (2pt);
        \draw[fill=black] (0,1) circle (2pt);
         \draw[fill=black] (0.5,0.866) circle (2pt);
         \draw[fill=black] (0.707,0.707) circle (2pt);
         \draw[fill=black] (0.809,0.587) circle (2pt);
         \draw[fill=black] (0.866,0.5) circle (2pt);
         \draw[fill=black] (0.9,0.433) circle (2pt);
         \draw[fill=black] (0.923,0.382) circle (2pt);
        \draw[line width=4pt] (0.923,0.382) -- (1,0);
        
        \tkzInit[xmax=2,ymax=2,xmin=-2,ymin=-2]
        \tkzGrid
        \tkzAxeXY
    \end{drawing}
    
    \item $m(i-1), \quad m=1,2,3,\dots$
    
    \begin{drawing}
        \draw[fill=black] (0.198,0.309) circle (2pt);
        \draw[fill=black] (-0.056,0.123) circle (2pt);
        \draw[fill=black] (-0.049,0.007) circle (2pt);
        
        
        \tkzInit[xmax=2,ymax=2,xmin=-2,ymin=-2]
        \tkzGrid
        \tkzAxeXY
    \end{drawing}
\end{enumerate}





\end{document}
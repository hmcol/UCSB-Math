\documentclass[12pt]{article}

% Packages
\usepackage[margin=1in]{geometry} % proper margins
\usepackage{enumitem} % custom numbering for lists
\usepackage{amsmath} % align, cases, eqref, matrices, dots, roots, delimiters, math mode functions, mod, arrows
\usepackage{amsthm} % theorems, proofs
\usepackage{amssymb} % fancy letters, niche relations/negations
\usepackage{mathrsfs} % much more loopy calligraphy font

% Theorems
\newtheorem{theorem}{Theorem}
\newtheorem{lemma}{Lemma}
\newtheorem{proposition}{Proposition}

% Problem Box
\setlength{\fboxsep}{4pt}
\newsavebox{\mybox}
\newenvironment{problem}
    {\begin{lrbox}{\mybox}\begin{minipage}{0.98\textwidth}}
    {\end{minipage}\end{lrbox}\framebox[\textwidth]{\usebox{\mybox}}}

% Formatting
\newcommand{\ds}{\displaystyle}
\newcommand{\isp}[1]{\quad\text{#1}\quad}

% Alternate Characters
\let\eps\varepsilon % double curved, rather than single curve with midline
\let\phi\varphi % single stroke loop, rather than vertical line through circle
\let\emptyset\varnothing % circular, rather than tall

% Named Sets
\newcommand{\N}{\mathbb{N}} % natural numbers
\newcommand{\Z}{\mathbb{Z}} % integers 
\newcommand{\Q}{\mathbb{Q}} % rational numbers
\newcommand{\R}{\mathbb{R}} % real numbers
\newcommand{\C}{\mathbb{C}} % complex numbers

% Fancy Characters
\newcommand{\F}{\mathbb{F}} % arbitrary field
\renewcommand{\P}{\mathbb{P}} % probability measure (apparently, sometimes prime numbers)
\newcommand{\FF}{\mathcal{F}} % sigma algebra
\newcommand{\BB}{\mathcal{B}} % Borel sigma algebra

% Paired Delimiters
\newcommand{\ceil}[1]{\left\lceil #1 \right\rceil} % ceiling
\newcommand{\floor}[1]{\left\lfloor #1 \right\rfloor} % floor
\newcommand{\<}{\left\langle} % left angle bracket
\renewcommand{\>}{\right\rangle} % right angle bracket

% Functions
\renewcommand{\Im}{\operatorname{Im}} % imaginary part of a complex number
\renewcommand{\Re}{\operatorname{Re}} % real part of a complex number
\newcommand{\Arg}{\operatorname{Arg}} % principal argument (angle) of complex number
\newcommand{\Log}{\operatorname{Log}} % principal log of complex number

% Simplified Notation
\newcommand{\id}[1]{\operatorname{id_{\mathnormal{#1}}}} % identity operator
\newcommand{\seq}[2][n]{\left\{#2\right\}_{#1\in\N}} % sequence
\renewcommand{\d}[1]{\operatorname{d}\!#1} % differential operator
%\renewcommand{\d}{\,d} % differential operator
\newcommand{\od}[3][1]{\ifnum#1=1{\frac{\d #2}{\d #3}}\else{\frac{\d^{#1}#2}{\d #3^{#1}}}\fi} % ordinary derivative
\newcommand{\pd}[3][1]{\ifnum#1=1{\frac{\partial #2}{\partial#3}}\else{\frac{\partial^{#1}#2}{\partial#3^{#1}}}\fi} % partial derivative
\newcommand{\intr}[1]{\accentset{\circ}{#1}} % interior of a set

% Renaming
\let\sm\setminus % set minus (difference)
\let\clo\overline % closure of a set
\let\conj\overline % conjugate of an object
\let\eqc\overline % equivalence class of an object
\let\teq\trianglelefteq % normal subgroup
\let\iso\cong % isomorphic (groups)

% Notes
% medskip for header-less paragraph
% intertext{} for short text inside big display structure
% dots is dynamic based on surroundings
% dfrac and tfrac to force large or small fractions
% operatorname for new operators instead of text
% consider using nath; it seems to break most formatting and is not compatible with many packages

\begin{document}
 
\title{Homework 4\\
    %\large MATH CS 121 Intro to Probability
    %\large MATH 111A Intro to Abstract Algebra
    \large MATH CS 122A Complex Analysis I
    %\large MATH 118A Intro to Real Analysis
    %\large MATH 104A Intro to Numerical Analysis
}
\author{Harry Coleman}
\date{November 25, 2020}
\maketitle

\section*{1 Exercise 4.1.1}
\begin{problem}
    Let $\gamma$ be the boundary of the triangle $\{0<y<1-x, 0<x<1\}$, with the usual counterclockwise orientation. Evaluate the following integrals
\end{problem}

\subsection*{Exercise 4.1.1(a)}
\begin{problem}
    \[\int_\gamma\Re z \d{z}\]
\end{problem}
\medskip

We first parameterize the three segments of $\gamma$ as follows:
\begin{align*}
    \gamma_1:[0,1] &\to \C & \gamma_2:[0,1] &\to \C & \gamma_3:[0,1] &\to \C\\
        t &\mapsto t + i0, & t &\mapsto (1-t) + it, & t &\mapsto 0 + i(1-t).
\end{align*}
Then we evaluate the integral.
\begin{align*}
    \int_\gamma\Re z \d{z}
        &= \int_{\gamma_1}\Re z \d{z} + \int_{\gamma_2}\Re z \d{z} + \int_{\gamma_3}\Re z \d{z} \\
        &= \int_0^1\Re(\gamma_1(t))\gamma_1'(t) \d{t} + \int_0^1\Re(\gamma_2(t))\gamma_2'(t) \d{t} + \int_0^1\Re(\gamma_3(t))\gamma_3'(t) \d{t} \\
        &= \int_0^1 t\cdot 1\d{t} + \int_0^1 (1-t)\cdot(-1+i)\d{t} + \int_0^1 0\cdot(-i)\d{t} \\
        &= \int_0^1 t \d{t} + (-1+i)\int_0^1(1-t)\d{t} + \int_0^1 0 \d{t} \\
        &= \frac12 + (-1+i)\frac12 + 0 \\
        &= i\frac12.
\end{align*}

\newpage
\subsection*{Exercise 4.1.1(b)}
\begin{problem}
    \[\int_\gamma\Im z \d{z}\]
\end{problem}
\medskip

Using the same parameterization as in Exercise 4.1.1(a), we evaluate the integral.
\begin{align*}
    \int_\gamma\Im z \d{z}
        &= \int_{\gamma_1}\Im z \d{z} + \int_{\gamma_2}\Im z \d{z} + \int_{\gamma_3}\Im z \d{z} \\
        &= \int_0^1\Im(\gamma_1(t))\gamma_1'(t) \d{t} + \int_0^1\Im(\gamma_2(t))\gamma_2'(t) \d{t} + \int_0^1\Im(\gamma_3(t))\gamma_3'(t) \d{t} \\
        &= \int_0^1 0\cdot1 \d{t} + \int_0^1 t\cdot(-1+i)\d{t} + \int_0^1 (1-t)\cdot(-i)\d{t} \\
        &= \int_0^1 0 \d{t} + (-1+i)\int_0^1 t \d{t} + (-i)\int_0^1 (1-t) \d{t} \\
        &= 0 + (-1+i)\frac12 + (-i)\frac12 \\
        &= -\frac12.
\end{align*}

\subsection*{Exercise 4.1.1(c)}
\begin{problem}
    \[\int_\gamma z \d{z}\]
\end{problem}
\medskip

Using the values solved for in Exercises 4.1.1(a) and (b),
\[\int_\gamma z \d{z} = \int_\gamma (\Re z + i \Im z) \d{z} = \int_\gamma \Re z \d{z} + i\int_\gamma \Im z \d{z} = i\frac12 + i\left(-\frac12\right) = 0.\]
This agrees with Cauchy's theorem on the triangle $\gamma$ and function $f(z)=z$, differentiable on domain $\C$.

\newpage
\section*{2 Exercise 4.1.2}
\begin{problem}
    Let $\gamma$ be the circle $\{|z| = R\}$, with the usual counterclockwise orientation. Evaluate the following integrals, for $m=0,\pm1,\pm2$,\dots.
\end{problem}

\subsection*{Exercise 4.1.2(a)}
\begin{problem}
    \[\int_\gamma z^m \d{z}\]
\end{problem}
\medskip

We parameterize $\gamma$ as follows
\begin{align*}
    \gamma:[0,2\pi] &\to \C \\
        \theta &\mapsto e^{i\theta}.
\end{align*}
Then the integral becomes
\begin{align*}
    \int_\gamma z^m \d{z}
        &= \int_0^{2\pi}\gamma(\theta)^m\gamma'(\theta) \d{\theta} \\
        &= \int_0^{2\pi} (e^{i\theta})^m\cdot ie^{i\theta} \d{\theta} \\
        &= i\int_0^{2\pi} e^{i(m+1)\theta} \d{\theta}.
\end{align*}
Now if $m=-1$, then
\begin{align*}
    \int_\gamma z^m \d{z}
        &= i\int_0^{2\pi} e^{i(-1+1)\theta} \d{\theta} \\
        &= i\int_0^{2\pi} e^{0} \d{\theta}\\
        &= i\int_0^{2\pi} 1 \d{\theta}\\
        &= i2\pi.
\end{align*}
And if $m\ne-1$, then
\begin{align*}
    \int_\gamma z^m \d{z}
        &= i\int_0^{2\pi} e^{i(m+1)\theta} \d{\theta} \\
        &= i\left(\frac{e^{i(m+1)2\pi}}{m+1} -  \frac{e^{i(m+1)0}}{m+1}\right)\\
        &= i\left(\frac{e^{i2\pi}}{m+1} -  \frac{e^{i0}}{m+1}\right)\\
        &= 0.
\end{align*}

For $m\geq0$, this agrees with Cauchy's theorem on the closed curve $\gamma$ and function $f(z)=z^m$, which is is complex differentiable on $\C$. Note, however, that Cauchy's theorem does not tell us anything when $m<0$. If $m<0$, then $f(z)=z^m$ is only complex differentiable on $\C\setminus\{0\}$, which is not simply connected. Moreover, any domain $D\subseteq\C$ containing $\gamma$ will not be simply connected, since $0$ is in the interior of $\gamma$. In particular, when $m=-1$, we do not obtain the result of Cauchy's theorem.

\subsection*{Exercise 4.1.2(b)}
\begin{problem}
    \[\int_\gamma\conj{z}^m \d{z}\]
\end{problem}
\medskip

We parameterize $\gamma$ the same as in Exercise 4.1.2(a), then evaluate the integral.
\begin{align*}
    \int_\gamma \conj{z}^m \d{z}
        &= \int_0^{2\pi}\left(\conj{\gamma(\theta)}\right)^m\gamma'(\theta) \d{\theta} \\
        &= \int_0^{2\pi} \left(\conj{e^{i\theta}}\right)^m\cdot ie^{i\theta} \d{\theta} \\
        &= i\int_0^{2\pi} \left(e^{\conj{i\theta}}\right)^m\cdot e^{i\theta} \d{\theta} \\
        &= i\int_0^{2\pi} e^{-im\theta}\cdot e^{i\theta} \d{\theta} \\
        &= i\int_0^{2\pi} e^{i(1-m)\theta} \d{\theta}.
\end{align*}
Now if $m=1$, then
\begin{align*}
    \int_\gamma \conj{z}^m \d{z}
        &= i\int_0^{2\pi} e^{-i(1-1)\theta} \d{\theta} \\
        &= i\int_0^{2\pi} e^{0} \d{\theta}\\
        &= i\int_0^{2\pi} 1 \d{\theta}\\
        &= i2\pi.
\end{align*}
And if $m\ne1$, then
\begin{align*}
    \int_\gamma \conj{z}^m \d{z}
        &= i\int_0^{2\pi} e^{-i(1-m)\theta} \d{\theta} \\
        &= i\left(\frac{e^{-i(1-m)2\pi}}{1-m} -  \frac{e^{-i(1-m)0}}{1-m}\right)\\
        &= i\left(\frac{e^{i2\pi}}{1-m} -  \frac{e^{i0}}{1-m}\right)\\
        &= 0.
\end{align*}

\subsection*{Exercise 4.1.2(c)}
\begin{problem}
    \[\int_\gamma\conj{z}^m|\d{z}|\]
\end{problem}
\medskip

Note that we have the following equivalent representation of our parameterization of $\gamma$:
\[\gamma(\theta) = e^{i\theta} = \cos\theta + i\sin\theta.\]
Then the integral becomes
\begin{align*}
    \int_\gamma\conj{z}^m|\d{z}|
        &= \int_0^{2\pi}\left(\conj{\gamma(\theta)}\right)^m\sqrt{\left(\od{}{\theta}\cos\theta\right)^2 + \left(\od{}{\theta}\sin\theta\right)^2}\d{\theta} \\
        &= \int_0^{2\pi}\left(\conj{e^{i\theta}}\right)^m\sqrt{(-\sin\theta)^2 + (\cos\theta)^2}\d{\theta} \\
        &= \int_0^{2\pi}\left(e^{\conj{i\theta}}\right)^m\sqrt{\sin^2\theta + \cos^2\theta}\d{\theta} \\
        &= \int_0^{2\pi}\left(e^{-i\theta}\right)^m\sqrt{1}\d{\theta} \\
        &= \int_0^{2\pi}e^{-im\theta}\d{\theta}.
\end{align*}
Now if $m=0$, then
\begin{align*}
    \int_\gamma\conj{z}^m|\d{z}|
        &= \int_0^{2\pi} e^{-i0\theta} \d{\theta} \\
        &= \int_0^{2\pi} e^0 \d{\theta} \\
        &= \int_0^{2\pi} 1 \d{\theta}\\
        &= 2\pi.
\end{align*}
And if $m\ne0$, then
\begin{align*}
    \int_\gamma\conj{z}^m|\d{z}|
        &= \int_0^{2\pi} e^{-im\theta} \d{\theta} \\
        &= \left(\frac{e^{-im2\pi}}{m} -  \frac{e^{-im0}}{m}\right)\\
        &= \left(\frac{e^{i2\pi}}{m} -  \frac{e^{i0}}{m}\right)\\
        &= 0.
\end{align*}


\newpage
\section*{3 Exercise 4.1.4}
\begin{problem}
    Show that if $D$ is a bounded domain with smooth boundary, then
    \[\int_{\partial D} \conj{z}\d{z} = 2i\operatorname{Area}(D).\]
\end{problem}

\begin{proof}
    Splitting the integral into its components, where $z=x+iy$, and and applying Green's theorem to the smooth simple closed curve $\partial D$, we obtain
    \begin{align*}
        \int_{\partial D} \conj{z}\d{z}
            &= \int_{\partial D} \conj{x + iy}(\d{x} + i \d{y}) \\e
            &= \int_{\partial D} (x - iy)\d{x} + (x - iy)i\d{y} \\
            &= \int_{\partial D} (x - iy)\d{x} + (y + ix)\d{y} \\
            &= \iint_D \left(\pd{}{x}(y + ix) - \pd{}{y}(x - iy)\right)\d{x}\d{y} \\
            &= \iint_D (i - (-i))\d{x}\d{y} \\
            &= 2i\iint_D \d{x}\d{y} \\
            &= 2i\operatorname{Area}(D).
    \end{align*}
    
\end{proof}

\newpage
\section*{4 Exercise 4.1.5}
\begin{problem}
    Show that
    \[\left|\oint_{|z-1|=1} \frac{e^z}{z+3}\d{z}\right| \leq 2\pi e^2.\]
\end{problem}

\begin{proof}
    For any $z\in\C$ such that $|z-1|=1$, we have
    \begin{align*}
        |z+3|
            &= |(z-1) - (-4)| \\
            &\geq ||z-1| - |-4|| \\
            &= |1-4| \\
            &= |-3| \\
            &= 3.
    \end{align*}
    Now for any $z\in\C$ with $z=x+iy$, we have
    \begin{align*}
        |e^z|
            &= |e^xe^{iy}| \\
            &= |e^x|\cdot|\cos y + i\sin y| \\
            &= e^x\cdot\sqrt{\cos^2 y + i\sin^2 y} \\
            &= e^x\cdot\sqrt{1} \\
            &= e^x.
    \end{align*}
    Additionally, if $|z-1|=1$, then
    \[|z-1| = \sqrt{(x-1)^2 + y^2} = 1 \quad\implies\quad 0\leq x \leq 2.\]
    And since the exponential function is increasing, we have
    \[e^x \leq e^2.\]
    Thus, for all $z\in\C$ such that $|z-1|=1$, we have
    \begin{align*}
        \left|\frac{e^z}{z+3}\right|
            &= \frac{|e^z|}{|z+3|} \\
            &= \frac{e^x}{|z+3|} \\
            &\leq \frac{e^x}{3} \\
            &\leq e^x \\
            &\leq e^2.
    \end{align*}
    And since the curve of $|z-1|=1$ is the unit circle centered at $1+i0$, its length is the circumference of a unit circle, which is $2\pi$. Now by the $ML$-inequality, we have
    \[\left|\oint_{|z-1|=1} \frac{e^z}{z+3}\d{z}\right| \leq e^2\cdot 2\pi = 2\pi e^2.\]
    
\end{proof}

\newpage
\section*{5 Exercise 4.1.6}
\begin{problem}
    Show that
    \[\left|\oint_{|z|=R} \frac{\Log z}{z^2}\d{z}\right| \leq 2\sqrt{2}\pi\frac{\log R}{R}, \qquad R > e^\pi.\]
\end{problem}

\begin{proof}
    For any $z\in\C$ such that $|z|=R$, we have
    \begin{align*}
        \left|\frac{\Log z}{z^2}\right|
            &= \frac{|\log|z| + i\Arg z|}{|z|^2} \\[1em]
            &= \frac{|\log R + i\Arg z|}{R^2}.
    \end{align*}
    Note that $\log R \in\R$ and $\Arg z \in[-\pi,\pi]\subseteq\R$. Then
    \begin{align*}
        \left|\frac{\Log z}{z^2}\right|
            &= \frac{\sqrt{(\log R)^2 + (\Arg z)^2}}{R^2} \\[1em]
            &\leq \frac{\sqrt{(\log R)^2 + (\pi)^2}}{R^2} \\[1em]
            &= \frac{\sqrt{(\log R)^2 + (\log e^\pi)^2}}{R^2} \\[1em]
            &\leq \frac{\sqrt{(\log R)^2 + (\log R)^2}}{R^2} \\[1em]
            &= \frac{\sqrt{2(\log R)^2}}{R^2} \\[1em]
            &= \frac{\sqrt{2}\log R}{R^2}.
    \end{align*}
    And since the circumference of the circle $|z|=R$ is $2\pi R$, then by the $ML$-inequality, we have
    \[\left|\oint_{|z|=R} \frac{\Log z}{z^2}\d{z}\right| \leq \frac{\sqrt{2}\log R}{R^2} \cdot 2\pi R = 2\sqrt{2}\pi\frac{\log R}{R}.\]
    
\end{proof}

\newpage
\section*{6 Exercise 4.1.8}
\begin{problem}
    Suppose the continuous function $f(e^{i\theta})$ on the unit circle satisfies $|f(e^{i\theta})| \leq M$ and $|\int_{|z|=1}f(z)\d{z}| = 2\pi M$. Show that $f(z)= c\conj{z}$ for some constant $c$ with modulus $|c| = M$.
\end{problem}
\medskip

It can be shown that $|f(e^{i\theta})| = M$ for all $\theta \in [0, 2\pi]$. If we assume for contradiction that it is not true, then it is strictly less than $M$ for some $\theta$ and we obtain that
\[\left|\int_{|z|=1}f(z)\d{z}\right| \leq \int_{|z|=1}|f(z)|\d{z} < 2\pi M,\]
which is a contradiction. I was unable to complete the proof after this.

\newpage
\section*{7 Exercise 4.3.3}
\begin{problem}
    Let $f(z)= c_0 + c_1 z + \cdots + c_n z^n$ be a polynomial.
\end{problem}

\subsection*{Exercise 4.3.3(a)}
\begin{problem}
    If the $c_k$'s are real, show that
    \[\int_{-1}^1 f(x)^2 dx \leq \pi \int_0^{2\pi}|f(e^{i\theta})|^2 \frac{d\theta}{2\pi} = \pi \sum_{k=0}^n c_k^2.\]
    \emph{Hint.} For the first inequality, apply Cauchy's theorem to the function $f(z)^2$ separately on the top half and the bottom half of the unit disk.
\end{problem}

\begin{proof}
    Suppose $c_k\in\R$ for $k = 0, \dots, n$. Then $f(x)$ is real for all $x \in \R$. Now because $f(x)^2 \geq 0$ for all $x \in \R$, then
    \[\int_{-1}^1 f(x)^2 \d{x} = \left| \int_{-1}^1 f(x)^2 \d{x} \right|.\]
    Denote by $\gamma_1$ the top half of the unit circle, i.e., the curve $e^{i\theta}$ for $\theta\in[0,\pi]$. Similarly, denote by $\gamma_2$ the bottom half of the unit circle, i.e., the curve $e^{i\theta}$ for $\theta\in[\pi,2\pi]$. Finally, denote by $\gamma_3$ the line segment from $-1$ to $1$, i.e., the curve $z = x + iy$ for $x \in [-1, 1]$ and $y = 0$. First, we see that the definition of $\gamma_3$ gives us
    \[\int_{\gamma_3} f(z)^2 \d{z} = \int_{\gamma_3} f(x + iy)^2 (\d{x} + i \d{y}) = \int_{-1}^1 f(x)^2 \d{x}.\]
    Now since $f^2$ is a polynomial, it is entire. So by Cauchy's theorem, we have
    \[\int_{\gamma_1} f(z)^2 \d{z} +  \int_{\gamma_3} f(z)^2 \d{z} = \int_{\gamma_1 \cup \gamma_3} f(z)^2 \d{z} = 0,\]
    \[- \int_{\gamma_3} f(z)^2 \d{z} + \int_{\gamma_2} f(z)^2 \d{z} = \int_{-\gamma_3 \cup \gamma_2} f(z)^2 = 0,\]
    where $\gamma_1 \cup \gamma_3$ is the piecewise smooth boundary of the top half of the unit disk, and $-\gamma_3 \cup \gamma_2$ the bottom half. These two equalities now imply that
    \[\int_{\gamma_3} f(z)^2 \d{z} = - \int_{\gamma_1} f(z)^2 \d{z} = \int_{\gamma_2} f(z)^2 \d{z},\]
    and, moreover, that
    \[\int_{\gamma_3} f(z)^2 \d{z} = - \frac12 \int_{\gamma_1} f(z)^2 \d{z} + \frac12 \int_{\gamma_2} f(z)^2 \d{z}.\]
    We now apply the regular triangle inequality and the triangle inequality for integrals to obtain
    \begin{align*}
        \int_{-1}^1 f(x)^2 \d{x}
            &= \left| - \frac12 \int_{\gamma_1} f(z)^2 \d{z} + \frac12 \int_{\gamma_2} f(z)^2 \d{z} \right| \\[1em]
            &\leq \frac12 \left| \int_{\gamma_1} f(z)^2 \d{z} \right| + \frac12 \left| \int_{\gamma_2} f(z)^2 \d{z} \right| \\[1em]
            &\leq \frac12 \int_{\gamma_1} |f(z)|^2 |\d{z}|  + \frac12  \int_{\gamma_2} |f(z)|^2 |\d{z}| \\[1em] 
            &= \frac12 \int_{\gamma_1 \cup \gamma_2} |f(z)|^2 |\d{z}|.
    \end{align*}
    Note that $\gamma_1 \cup \gamma_2$ is precisely the unit circle, i.e., the curve $e^{i\theta}$ for $\theta \in [0, 2\pi]$. Thus, we obtain the desired inequality
    \begin{align*}
        \int_{-1}^1 f(x)^2 \d{x}
            &\leq \frac12 \oint_{|z| = 1} |f(z)|^2 |\d{z}| \\[1em]
            &= \pi \cdot \frac1{2\pi} \int_0^{2\pi} |f(e^{i\theta})|^2 \d{\theta} \\[1em]
            &= \pi \int_0^{2\pi} |f(e^{i\theta})|^2 \frac{\d{\theta}}{2\pi}.
    \end{align*}
    We now begin solving for the desired equality:
    \allowdisplaybreaks
    \begin{align*}
        \pi \int_0^{2\pi} |f(e^{i\theta})|^2 \frac{\d{\theta}}{2\pi}
            &= \frac12 \int_0^{2\pi} |f(e^{i\theta})|^2 \d{\theta} \\[1em]
            &= \frac12 \int_0^{2\pi} \left| \sum_{k=0}^n c_k e^{ik\theta} \right|^2 \d{\theta} \\[1em]
            &= \frac12 \int_0^{2\pi} \left( \sum_{k=0}^n c_k e^{ik\theta} \right) \left(\conj{ \sum_{k=0}^n c_k e^{ik\theta} }\right) \d{\theta} \\[1em]
            &= \frac12 \int_0^{2\pi} \left( \sum_{k=0}^n c_k e^{ik\theta} \right) \left(\sum_{k=0}^n \conj{c_k} e^{-ik\theta} \right) \d{\theta} \\[1em]
            &= \frac12 \int_0^{2\pi} \left( \sum_{k=0}^n c_k e^{ik\theta} \right) \left(\sum_{k=0}^n c_k e^{-ik\theta} \right) \d{\theta} \\[1em]
            &= \frac12 \int_0^{2\pi} \sum_{k=0}^n \sum_{\ell=0}^n c_k e^{ik\theta} c_\ell e^{-i\ell\theta} \d{\theta} \\[1em]
            &= \frac12 \sum_{k=0}^n \sum_{\ell=0}^n c_kc_\ell \int_0^{2\pi} e^{i(k - \ell)\theta} \d{\theta}.
    \end{align*}
    If $k = \ell$, then
    \begin{align*}
        \int_0^{2\pi} e^{i(k - \ell)\theta} \d{\theta}
            &= \int_0^{2\pi} e^{i(0)\theta} \d{\theta} \\[1em]
            &= \int_0^{2\pi} 1 \d{\theta} \\[1em]
            &= 2\pi.
    \end{align*}
    If $k \ne \ell$, then
    \begin{align*}
        \int_0^{2\pi} e^{i(k - \ell)\theta} \d{\theta}
            &= \frac{-i}{k-\ell} \left[ e^{i(k - \ell)\theta} \right]_0^{2\pi} \\[1em]
            &= \frac{-i}{k-\ell} \left( e^{i(k - \ell)2\pi} - e^{i(k - \ell)20} \right) \\[1em]
            &= \frac{-i}{k-\ell} \left( 1 - 1 \right) \\[1em]
            &= 0.
    \end{align*}
    Therefore, we have
    \begin{align*}
        \pi \int_0^{2\pi} |f(e^{i\theta})|^2 \frac{\d{\theta}}{2\pi}
            &= \frac12 \sum_{k=0}^n \sum_{\ell=0}^n c_kc_\ell (2\pi \delta_{k\ell}) \\[1em]
            &= \pi \sum_{k=0}^n \sum_{\ell=0}^n c_kc_\ell \delta_{k\ell},
    \end{align*}
    where $\delta_{k\ell}$ is the Kronecker delta function. Thus, we obtain the desire equality
    \[\pi \int_0^{2\pi} |f(e^{i\theta})|^2 \frac{\d{\theta}}{2\pi}= \pi \sum_{k=0}^n c_k^2.\]
    
\end{proof}

\newpage
\subsection*{Exercise 4.3.3(b)}
\begin{problem}
    If the $c_k$'s are complex, show that
    \[\int_{-1}^1 |f(x)|^2 \d{x}  \leq \pi \int_0^{2\pi}|f(e^{i\theta})|^2 \frac{\d{\theta}}{2\pi} = \pi \sum_{k=0}^n|c_k|^2.\]
\end{problem}

\begin{proof}
    If each $c_k \in \C$, then we define $a_k = \Re c_k$ and $b_k = \Im c_k$ for $k = 0, \dots, n$. Now if $f = u + iv$ for real-valued functions $u$ and $v$, then for any $x \in \R$, we have
    \begin{align*}
        f(x)
            &= \sum_{k=0}^n c_k x^k \\
            &= \sum_{k=0}^n (a_k + ib_k) x^k \\
            &= \sum_{k=0}^n a_kx^k + i\sum_{k=0}^n b_k x^k \\
            &= u(x) + i v(x).
    \end{align*}
    Then
    \begin{align*}
        \int_{-1}^1 |f(x)|^2 \d{x} 
            &= \int_{-1}^1 |u(x) + iv(x)|^2 \d{x}  \\
            &= \int_{-1}^1 \sqrt{u(x)^2 + v(x)^2}^2 \d{x}  \\
            &= \int_{-1}^1 u(x)^2 + v(x)^2 \d{x} \\
            &= \int_{-1}^1 u(x)^2 \d{x}+ \int_{-1}^1 v(x)^2 dx \\
    \end{align*}
    Then by Exercise 4.3.3(a), we have
    \allowdisplaybreaks
    \begin{align*}
        \int_{-1}^1 |f(x)|^2 \d{x} 
            &\leq \pi \sum_{k=0}^n a_k^2 + \pi \sum_{k=0}^n b_k^2 \\
            &= \pi \sum_{k=0}^n a_k^2 + b_k^2 \\
            &= \pi \sum_{k=0}^n \sqrt{a_k^2 + b_k^2}^2 \\
            &= \pi \sum_{k=0}^n |a_k + ib_k|^2 \\
            &= \pi \sum_{k=0}^n |c_k|^2.
    \end{align*}
    The proof of the equality is the same as the proof of the equality from Exercise 4.3.3(a), but without the assumption that $\conj{c_k}$ is equal to $c_k$. In this case, the result is
    \[\pi \int_0^{2\pi}|f(e^{i\theta})|^2 \frac{\d{\theta}}{2\pi} = \pi \sum_{k=0}^n c_k\conj{c_k}.\]
    And since $c_k\conj{c_k} = |c_k|^2$, then the desired equality is obtained.
    
    
\end{proof}

\subsection*{Exercise 4.3.3(c)}
\begin{problem}
    Establish the following variant of \textbf{Hilbert's inequality}, that
    \[\left|\sum_{j,k=0}^n \frac{c_jc_k}{j+k+1}\right| \leq \pi \sum_{k=0}^n|c_k|^2,\]
    with strict inequality unless the complex numbers $c_0,\dots,c_n$ are all zero. \emph{Hint.} Start by evaluating $\int_0^1f(x)^2 dx$.
\end{problem}

\begin{proof}
    If all the $c_k$'s are zero, then the inequality is trivial, as it is simply the equality $0 = 0$. Without loss of generality, we assume some of the $c_k$'s are nonzero. We first evaluate the following integral:
    \begin{align*}
        \int_0^1f(x)^2 \d{x}
            &= \int_0^1 \left( \sum_{k=0}^n c_k x^k \right)^2 \d{x} \\
            &= \int_0^1 \sum_{k,\ell = 0}^n c_k c_\ell x^{k + \ell}  \d{x} \\
            &= \sum_{k,\ell = 0}^n c_k c_\ell \int_0^1 x^{k + \ell}  \d{x} \\
            &= \sum_{k,\ell = 0}^n c_k c_\ell \frac{1}{k + \ell + 1} \\
            &= \sum_{k,\ell = 0}^n \frac{c_k c_\ell }{k + \ell + 1}.
    \end{align*}
    Therefore, we have
    \[\left| \sum_{k,\ell = 0}^n \frac{c_k c_\ell }{k + \ell + 1} \right| = \left| \int_0^1f(x)^2 \d{x} \right| \leq \int_0^1 |f(x)|^2 \d{x}.\]
    Since we assumed that $f$ is not the zero polynomial, then it has at most finitely many zeros on the real interval $[-1, 0]$. Therefore,
    \[0 < \int_{-1}^0 |f(x)|^2 \d{x},\]
    and we have the strict inequality
    \[\left| \sum_{k,\ell = 0}^n \frac{c_k c_\ell }{k + \ell + 1} \right| < \int_0^1 |f(x)|^2 \d{x} + \int_{-1}^0 |f(x)|^2 \d{x} = \int_{-1}^1 |f(x)|^2 \d{x}.\]
    Exercise 4.3.3(b) now gives us that
    \[\int_{-1}^1 |f(x)|^2 \d{x} \leq \pi \sum_{k=0}^n |c_k|^2,\]
    and we obtain the desired inequality
    \[\left| \sum_{k,\ell = 0}^n \frac{c_k c_\ell }{k + \ell + 1} \right| < \pi \sum_{k=0}^n |c_k|^2.\]
    
\end{proof}



\end{document}
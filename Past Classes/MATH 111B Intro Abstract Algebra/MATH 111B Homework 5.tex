\documentclass[12pt]{article}

% Packages
\usepackage[margin=1in]{geometry}
\usepackage{amsmath, amsthm, amssymb, physics, enumitem}

% Problem Box
\setlength{\fboxsep}{4pt}
\newsavebox{\savefullbox}
\newenvironment{fullbox}{\begin{lrbox}{\savefullbox}\begin{minipage}{\dimexpr\textwidth-2\fboxsep\relax}}{\end{minipage}\end{lrbox}\begin{center}\framebox[\textwidth]{\usebox{\savefullbox}}\end{center}}
\newenvironment{pbox}[1][]{\begin{fullbox}\ifx#1\empty\else\paragraph{#1}\fi}{\end{fullbox}}

% Options
\allowdisplaybreaks
\addtolength{\jot}{4pt}
\theoremstyle{definition}

% Default Commands
\newtheorem{proposition}{Proposition}
\newtheorem{lemma}{Lemma}
\newcommand{\ds}{\displaystyle}
\newcommand{\isp}[1]{\quad\text{#1}\quad}
\newcommand{\N}{\mathbb{N}}
\newcommand{\Z}{\mathbb{Z}}
\newcommand{\Q}{\mathbb{Q}}
\newcommand{\R}{\mathbb{R}}
\newcommand{\C}{\mathbb{C}}
\newcommand{\eps}{\varepsilon}
\renewcommand{\phi}{\varphi}
\renewcommand{\emptyset}{\varnothing}

% Extra Commands
\newcommand{\isom}{\cong}
\newcommand{\eqc}{\overline}

% Document Info
\title{\vspace{-0.5in}Homework 5\\
    \large MATH 111B
}
\author{Harry Coleman}
\date{February 12, 2021}

% Begin Document
\begin{document}
\maketitle


\begin{pbox}[Q1 Problem 8.2.5]
    Let $R$ be the quadratic integer ring $\Z[\sqrt{-5}]$. Define the ideals $I_2=(2,1+\sqrt{-5})$, $I_3=(3,2+\sqrt{-5})$, and $I_3'=(3,2-\sqrt{-5})$
\end{pbox}

\begin{pbox}[(a)]
    Prove that $I_2, I_3,$ and $I_3'$ are nonprincipal ideals in $R$. (Note that Example 2 following Proposition 1 proves this for $I_3$.)
\end{pbox}

\begin{proof}
    The proofs for $I_3$ and $I_3'$ are similar. Suppose, for contradiction, that $I_3' = (a + b\sqrt{-5})$ is principal. Then for some $\alpha, \beta \in \Z[\sqrt{-5}]$, we have
    \[
        3 = \alpha(a + b\sqrt{-5}) \isp{and} 2 - \sqrt{-5} = \beta(a + b\sqrt{-5}).
    \]
    Applying the associated norm, which is multiplicative, to both sides of the first equation, we find
    \[
        9 = N(3) = N(\alpha)N(a + b\sqrt{-5}) = N(\alpha)(a^2 + 5b^2).
    \]
    Since $a^2 + 5b^2$ is a positive integer, then it must be a factor of $9$, i.e., $1$, $3$, or $9$. It cannot be $3$, as $b \ne 0$ implies $a^2 + 5b^2 \geq 5$, and $b = 0$ would imply that $3$ is the square of $a$. If it is $9$, then $N(\alpha) = 1$, which implies $\alpha = \pm1$. Then $a + b\sqrt{-5} = \pm 3$, but this is impossible as the coefficients of $2 - \sqrt{-5}$ are not divisible by $3$. Lastly, if $N(a + b\sqrt{-5}) =a^2 + 5b^2 = 1$, then we would have $a + b\sqrt{-5} = \pm1$, and $1 = (a + b\sqrt{-5})^2 \in I_3'$. Then for some $\gamma, \delta \in \Z[\sqrt{-5}]$, we have
    \begin{align*}
        3\gamma + (2 - \sqrt{-5})\delta &= 1 \\
        3\gamma(2 + \sqrt{-5}) + (4 + 5)\delta &= 2 + \sqrt{-5} \\
        3(\gamma(2 + \sqrt{-5}) + 3\delta) &= 2 + \sqrt{-5},
    \end{align*}
    which is a contradiction. Hence $I_3'$ is not a principal ideal of $\Z[\sqrt{-5}]$.
    
    Suppose, again for contradiction, that $I_2 = (a + b\sqrt{-5})$. Then for some $\alpha\in \Z[\sqrt{-5}]$, we have $2 = \alpha(a + b\sqrt{-5})$. Taking norms, we find $2 = N(\alpha)(a^2 + 5b^2)$. Then $a^2 + 5b^2$ must be $1$ or $2$. In either case, $b = 0$. It cannot be $2$, since $2$ is not a perfect square. If it is $1$, then $1 = (a + b\sqrt{-5})^2 \in I_2$. Then for some $\gamma, \delta \in \Z[\sqrt{-5}]$,
    \begin{align*}
        2\gamma + (1 + \sqrt{-5})\delta &= 1 \\
        2\gamma(1 - \sqrt{-5}) + (1 + 5)\delta &= 1 - \sqrt{-5} \\
        2(\gamma(1 - \sqrt{-5}) + 3\delta) &= 1 - \sqrt{-5},
    \end{align*}
    which is a contradiction. Hence $I_2$ is not a principal ideal of $\Z[\sqrt{-5}]$.
    
\end{proof}

\begin{pbox}[(b)]
    Prove that the product of two nonprincipal ideals can be principal by show that $I_2^2$ is the principal ideal generated by 2, i.e., $I_2^2 = (2)$.
\end{pbox}

\begin{proof}
    The elements of $I_2^2$ are the elements of the form
    \[
        2^2\alpha + 2(1 + \sqrt{-5})\beta + (1 + \sqrt{-5})^2\gamma
    \]
    for $\alpha, \beta, \gamma \in \Z[\sqrt{-5}]$. Moreover,
    \begin{align*}
        2^2\alpha + 2(1 + \sqrt{-5})\beta + (1 + \sqrt{-5})^2\gamma
            &= 4\alpha + 2(1 + \sqrt{-5})\beta + (1 + 2\sqrt{-5} - 5)\gamma \\
            &= 4\alpha + 2(1 + \sqrt{-5})\beta + (-4 + 2\sqrt{-5})\gamma \\
            &= 2(2\alpha + (1 + \sqrt{-5})\beta + (-2 + \sqrt{-5})\gamma),
    \end{align*}
    so $I_2^2 \subseteq (2)$. In particular, taking $\alpha = -4$, $\beta = 0$, and $\gamma = -2 + \sqrt{-5}$, we obtain the element of $I_2^2$
    \begin{align*}
        2(2(-4) + (-2 + \sqrt{-5})(-2 - \sqrt{-5}))
            = 2(-8 + 4 + 5) 
            = 2.
    \end{align*}
    Hence, $2 \in I_2^2$, so $(2) \subseteq I_2^2$, giving us equality.
    
\end{proof}

\begin{pbox}[(c)]
    Prove similarly that $I_2 I_3= (1-\sqrt{-5})$ and $I_2 I_3' = (1+\sqrt{-5})$ are principal. Conclude that the principal ideal $(6)$ is the product of 4 ideals: $(6)=I_2^2 I_3 I_3'$.
\end{pbox}

\begin{proof}
    The elements of $I_2 I_3$ are the elements of the form
    \[
        6\alpha + 2(2 + \sqrt{-5})\beta + 3(1 + \sqrt{-5})\gamma + (1 + \sqrt{-5})(2 + \sqrt{-5})\delta
    \]
    for $\alpha, \beta, \gamma, \delta \in \Z[\sqrt{-5}]$. First,
    \[
        6\alpha = (1 - \sqrt{-5})(1 + \sqrt{-5})\alpha \in (1 - \sqrt{-5}).
    \]
    In particular, $6 \in (1 - \sqrt{-5})$, which tells us
    \[
        2(2 + \sqrt{-5})\beta = (6 - 2(1 - \sqrt{-5}))\beta \in (1 - \sqrt{-5}),
    \]
    and
    \[
         3(1 + \sqrt{-5})\gamma = (6 - 3(1 - \sqrt{-5}))\gamma \in (1 - \sqrt{-5}).
    \]
    Lastly,
    \[
         (1 + \sqrt{-5})(2 + \sqrt{-5})\delta = -3(1 - \sqrt{-5})\delta \in (1 - \sqrt{-5}),
    \]
    so $I_2I_3 \subseteq (1 - \sqrt{-5})$. Taking $\beta = 1$ and $\gamma = -1$, we obtain the element of $I_2I_3$
    \begin{align*}
        2(2 + \sqrt{-5})\beta + 3(1 + \sqrt{-5})\gamma
            &= 4 + 2\sqrt{-5} - 3 - 3\sqrt{-5} \\
            &= 1 - \sqrt{-5}.
    \end{align*}
    Hence, $1 - \sqrt{-5} \in I_2I_3$, so $(1 - \sqrt{-5}) \subseteq I_2I_3$, giving us equality.
    
    The elements of $I_2 I_3'$ are the elements of the form
    \[
        6\alpha + 2(2 - \sqrt{-5})\beta + 3(1 + \sqrt{-5})\gamma + (1 + \sqrt{-5})(2 - \sqrt{-5})\delta
    \]
    for $\alpha, \beta, \gamma, \delta \in \Z[\sqrt{-5}]$. First,
    \[
        6\alpha = (1 - \sqrt{-5})(1 + \sqrt{-5})\alpha \in (1 + \sqrt{-5}).
    \]
    In particular, $6 \in (1 + \sqrt{-5})$, which tells us
    \[
        2(2 - \sqrt{-5})\beta = (6 - 2(1 + \sqrt{-5}))\beta \in (1 + \sqrt{-5}).
    \]
    Clearly,
    \[
         3(1 + \sqrt{-5})\gamma \in (1 + \sqrt{-5}),
    \]
    and
    \[
         (1 + \sqrt{-5})(2 - \sqrt{-5})\delta \in (1 - \sqrt{-5}),
    \]
    so $I_2I_3' \subseteq (1 + \sqrt{-5})$. Taking $\alpha = -1$ and $\delta = 1$, we obtain the element of $I_2I_3'$
    \begin{align*}
        6\alpha + (1 + \sqrt{-5})(2 - \sqrt{-5})\delta
            &= -6 + 2 + \sqrt{-5} + 5 \\
            &= 1 + \sqrt{-5}.
    \end{align*}
    Hence, $1 + \sqrt{-5} \in I_2I_3'$, so $(1 + \sqrt{-5}) \subseteq I_2I_3'$, giving us equality.
    
    Then the elements of $I_2^2 I_3 I_3' = I_2I_3I_2I_3' = (1 - \sqrt{-5})(1 + \sqrt{-5})$ are the elements of the form
    \[
        (1 - \sqrt{-5})(1 + \sqrt{-5})\alpha = (1 + 5) \alpha = 6\alpha,
    \]
    for $\alpha \in \Z[\sqrt{-5}]$. These are precisely the elements of $(6)$, so they are equal.
        
\end{proof}


\newpage
\begin{pbox}[Q2]
    Find all the ideals in $\Q[x]$ containing $x^2 - 3x + 2$.
\end{pbox}

Trivially, we have ideals $(x^2 - 3x + 2)$ and $\Q[x]$ containing $x^2 - 3x + 2$. Since
\[
    x^2 - 3x + 2 = (x - 1)(x - 2),
\]
then the ideals $(x-1)$ and $(x-2)$ contain $x^2 - 3x +2$. We claim that these are the only ideals.

Suppose $I$ is an ideal of $\Q[x]$ containing $x^2 - 3x + 2$. Since $\Q[x]$ is a principal ideal domain, then $I = (p(x))$ for some nonzero $p(x) \in \Q[x]$. If $\deg p(x) = 0$, then $p(x) \in \Q^\times = (\Q[x])^\times$, so $(p(x)) = \Q[x]$. If $\deg p(x) = 2$, then $x^2 - 3x + 2 \in (p(x))$ implies
\[
    x^2 - 3x + 2 = ap(x)
\]
for some $a \in \Q^\times$. Multiplying by $a^{-1}$ on both sides gives us $p(x) \in (x^2 - 3x + 2)$, and we deduce that $(p(x)) = (x^2 - 3x + 2)$. Lastly, if $\deg p(x) = 1$, then there is some $q(x) \in \Q[x]$ with $\deg q(x) = 1$ and
\[
    x^2 - 3x + 2 = p(x)q(x).
\]
This is another factorization of $x^2 - 3x + 2$ into irreducibles, so for some $a \in \Q^\times$, we have $p(x) = a(x-1)$, in which case $(p(x)) = (x-1)$, or $p(x) = a(x-2)$, in which case $(p(x)) = (x-2)$.



\newpage
\begin{pbox}[Q3]
    Find all the ideals in $\Z[x]$ containing $2$ and $x^2 + 1$.
\end{pbox}

We have $\Z[x]/(2) \isom (\Z/2\Z)[x]$ and $(2, x^2 +1)/(2) \isom (x^2 + \eqc{1}) \subseteq (\Z/2\Z)[x]$. Then $(2) \subseteq (2, x^2 +1)$, so the third isomorphism theorem gives us
\[
    \Z[x]/(2, x^2+1) \isom (\Z[x]/(2))/((2, x^2+1)/(2)) \isom (\Z/2\Z)[x]/(x^2 + \eqc{1}) = S.
\]
By the fourth isomorphism theorem, the map between the ideals of $\Z[x]$ containing $(2, x^2 + 1)$ and the ideals of $S$ induced by the natural projection is an inclusion preserving bijection. In $S$, we have the elements $\eqc{0}, \eqc{1}, \eqc{x}, \eqc{x+1}$. The elements $\eqc{1}$ and $\eqc{x}$ are units, and we have
\begin{align*}
    \eqc{x+1} \cdot \eqc{1} &= \eqc{x+1} \\
    \eqc{x+1} \cdot \eqc{x} &= \eqc{x^2 + x} = \eqc{x - 1} = \eqc{x+1} \\
    \eqc{x+1} \cdot \eqc{x+1} &= \eqc{x^2 + 2x + 1} = \eqc{x^2 + 1} = \eqc{0}.
\end{align*}
So there are three ideals, namely $\{\eqc{0}\}$, $\{\eqc{0}, \eqc{x+1}\}$, and $\{\eqc{0}, \eqc{1}, \eqc{x}, \eqc{x+1}\}$. Taking the preimages under the natural projection gives us the ideals $(2, x^2+1)$, $(2, x^2 + 1, x+1)$, and $\Z[x]$. More specifically,
\[
    x^2 + 1 = -2x + (x+1)^2 \in (2, x+1) \subseteq (2, x^2 + 1, x+1),
\]
so $(2, x^2 + 1, x+1) = (2, x+1)$.



\end{document}
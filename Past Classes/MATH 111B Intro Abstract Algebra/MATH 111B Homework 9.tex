\documentclass[12pt]{article}

% Packages
\usepackage[margin=1in]{geometry}
\usepackage{fancyhdr}
\usepackage{amsmath, amsthm, amssymb, physics}

% Page Style
\fancypagestyle{plain}{
    \fancyhf{}
    \renewcommand{\headrulewidth}{0pt}
    \renewcommand{\footrulewidth}{0pt}
    \fancyfoot[R]{\thepage}
}
\pagestyle{plain}

% Problem Box
\setlength{\fboxsep}{4pt}
\newsavebox{\savefullbox}
\newenvironment{fullbox}{\begin{lrbox}{\savefullbox}\begin{minipage}{\dimexpr\textwidth-2\fboxsep\relax}}{\end{minipage}\end{lrbox}\begin{center}\framebox[\textwidth]{\usebox{\savefullbox}}\end{center}}
\newenvironment{pbox}[1][]{\begin{fullbox}\ifx#1\empty\else\paragraph{#1}\fi}{\end{fullbox}}

% Options
\renewcommand{\thesubsection}{\thesection(\alph{subsection})}
\allowdisplaybreaks
\addtolength{\jot}{4pt}
\theoremstyle{definition}

% Default Commands
\newtheorem{proposition}{Proposition}
\newtheorem{lemma}{Lemma}
\newcommand{\ds}{\displaystyle}
\newcommand{\isp}[1]{\quad\text{#1}\quad}
\newcommand{\N}{\mathbb{N}}
\newcommand{\Z}{\mathbb{Z}}
\newcommand{\Q}{\mathbb{Q}}
\newcommand{\R}{\mathbb{R}}
\newcommand{\C}{\mathbb{C}}
\newcommand{\eps}{\varepsilon}
\renewcommand{\phi}{\varphi}
\renewcommand{\emptyset}{\varnothing}
\newcommand{\pfrac}[2]{\left(\frac{#1}{#2}\right)}

% Extra Commands
\newcommand{\isom}{\cong}
\newcommand{\eqc}{\overline}
\newcommand{\Mat}[1]{\operatorname{M}_{#1, #1}}
\newcommand{\dsum}{\oplus}
\newcommand{\Dsum}{\bigoplus}
\DeclareMathOperator{\Tor}{Tor}
\newcommand{\Hom}{\operatorname{Hom}}

% Document Info
\fancypagestyle{title}{
    \renewcommand{\headrulewidth}{0.4pt}
    \setlength{\headheight}{15pt}
    \fancyhead[R]{Harry Coleman}
    \fancyhead[L]{MATH 111B Homework 9}
    \fancyhead[C]{March 12, 2021}
}

% Begin Document
\begin{document}
\thispagestyle{title}




\begin{lemma}
    Let $R$ be an integral domain and $a, b \in R$ nonzero. Then $a(R/(b)) = (a, b)/(b)$ as $R$-modules, i.e., the identity map is an $R$-module isomorphism between the two. 
\end{lemma}

\begin{proof}
    First, $a(R/(b))$ is an $R$-submodule of $R/(b)$. It is nonempty since $a0 = 0 \in a(R/(b))$. And for any $r \in R$ and $a\eqc{x}, a\eqc{y} \in a(R/(b))$, we have $r(a\eqc{x}) + a\eqc{y} = a\eqc{rx + y} \in a(R/(b))$. Second, $(a)$ and $(b)$ are $R$-submodules of $R$, so $(a, b)$ is an $R$-submodule of $R$. Hence, $(a, b)/(b)$ is an $R$-submodule of $R/(b)$.
    
    We will now show that $a(R/(b)) = (a, b)/(b)$ as sets. Let $a\eqc{x} \in a(R/(b))$, then $ax + b0 \in (a, b)$, so
    \[
        a\eqc{x} = \eqc{ax} + \eqc{b0} = \eqc{ax + b0} \in (a, b)/(b).
    \]
    Let $\eqc{ax + by} \in (a, b)/(b)$, then
    \[
        \eqc{ax + by} = \eqc{ax} + \eqc{by} = a\eqc{x} \in a(R/(b)).
    \]
    Hence, the identity map is a bijection between the $R$-modules. Since the identity map is trivially an $R$-module homomorphism, then it is in fact an isomorphism.
    
\end{proof}

\begin{lemma}(Problem 11)
    Let $R$ be a PID, let $a$ be a nonzero element of $R$ and let $M = R/(a)$. For any prime $p$ of $R$,
    \[
        p^{k-1}M/p^kM \isom \begin{cases}
            R/(p) &\text{if } k \leq n \\
            0 &\text{if } k > n,
        \end{cases}
    \]
    where $n$ is the power of $p$ dividing $a$ in $R$.
\end{lemma}

\begin{proof}
    By Lemma 1 and the third isomorphism theorem for $R$-modules, we have
    \[
        p^{k-1}M/p^kM 
            = ((p^{k-1}, a)/(a))/((p^k, a)/(a))
            \isom (p^{k-1}, a)/(p^k, a).
    \]
    Let $b \in R$ such that $a = p^nb$ and $p \nmid b$. If $k \leq n$, then
    \[
        (a) = (p^nb) \subseteq (p^n) \subseteq (p^k) \subseteq (p^{k-1}).
    \]
    This implies $(p^{k-1}, a) = (p^{k-1})$ and $(p^k, a) = (p^k)$, so
    \[
        p^{k-1}M/p^kM  \isom (p^{k-1})/(p^k).
    \]
    The map $\phi : R \to (p^{k-1})$ defined by $\phi(x) = p^{k-1}x$ is a surjective $R$-module homomorphism, since $R$ is commutative. The natural projection $\pi : (p^{k-1}) \to (p^{k-1})/(p^k)$ is also a surjective $R$-module homomorphism and, therefore, so is the composition $\psi = \pi \circ \phi$. We now examine the kernel of $\psi$. $x \in \ker \psi$ if and only if $p^{k-1}x = \phi(x) \in \ker\pi = (p^k)$, which is equivalent to $x \in (p)$. Thus, the first isomorphism theorem gives us $R/(p) \isom p^{k-1}M/p^kM$.
    
    Now suppose $k > n$, so $p^n \mid p^{k-1} \mid p^k$. Assume $d \in R$ is not a unit with $d \mid p^{k-1}$ and $d \mid a$. Since $p$ is prime, then the first condition implies $d$ is a power of $p$, and the second condition implies that it is no more than $p^n$, so $p^n \mid d$. And since $p^n$ divides both $p^{k-1}$ and $a$, then $p^n$ is a greatest common divisor of $p^{k-1}$ and $a$. Since $R$ is a principal ideal domain, then in fact $(p^{k-1}, a) = (p^n)$. For the same reasons, we have $(p^k, a) = (p^n)$. Thus,
    \[
          p^{k-1}M/p^kM 
            \isom (p^{k-1}, a)/(p^k, a)
            = (p^n)/(p^n)
            = 0.
    \]
    
\end{proof}


\begin{lemma}(Problem 7)
    If $A_1, \dots, A_m$ are $R$-modules and $B_i$ is an $R$-submodule of $A_i$, then
    \[
        (A_1 \dsum \cdots \dsum A_m)/(B_1 \dsum \cdots \dsum B_m) \isom (A_1/B_1) \dsum \cdots \dsum (A_m/B_m).
    \]
\end{lemma}

\begin{proof}
    Let $\pi_i : A_i \to A_i/B_i$ be the natural projection and define the map $\pi = \pi_1 \times \cdots \times \pi_m$, implicitly using the isomorphism between the direct product and direct sum, by
    \begin{align*}
        \pi : A_1 \dsum \cdots \dsum A_m &\to (A_1/B_1) \dsum \cdots \dsum (A_m/B_m) \\
            (x_1, \dots, x_m) &\mapsto (\pi_i(x_1), \dots, \pi_m(x_m)).
    \end{align*}
    Since each component $\pi_i$ is a surjective $R$-module homomorphism with $\ker\pi_i = B_i$, then it follows that $\pi$ is also a surjective $R$-module homomorphism with
    \[
        \ker\pi
            = \ker\pi_1 \times \cdots \times \ker\pi_m
            \isom B_1 \dsum \cdots \dsum B_m.
    \]
    Hence, the first isomorphism theorem implies the result.

\end{proof}


\begin{pbox}[Q1 Problem 12.1.12]
    Let $R$ be a PID and let $p$ be a prime in $R$.
\end{pbox}

\begin{pbox}[(a)]
    Let $M$ be a finitely generated torsion $R$-module. Use the previous exercise to prove that $p^{k-1}M/p^kM \isom F^{n_k}$ where $F$ is the field $R/(p)$ and $n_k$ is the number of elementary divisors of $M$ which are powers $p^\alpha$ with $\alpha \geq k$.
\end{pbox}

\begin{proof}
    We first see that the free rank of $M$ is zero. Suppose $M$ has the invariant factor form
    \[
        M \isom R^r \dsum R/(a_1) \dsum \cdots \dsum R/(a_m).
    \]
    Since $M$ is a torsion $R$-module and $R$ is an integral domain, then
    \[
        M = \Tor_R(M) \isom R/(a_1) \dsum \cdots \dsum R/(a_m),
    \]
    so we must have $r = 0$. We now consider the elementary divisor form of $M$ given by
    \[
        M \isom \Dsum_{i=1}^{n_k} R/(p^{\alpha_i}) \dsum \Dsum_{j=1}^{m} R/(p_j^{\beta_j}).
    \]
    where $\alpha_i \geq k$ for $i = 1, \dots, n_k$, and either $p_j$ is not associate to $p$ or $\beta_j < k$, for $j = 1, \dots, m$, i.e., $(p^{\alpha_i}) \subseteq (p^k)$ but $(p_j^{\beta_j}) \not\subseteq (p^k)$. In particular, the power of $p$ dividing each $p_j^{\beta_j}$ is less than $k$. Define $M_i = R/(p^{\alpha_i})$ and $N_j = R/(p_j^{\beta_j})$. Then, as instances of exercise 11, we have the $R$-module isomorphisms
    \[
        p^{k-1}M_i/p^kM_i \isom R/(p) = F
        \isp{and}
        p^{k-1}N_j/p^kN_j \isom 0.
    \]
    By Lemma 3 (Problem 12.1.7), we have
    \[
        p^{k-1}M/p^kM \isom \Dsum_{i=1}^{n_k} p^{k-1}M_i/p^kM_i \dsum \Dsum_{j=1}^{m} p^{k-1}N_j/p^kN_j,
    \]
    and by Lemma 2 (Problem 12.1.11),
    \[
        p^{k-1}M/p^kM
            \isom \Dsum_{i=1}^{n_k} F \dsum \Dsum_{j=1}^{m} 0
            \isom F^{n_k}.
    \].
    
\end{proof}

\newpage
\begin{pbox}[(b)]
    Suppose $M_1$ and $M_2$ are isomorphic finitely generated torsion $R$-modules. Use (a) to prove that, for every $k \geq 0$, $M_1$ and $M_2$ have the same number of elementary divisors $p^\alpha$ with $\alpha \geq k$. Prove that this implies $M_1$ and $M_2$ have the same set of elementary divisors.
\end{pbox}

\begin{proof}
    For a fixed $k \geq 0$, let $n$ be the number of elementary divisors of $M_1$ which are powers $p^\alpha$ with $\alpha \geq k$, and let $m$ be the same for $M_2$. Then by part (a), we have
    \[
        F^n \isom p^{k-1}M_1/p^kM_2 \isom p^{k-1}M_2/p^kM_2 \isom F^m,
    \]
    as $R$-modules. We equip $F^n$ with the $F$-action defined by
    \[
        \eqc{r}(\eqc{x_1}, \dots, \eqc{x_n}) = r(\eqc{x_1}, \dots, \eqc{x_n}),
    \]
    where the right-hand side is the natural $R$-action on $F^n = (R/(p))^n$, i.e.,
    \[
        r(\eqc{x_1}, \dots, \eqc{x_n})
            = (r\eqc{x_1}, \dots, r\eqc{x_n})
            = (\eqc{rx_1}, \dots, \eqc{rx_n}).
    \]
    We check that this $F$-action is well-defined. Suppose $\eqc{a}, \eqc{b} \in F$ with $\eqc{a} = \eqc{b}$, then
    \begin{align*}
        \eqc{a}(\eqc{x_1}, \dots, \eqc{x_n})
            &= a(\eqc{x_1}, \dots, \eqc{x_n}) \\
            &= (\eqc{ax_1}, \dots, \eqc{ax_n}) \\
            &= (\eqc{bx_1}, \dots, \eqc{bx_n}) \\
            &= b(\eqc{x_1}, \dots, \eqc{x_n}) \\
            &= \eqc{b}(\eqc{ax_1}, \dots, \eqc{ax_n}).
    \end{align*}
    We equip $F^m$ with the same sort of $F$-action. Let $\phi : F^n \to F^m$ be an $R$-module isomorphism, we will show that it is also an $F$-module isomorphism. Since we already know $\phi$ to be a bijection, it suffices to show that it is an $F$-module homomorphism. Let $\eqc{r} \in F$ and $x, y \in F^n$, the
    \begin{align*}
        \phi(\eqc{r}x + y)
            &= \phi(rx + y) \\
            &= r\phi(x) + y \\
            &= \eqc{r}\phi(x) + \phi(y).
    \end{align*}
    Hence $\phi$ is an $F$-module homomorphism and, therefore, an $F$-module isomorphism. This tells use $F^n \isom F^m$ as free $F$-modules, implying that $n = m$.
    
    Define $n_k$ to be the number of elementary divisors of $M_1$ of the form $p^\alpha$ with $\alpha \geq n$, and define $m_k$ to be the same for $M_2$. Then by the above result, $n_k - n_{k+1} = m_k - m_{k+1}$ is the number of elementary divisors of both $M_1$ and $M_2$ of the form $p^k$.
    
    Hence, for all primes $q \in F$ and integers $k \geq 0$, we know that $M_1$ and $M_2$ have the same number of elementary divisors of the form $q^k$. 
    
    
\end{proof}


\newpage
\begin{pbox}[Q2 Problem 12.3.10]
    Find all Jordan canonical forms of $2 \times 2$, $3 \times 3$, and $4 \times 4$ matrices over $\C$.
\end{pbox}

Jordan canonical forms are defined up to the order of Jordan blocks. So the possible shapes of $n \times n$ Jordan canonical forms correspond bijectively to the additive partitions of the number $n$. 

For $n = 2$, we have the partitions $2$ and $1 + 1$, which correspond to the $2 \times 2$ Jordan canonical forms
\[
    \mqty[\lambda_1 & 1 \\ & \lambda_1]
    \isp{and}
    \mqty[\lambda_1 \\ & \lambda_2].
\]
For $n = 3$, we have the partitions $3$, $2 + 1$, and $1 + 1 + 1$, corresponding to
\[
    \mqty[\lambda_1 & 1 \\ & \lambda_1 & 1 \\ & & \lambda_1],
    \quad
    \mqty[\lambda_1 & 1 \\ & \lambda_1 \\ & & \lambda_2],
    \quad
    \mqty[\lambda_1 \\ & \lambda_2 \\ & & \lambda_3].
\]
For $n = 4$, we have the partitions
\[
    4, \quad 3 + 1, \quad 2 + 2, \quad 2 + 1 + 1, \quad 1 + 1 + 1 + 1,
\]
corresponding to
\[
    \mqty[\lambda_1 & 1 \\ & \lambda_1 & 1 \\ & & \lambda_1 & 1 \\ & & & \lambda_1],
    \quad
    \mqty[\lambda_1 & 1 \\ & \lambda_1 & 1 \\ & & \lambda_1 \\ & & & \lambda_2],
    \quad
    \mqty[\lambda_1 & 1 \\ & \lambda_1 \\ & & \lambda_2 & 1 \\ & & & \lambda_2],
\]
\[
    \mqty[\lambda_1 & 1 \\ & \lambda_1 \\ & & \lambda_2 \\ & & & \lambda_3],
    \quad
    \mqty[\lambda_1 \\ & \lambda_2 \\ & & \lambda_3 \\ & & & \lambda_4].
\]
Note that the eigenvalues $\lambda_1, \lambda_2, \lambda_3, \lambda_4 \in \C$ need not be distinct.


\newpage
\begin{pbox}[Q3]
    Find the Jordan canonical form of $\mqty(1 & 0 & 0 \\ 0 & 0 & 5 \\ 0 & 1 & 3) \in \Mat{3}(\Z/7\Z)$.
\end{pbox}

Implicitly, integers will represent their equivalence class mod $7$. Let
\[
    A
        = \mqty[1 & 0 & 0 \\ 0 & 0 & 5 \\ 0 & 1 & 3]
        = \mqty[1 & 0 & 0 \\ 0 & 0 & -2 \\ 0 & 1 & 3].
\]
First, we find the characteristic polynomial of $A$.
\begin{align*}
    c_A(x) 
        &= \det(xI_3 - A) \\
        &= \det\mqty[x - 1 & 0 & 0 \\ 0 & x & 2 \\ 0 & -1 & x - 3] \\
        &= (x - 1)\det\mqty[x & 2 \\ -1 & x - 3] \\
        &= (x - 1)(x^2 - 3x + 2) \\
        &= (x - 1)(x - 1)(x - 2).
\end{align*}
Then there are two possible cases for the elementary divisors of $A$. In the first case, the elementary divisors are $(x - 1)^2$ and $x - 2$, and the minimal polynomial is $(x - 1)^2(x - 2)$. In the second case, the elementary divisors are $x - 1$, $x - 1$, and $x - 2$, and the minimal polynomial is $(x - 1)(x - 2)$. Since the latter divides the former, it suffices to check whether the latter evaluates to zero at $A$. If so, it is the minimal polynomial and, if not, then the former is the minimal polynomial.
\begin{align*}
    (A - 1)(A - 2)
        &= \mqty[0 & 0 & 0 \\ 0 & -1 & 5 \\ 0 & 1 & 2]\mqty[-1 & 0 & 0 \\ 0 & -2 & 5 \\ 0 & 1 & 1] \\
        &= \mqty[0 \\ 0 \\ 0]\mqty[-1 & 0 & 0]
            + \mqty[0 \\ -1 \\ 1]\mqty[0 & -2 & 5]
            + \mqty[0 \\ 5 \\ 2]\mqty[0 & 1 & 1] \\
        &= 0 + \mqty[0 & 0 & 0 \\ 0 & 2 & -5 \\ 0 & -2 & 5] 
            + \mqty[0 & 0 & 0 \\ 0 & 5 & 5 \\ 0 & 2 & 2] \\
        &= \mqty[0 & 0 & 0 \\ 0 & 7 & 0 \\ 0 & 0 & 7] \\
        &= 0.
\end{align*}
Hence, $m_A(x) = (x - 1)(x - 2)$. So the elementary divisors are $x - 1$, $x - 1$, and $x - 2$, and $A$ has the Jordan canonical form (i.e., is similar to)
\[
    \mqty[\dmat[0]{1, 1, 2}].
\]


\newpage
\begin{pbox}[Q4]
    Find the Jordan canonical form of $\mqty(2 & -2 & 0 \\ 1 & -2 & 1 \\ -1 & 1 & 0) \in \Mat{3}(\C)$.
\end{pbox}

Let
\[
    A = \mqty[2 & -2 & 0 \\ 1 & -2 & 1 \\ -1 & 1 & 0].
\]
First, we find the characteristic polynomial of $A$.
\begin{align*}
    c_A(x) 
        &= \det(xI_3 - A) \\
        &= \det\mqty[x - 2 & 2 & 0 \\ -1 & x + 2 & -1 \\ 1 & -1 & x] \\
        &= (x - 2)\det\mqty[x + 2 & -1 \\ -1 & x] - 2\det\mqty[-1 & -1 \\ 1 & x] \\
        &= (x - 2)(x^2 + 2x - 1) - 2(-x + 1) \\
        &= x(x^2 - 3) \\
        &= x(x - \sqrt{3})(x + \sqrt{3}).
\end{align*}
So the elementary divisors of $A$ are $x$, $x - \sqrt{3}$, and $x + \sqrt{3}$. Thus, the Jordan canonical form of $A$ is
\[
    \mqty[\dmat[0]{\sqrt{3}, -\sqrt{3}, 0}].
\]

\end{document}
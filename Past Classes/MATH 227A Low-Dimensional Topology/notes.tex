\documentclass[12pt]{article}

% Packages
\usepackage[margin=1in]{geometry}
\usepackage{parskip}
\usepackage{amsmath, amsthm, amssymb}
\usepackage{tikz, tikz-cd}
\usepackage[shortlabels]{enumitem}

\usepackage{bbm}

\usepackage{suffix}
\usetikzlibrary{decorations.pathmorphing}

% Problem Box
\setlength{\fboxsep}{4pt}
\newlength{\myparskip} 
\setlength{\myparskip}{\parskip}
\newsavebox{\savefullbox}
\newenvironment{fullbox}{\begin{lrbox}{\savefullbox}\begin{minipage}{\dimexpr\textwidth-2\fboxsep\relax}\setlength{\parskip}{\myparskip}}{\end{minipage}\end{lrbox}\framebox[\textwidth]{\usebox{\savefullbox}}}

% Environments
\setlist[enumerate]{nosep}
\newcommand{\keyword}[1]{\textbf{#1}}
\newcommand{\sepline}{\rule{\textwidth}{0.4pt}}

% Tikz Environments
\newenvironment{drawing}{\begin{center}\begin{tikzpicture}}{\end{tikzpicture}\end{center}}
% \tikzcdset{row sep/normal=0pt}
\newenvironment{cd}{\begin{center}\begin{tikzcd}}{\end{tikzcd}\end{center}}


% Document Formatting
\newtheoremstyle{mythmstyle}% name of the style to be used
{ }% measure of space to leave above the theorem. E.g.: 3pt
{ }% measure of space to leave below the theorem. E.g.: 3pt
{ }% name of font to use in the body of the theorem
{ }% measure of space to indent
{\scshape}% name of head font
{.}% punctuation between head and body
{ }% space after theorem head; " " = normal interword space
{\thmname{#1}\thmnumber{ #2}\thmnote{ (#3)}}% Manually specify head

\theoremstyle{definition}
\newtheorem{theorem}{Theorem}
\newtheorem{corollary}{Corollary}
\newtheorem{lemma}{Lemma}
\newtheorem{proposition}{Proposition}
\newtheorem{claim}{Claim}


% Math Formatting
\newcommand{\isp}[1]{\quad\text{#1}\quad}

% mathbb
\newcommand{\N}{\mathbb{N}}
\newcommand{\Z}{\mathbb{Z}}
\newcommand{\Q}{\mathbb{Q}}
\newcommand{\R}{\mathbb{R}}
\newcommand{\C}{\mathbb{C}}
\renewcommand{\k}{\mathbbm{k}}

% mathcal
\newcommand{\LL}{\mathcal{L}}
\newcommand{\FF}{\mathcal{F}}

% Symbols
\newcommand{\eps}{\varepsilon}
\renewcommand{\phi}{\varphi}
\renewcommand{\emptyset}{\varnothing}

% Delimiters
\newcommand{\<}{\left\langle}
\renewcommand{\>}{\right\rangle}

% Relations
\newcommand{\iso}{\cong}
\newcommand{\seq}{\subseteq}

\newcommand{\inc}{\hookrightarrow}
\newcommand{\To}{\longrightarrow}
\newcommand{\Mapsto}{\longmapsto}



% Math Roman
\newcommand{\dd}{\mathrm{d}}
\newcommand{\DD}{\mathrm{D}}


\DeclareMathOperator{\im}{im}

\newcommand{\Isom}{\mathrm{Isom}}
\newcommand{\Stab}{\mathrm{Stab}}
\newcommand{\orb}{\mathrm{orb}}

% Other
\newcommand{\pdv}[2]{\frac{\partial #1}{\partial #2}}
\newcommand{\mat}[1]{\begin{bmatrix}#1\end{bmatrix}}
\newcommand{\clo}{\overline}
\newcommand{\conj}{\overline}



\title{Geometry \& Topology in Low Dimensions \\
    \large Fall 2022
}
\author{}
\date{}


\begin{document}
\maketitle

this requires some topology, maybe algebraic topology, linear algebra, a little group theory

\sepline

9/22/2022

we'll cover some hyperbolic geometry and topology, but will start with Euclidean

\sepline

Euclidean Geometry

Talking about $\R^n$ with norm $\|x\| = x_1^2 + \cdots + x_n^2$, then Euclidean metric
$d(x, y) = \|x - y\|$.

A map $T : \R^n \to \R^n$ is an \keyword{isometry} if for all $x, y \in \R^n$ we have $d(Tx, Ty) = d(x, y)$.

Examples:
\begin{itemize}
    \item translation: $Tx = x + t$ for some constant $t \in \R^n$.
    \item orthogonal: $Tx = Ax$ where $A \in M_n(\R)$ and $n \times n$ matrix with $A^TA = I$.
    Check
    \[
        \|Ax\|^2
            = \<Ax, Ax\>
            = (Ax)^T(Ax)
            = x^T(A^TA)x
            = x^Tx
            = \|x\|^2.
    \]
    Therefore,
    \[
        d(x, y)^2
        = \|Tx - Ty\|^2
        = \|Ax - Ay\|^2
        = \|A(x - y)\|^2
        = \|x - y\|^2
        = d(x, y)^2
    \]
    \item composition.
    In fact, the set of isometries of $\R^n$ is a group under composition.
\end{itemize}

\begin{theorem}
    If $T$ is an isometry of $\R^n$ of then there exists $b \in \R^n$ and $A \in O(n)$ such that $Tx = Ax b$.
\end{theorem}

\begin{proof}
    Suppose $T$ is an isometry, set $b = T(0)$.

    Define $S(x) = x - b$, a translation.

    Define $T' = S \circ T$, an isometry with $T'(0) = 0$.

    Suffices to prove $T'x = Ax$.
    
    Without loss of generality $T(0) = 0$.

    In $\R^n$, the distances of a point to $0, e_1, \dots, e_n$ uniquely determines the point.

    Define $v_i = T(e_i)$, then $\|v_i\| = 1$ since $T$ isometry.

    For $i \ne j$, we have $d(v_i, v_j) = d(e_i, e_j) = \sqrt{2}$, so $\<v_i, v_j\> = 0$.

    Hence, $T$ sends $e_i$'s to $v_i$'s, which form an orthonormal basis.

    Now define matrix
    \[
        A = \mat{| & | & & | \\ v_1 & v_2 & \cdots & v_n \\ | & | & & |}.
    \]
    This matrix has $A^TA = I$ so $A \in O(n)$.
    Moreover, 
    \[
        (A^{-1} \circ T)e_i
            = A^{-1}(Te_i)
            = A^{-1}v_i
            = e_i.
    \]
    Therefore, $A^{-1} \circ T$ fixes $0, e_1, \dots, e_n$, so $A^{-1} \circ T = I$, so $T = A$.
\end{proof}

\sepline

Classification of isometries

dim $n = 1$.
\begin{enumerate}
    \item \keyword{reflection}: $Tx = 2c - x$ for some $c \in \R$
    \item \keyword{translation}: $Tx = x + t$ for some $t \in \R$
\end{enumerate}

notice that the composition of two reflections is a translation: For $Tx = x + c$ and $T'x = x + c'$, we have $(T \circ T')x = x + 2(c - c')$.

Group of isometries:
\[
    \Isom(\R)
        = \{x \mapsto ax + b \mid a = \pm1, b \in \R\}
        \iso \mat{\pm1 & \R \\ 0 & 1}
        \iso Z_2 \ltimes \R.
\]
Let $S$ be a reflection, $T$ a translation.
Note $S = S^{-1}$, then
\[
    S \circ T \circ S^{-1}
        = S \circ T \circ S
        = T^{-1}.  
\]

dim $n = 2$
\begin{enumerate}
    \item \keyword{reflection} across a line
    \item \keyword{translation}
    \item \keyword{rotation} around a point
    \item \keyword{glide reflection}: reflect across a line and translate along the line
\end{enumerate}

dim $n = 3$
\begin{enumerate}
    \item \keyword{translation}
    \item \keyword{rotation} around a line
    \item \keyword{reflection} across a plane
    \item \keyword{screw}: rotate around a line and translate along the line
    \item \keyword{glide reflection}: reflect across a plane and translate along a line in the plane
\end{enumerate}

dim $n = n$ (up to conjugacy)
\begin{itemize}
    \item Case 1: there exists a fixed point.
    Then the isometry is conjugate to an orthogonal transformation.
    \item Case 2: no fixed point.
    Conjugate to $Tx = Ax + b$, with $A \in O(n)$ such that $Ab = b$.
    \[
        b^\perp = \{x \in \R^n \mid \<x, b\> = 0\} \iso \R^{n-1}
    \]
    So $A$ preserves the copy of $\R^{n-1}$ orthogonal to $b$.
    Say $T$ ``translates along an axis and `rotates' (really orthogonal) in hyperplane orthogonal to axis''
\end{itemize}

HW 1: prove all this

\begin{corollary}
    \[
        \Isom(\R^n) \iso O(n) \ltimes \R^n = \mat{A & b \\ 0 & 1}
    \]
    with $A \in O(n), b \in \R^n$.
\end{corollary}


\sepline

Composition of Rotation

In a plane, take points $p$ and $q$

\begin{drawing}
    \fill (0, 0) circle (2pt) node[left] {$p$};
    \fill (2, 0) circle (2pt) node[right] {$q$};

    \draw (0, 0) -- (1.5, 1);
    \draw (0, 0) -- (1.5, -1);

    \draw (2, 0) -- (1.5, 1);
    \draw (2, 0) -- (1.5, -1);

    \draw (0, 0) -- (2, 0);
\end{drawing}

\sepline

Euclidean Manifolds

A metric manifold $M^n$ is \keyword{Euclidean} if it is locally isometric to Euclidean space.

\begin{theorem}
    Every complete simply-connected Euclidean manifold is isometric to $\R^n$.
\end{theorem}

\begin{proof}[Proof Sketch]
    Let $N$ be simply-connected complete Euclidean manifold.

    Goal: construct isometry $\mathrm{dev} : N \to \R^n$, called the ``developing map.''

    Know every point has a little neighborhood $U \seq N$ with isometry $h : U \to h(U) \seq \R^n$.

    For another such isometry $k : V \to k(V) \seq \R^n$.

    Then look at $h(U \cap V)$ and $k(U \cap V)$; would be great if these were the same in $\R^n$.

    There exists an isometry $T : \R^n \to \R^n$ such that $T|_{k(U \cap V)} = h \circ k^{-1}$.

    We then map $V$ using $T \circ k$ instead of $k$.

    Can continue this process for countably many open sets and charts.

    Take an atlas of isometries and adjust the charts so that the charts of intersection open sets agree.

    Simply-connected will ensure that this is well-defined.

    Completeness implies the map is surjective.

    Geometry gives injectivity.
    If two points in $N$ were sent to the same point in $\R^n$ then straight lines bad.
\end{proof}

\begin{corollary}
    If $M^n$ is a closed (complete?) Euclidean manifold, then universal cover $\widetilde{M}$ is a simply-connected complete Euclidean manifold.
    Therefore, $\widetilde{M} = \R^n$.
\end{corollary}

Covering transformations of $\widetilde{M}$ are isometries of $\R^n$.
Then holonomy map $\pi_1M \to \Isom(\R^n)$.
Then
\[
    M = \R^n / \mathrm{hol}(\pi_1M).
\]


\sepline

9/27/22

Examples
\begin{enumerate}
    \item $\R^n$, $G = 1$
    \item $S^1 = \R/\Z$
    \item $T^2 = \R^2/\Z^2$
    \item Klein Bottle: $K = \R^2/G$ where $G = \<\alpha, \beta : \text{ relations}\>$ with
    \[
        \beta(x, y) = (x, y + 2) \isp{and} \alpha(x, y) = (x + 2, -y),
    \]
    a translation and glide reflection, respectively.
    There is a twofold (and cyclic) cover $T^2 \to K = \R^2/G$.
    $T^2 = \R^2/H$ with $H = \mathrm{ab}\<\alpha^2, \beta\> \leq G$. 
\end{enumerate}

Exercise: only compact Euclidean $2$-manifolds are torus and Klein bottle (up to homeomorphism).

\sepline

\begin{theorem}[Baberbach]
    \begin{enumerate}
        \item Every closed Euclidean $n$-manifold is finitely covered by an $n$-torus ($T^n = S^1 \times \cdots \times S^1$).
        \item Up to homeomorphism, there are finitely many.
    \end{enumerate}
\end{theorem}

\begin{center}
    \begin{tabular}{r|l}
        dim & \# closed Euclidean $n$-manifolds \\
        \hline
        1 & 1 \\
        2 & 2 \\
        3 & 10 \\
        4 & 74 \\
        5 & 1060 \\
        6 & 38746
    \end{tabular}
\end{center}

\sepline

Example: $M^3$ place diagram
$G = \<\alpha, \beta, \gamma\> \leq \Isom(\R^3)$ with
\begin{align*}
    \alpha(x, y, z) &= (x + 1, y, z) \\
    \beta(x, y, z) &= (x, y + 1, z) \\
    \gamma(x, y, z) &= (-y, x, z + 1).
\end{align*}
Here, $\gamma$ is a screw motion.

There is a $4$-fold cyclic cover $T^3 \to M^3$.

\sepline

Example: Hexagonal torus. place picture

\sepline

Torus Bundle

\begin{cd}
    T^{n - 1} \rar & M^n \dar["p"] \\
    & S^1
\end{cd}
$p$ is a submersion; for all $x \in S^1$, $p^{-1}(x) \iso T^{n-1}$.

Then $M = T^{n-1} \times [0, 1] / {\sim}$

\sepline

IF $M^n$ is a closed Euclidean manifold, then $M = \R^n/G$ and $G \leq \Isom(\R^n)$.
There exists a short exact sequence of groups
\begin{cd}[row sep=tiny]
    0 \rar & \R^n \rar & \Isom(\R^n) \rar & O(n) \rar & 1 \\
    & \mat{I & b \\ 0 & 1} \rar[mapsto] & \mat{A & b \\ 0 & 1} \rar[mapsto] & A
\end{cd}
The $A$ is called the linear or rotational part of isometry.

\sepline

\begin{theorem}[Baberbach]
    There is a maximal subgroup $\Z^n \leq G$ and there is a short exact sequence
    \begin{cd}
        0 \rar & \Z^n \rar & G \rar & F \rar & 1
    \end{cd}
    where $F$ is a finite subgroup of $O(n)$.
\end{theorem}

\sepline

Remark: these isometries of $R^n$ never have a fixed point.
for, say, a rotation, the point of rotation will have a nasty neighborhood.

\sepline

A manifold $Q$ is a \keyword{Euclidean Orbifold} if $Q = \R^n/G$ and $G$ is a discrete subgroup of $\Isom(\R^n)$.

\sepline

Example

$n = 1$, $G = \<\sigma, \tau : \sigma^2 = \tau^2 = 1\> \iso D_\infty$.

$\sigma(x) = -x$, $\tau(x = 2 - x)$, $\sigma\tau(x) = x - 2$ (translation)

$\R^1/G = [0, 1]$.

$p^{-1}(0) = 2\Z$

Given $x \in \R$, $\Stab(x) = \{g \in G : g(x) = x\}$.
Zero is a singular point, i.e., does not have a little manifold neighborhood around it.

\sepline

Example

$n = 2$.
If $Q$ is compact, $G$ is called a \keyword{wallpaper group} (there are $17$ up to conjugacy).
There are $17$ compact Euclidean orbifolds.

Could have $(\R/G) \times (\R/G) = \R \times \R / G \times G$ where $G$ is from previous example.

This is $[0, 1]^2$... put drawing

Every point in a EUclidean orbifold has some stabilizer group associated to it. most (generic) points have the trivial group and therefore have a little euclidean neighborhood.
Other points do not, and these are called \keyword{singular locus} points.

In $2$-dimensions, singular points can be mirror, corner, or cone.

\sepline

Example

$H$ is orientation-preserving subgroup of $G \times G$.

This contains $\pi$ rotations about $\Z \oplus \Z$.

Then $\R^2/H = S^2$ with singular points of $Z_2$; cone points.

Cone points are $D^2$ quotient by a rotation of finite order.

\sepline

Orbifold fundamental group

Let $Q = \R^n/G$ be a Euclidean orbifold.

If $Q$ happens to be a manifold, then $\R^n$ is universal cover and $G$ is group of covering transformations, then $G$ is the fundamental group of $Q$.

The orbifold fundamental group of $Q$ is defined as
\[
    \pi^{\mathrm{orb}}_1(Q) = G.
\]

Remark: if $G$ acts freely, then $G$ is the group of covering transformations of $Q$ as a manifold, so $G = \pi_1(Q)$.

\sepline

Let $\pi : \R^n \to Q = \R^n/G$ be orbifold projection

The \keyword{singular locus} of $Q$ is the set
\[
    \Sigma(Q) = \pi\{x \in \R^n : \text{there exists } 1 \ne g \in G \text{ such that } g(x) = x\}.
\]

For $n = 2$ there are three kinds of singular points:
\begin{enumerate}
    \item cone point: $D^2$ mod rotation by $2\pi/n$
    \item mirror neighborhood: $D^2$ mod reflection
    \item corner point: $D^2/D_{2n}$
\end{enumerate}

To see that these are the only possibilities, check the finite subgroups of $O(2)$.

\sepline

Calculating the orbifold fundamental group in a special case

$Q$ a surface of genus 2 with cone points $x_1, x_2, \dots, x_k$ of orders $n_1, n_2, \dots, n_k$.

\[
    \pi^\orb_1(Q) = \<\pi_1(Q \setminus \Sigma(Q)) \mid \alpha_i^{n_1} = 1\>
\]

\sepline

Example

Pillowcase

\sepline

\begin{theorem}[Holden, Lozano, Montesinos]
    Every closed orientable $3$-manifold is a branched cover of $S^3$ branched over Boromean rings
\end{theorem}



\end{document}
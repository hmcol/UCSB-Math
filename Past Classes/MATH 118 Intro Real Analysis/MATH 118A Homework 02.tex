\documentclass[12pt]{article}

% packages
\usepackage{kantlipsum}
\usepackage[margin=1in]{geometry}
\usepackage[labelfont=it]{caption}
\usepackage[table]{xcolor}
\usepackage{subcaption,framed,colortbl,multirow,enumitem}
\usepackage{amsmath,amsthm,amssymb,wasysym,mathrsfs,mathtools,babel}
\usepackage{tikz,graphicx,pgf,pgfplots}
\usetikzlibrary{arrows, angles, quotes, decorations.pathreplacing, math, patterns, calc}
\pgfplotsset{compat=1.16}

% Theorems
\newtheorem{theorem}{Theorem}
\newtheorem{lemma}{Lemma}
\newtheorem{proposition}{Proposition}

% Problem Box
\setlength{\fboxsep}{4pt}
\newsavebox{\mybox}
\newenvironment{problem}
    {\begin{lrbox}{\mybox}\begin{minipage}{\textwidth-10pt}}
    {\end{minipage}\end{lrbox}\framebox[6.5in]{\usebox{\mybox}}}

% Environments
\newenvironment{drawing}{\begin{center}\begin{tikzpicture}}{\end{tikzpicture}\end{center}}
\newenvironment{response}{\paragraph{}}{}

% Formatting
\newcommand{\ds}{\displaystyle}
\newcommand{\isp}[1]{\quad\text{#1}\quad}
\newcommand{\seq}[2]{\left\{#1\right\}_{#2=1}^\infty}
\newcommand{\clo}[1]{\overline{#1}}

% Paired Delimiters
\DeclarePairedDelimiter{\ceil}{\lceil}{\rceil}
\DeclarePairedDelimiter\floor{\lfloor}{\rfloor}
\DeclarePairedDelimiter{\ang}{\langle}{\rangle}

% Sets
\newcommand{\N}{\mathbb{N}}
\newcommand{\Z}{\mathbb{Z}}
\newcommand{\I}{\mathbb{I}}
\newcommand{\R}{\mathbb{R}}
\newcommand{\Q}{\mathbb{Q}}
\newcommand{\C}{\mathbb{C}}
\newcommand{\F}{\mathbb{F}}

% Misc Characters
\newcommand{\powerset}{\raisebox{.15\baselineskip}{\Large\ensuremath{\wp}}}
\let\eps\varepsilon
\let\emptyset\varnothing

% Functions
\newcommand{\id}[1]{\mathsf{id}_{#1}}

% Babel Shorthands
\useshorthands*{"}
\defineshorthand{"-}{\setminus}
\defineshorthand{"d}{\partial}

% Probability
\newcommand{\FF}{\mathcal{F}}
\renewcommand{\P}{\mathbb{P}}
 
\begin{document}
 
\title{Homework 2\\
    \large MATH 118A Intro to Real Analysis
}
\author{Harry Coleman}
\date{October 15, 2020}
\maketitle

\section*{Exercise 1}
\begin{problem}
    Let $x \in \mathbb{R}$ and assume that $0 \le x <  \eps$ for all $\eps >0$. Prove that $x=0$.
\end{problem}

\begin{proof}
    Suppose that $0<x$. Then
    \[0<\frac x2 < x,\]
    which contradicts the fact that $0\leq x < \eps$ for all $\eps>0$. Therefore, $0\leq x\leq 0$, so $x=0$.
    
\end{proof}

\section*{Exercise 2}
\begin{problem}
    Given nonempty subsets $A$ and $B$ of positive real numbers, let $C$ denote the set
    \[
     C = \{ xy:\ x \in A,\ y\in B\}.
    \]
    If each $A$ and $B$ has a supremum, then prove that $C$ has a supremum and 
    \[
     \sup C = \sup A \cdot  \sup B.
    \]
\end{problem}

\begin{proof}
    Note that since $A$ and $B$ are strictly positive and nonempty, then each has some positive element, so the supremum of each is also positive. Suppose $xy\in C$ with $x\in A$ and $y\in B$, then
    \[xy \leq x\sup B \leq \sup A \sup B,\]
    so $\sup A\sup B$ is an upper bound for $C$. To show it is the least upper bound, suppose $z<\sup A\sup B$. By definition of supremum, there exists some $x\in A$ such that
    \[\frac z{\sup B} < z \leq \sup A,\]
    which implies that
    \[\frac zx < \sup B.\]
    Again, by definition of supremum, there exists some $y\in B$ such that
    \[\frac zx < y \leq \sup B,\]
    which implies that $z<xy$. And since $xy\in C$, then $z$ is not an upper bound of $C$. Therefore, $\sup C = \sup A \sup B$.
    
\end{proof}

\section*{Exercise 3}
\begin{problem}
    Check if the following sets are bounded, and if so, find their suprema and infima. Determine whether the suprema and infima belong to the set:
\end{problem}

\subsection*{Exercise 3(a)}
\begin{problem}
    $A = \left \{ (x+1)/(|x|+2):\ x \in \mathbb{R} \right \}$.
\end{problem}

\begin{proposition}
    $A$ is bounded above and below with $\sup A = 1$ and $\inf A = -1$. Additionally, $\sup A, \inf A \notin A$.
\end{proposition}

\begin{proof}
    Suppose $A$ is not strictly bounded above by $1$, then there exists some $x\in\R$ such that
    \[1 \leq \frac{x+1}{|x|+2}.\]
    However, this implies that
    \[x + 2 \leq |x|+2 \leq x+1,\]
    giving us $2\leq1$. Similarly, suppose $A$ is not strictly bounded below by $-1$, so there exists some $x\in\R$ such that
    \[\frac{x+1}{|x|+2} \leq-1.\]
    However, this implies that
    \[x+1 \leq-|x|-2 \leq x -2,\]
    giving us $1\leq-2$. Now since
    \[\lim_{x\to\infty}\frac{x+1}{|x|+2} = 1 \isp{and} \lim_{x\to-\infty}\frac{x+1}{|x|+2} = -1,\]
    then we have $\sup A=1$ and $\inf A=-1$, and because these are strict bounds on $A$, then they are not elements of $A$.
    
\end{proof}

\subsection*{Exercise 3(b)}
\begin{problem}
    $B = \left \{ 2/n - 3/m:\ n,m\in \mathbb{N} \right \}$.
\end{problem}

\begin{proposition}
    $B$ is bounded above and below with $\sup B = 2$ and $\inf B = -3$. Additionally, $\sup B, \inf B \notin B$.
\end{proposition}

\begin{proof}
    For all $n,m\in\N$, we have the strict upper bound
    \[\frac2n-\frac3m < \frac2n-0 \leq 2.\]
    This is the least upper bound since if we fix $n=1$,
    \[\lim_{m\to\infty}\frac2n-\frac3m = \lim_{m\to\infty}\frac21-\frac3m = 2-0 = 2.\]
    For all $n,m\in\N$, we have the strict lower bound
    \[-3 \leq 0 - \frac3m < \frac2n - \frac3m.\]
    This is the greatest lower bound since if we fix $m=1$,
    \[\lim_{n\to\infty}\frac2n-\frac3m = \lim_{n\to\infty}\frac2n-\frac31 = 0-3 = -3.\]
    
\end{proof}

\subsection*{Exercise 3(c)}
\begin{problem}
    $C = \left \{ x \in \mathbb{R}:\ \Big | |x-1| - |x-2| \Big | < 2\right \}$.
\end{problem}

\begin{proposition}
    $C=\R$, i.e., $C$ is unbounded.
\end{proposition}

\begin{proof}
    By definition, $C\subseteq\R$. Let $x\in\R$, then by the reverse triangle inequality,
    \[||x-1|-|x-2|| \leq |(x-1)-(x-2)| = |1| < 2.\]
    Therefore, $x\in C$, so $\R\subseteq C$.
    
\end{proof}

\section*{Exercise 4}
\begin{problem}
    Given nonempty subsets $S$ and $T$ of $\mathbb{R}$ such that $s \le t$ for all $s\in S$ and all $t \in T$. Prove that $S$ has a supremum, $T$ has a infimum, and 
    \[
     \sup S \le \inf T.
    \]
\end{problem}

\begin{proof}
    Let $t\in T$, so $s\leq t$ for all $s\in S$. So every element in $T$ is an upper bound for $S$. Similarly, every element of $S$ is a lower bound for $T$. Since $S$ and $T$ are bounded above and below, respectively, then $\sup S$ and $\inf T$ exist. Now since for every $t\in T$, $t$ is an upper bound for $S$, we have $\sup S \leq t$. This being true for all $t\in T$ then implies that $\sup S \leq \inf T$.
    
\end{proof}




\end{document}
\documentclass[12pt]{article}

% packages
\usepackage{kantlipsum}
\usepackage[margin=1in]{geometry}
\usepackage[labelfont=it]{caption}
\usepackage[table]{xcolor}
\usepackage{subcaption,framed,colortbl,multirow}
\usepackage{amsmath,amsthm,amssymb,wasysym,mathrsfs,mathtools}
\usepackage{tikz,graphicx,pgf,pgfplots}
\usetikzlibrary{arrows, angles, quotes, decorations.pathreplacing, math, patterns, calc}
\pgfplotsset{compat=1.16}

% custom commands
\newcommand{\N}{\mathbb{N}}
\newcommand{\Z}{\mathbb{Z}}
\newcommand{\I}{\mathbb{I}}
\newcommand{\R}{\mathbb{R}}
\newcommand{\Q}{\mathbb{Q}}
\newcommand{\C}{\mathbb{C}}
\newcommand{\F}{\mathbb{F}}
\newcommand{\B}{\mathcal{B}}
\newcommand{\p}{^{\prime}}
\newcommand{\powerset}{\raisebox{.15\baselineskip}{\Large\ensuremath{\wp}}}
\DeclarePairedDelimiter{\ceil}{\lceil}{\rceil}
\DeclarePairedDelimiter\floor{\lfloor}{\rfloor}

\setlength{\fboxsep}{4pt}
\newcommand{\exercise}[2]{\section*{Exercise #1}\begin{center}\framebox{\begin{minipage}{\textwidth-10pt}#2\end{minipage}}\end{center}}
\newcommand{\problem}[2]{\section*{Problem #1}\begin{center}\framebox{\begin{minipage}{\textwidth-10pt}#2\end{minipage}}\end{center}}
\newcommand{\generic}[2]{\section*{#1}\begin{center}\framebox{\begin{minipage}{\textwidth-10pt}#2\end{minipage}}\end{center}}

 
\begin{document}
 
\title{Homework 15\\
    \large MATH CS 108A Linear Algebra I}
\author{Harry Coleman}
\date{February 21, 2020}
\maketitle


\exercise{1}{
    Show that rotations are linear transformations.
}

Let $T$ be the function which rotates vectors in $\R^2$ counterclockwise by an angle $\theta$. We want to show that $T$ is linear. Let $(x,y)\in\R^2$ be the vector with angle $\alpha$ and magnitude $r$. This means that
\begin{align*}
    x &= r\cos(\alpha), \\
    y &= r\sin(\alpha).
\end{align*}
Under $T$, we have
\begin{align*}
    T(x,y)  &= T(r\cos(\alpha), r\sin(\alpha)), \\
            &= (r\cos(\alpha+\theta), r\sin(\alpha+\theta)), \\
            &= (r\cos\alpha\cos\theta - r\sin\alpha\sin\theta), r\sin\alpha\cos\theta + r\cos\alpha\sin\theta), \\
            &= (x\cos\theta - y\sin\theta, x\sin\theta + y\cos\theta).
\end{align*}
Which we can represent as the matrix multiplication
\[T(x,y) = \begin{bmatrix}
    \cos\theta & - \sin\theta \\
    \sin\theta &  \cos\theta
\end{bmatrix}
\begin{bmatrix}
    x \\
    y
\end{bmatrix}.
\]
Since multiplying the vector $(x,y)$ by a matrix is a linear transformation, $T$ is a linear transformation.


\newpage
\exercise{2}{
    Let $T: V \rightarrow W$ be a function. Show that $T$ is linear if and only if $T(\alpha x + y) = \alpha T(x) + T(y)$.
}

Suppose $T$ is linear. Let $x,y\in V$, $\alpha\in\F$.  By the linearity of $T$,
\begin{align*}
    T(\alpha x + y)     &= T(\alpha x) + T(y), \\
                        &= \alpha T(x) + T(y).
\end{align*}
This proves the rightward conditional.
Now suppose that for all $x,y\in V$ and $\alpha\in\F$, $T(\alpha x + y) = \alpha T(x) + T(y)$. Let $\alpha=1$. Substituting, we find
\begin{align*}
    T(1 x + y) &= 1 T(x) + T(y), \\
    T(x+y)      &= T(x) + T(y).
\end{align*}
Now let $y=0_V$. Substituting, we find
\begin{align*}
    T(\alpha x + 0_V) &= \alpha T(x) + T(0_V), \\
    T(\alpha x) &= \alpha T(x) + 0_W, \\
    T(\alpha x) &= \alpha T(x).
\end{align*}
So $T$ is linear, which completes the biconditional.

\exercise{7}{
    \begin{enumerate}
        \item Show that the composition of linear transformations is linear.
        \item Show that the inverse of a bijective linear transformation is a linear transformation.
    \end{enumerate}
}

\subsection*{1. Composition of Linear Transformations}

Let $X,Y,Z$ be $\F$-vector spaces, $F\in L(X,Y)$ and $G\in L(Y,Z)$. We want to show that $G\circ F$ is a linear transformation. Let $u,v\in X$ and $\alpha\in\F$. By the linearity of $F$ and $G$,
\begin{align*}
    (G\circ F)(u + v)   &= G(F(u+v)), \\
                        &= G(F(u) + F(v)), \\
                        &= G(F(u)) + G(F(v)), \\
                        &= (G\circ F)(u) + (G \circ F)(v).
\end{align*}
Similarly,
\begin{align*}
    (G\circ F)(\alpha u)    &= G(F(\alpha u)), \\
                            &= G(\alpha F(u)), \\
                            &= \alpha G(F(u)), \\
                            &= \alpha (G\circ F)(u).\\
\end{align*}
Therefore, $G\circ F$ is a linear transformation in $L(X,Z)$.



\exercise{8}{
    Let $V$ and $W$ be two $\F$-vector spaces.
    \begin{enumerate}
        \item Show that $L(V, W)$ is an $\F$-vector space on its own right when we consider the following operations:
        \[(T + H)(v) = T(v) + H(v), T, H \in L(V, W), v \in V.\]
        \[(\alpha T)(v) = \alpha T(v), T \in L(V, W), \alpha \in F, v \in V.\]
        
        \item Assume that $V$ and $W$ are finite-dimensional vector spaces. Show that $L(V, W)$ is also finite-dimensional. What is its dimension? Give a basis for $L(V, W)$.
    \end{enumerate}
}

If $F, H \in L(V,W)$, we will say $F=H$ if and only if $F(v) = H(v)$ for all $v\in V$.

\subsection*{1. $L(V, W)$ is an $\F$-vector space}
Let $W^V$ be the vector space of functions with domain $V$ and codomain $W$. Since $L(V, W)\subseteq W^V$, we will show that $L(V,W)$ is a subspace of $W^V$, and therefore an $\F$-vector space. First, consider the zero element of $W_V$,
\begin{align*}
    \textbf{0}:    & V\rightarrow W \\
                & v \mapsto 0_W.
\end{align*}
To show that $\textbf{0}\in L(V,W)$, we must show \textbf{0} is a linear transformation. Let $x,y\in V$ and $\alpha\in\F$. We show that
\begin{align*}
    \textbf{0}(\alpha x+y)      &= 0_W \\
                                &= \alpha 0_W + 0_W \\
                                &= \alpha \textbf{0}(x) + \textbf{0}(y).
\end{align*}
So \textbf{0} is linear and therefore in $L(V,W)$. Now let
\begin{align*}
    x,y             &\in V, \\
    T,H             &\in L(V,W), \\
    \alpha, \beta   &\in\F.
\end{align*}
We use the given definitions of addition and scalar multiplication, with the linearity of $T$ and $H$ to show that $G=(\beta T + H)$ is linear\textemdash that is,
\begin{align*}
    G(\alpha x + y)     &= (\beta T + H)(\alpha x + y) \\
                        &= \beta [T(\alpha x + y)] + H(\alpha x + y) \\
                        &= \beta [\alpha T(x) + T(y)] + \alpha H(x) + H(y) \\
                        &= \alpha [\beta T(x) + H(x)] + T(y) + H(y) \\
                        &= \alpha (\beta T + H)(x) + (\beta T + H)(y) \\
                        &= \alpha G(x) + G(y).
\end{align*}
So $G=(\beta T + H)\in L(V,W)$, therefore $L(V,W)$ is a vector space.


\subsection*{2. Finite Basis for $L(V,W)$}





\end{document}
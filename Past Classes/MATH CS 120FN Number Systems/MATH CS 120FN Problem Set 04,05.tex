\documentclass[12pt]{article}
 
\usepackage[margin=1in]{geometry} 
\usepackage{amsmath,amsthm,amssymb}
\usepackage{listings}
\usepackage{tikz}
\usepackage{colortbl}

\lstset{basicstyle=\footnotesize}
\usetikzlibrary{calc}

 
\newenvironment{theorem}[2][Theorem]{\begin{trivlist}
\item[\hskip \labelsep {\bfseries #1}\hskip \labelsep {\bfseries #2.}]}{\end{trivlist}}
\newenvironment{lemma}[2][Lemma]{\begin{trivlist}
\item[\hskip \labelsep {\bfseries #1}\hskip \labelsep {\bfseries #2.}]}{\end{trivlist}}
\newenvironment{exercise}[2][Exercise]{\begin{trivlist}
\item[\hskip \labelsep {\bfseries #1}\hskip \labelsep {\bfseries #2.}]}{\end{trivlist}}
\newenvironment{problem}[2][Problem]{\begin{trivlist}
\item[\hskip \labelsep {\bfseries #1}\hskip \labelsep {\bfseries #2.}]}{\end{trivlist}}
\newenvironment{question}[2][Question]{\begin{trivlist}
\item[\hskip \labelsep {\bfseries #1}\hskip \labelsep {\bfseries #2.}]}{\end{trivlist}}
\newenvironment{corollary}[2][Corollary]{\begin{trivlist}
\item[\hskip \labelsep {\bfseries #1}\hskip \labelsep {\bfseries #2.}]}{\end{trivlist}}

\newenvironment{solution}{\begin{proof}[Solution]}{\end{proof}}
 
\begin{document}
 
\title{Problem Sets 4 and 5\\
    \large CS120FN Number Systems}
\author{Harry Coleman}
\date{October 15, 2019}

\maketitle

\section*{Problem Set 4 - Exercise 3}
\fbox{
    \parbox{\textwidth}{
        (a) Find integers $x,y,z$ that satisfy $6x+15y+20z=1$.
        
        (b) Under what conditions is it true that the equation $$ax+by+cz=1$$  has a solution?
    }
}

\subsection*{a}

Solution
\[(x, y, z) = (1, 1, -1)\]

If we subsitutite into our equation
\[6(1) + 15(1) + 20(-1) = 1\]
\[6 + 15 - 20 = 1\]
\[1 = 1\]

Therefore
\[(x, y, z) = (1, 1, -1)\]
is a solution to our equation.


\subsection*{b}

The equation
\[ax + by + cz = 1\]
has a solution when $a, b, c$ are coprime.

To prove this, we will take the greatest common divisor of $a, b, c$
\[n = \text{gcd}(a, b, c)\]

So we can write each as
\begin{align*}
    a &= An \\
    b &= Bn \\
    c &= Cn
\end{align*}
where $A, B, C \in \mathbb{Z}$. Substitution this into our original equation, we get
\[Anx + Bny + Cnz = 1\]
\[n(Ax + By + Cz) = 1\]

We can see that $(Ax + By + Cz)$ is some integer because of the closure of integers over addition and multiplication. So $n$ is some integer factor of 1. The only factor of 1 is 1. Therefore,
\[\text{gcd}(a, b, c) = 1\]

Thus, $a, b, c$ must be coprime for the equation to have a solution.


\section*{Problem Set 5 - Exercise 1}
\fbox{
    \parbox{\textwidth}{
        Use the second proof of the infinitude of primes we talked about in class today to prove that for all $x\geq 1$, $$\pi(x)>\frac{\ln(x)}{2\ln(2)},$$ and also that $p_n\leq 4^n.$ (Hint: Choose $j$ wisely.) 
    }
}
\\

We'll define a set of the first $j$ primes.
\[\{p_1, p_2, \cdots, p_j\}\]


We'll also define $N(x)$ to be a function which returns the number of Natural numbers less than or equal to $x$ which have no prime divisors greater than the $j^\text{th}$ prime, $p_j$.

Every number counted by $N(x)$ can be written as
\[n^2 p_1^{a_1} p_2^{a_2} \cdots p_j^{a_j}\]
where $n \in \mathbb{N}$ and each $a_i \in \{0,1\}$. For instance, if we take $j=3$ and some number with prime factorization
\[p_1^3 p_2^4 p_3^1 = (p_1^2 p_2^4) p_1^1 p_2^0 p_3^1 = (p_1^1 p_2^2)^2 p_1^1 p_2^0 p_3^1\]

We know that there are at most $\sqrt{x}$ perfect squares less than or equal to $x$. Since for any $n \leq \sqrt{x}$, we know $n^2 \leq x$. Giving us no more than $\sqrt{x}$ possibilities for $n^2$.

And for each prime factor in $\{p_1, p_2, \cdots, p_j\}$ after the perfect square, the power can either be 0 or 1, giving each 2 possibilities. Which gives us $2^j$ possible values here. So we can say that
\[N(x) \leq \sqrt{x}2^j\]

At this point, we will define $j$ to be the number of primes less than to equal to $x$, so $j = \pi(x)$. Since out $N(x)$ now considers all prime numbers less than or equal to $x$, it counts all numbers less than or equal to $x$, so
$N(x) = x$

With this we can restate our inequality as
\[x \leq \sqrt{x}2^{\pi(x)}\]

To make this an exclusive inequality, we'll reconsider how we derived the right side. We want to show that our estimate for the upper bound for $x$ is strictly greater than $x$. We considered all possible numbers less than or equal to $x$ expressed as
\[n^2 p_1^{a_1} p_2^{a_2} \cdots p_j^{a_j}\]
where $n \in \mathbb{N}$ and each $a_i \in \{0,1\}$. This must consider the number where $n^2$ is the largest perfect square less than or equal to $x$ and each $a_i=1$. Ignoring $x=1$ for now, we'll just consider $x \geq 2$, which is $p_1$. So for any $x \geq 2$, our upper bound considers the number which is the product of the greatest perfect square less than or equal to $x$ and $p_1^1 = 2$.

So we want to show that $x$ is always less than the product of the greatest perfect square less than or equal to $x$ and 2. The worst case for this would be when $x$ is 1 less than a perfect square. So we want to see for what $n \in \mathbb{N}$ values make the following true.
\[2n^2 > (n+1)^2 - 1 = x\]
\[2n^2 > n^2 + 2n + 1 -1\]
\[n^2 - 2n > 0\]
\[n(n - 2) > 0\]

So for $n > 2$, or
\[(n+1)^2 - 1 = x\]
\[(n+1)^2 = x + 1\]
\[n+1 = \sqrt{x + 1}\]
\[n = \sqrt{x + 1} - 1 > 2\]
\[\sqrt{x + 1} > 3\]
\[x + 1 > 9\]
\[x > 8\]
we know that out upper bound for $N(x)$ considers values greater than $x$. So
\[x < \sqrt{x}2^{\pi(x)}\]
for $x>8$. Since we want this for $x \geq 1$, we'll show $x$ values 1 to 8, manually.

\begin{center}
    \begin{tabular}{|c|c|c|}
        \hline
        $x$ & $\pi(x)$ & $\sqrt{x}2^{\pi(x)}$ \\
        \hline
        1 & 0 & 2\\
        2 & 1 & 2.8...\\
        3 & 2 & 6.9...\\
        4 & 2 & 8\\
        5 & 3 & 17.8...\\
        6 & 3 & 19.5...\\
        7 & 4 & 42.3...\\
        8 & 4 & 45.2...\\
        \hline
    \end{tabular}
\end{center}

So we know that for $x \geq 1$
\[x < \sqrt{x}2^{\pi(x)}\]
which we can rearrange
\begin{align*}
    \sqrt{x} &< 2^{\pi(x)} \\
    \ln \sqrt{x} &< \ln 2^{\pi(x)} \\
    \frac{\ln x}{2} &< \pi(x)\ln 2 \\
    \pi(x) &> \frac{\ln x}{2\ln 2} \\
\end{align*}

For the second part, we'll take the $n^\text{th}$ prime $p_n = \pi^{-1}(n)$ so 

\begin{align*}
    n &> \frac{\ln p_n}{2\ln 2} \\
    2n &> \frac{\ln p_n}{\ln 2} \\
    2n &> \log_2 p_n \\
    2^{2n} &> p_n \\
    p_n &< 4^n
\end{align*}
which can be easily generalized to
\[p_n \leq 4^n\]



\end{document}


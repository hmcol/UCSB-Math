\documentclass[12pt]{article}

% Packages
\usepackage[margin=1in]{geometry}
\usepackage{fancyhdr, parskip}
\usepackage{amsmath, amsthm, amssymb}
\usepackage{tikz, tikz-cd}
\usepackage[shortlabels]{enumitem}

% Page Style
\makeatletter
\fancypagestyle{title}{
    \renewcommand{\headrulewidth}{0.4pt}
    \setlength{\headheight}{15pt}
    \fancyhead[R]{\@author}
    \fancyhead[L]{\@title}
    \fancyhead[C]{\@date}
}
\makeatother
\renewcommand{\maketitle}{\thispagestyle{title}}
\fancypagestyle{plain}{
    \fancyhf{}
    \renewcommand{\headrulewidth}{0pt}
    \renewcommand{\footrulewidth}{0pt}
    \fancyfoot[R]{\thepage}
}
\pagestyle{plain}

% Problem Box
\setlength{\fboxsep}{4pt}
\newlength{\myparskip}
\setlength{\myparskip}{\parskip}
\newsavebox{\savefullbox}
\newenvironment{fullbox}{\begin{lrbox}{\savefullbox}\begin{minipage}{\dimexpr\textwidth-2\fboxsep\relax}\setlength{\parskip}{\myparskip}}{\end{minipage}\end{lrbox}\framebox[\textwidth]{\usebox{\savefullbox}}}
\newenvironment{pbox}[1][]{\begin{fullbox}\def\temp{#1}\ifx\temp\empty\else\paragraph{#1}\phantom{}\fi}{\end{fullbox}}

% Theorem Environments
\theoremstyle{definition}
\newtheorem{lemma}{Lemma}

% Tikz Environments
\newenvironment{drawing}{\begin{center}\begin{tikzpicture}}{\end{tikzpicture}\end{center}}
% \tikzcdset{row sep/normal=0pt}
\newenvironment{cd}{\begin{center}\begin{tikzcd}}{\end{tikzcd}\end{center}}

% Default Commands
\newcommand{\isp}[1]{\quad\text{#1}\quad}
\newcommand{\N}{\mathbb{N}} 
\newcommand{\Z}{\mathbb{Z}}
\newcommand{\Q}{\mathbb{Q}}
\newcommand{\R}{\mathbb{R}}
\newcommand{\C}{\mathbb{C}}
\newcommand{\A}{\mathbb{A}}
\renewcommand{\P}{\mathbb{P}}
\newcommand{\eps}{\varepsilon}
\renewcommand{\phi}{\varphi}
\renewcommand{\emptyset}{\varnothing}
\newcommand{\<}{\langle}
\renewcommand{\>}{\rangle}
\newcommand{\iso}{\cong}
\newcommand{\eqc}{\overline}
\newcommand{\clo}{\overline}
\renewcommand{\tilde}{\widetilde}
\renewcommand{\hat}{\widehat}
\newcommand{\seq}{\subseteq}
\newcommand{\teq}{\trianglelefteq}
\DeclareMathOperator{\id}{id}
\DeclareMathOperator{\im}{im}
\newcommand{\inc}{\hookrightarrow}
\newcommand{\dd}{\mathrm{d}}
\newcommand{\mat}[1]{\begin{bmatrix}#1\end{bmatrix}}

% Extra Commands
\DeclareMathOperator{\supp}{supp}
\newcommand{\pdv}[2]{\frac{\partial #1}{\partial #2}}

% Document
\begin{document}
\title{MATH 240A Homework 4}
\author{Harry Coleman}
\date{Octobter 21, 2022}
\maketitle

\begin{pbox}[1]
    Let $M$ be a smooth manifold.
\end{pbox}

\begin{pbox}[(a)]
    Use partition of unity to show that, if $F_1, F_2$ are two closed subsets of $M$ such that $F_1 \cap F_2 = \emptyset$, then there is a smooth function $f$ on $M$ such that $f=1$ on $F_1$ and $f=0$ on $F_2$.
\end{pbox}

Note that $\{M \setminus F_1, M \setminus F_2\}$ is an open cover of $M$.
Let $\{\psi_1, \psi_2\}$ be a smooth partition of unity subordinate to this cover.
By construction, the support of $\psi_1$ is contained in $M \setminus F_1$, which means that $\psi_1 \equiv 0$ on $F_1$.
Similarly, $\psi_2 \equiv 0$ on $F_2$.
Since $\psi_1 + \psi_2 \equiv 1$, we deduce that $\psi_2 \equiv 1$ on $F_1$.
Hence, $f = \psi_2$ is the desired function.



\begin{pbox}[(b)]
    Now let $U$ be an open set of $M$ (which then inherits a smooth manifold structure) and $F$ be a closed subset such that $F\subseteq U$. Show that for any smooth function $f$ on $U$, there is a smooth function $\overline{f}$ on $M$ such that $\overline{f}|_F = f|_F$. (We have made use of this extension result in HW2.)
\end{pbox}

Applying part (a), choose $h \in C^\infty(M)$ such that $h \equiv 1$ on $F$ and $h \equiv 0$ on $M \setminus U$.
We now define the smooth function $\overline{f} = f \cdot h$ on $U$, which we can extend to all of $M$ by defining $\overline{f} \equiv 0$ outside of $U$.

By construction, $\overline{f}$ is smooth inside of $U$.
For a point $x \in M \setminus U$, we have $x \notin \supp f$.
Since the support of $f$ is a closed set, there is a neighborhood $V$ of $x$ which is entirely outside of $\supp f$.
In which case, $f \equiv 0$ on $V$, which is clearly smooth.




\newpage
\begin{pbox}[2]
    Let $S^n$ be the standard sphere in $\R^{n+1}$. Show that, for any $p\in S^n$, $T_p S^n$ can be naturally identified with the subspace of vectors in $\R^{n+1}$ perpendicular to $p$.
\end{pbox}

Consider the inclusion $\iota : S^n \inc \R^{n+1}$.

We claim that the differential $\iota_* : T_pS^n \to T_p\R^{n+1} \iso \R^{n+1}$ is injective.
The radial projection $\pi : \R^{n+1} \setminus \{0\} \to S^n$ is a retraction, i.e., $\pi \circ \iota = \id$.
Then functorality of the differential gives us $\pi_* \circ \iota_* = \id$, hence $\iota_*$ is injective.




\newpage
\begin{pbox}[3]
    Recall that the graph $\Gamma$ of the function $f(x) = |x|$, $x\in \R$, is a smooth manifold.
\end{pbox}

\begin{pbox}[(a)]
    Show that it is diffeomorphic to $\R$.
\end{pbox}

Note that $\R$ is a smooth manifold with a global chart $\id_\R : \R \to \R$.

There is a global chart $\phi : \Gamma \to \R$ given by $(x, |x|) \mapsto x$.
By definition, this is a smooth map of manifolds.

Moreover, the inverse $\phi^{-1} : \R \to \Gamma$ is also smooth:
\[
    \widehat{\phi^{-1}}
        = \phi \circ \phi^{-1} \circ \id_\R^{-1}
        = \id_\R
\]

Hence, $\phi$ is a diffeomorphism.

\begin{pbox}[(b)]
    Show that, however, there is no diffeomorphism $F: \R^2\to \R^2$ taking $\R$ to $\Gamma$ (or vice versa). Here you can take $\R$ to be the real axis of $\R^2$. (\textbf{Hint}: By contradiction, assume that there is such an $F$. Without loss of generality, we can assume that $F(0,0) = (0,0)$ (why?). Write $F(x,y) = (g(x,y), h(x,y))$. What does $F$ taking $\R$ to $\Gamma$ tell you about the form of $F(x,0)$? This will depend on which side you are approaching the origin. Now what does this information translate to $F_*(\frac{\partial}{\partial x})$? )
\end{pbox}

Per the hint, assume $F$ is such a diffeomorphism.
We may assume $F(0, 0) = (0, 0)$ since translation on $\R^2$ is a diffeomorphism.

The fact that $F$ takes $\R$ to $\Gamma$ tells us that $F(x, 0) = (g(x, 0), |g(x, 0)|)$.

At the point $(a, 0)$ we compute
\[\textstyle
    F_*(\pdv{}{x})(x) = \pdv{}{x}\big|_{(a,0)}g
    \isp{and}
    F_*(\pdv{}{x})(y) = \pdv{}{x}\big|_{(a,0)}|g|.
\]
Therefore, we have
\[\textstyle
    F_*(\pdv{}{x}) = \left(\pdv{}{x}\big|_{(a,0)}g\right)\pdv{}{x} + \left(\pdv{}{x}\big|_{(a,0)}|g|\right)\pdv{}{y}.
\]
For $a < 0$ we get 
\[\textstyle
    F_*(\pdv{}{x}) = \left(\pdv{}{x}\big|_{(a,0)}g\right)\pdv{}{x} - \left(\pdv{}{x}\big|_{(a,0)}g\right)\pdv{}{y}
\]
and for $a > 0$ we get
\[\textstyle
    F_*(\pdv{}{x}) = \left(\pdv{}{x}\big|_{(a,0)}g\right)\pdv{}{x} + \left(\pdv{}{x}\big|_{(a,0)}g\right)\pdv{}{y}.
\]
Since $F$ is a diffeomorphism, $F_*$ is an isomorphism, so these are nonzero for all values of $a$.
But then taking the limit as $a \to 0$ gives us different answers from the left and the right, which is a contradiction.




\newpage
\begin{pbox}[4 Lee 3-5]
    (second part only) Let $S^1 \subseteq \R^2$ be the unit circle, and let $K\subseteq \R^2$ be the boundary of the square of side 2 centered at the origin: $K=\{(x,y): \max(|x|,|y|)=1\}$. Show that there is a homeomorphism $F: \R^2\to \R^2$ such that $F(S^1)=K$, but there is no \textit{diffeomorphism} with the same property. [Hint: let $\gamma$ be a smooth curve whose image lies in $S^1$, and consider the action of $\mathrm{d} F(\gamma'(t))$ on the coordinate functions $x$ and $y$.]
\end{pbox}


\end{document}
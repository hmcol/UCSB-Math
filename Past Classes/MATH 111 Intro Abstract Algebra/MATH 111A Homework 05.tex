\documentclass[12pt]{article}

% Packages
\usepackage[margin=1in]{geometry} % proper margins
\usepackage{enumitem} % custom numbering for lists
\usepackage{amsmath} % align, cases, eqref, matrices, dots, roots, delimiters, math mode functions, mod, arrows
\usepackage{amsthm} % theorems, proofs
\usepackage{amssymb} % fancy letters, niche relations/negations
\usepackage{mathrsfs} % much more loopy calligraphy font

% Theorems
\newtheorem{theorem}{Theorem}
\newtheorem{lemma}{Lemma}
\newtheorem{proposition}{Proposition}

% Problem Box
\setlength{\fboxsep}{4pt}
\newsavebox{\mybox}
\newenvironment{problem}
    {\begin{lrbox}{\mybox}\begin{minipage}{0.98\textwidth}}
    {\end{minipage}\end{lrbox}\framebox[\textwidth]{\usebox{\mybox}}}

% Formatting
\newcommand{\ds}{\displaystyle}
\newcommand{\isp}[1]{\quad\text{#1}\quad}

% Alternate Characters
\let\eps\varepsilon % double curved, rather than single curve with midline
\let\phi\varphi % single stroke loop, rather than vertical line through circle
\let\emptyset\varnothing % circular, rather than tall

% Named Sets
\newcommand{\N}{\mathbb{N}} % natural numbers
\newcommand{\Z}{\mathbb{Z}} % integers 
\newcommand{\Q}{\mathbb{Q}} % rational numbers
\newcommand{\R}{\mathbb{R}} % real numbers
\newcommand{\C}{\mathbb{C}} % complex numbers

% Fancy Characters
\newcommand{\F}{\mathbb{F}} % arbitrary field
\renewcommand{\P}{\mathbb{P}} % probability measure (apparently, sometimes prime numbers)
\newcommand{\FF}{\mathcal{F}} % sigma algebra

% Paired Delimiters
\newcommand{\ceil}[1]{\left\lceil #1 \right\rceil} % ceiling
\newcommand{\floor}[1]{\left\lfloor #1 \right\rfloor} % floor
\newcommand{\<}{\left\langle} % left angle bracket
\renewcommand{\>}{\right\rangle} % right angle bracket

% Functions
\renewcommand{\Im}{\operatorname{Im}} % imaginary part of a complex number
\renewcommand{\Re}{\operatorname{Re}} % real part of a complex number
\newcommand{\Arg}{\operatorname{Arg}} % principal argument (angle) of complex number

% Simplified Notation
\newcommand{\id}[1]{\operatorname{id_{\mathnormal{#1}}}} % identity operator
\newcommand{\seq}[2][n]{\left\{#2\right\}_{#1\in\N}} % sequence
\newcommand{\pdv}[3][1]{\ifnum#1=1{\frac{\partial #2}{\partial#3}}\else{\frac{\partial^{#1}#2}{\partial#3^{#1}}}\fi} % partial derivative
\newcommand{\intr}[1]{\accentset{\circ}{#1}} % interior of a set

% Renaming
\let\sm\setminus % set minus (difference)
\let\clo\overline % closure of a set
\let\conj\overline % conjugate of an object
\let\eqc\overline % equivalence class of an object
\let\teq\trianglelefteq % normal subgroup
\let\iso\cong % isomorphic (groups)


\begin{document}
 
\title{Homework 5\\
    %\large MATH CS 121 Intro to Probability
    \large MATH 111A Intro to Abstract Algebra
    %\large MATH CS 122A Complex Analysis I
    %\large MATH 118A Intro to Real Analysis
    %\large MATH 104A Intro to Numerical Analysis
}
\author{Harry Coleman}
\date{November 12, 2020}
\maketitle

\section*{Q1 Exercise 3.4.1}
\begin{problem}
    Prove that if $G$ is an abelian simple group then $G \cong Z_p$ for some prime $p$ (do not assume $G$ is a finite group).
\end{problem}

\begin{proof}
    Since $G$ is abelian, then all subgroups of $G$ are normal subgroups. And since $G$ is simple, then $G$ has no nontrivial normal subgroups. Therefore, $G$ has no nontrivial subgroups, that is, the only subgroups of $G$ are $\{1\}$ and $G$. Additionally, since $G$ is simple, $|G|>1$, so $G$ is not the trivial group. Let $x\in G$ such that $x\ne 1$, then $\<x\>\leq G$. Since $x\ne1$, then $\<x\>\ne\{1\}$, so $\<x\>=G$. Therefore, $G$ is a cyclic group, so
    \[G \iso \begin{cases}
        (\Z, +) &\text{if $|G|=\infty$,} \\
        Z_n &\text{if $|G|=n$.}
    \end{cases}\]
    
    We claim that $|G|<\infty$. Suppose, for contradiction, that $|G|=\infty$. In which case, we have an isomorphism $\phi:\Z\to G$. If we now consider the subgroup $2\Z\leq\Z$ of even integers, then the image $\phi(2\Z) \leq \phi(\Z) = G$. Clearly, $\phi(2\Z)$ is not the trivial subgroup, $\{1\}$. However, because $\phi$ is a bijection from $\Z$ to $G$ and $2\Z$ is a strict subset of $\Z$, its image $\phi(2\Z)$ must be a strict subset of $\phi(\Z)=G$. This is a contradiction since $\phi$ being a homomorphism gives us $\phi(2\Z)$ to be a subgroup of $G$ which is neither the trivial subgroup nor $G$, itself.
    
    Let $n\in\N$ such that $|G|=n$ and $G\iso Z_n$. Recall that $G=\<x\>$, so we have the property
    \[\<x^a\> = \<x^b\> \iff (a,n) = (b,n),\]
    for any $a,b\in\{1,\dots,n-1\}$. In particular, since $G$ has no nontrivial subgroups,
    \[\<x\> = \<x^b\>\]
    is true for all $b\in\{1,\dots,n-1\}$. Therefore,
    \[1 = (1,n) = (b,n)\]
    for all $b\in\{1,\dots,n-1\}$, so $n$ is prime. 
        
\end{proof}

\section*{Q2 Exercise 3.3.4}
\begin{problem}
    Let $C$ be a normal subgroup of the group $A$ and let $D$ be a normal subgroup of the group $B$. Prove that $(C\times D)\teq(A\times B)$ and $(A\times B)/(C\times D) \iso (A/C)\times(B/D)$.
\end{problem}

\begin{proof}
    Consider the following map between groups:
    \begin{align*}
        \phi : (A\times B) &\to (A/C)\times(B/D) \\
        (a,b) &\mapsto (aC,bD).
    \end{align*}
    For any $(aC,bD) \in (A/C)\times(B/D)$, we have $\phi((a,b)) = (aC,bD)$, so this map is surjective. Next, we prove that $\phi$ is a homomorphism. If $(a_1,b_1),(a_2,b_2)\in(A\times B)$, then
    \begin{align*}
        \phi((a_1,b_1)(a_2, b_2)) 
            &= \phi((a_1a_2, b_1b_2)) \\
            &= ((a_1a_2)C, (b_1b_2)D) \\
            &= (a_1C \cdot a_2C, b_1D \cdot b_2D) \\
            &= (a_1C,b_1D)(a_2C,b_2D) \\
            &= \phi((a_1,b_2))\phi((a_2,b_2)).
    \end{align*}
    Note that $(cC,dD)$ is the identity of $(A/C)\times(B/D)$ for any $c\in C$ and $d\in D$, since for any $(aC,bD)\in(A/C)\times(B/D)$, we have
    \begin{align*}
        (aC,bD)(cC,dD)
            &= (aC,bD)(1_AC,1_BD) \\
            &= (aC \cdot 1_AC, bD \cdot 1_BD) \\
            &= ((a1_A)C, (b1_B)D) \\
            &= (aC, bD).
    \end{align*}
    Therefore, if $(a,b)\in\ker\phi$, then
    \[\phi((a,b)) = (aC,bD) = (cC, dD),\]
    for some $c\in C$ and $d\in D$. This would imply that
    \[aC = cC = C \isp{and} bD = dD = D,\]
    giving us $a\in C$ and $b\in D$. In other words, $\ker\phi \subseteq (C\times D)$. Now for any $(c,d)\in (C\times D)$, we have
    \[\phi((c,d)) = (cC,dD),\]
    where $c\in C$ and $d\in D$. Therefore, $(C\times D)\subseteq\ker\phi$ and we, in fact, have equality $\ker\phi = (C\times D)$. Then by the first isomorphism theorem, we have
    \[\ker\phi = (C\times D) \teq (A\times D)\]
    and
    \[(A\times B)/(C\times D) = (A\times B)/\ker\phi \iso \phi(A\times B) = (A/C)\times(B/D).\]
    
    
    
\end{proof}


\newpage
\section*{Q3}
\begin{problem}
    Let $G$ be a group and $X$ be a nonempty set. Let $A=\{\text{functions from $X$ to $G$}\}$. Given $f_1,f_2\in A$, define $f_1f_2$ as
    \[(f_1f_2)(x) = f_1(x)f_2(x).\]
    This is a binary operation on $A$ under which $A$ is a group. Fix $x_0\in X$, and define
    \[A_0 = \{f\in A \mid f(x_0) = 1\}.\]
\end{problem}

\subsection*{Q3.1}
\begin{problem}
    Show that $A_0$ is a normal subgroup of $A$.
\end{problem}

\begin{proof}
    Consider the map
    \begin{align*}
        \phi : A &\to G \\
        f &\mapsto f(x_0).
    \end{align*}
    For any $g\in G$, we define the function $f_g: X \to G$, where $f_g(x) = g$ for all $x\in X$. Then we have $f_g\in A$ and $\phi(f_g) = f_g(x_0) = g$, so $\phi$ is surjective. Now for any $f_1,f_2\in A$, we have
    \[\phi(f_1f_2) = (f_1f_2)(x_0) = f_1(x_0)f_2(x_0) = \phi(f_1)\phi(f_2),\]
    so $\phi$ is a homomorphism. Now for any $f\in A$, we have $f\in\ker\phi$ if and only if $\phi(f) = f(x_0) = 1_G$, which is the case if and only if $f\in A_0$. Therefore, $\ker\phi = A_0$, and by the first isomorphism theorem, we have $A_0\teq A$.

\end{proof}

\subsection*{Q3.2}
\begin{problem}
    Show that $A/A_0 \cong G$.
\end{problem}

\begin{proof}
    We take the same map $\phi:X\to G$, which is a surjective homomorphism whose kernel is $A_0$. Then by the first isomorphism theorem, $A/A_0 \iso G$.
    
\end{proof}



\end{document}
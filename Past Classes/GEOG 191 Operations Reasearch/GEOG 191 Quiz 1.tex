\documentclass[12pt]{article}

% Packages
\usepackage[margin=1in]{geometry}
\usepackage{amsmath, amsthm, amssymb, physics}

% Problem Box
\setlength{\fboxsep}{4pt}
\newsavebox{\mybox}
\newenvironment{problem}
    {\begin{lrbox}{\mybox}\begin{minipage}{0.98\textwidth}}
    {\end{minipage}\end{lrbox}\begin{center}\framebox[\textwidth]{\usebox{\mybox}}\end{center}}

% Options
\renewcommand{\thesubsection}{\thesection(\alph{subsection})}
\allowdisplaybreaks
\addtolength{\jot}{1em}
\theoremstyle{definition}

% Default Commands
\newtheorem{proposition}{Proposition}
\newtheorem{lemma}{Lemma}
\newcommand{\ds}{\displaystyle}
\newcommand{\isp}[1]{\quad\text{#1}\quad}
\newcommand{\N}{\mathbb{N}}
\newcommand{\Z}{\mathbb{Z}}
\newcommand{\Q}{\mathbb{Q}}
\newcommand{\R}{\mathbb{R}}
\newcommand{\C}{\mathbb{C}}
\newcommand{\eps}{\varepsilon}
\renewcommand{\phi}{\varphi}
\renewcommand{\emptyset}{\varnothing}

% Extra Commands



% Document Info
\title{Quiz 1\\
    \large GEOG 191
}
\author{Harry Coleman}
\date{January 12, 2021}

% Begin Document
\begin{document}
\maketitle

\noindent
To maximize the total number of items purchased, we use the following linear program:
\begin{align*}
    \textbf{maximize } & x_1 + x_2 + 2x_3 + 3x_4 + x_5 + x_6, \\
    \textbf{subject to } & 4.50x_1 + 4.15x_2 + 8.20x_3 + 7.20x_4 + 8.19x_5 + 13.00x_6 \leq 20, \\
                        & x_i \in \{0, 1\} \quad i = 1, \dots 6.
\end{align*}
In matrix notation, define
\[
    a = \mqty[4.50 \\ 4.15 \\ 8.20 \\ 7.20 \\ 8.19 \\ 13.00] \quad c = \mqty[1 \\ 1 \\ 2 \\ 3 \\ 1 \\ 1] \quad b = 20.
\]
Then the above program can be rewritten as
\begin{align*}
    \textbf{maximize } & c^TX, \\
    \textbf{subject to } & a^TX \leq b, \\
                        & X \in \{0, 1\}^6.
\end{align*}



\end{document}
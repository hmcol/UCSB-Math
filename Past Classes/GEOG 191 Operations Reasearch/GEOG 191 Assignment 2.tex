\documentclass[12pt]{article}

% Packages
\usepackage[margin=1in]{geometry}
\usepackage{amsmath, amsthm, amssymb, physics}

% Problem Box
\setlength{\fboxsep}{4pt}
\newsavebox{\mybox}
\newenvironment{problem}
    {\begin{lrbox}{\mybox}\begin{minipage}{0.98\textwidth}}
    {\end{minipage}\end{lrbox}\begin{center}\framebox[\textwidth]{\usebox{\mybox}}\end{center}}

% Options
\renewcommand{\thesubsection}{\thesection(\alph{subsection})}
\allowdisplaybreaks
\addtolength{\jot}{1em}
\theoremstyle{definition}

% Default Commands
\newtheorem{proposition}{Proposition}
\newtheorem{lemma}{Lemma}
\newcommand{\ds}{\displaystyle}
\newcommand{\isp}[1]{\quad\text{#1}\quad}
\newcommand{\N}{\mathbb{N}}
\newcommand{\Z}{\mathbb{Z}}
\newcommand{\Q}{\mathbb{Q}}
\newcommand{\R}{\mathbb{R}}
\newcommand{\C}{\mathbb{C}}
\newcommand{\eps}{\varepsilon}
\renewcommand{\phi}{\varphi}
\renewcommand{\emptyset}{\varnothing}

% Extra Commands



% Document Info
\title{Assignment 2\\
    \large GEOG 191
}
\author{Harry Coleman}
\date{January 18, 2021}

% Begin Document
\begin{document}
\maketitle

\section*{Exercise 1}
\begin{problem}
    A firm operates two distribution centers, one in Phoenix and the other is Chicago. A customer in College Station, TX needs solar panels produced by the firm. The cost to produce a solar panel in Phoenix is \$349, and shipment to College Station is \$63. The cost to produce a solar panel in Chicago is \$349, and shipment to College Station is \$47. Phoenix can supply at most 13 panels, and Chicago up to 29. The customer wants 21 panels. Structure this as an optimization model? 
\end{problem}

Let $X_1$ and $X_2$ be the decision variables representing the number of solar panels to be purchased from Phoenix and Chicago, respectively. The cost for each solar panel is the sum of the production and shipment costs. We aim to minimize the total cost, which is given by
\[
    (349 + 63)X_1 + (349 + 47)X_2 = 412X_1 + 396X_2.
\]
Since Phoenix can supply at most $13$ panels, then the number of solar panels purchased from Phoenix can be at most $13$, i.e.,$X_1 \leq 13$. Similarly, Chicago can supply at most $29$ panels, so $X_2 \leq 29$. The customer wants $21$ panels total, so $X_1 + X_2 = 21$. Lastly, the number of panels purchased from each location must be nonnegative integers. Hence, we have the following integer linear program:
\[
    \begin{array}{lrcr}
        \textbf{minimize}   & 412X_1 + 396X_2 & & \\
        \textbf{subject to} & X_1 & \leq & 13 \\
                            & X_2 & \leq & 29 \\
                            & X_1 + X_2 & = & 21 \\
                            & X_1, X_2 & \geq & 0 \\
    \end{array}
\]

\newpage
\section*{Exercise 2}
\begin{problem}
    How does the problem in \#1 change if the cost to produce a panel in each city changes? Specifically, the cost to produce a solar panel in Phoenix is \$349 and the cost to produce a solar panel in Chicago is \$379. Structure this optimization model.
\end{problem}

The total cost becomes, instead,
\[
    (349 + 63)X_1 + (379 + 47)X_2 = 412X_1 + 426X_2.
\]
We obtain a similar integer linear program, with only the objective function changed:
\[
    \begin{array}{lrcr}
        \textbf{minimize}   & 412X_1 + 426X_2 & & \\
        \textbf{subject to} & X_1 & \leq & 13 \\
                            & X_2 & \leq & 29 \\
                            & X_1 + X_2 & = & 21 \\
                            & X_1, X_2 & \geq & 0 \\
    \end{array}
\]





\end{document}
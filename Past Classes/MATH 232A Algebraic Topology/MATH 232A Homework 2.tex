\documentclass[12pt]{article}

% Packages
\usepackage[margin=1in]{geometry}
\usepackage{fancyhdr, parskip}
\usepackage{amsmath, amsthm, amssymb}
\usepackage{tikz, tikz-cd}
\usepackage[shortlabels]{enumitem}

% Page Style
\makeatletter
\fancypagestyle{title}{
    \renewcommand{\headrulewidth}{0.4pt}
    \setlength{\headheight}{15pt}
    \fancyhead[R]{\@author}
    \fancyhead[L]{\@title}
    \fancyhead[C]{\@date}
}
\makeatother
\renewcommand{\maketitle}{\thispagestyle{title}}
\fancypagestyle{plain}{
    \fancyhf{}
    \renewcommand{\headrulewidth}{0pt}
    \renewcommand{\footrulewidth}{0pt}
    \fancyfoot[R]{\thepage}
}
\pagestyle{plain}

% Problem Box
\setlength{\fboxsep}{4pt}
\newlength{\myparskip}
\setlength{\myparskip}{\parskip}
\newsavebox{\savefullbox}
\newenvironment{fullbox}{\begin{lrbox}{\savefullbox}\begin{minipage}{\dimexpr\textwidth-2\fboxsep\relax}\setlength{\parskip}{\myparskip}}{\end{minipage}\end{lrbox}\framebox[\textwidth]{\usebox{\savefullbox}}}
\newenvironment{pbox}[1][]{\begin{fullbox}\def\temp{#1}\ifx\temp\empty\else\paragraph{#1}\phantom{}\fi}{\end{fullbox}}

% Theorem Environments
\theoremstyle{definition}
\newtheorem{lemma}{Lemma}

% Tikz Environments
\newenvironment{drawing}{\begin{center}\begin{tikzpicture}}{\end{tikzpicture}\end{center}}
% \tikzcdset{row sep/normal=0pt}
\newenvironment{cd}{\begin{center}\begin{tikzcd}}{\end{tikzcd}\end{center}}

% Default Commands
\newcommand{\isp}[1]{\quad\text{#1}\quad}
\newcommand{\N}{\mathbb{N}} 
\newcommand{\Z}{\mathbb{Z}}
\newcommand{\Q}{\mathbb{Q}}
\newcommand{\R}{\mathbb{R}}
\newcommand{\C}{\mathbb{C}}
\newcommand{\A}{\mathbb{A}}
\renewcommand{\P}{\mathbb{P}}
\newcommand{\eps}{\varepsilon}
\renewcommand{\phi}{\varphi}
\renewcommand{\emptyset}{\varnothing}
\newcommand{\<}{\langle}
\renewcommand{\>}{\rangle}
\newcommand{\iso}{\cong}
\newcommand{\eqc}{\overline}
\newcommand{\clo}{\overline}
\renewcommand{\tilde}{\widetilde}
\renewcommand{\hat}{\widehat}
\newcommand{\seq}{\subseteq}
\newcommand{\teq}{\trianglelefteq}
\DeclareMathOperator{\id}{id}
\DeclareMathOperator{\im}{im}
\newcommand{\inc}{\hookrightarrow}
\newcommand{\dd}{\mathrm{d}}
\newcommand{\mat}[1]{\begin{bmatrix}#1\end{bmatrix}}

% Extra Commands
\newcommand{\B}{\mathsf{B}}

% Document
\begin{document}
\title{MATH 232A Homework 2}
\author{Harry Coleman}
\date{October 13, 2022}
\maketitle

\begin{pbox}[1]
    A group $G$ can be thought of as a category which has just one object. The morphisms are the group elements and composition of morphisms is the group operation. What can you say about functors between two such categories?
\end{pbox}


Say $\B{G}$ is the category corresponding to $G$; denote its object by $\bullet$.

Note that the identity element $e \in G$ corresponds to the identity morphism $\id_\bullet$ in $\B{G}$:
For any element $g \in G$, we have
\[
    g \circ e = ge = g = ge = e \circ g.
\]
In other words, $e$ acts as the identity morphism on $\bullet$, so indeed $e = \id_\bullet$.

Let $H$ be another group and $\B{H}$ be the corresponding category with the element $*$.

A functor $F : \B{G} \to \B{H}$ corresponds to a group homomorphism $G \to H$ which maps elements of $G$ in the same way that $F$ maps the morphisms of $\B{G}$.
By the axioms of a functor, we must have $Fe_G = F\id_\bullet = \id_* = e_H$.
Moreover, for any $g_1, g_2 \in G$ we have
\[
    F(g_1 \circ g_2) = F(g_1g_2) = F(g_1)F(g_2) = F(g_1) \circ F(g_2).
\]
So indeed, $F$ induces a group homomorphism $G \to H$.


\newpage
\begin{pbox}[2]
    Recall that the \textit{center} of a group $G$ is the subgroup $Z(G)$ of elements which commute with every element of $G$. The \textit{commutator subgroup} of $G$ is the subgroup $[G, G]$ generated by elements of the form $ghg^{-1}h^{-1}$ for $g,h\in G$. Does $G\mapsto Z(G)$ (with homomorphisms going to homomorphisms in some sensible way) define a functor from groups to groups? 
\end{pbox}

No.

Write the dihedral group $D_6 = \<r, s \mid r^3 = s^2 = (sr)^2 = 1\>$.

Alternatively, we can write it as the semidirect product $D_6 = Z_3 \rtimes Z_2$ with $Z_2 = \<s\>$ and $Z_3 = \<r\>$
Hence, we have an inclusion $Z_2 \inc D_6$ and a quotient map $D_6 \to D_6/Z_3 = Z_2$.
Moreover, one can check that the following diagram commutes:
\begin{cd}
    Z_2 \rar[hook] \ar[rr, bend right=30, "\id"'] & D_6 \rar[two heads] & Z_2
\end{cd}

Supposing the center operation was a functor, this diagram would be sent to a commutative diagram of the form
\begin{cd}
    Z_2 \rar \ar[rr, bend right=30, "\id"'] & Z(D_6)=0 \rar & Z_2
\end{cd}
But clearly no such diagram can exist because the identity on $Z_2$ cannot factor through the trivial map.

\begin{pbox}[]
    What about $G\mapsto [G, G]$?
\end{pbox}

Yes.

If $f : G \to H$ is a homomorphism of groups, the commutator functor should send it to the restricted map $f : [G, G] \to [H, H]$.
This is well defined since for $[g, h] = ghg^{-1}h^{-1} \in [G, G]$ we have 
\[
    f([g, h]) = f(g)f(h)f(g)^{-1}f(h)^{-1} = [f(g), f(h)] \in [H, H].
\]
And since it is simply a restriction of a group homomorphism, this is still a group homomorphism.
One can check that this construction satisfies the functorality axioms.



\newpage
\begin{pbox}[3 Hatcher 2.11]
    Show that if $A$ is a retract of $X$, then the map $H_n(A) \to H_n(X)$ induced by the inclusion $A\subseteq X$ is injective.
\end{pbox}

\begin{proof}
    Let $r : X \to A$ be a retraction and $\iota : A \inc X$ be the inclusion.
    Then by definition, we have $r \circ \iota = \id_A$.
    Then we have induced maps
    \[
        \id_{H_n(A)} = (\id_A)_* = (r \circ \iota)_* = r_* \circ \iota_*.
    \]
    Since $\id_{H_n(A)}$ is injective, then so must $\iota_*$
\end{proof}

\end{document}
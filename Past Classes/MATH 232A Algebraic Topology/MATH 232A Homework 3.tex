\documentclass[12pt]{article}

% Packages
\usepackage[margin=1in]{geometry}
\usepackage{fancyhdr, parskip}
\usepackage{amsmath, amsthm, amssymb}
\usepackage{tikz, tikz-cd}
\usepackage[shortlabels]{enumitem}

% Page Style
\makeatletter
\fancypagestyle{title}{
    \renewcommand{\headrulewidth}{0.4pt}
    \setlength{\headheight}{15pt}
    \fancyhead[R]{\@author}
    \fancyhead[L]{\@title}
    \fancyhead[C]{\@date}
}
\makeatother
\renewcommand{\maketitle}{\thispagestyle{title}}
\fancypagestyle{plain}{
    \fancyhf{}
    \renewcommand{\headrulewidth}{0pt}
    \renewcommand{\footrulewidth}{0pt}
    \fancyfoot[R]{\thepage}
}
\pagestyle{plain}

% Problem Box
\setlength{\fboxsep}{4pt}
\newlength{\myparskip}
\setlength{\myparskip}{\parskip}
\newsavebox{\savefullbox}
\newenvironment{fullbox}{\begin{lrbox}{\savefullbox}\begin{minipage}{\dimexpr\textwidth-2\fboxsep\relax}\setlength{\parskip}{\myparskip}}{\end{minipage}\end{lrbox}\framebox[\textwidth]{\usebox{\savefullbox}}}
\newenvironment{pbox}[1][]{\begin{fullbox}\def\temp{#1}\ifx\temp\empty\else\paragraph{#1}\phantom{}\fi}{\end{fullbox}}

% Theorem Environments
\theoremstyle{definition}
\newtheorem{lemma}{Lemma}

% Tikz Environments
\newenvironment{drawing}{\begin{center}\begin{tikzpicture}}{\end{tikzpicture}\end{center}}
% \tikzcdset{row sep/normal=0pt}
\newenvironment{cd}{\begin{center}\begin{tikzcd}}{\end{tikzcd}\end{center}}

% Default Commands
\newcommand{\isp}[1]{\quad\text{#1}\quad}
\newcommand{\N}{\mathbb{N}} 
\newcommand{\Z}{\mathbb{Z}}
\newcommand{\Q}{\mathbb{Q}}
\newcommand{\R}{\mathbb{R}}
\newcommand{\C}{\mathbb{C}}
\newcommand{\A}{\mathbb{A}}
\renewcommand{\P}{\mathbb{P}}
\newcommand{\eps}{\varepsilon}
\renewcommand{\phi}{\varphi}
\renewcommand{\emptyset}{\varnothing}
\newcommand{\<}{\langle}
\renewcommand{\>}{\rangle}
\newcommand{\iso}{\cong}
\newcommand{\eqc}{\overline}
\newcommand{\clo}{\overline}
\renewcommand{\tilde}{\widetilde}
\renewcommand{\hat}{\widehat}
\newcommand{\seq}{\subseteq}
\newcommand{\teq}{\trianglelefteq}
\DeclareMathOperator{\id}{id}
\DeclareMathOperator{\im}{im}
\newcommand{\inc}{\hookrightarrow}
\newcommand{\dd}{\mathrm{d}}
\newcommand{\mat}[1]{\begin{bmatrix}#1\end{bmatrix}}

% Extra Commands


% Document
\begin{document}
\title{MATH 232A Homework 3}
\author{Harry Coleman}
\date{October 20, 2022}
\maketitle

\begin{pbox}[1]
    Redo \#1 from Homework 1, but this time use the snake lemma: Let $A$ be a $\Delta$-complex and build a new $\Delta$-complex $X$ by adding a single new $n$-simplex $D$. (\textbf{Clarification:} $D$ is glued to $A$ via a continuous map from the boundary of $D$ to $A$.) Using simplicial homology, compute the difference between $H_*(A)$ and $H_*(X)$. (This will depend heavily on the chain $\partial D$.)
\end{pbox}

\begin{pbox}[2 Hatcher 2.1.12]
    Show that chain homotopy of chain maps is an equivalence relation.
\end{pbox}

\begin{pbox}[3 Hatcher 2.1.14]
    Determine whether there exists a short exact sequence $0\to \Z_4 \to \Z_8 \oplus \Z_2 \to \Z_4 \to 0$. More generally, determine which abelian groups $A$ fit into a short exact sequence $0\to \Z_{p^m}\to A \to \Z_{p^n}\to 0$ with $p$ prime. What about the case of short exact sequences $0\to \Z \to A \to \Z_n \to 0$?
\end{pbox}

\begin{pbox}[4 Hatcher 2.1.16 \\ (a)]
    Show that $H_0(X,A)=0$ iff $A$ meets each path-component of $X$.
\end{pbox}

\begin{pbox}[(b)]
    Show that $H_1(X,A)=0$ iff $H_1(A) \to H_1(X)$ is surjective and each path-component of $X$ contains at most one path-component of $A$.
\end{pbox}

\begin{pbox}[5 Hatcher 2.1.18]
    Show that for the subspace $\Q\subseteq \R$, the relative homology group $H_1(\R,\Q)$ is free abelian and find a basis.
\end{pbox}

\end{document}
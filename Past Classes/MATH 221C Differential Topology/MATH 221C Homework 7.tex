\documentclass[12pt]{article}

% Packages
\usepackage[margin=1in]{geometry}
\usepackage{fancyhdr, parskip}
\usepackage{amsmath, amsthm, amssymb}
\usepackage{tikz, tikz-cd}

% Page Style
\makeatletter
\fancypagestyle{title}{
    \renewcommand{\headrulewidth}{0.4pt}
    \setlength{\headheight}{15pt}
    \fancyhead[R]{\@author}
    \fancyhead[L]{\@title}
    \fancyhead[C]{\@date}
}
\makeatother
\renewcommand{\maketitle}{\thispagestyle{title}}
\fancypagestyle{plain}{
    \fancyhf{}
    \renewcommand{\headrulewidth}{0pt}
    \renewcommand{\footrulewidth}{0pt}
    \fancyfoot[R]{\thepage}
}
\pagestyle{plain}

% Problem Box
\setlength{\fboxsep}{4pt}
\newlength{\myparskip}
\setlength{\myparskip}{\parskip}
\newsavebox{\savefullbox}
\newenvironment{fullbox}{\begin{lrbox}{\savefullbox}\begin{minipage}{\dimexpr\textwidth-2\fboxsep\relax}\setlength{\parskip}{\myparskip}}{\end{minipage}\end{lrbox}\framebox[\textwidth]{\usebox{\savefullbox}}}
\newenvironment{pbox}[1][]{\begin{fullbox}\def\temp{#1}\ifx\temp\empty\else\paragraph{#1}\phantom{}\fi}{\end{fullbox}}

\newcommand{\pnum}[1]{\paragraph{#1}}

% Theorem Environments
\theoremstyle{definition}
\newtheorem{lemma}{Lemma}
\newtheorem{theorem}{Theorem}

% Tikz Environments
\newenvironment{drawing}{\begin{center}\begin{tikzpicture}}{\end{tikzpicture}\end{center}}
% \tikzcdset{row sep/normal=0pt}
\newenvironment{cd}{\begin{center}\begin{tikzcd}}{\end{tikzcd}\end{center}}

% Default Commands
\newcommand{\isp}[1]{\quad\text{#1}\quad}
\newcommand{\N}{\mathbb{N}} 
\newcommand{\Z}{\mathbb{Z}}
\newcommand{\Q}{\mathbb{Q}}
\newcommand{\R}{\mathbb{R}}
\newcommand{\C}{\mathbb{C}}
\newcommand{\A}{\mathbb{A}}
\renewcommand{\P}{\mathbb{P}}
\newcommand{\eps}{\varepsilon}
\renewcommand{\phi}{\varphi}
\renewcommand{\emptyset}{\varnothing}
\newcommand{\<}{\langle}
\renewcommand{\>}{\rangle}
\newcommand{\iso}{\cong}
\newcommand{\eqc}{\overline}
\newcommand{\clo}{\overline}
\newcommand{\seq}{\subseteq}
\newcommand{\teq}{\trianglelefteq}
\DeclareMathOperator{\id}{id}
\DeclareMathOperator{\im}{im}
\newcommand{\inc}{\hookrightarrow}
\newcommand{\dd}{\mathrm{d}}
\newcommand{\mat}[1]{\begin{bmatrix}#1\end{bmatrix}}

% Extra Commands
\newcommand{\bd}{\partial}
\newcommand{\htpy}{\simeq}
\DeclareMathOperator{\inter}{int}


% Document
\begin{document}
\title{MATH 221C Homework 7}
\author{Harry Coleman}
\date{May 15, 2022}
\maketitle

For any map $g : X \to \R^n$ with $z \notin \im g$, define the map $u_{g,z} : X \to S^{n-1}$ by
\[
    u_{g,z}(x) = \frac{g(x) - z}{\|g(x) - z\|}.
\]
We use simply $u_g$ when the point $z$ is clear from context.

\pnum{Exercise 2.5.1}

By definition, $W_2(f, z) = \deg_2(u_f) = \#u_f^{-1}(y)$, where $y \in S^{n-1}$ is a regular value of $u_f$
Since $z \notin \im F$, we can extend $u_f$ to the map $u_F : D \to S^{n-1}$.
If $y \in S^{n-1}$ is a regular value of $u_F$, the preimage $u_F^{-1}(y) \seq D$ is a compact $1$-manifold, consisting of finitely many arcs and circles.
Each arc has two endpoints, so the boundary
\[
    \bd (u_F^{-1}(y))
        = u_F^{-1}(y) \cap \bd D
        = u_f^{-1}(y)
\]
consists of an even number of points.
Hence, $W_2(f, z) = \#u_f^{-1}(y) \bmod{2} = 0$.

\pnum{Exercise 2.5.2}

Define the compact submanifold
\[
    E = D \setminus \bigcup_{i=1}^{\ell} \inter B_i
\]
which has boundary
\[
    \bd E = \bd D \sqcup \bigsqcup_{i=1}^{\ell} \bd B_i.
\]
Define the restrictions $G = F|_E$ and $g = \bd G$.
By construction, $F^{-1}(z) \seq D \setminus E$, which means that $G$ does not hit $z$, so by Exercise 2.5.1 we have $W_2(g, z) = 0$.

By definition, we have $W_2(g, z) = \#u_g^{-1}(y)$, where $y \in S^{n-1}$ is a regular value of $u_g$.
Moreover, since $u_g^{-1} \seq \bd E$ we have
\[
    u_g^{-1}(y)
        = \left(u_g^{-1}(y) \cap \bd E\right) \sqcup \bigsqcup_{i=1}^{\ell} \left(u_g^{-1}(y) \cap \bd B_i\right)
        = u_f^{-1}(y) \sqcup \bigsqcup_{i=1}^{\ell} u_{f_i}^{-1}(y).
\]
It follows that
\[
    0 = W_2(g, z) = W_2(f, z) + \sum_{i=1}^{\ell} W_2(f_i, z) \mod{2},
\]
hence
\[
    W_2(f, z) = \sum_{i=1}^{\ell} W_2(f_i, z) \mod{2}.
\]

\pnum{Exercise 2.5.3}

By the Stack of Records Theorem (Homework 5 Exercise 1.5.7), there is an open neighborhood $V$ of $z$ in $\R^n$ such that $F^{-1}(V) = U_1 \sqcup \cdots \sqcup U_\ell$, where each $U_i$ is an open neighborhood of $y_i$ and $F$ restricts to a diffeomorphism from $U_i$ to $V$.
Let $\phi_i : V \to U_i$ be a smooth inverse of $F|_{U_i}$, then $\phi_i$ is a local parameterization of of $D$ at $y_i$.
Choose a closed ball $\bar{B}_r(z) \seq V$ with $r > 0$ and put $B_i = \phi_i(\bar{B}_r(z))$.
Then $f_i$ is a diffeomorphism from $\bd B_i$ to the sphere $\bd \bar{B}_r(z)$.
Since $\bd\bar{B}_{r}(z)$ is a sphere, the shift and scale map $w \mapsto (w - z)/r$ is a bijection to the unit sphere $S^{n-1}$.
But $u_{f_i}$ is simply the composition of this map with $f_i$, so $u_{f_i}$ is a bijection.
So then $\#u_{f_i}^{-1}(y) = 1$ for all $y \in S^{n-1}$, hence $W_2(f_i, z) = 1$.

Combining these results, we obtain
\[
    W_2(f, z)
        \equiv \sum_{i=1}^{\ell} W_2(f_i, z)
        \equiv \sum_{i=1}^{\ell} 1
        \equiv \ell
        \equiv \#F^{-1}(z) \pmod{2}.
\]

\pnum{Exercise 2.5.4}

Let $A \seq X$ be the set of points $x \in X$ for which the statement is true.

We first check that $A$ is nonempty.
Since $X$ is compact, there is a point $x \in X$ for which the Euclidean distance $d(x, z)$ is minimized over all of $X$.
Let $\gamma : I \to \R^n$ be the straight line segment from $z$ to $x$.
Then for any neighborhood $U$ of $x$ in $\R^n$, the preimage $\gamma^{-1}(U)$ is an open neighborhood of $1$ in $I$.
Choose some $t \in \gamma^{-1}(U) \setminus \{1\}$, then $\gamma$ restricted to $[0, t]$ is the straight line segment from $z$ to $\gamma(t) \in U$.
This curve does not intersect $X$, otherwise there would be a point in $X$ strictly closer to $z$ than $x$.
Hence, $x \in A$.

Next, we check that $A$ is closed.
Let $\clo{A}$ be the closure of $A$ inside $X$.
Let $x \in \clo{A}$ and $U$ be an open neighborhood of $x$ in $\R^n$.
Then $U \cap X$ is an open neighborhood of $x$ in $X$, so there is some $a \in U \cap A$.
But then $U$ is an open neighborhood of $a$ in $\R^n$, so by definition of $A$ there is a curve in $\R^n \setminus X$ between $z$ and a point of $U$.
Hence, $x \in A$, which implies $A = \clo{A}$.

Lastly, we check that $A$ is open.
Let $x \in A$ be any point.
The derivative of the inclusion $\iota : X \inc \R^n$ is the inclusion of tangent spaces $\dd{\iota}_x : T_xX \inc T_x\R^n = \R^n$.
In particular, $\iota$ is an immersion at $x$ so the local immersion theorem gives us local coordinates at $x$ such that $\iota$ looks like the standard immersion.
We restrict these local coordinates to a ball $B = B_r(0) \seq \R^n$ with a parameterization $\phi : B \to \R^n$ of $\R^n$ at $x$ such that the restriction to $U = \R^{n-1} \cap B$ is a parameterization of $X$ at $x$.
Then $B \setminus \R^{n-1}$ consists of an upper component $V^+$ and a lower component $V^-$.
Each is an open half of the $n$-ball, so each image $\phi(V^\pm)$ is path-connected in $\R^n \setminus X$.

Then $\phi(U)$ is an open neighborhood of $x$ in $X$, which we claim is contained in $A$.
Let $y \in \phi(U)$ and $W \seq \R^n$ be an open neighborhood of $y$.
Since $\phi(B)$ is an open neighborhood of $x$, then $z$ is connected to a point $p \in \phi(B)$ via a path outside $X$.
Either $p \in \phi(V^+)$ or $p \in \phi(V^-)$; without loss of generality we assume the former.
Choosing a point $q \in W \cap \phi(V^+)$, there is a path from $p$ to $q$ inside of $V^+$.
Performing a smooth concatenation, we obtain a path from $z$ to $q$ outside of $X$, hence $y \in A$.

Since $X$ is connected but $A \seq X$ is nonempty and clopen, we conclude that $X = A$.

\pnum{Exercise 2.5.5}

Let $\phi : B \to \R^n$ at $x$ be as in the proof of Exercise 2.5.4.
Then for any point $z \in \R^n \setminus X$, there is a path from $z$ to a point $z' \in \phi(B)$ outside of $X$.
Then $z'$ is either in $\phi(V^+)$ or $\phi(V^-)$, which implies $z$ is either in the connected component of $\phi(V^+)$ or $\phi(V^-)$.
Hence, there are at most two components---one containing $\phi(V^+)$ and one containing $\phi(V^-)$.

\pnum{Exercise 2.5.6}

Denote the inclusion $\iota : X \inc \R^n$, then for any $z \in \R^n \setminus X$ we have
\[
    W_2(X, z) = W_2(\iota, z) = \deg_2 u_{\iota, z}.
\]
A path from $z_0$ to $z_1$ outside $X$ gives homotopy from $u_{\iota, z_0}$ to $u_{\iota, z_1}$.
Then degree mod $2$ is homotopy invariant, so
\[
    W_2(X, z_0) = \deg_2 u_{\iota, z_0} = \deg_2 u_{\iota, z_1} = W_2(X, z_1).
\]

\pnum{Exercise 2.5.7}

By definition, $v$ is a regular value of $u$ if and only if $u$ is transverse to the $0$-manifold $\{v\}$.
By Homework 5 Exercise 1.5.7, this is the case if and only if $\id_X$ is transverse to $u^{-1}(v)$, i.e., if $X$ is transverse to $u^{-1}(v) = r$.

\pnum{Exercise 2.5.8}

By Exercise 2.5.7, the direction vector $v$ of $r$ is a regular value of $u_{\iota, z_i}$, so
\[
    W_2(X, z_i)
        = \deg_2 u_{\iota, z_i}
        = \#u_{\iota, z_i}^{-1}(v) \mod{2}.
\]
Note that $u_{\iota, z_i}^{-1}(v)$ is precisely the intersection of the ray $\{z_i + tv : t \geq 0\}$ with $X$.
But the ray emanating from $z_0$ intersects $X$ at the same points as the ray emanating from $z_1$, plus the $\ell$ additional points on the segment from $z_0$ and $z_1$.
So
\[
    \#\big(\{z_0 + tv : t \geq 0\} \cap X\big) = \#\big(\{z_1 + tv : t \geq 0\} \cap X\big) + \ell,
\]
and taking this equation modulo $2$ gives us the result.


\pnum{Exercise 2.5.9}

We first show that there does in fact exist a ray $r$ originating from a point $z_0 \in \R^n \setminus X$ which intersects $X$ both transversally and nontrivially.
Let $x \in X$ be any point.
After translating and rotating, assume $x = e_n$ and $T_{e_1}X = \R^{n-1}$.
Then consider the normalization map $u = u_{\iota, 0} : X \to S^{n-1}$.
The derivative of this map $\dd{u}_{e_1}$ is the identity on the tangent spaces $T_{e_1}X = \R^{n-1} = T_{e_1}S^{n-1}$.
By the inverse function theorem, $u$ is a local diffeomorphism at $e_n$, so in particular the image of $u$ contains an open set in $S^{n-1}$.
Since almost every point of $S^{n-1}$ is a regular value of $u$, we can find a regular value $v \in \im u$.
Then $v$ defines the desired ray.


Then let $z_1$ be a point on the ray $r$ between its first and second intersection with $X$.
Then by Exercise 2.5.8, we know that $W_2(X, z_0) = W_2(X, z_1) + 1 \mod{2}$.
In particular, $W_2(X, z_0) \ne W_2(X, z_1)$ so Exercise 2.5.6 implies that $z_0$ and $z_1$ must be in separate connected components of $\R^n \setminus X$, hence there are at least two components.
Since Exercise 2.5.5 tells us there are at most two components, we conclude that there are exactly two components.
Since the winding number can only ever be $0$ or $1$, it follows that two points are in the same connected component if and only if their winding numbers are the same.


\pnum{Exercise 2.5.10}

Since $X$ is a compact subset of $\R^n$, it is bounded, i.e., $X \seq B_R(0)$ for some radius $R > 0$.
Let $z \in X \setminus B_R(0)$, i.e., $|z| \geq R$.
After a translation and rotation, we can put $z$ at the origin and $B_R(0)$ in the upper half space $H^n$.
Then $u_{\iota, z} : X \to S^{n-1}$ is just the normalization map.
But $X \seq H^n$ implies the image of $X$ is still inside the upper half space.
In particular, $-e_n$ is a regular value of $u_{\iota, z}$ so
\[
    W_2(X, z) = \deg_2 u_{\iota, z} = \# u_{\iota, z}^{-1}(-e_n) \mod{2} = 0.
\]


\pnum{Exercise 2.5.11}

Exercise 2.5.9 tells us that $\R^n \setminus X$ consists of the two connected components $D_0$ and $D_1$.
If $z \in D_i$, then $X$ being closed implies $B_r(z) \seq \R^n \setminus X$ for some radius $r > 0$.
But $B_r(z)$ is convex, so $B_r(z) \seq D_i$, hence $D_i$ is open.
Exercise 2.5.10 tells us $D_1 \seq B_R(0)$, so $D_1$ is bounded, hence $\clo{D_1}$ is compact.


Since $D_1 \seq \R^n$ is open, the identity map $D_1 \to D_1$ is a local parameterization of $\clo{D_1}$ at all the points in $D_1$.
Let $\phi : B \to \R^n$ be as in the proof of Exercise 2.5.4 at a point $x \in X$.
Assume $\phi(V^+) \seq D_1$; otherwise precompose $\phi$ with the map $\R^n \to \R^n$ which takes the opposite of the $n$th coordinate.
Since we could make $\phi(B)$ arbitrary small but it always intersects $D_1$ at $\phi(V^+)$, we deduce that $x \in \clo{D_1}$.
Then the restriction of $\phi$ to $H^n \cap B$ is a local parameterization of $\clo{D_1}$ at $x$, and tells us that $x \in \bd\clo{D_1}$.
Lastly, $D_1 \cup X$ is a closed set containing $D_1$ and contained on the closure, so we must have $\clo{D_1} = D_1 \cup X$.
Therefore, we have found local parameterizations everywhere in $\clo{D_1}$, making it a manifold with boundary, and $\bd\clo{D_1} = X$.




\pnum{Exercise 2.5.12}

Since $X$ is compact, assume $X \seq B_R(0)$ and choose $z' \in r$ with $\|z'\| > R$.
Then $r$ does not cross $X$ beyond $z'$, i.e., $r$ intersects $X$ only at points between $z$ and $z'$.
By Exercise 2.5.10 we know $W_2(X, z') = 0$.
Then Exercise 2.5.8 tells us $W_2(X, z) = \ell \bmod 2$.
Then by definition $x$ is inside $X$, i.e., $x \in D_1$, if and only if $\ell$ is odd.

\end{document}
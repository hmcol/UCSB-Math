\documentclass[12pt]{article}

% packages
\usepackage{kantlipsum}
\usepackage[margin=1in]{geometry}
\usepackage[labelfont=it]{caption}
\usepackage[table]{xcolor}
\usepackage{subcaption,framed,colortbl,multirow,enumitem}
\usepackage{amsmath,amsthm,amssymb,wasysym,mathrsfs,mathtools,babel}
\usepackage{tikz,graphicx,pgf,pgfplots}
\usetikzlibrary{arrows, angles, quotes, decorations.pathreplacing, math, patterns, calc}
\pgfplotsset{compat=1.16}

% Theorems
\newtheorem{theorem}{Theorem}
\newtheorem{lemma}{Lemma}
\newtheorem{proposition}{Proposition}

% Problem Box
\setlength{\fboxsep}{4pt}
\newsavebox{\mybox}
\newenvironment{problem}
    {\begin{lrbox}{\mybox}\begin{minipage}{\textwidth-10pt}}
    {\end{minipage}\end{lrbox}\framebox[6.5in]{\usebox{\mybox}}}

% Environments
\newenvironment{drawing}{\begin{center}\begin{tikzpicture}}{\end{tikzpicture}\end{center}}
\newenvironment{response}{\paragraph{}}{}

% Formatting
\newcommand{\ds}{\displaystyle}
\newcommand{\isp}[1]{\quad\text{#1}\quad}
\newcommand{\seq}[2]{\left\{#1\right\}_{#2=1}^\infty}
\newcommand{\clo}[1]{\overline{#1}}

% Paired Delimiters
\DeclarePairedDelimiter{\ceil}{\lceil}{\rceil}
\DeclarePairedDelimiter\floor{\lfloor}{\rfloor}
\DeclarePairedDelimiter{\ang}{\langle}{\rangle}

% Sets
\newcommand{\N}{\mathbb{N}}
\newcommand{\Z}{\mathbb{Z}}
\newcommand{\I}{\mathbb{I}}
\newcommand{\R}{\mathbb{R}}
\newcommand{\Q}{\mathbb{Q}}
\newcommand{\C}{\mathbb{C}}
\newcommand{\F}{\mathbb{F}}

% Misc Characters
\newcommand{\powerset}{\raisebox{.15\baselineskip}{\Large\ensuremath{\wp}}}
\let\eps\varepsilon
\let\emptyset\varnothing

% Functions
\newcommand{\id}[1]{\mathsf{id}_{#1}}

% Babel Shorthands
\useshorthands*{"}
\defineshorthand{"-}{\setminus}
\defineshorthand{"d}{\partial}

% Topology
\newcommand{\T}{\mathscr{T}}
\renewcommand{\S}{\mathscr{S}}
\newcommand{\B}{\mathscr{B}}
\renewcommand{\int}{\text{int}}
\newcommand{\diam}{\text{diam}}
 
\begin{document}
 
\title{Exam 1\\
    \large MATH 145 Intro to Topology
}
\author{Harry Coleman}
\date{July 17, 2020}
\maketitle

\section*{Problem 1(a)}
\begin{problem}
    State the definition of a topological space.
\end{problem}

\begin{response}
    Given a set $X$, a topology $T$ on $X$ is a collection of subsets of $X$ such that:
    \begin{enumerate}
        \item $\emptyset, X \in T$,
        \item If $\{U_\alpha : \alpha\in I\}\subseteq T$, then $\bigcup_{\alpha\in I}U_\alpha \in T$,
        \item If $\{U_i : i=1,\dots,n\}\subseteq T$, then $\bigcap_{i=1}^nU_i \in T$.
    \end{enumerate}
    We say the pair $(X,T)$ is a topological space.
\end{response}

\section*{Problem 1(b)}
\begin{problem}
    Prove that every metric space is a topological space.
\end{problem}

\begin{proof}
    Suppose $(X,d)$ is a metric space. We define the metric topology $T$ on $X$ the be the collection of subsets of $X$ which are open with respect to the metric $d$. The empty set is vacuously open in the metric, and therefore $\emptyset\in T$. Let $x\in X$, then for any $\eps>0$, the open ball $B(\eps, x)\subseteq X$. So $X$ is open in the metric, and therefore $X\in T$.
    
    Now suppose $\{U_\alpha : \alpha\in I\}\subseteq T$, i.e, that each $U_\alpha$ is open in the metric space. Then for any $x\in\bigcup_{\alpha\in I}U_\alpha$, there is a particular $U_\alpha$ such that $x\in U_\alpha$, and an open ball $B(\eps, x)\subseteq U_\alpha \subseteq \bigcup_{\alpha\in I}U_\alpha$. Therefore $\bigcup_{\alpha\in I}U_\alpha$ is open in the metric space and therefore in $T$.
    
    Now suppose $\{U_i : i=1,\dots,n\}\subseteq T$, i.e, that each $U_i$ is open in the metric space. The for any $x\in\bigcap_{i=1}^nU_i$, we define the open ball $B(\eps_i, x)$ for each $i=1,\dots,n$ such that $B(\eps_i, x)\subseteq U_i$. Then if $\eps = \min\{\eps_i : i=1,\dots,n\}$, then $B(\eps,x)\subseteq\bigcap_{i=1}^nU_i$. Therefore $\bigcap_{i=1}^nU_i$ is open in the metric space and therefore in $T$. Thus, $(X,T)$ is a topology.
    
\end{proof}

\section*{Problem 1(c)}
\begin{problem}
    Is the converse of part (b) true? That is, is every topological space a metric space? Justify your answer.
\end{problem}

\begin{response}
    No, not every topological space is a metric space. In particular, a topological space $X$ is metrizable if and only if for every $p,q\in X$ with $p\ne q$, there exist open subsets $U,v\subseteq X$ such that $p\in U$, $q\in V$, and $U\cap V=\emptyset$. For example, the cofinite topology on an infinite set does not satisfy this condition, and is therefore not metrizable.
\end{response}

\section*{Problem 2(a)}
\begin{problem}
    State the definition of a continuous map between two topological spaces.
\end{problem}

\begin{response}
    Given topological spaces $X$ and $Y$, a function $f:X\to Y$ is continuous if $f^{-1}(U)$ is an open subset of $X$ for every open subset $U$ of $Y$.
\end{response}

\section*{Problem 2(b)}
\begin{problem}
    Suppose that $f : X \to Y$ is a continuous map, determine whether each of the following statements is true:
\end{problem}

\subsection*{Problem 2(b)(i)}
\begin{problem}
    If $Z\subset Y$ is a closed subset, then $f^{-1}(Z)$ is always closed in $X$.
\end{problem}

\begin{response}
    True.
\end{response}

\subsection*{Problem 2(b)(ii)}
\begin{problem}
    If $Z\subset Y$ is a compact subset, then $f^{-1}(Z)$ is always compact.
\end{problem}

\begin{response}
    False. Consider the constant function $f:\R\to\R$ which maps all points of $\R$ to $1$; $f$ is continuous. However, $\{1\}$ is a compact subset of $\R$ and $f^{-1}(\{1\})=\R$ is not compact.
\end{response}

\subsection*{Problem 2(b)(iii)}
\begin{problem}
    If $Z\subset X$ is a closed subset, then $f(Z)$ is always closed in $Y$.
\end{problem}

\begin{response}
    False. Consider the function $f:\R\to \R$ given by $f(x)=2^x$; $f$ is continuous. However, $\R$ is closed and $f(\R)=(0,\infty)$ is not closed. 
\end{response}

\subsection*{Problem 2(b)(iv)}
\begin{problem}
    If $Z\subset X$ is a compact subset, then $f(Z)$ is always compact.
\end{problem}

\begin{response}
    True.
\end{response}

\begin{proof}
    Suppose $Z\subseteq X$ is compact and let $\{U_\alpha : \alpha\in I\}$ be an open cover of $f(Z)$. Then $\{f^{-1}(U_\alpha) : \alpha \in I\}$ is an open cover of $Z$ and has a finite subcover $\{f^{-1}(U_i) : i=1,\dots,n\}$. This gives us the finite cover $\{U_i : i=1,\dots,n\}$ of $Z$, which is a subcover of the original cover for $Z$. Therefore $Z$ is compact.
    
\end{proof}

\section*{Problem 2(c)}
\begin{problem}
    Justify your answer for statement (i) in part (b).
\end{problem}

\begin{proof}
    Let $Z\subseteq Y$ be closed. Then $Z^C$ is open, which implies $f^{-1}(Z^C)$ is open. Therefore $f^{-1}(Z^C)^C=f^{-1}(Z)$ is closed.
    
\end{proof}

\newpage
\section*{Problem 3}
\begin{problem}
    Let $X$ be a Hausdorff topological space, $E,F\subset X$ be two disjoint compact subsets, prove that there exists open subsets $U,V\subset X$ such that $E\subset U, F\subset V$, and $U\cap V=\emptyset$.
\end{problem}

\begin{proof}
    For each $x\in E$ and $y\in F$, there exist disjoint open subsets $U_{x,y},V_{x,y}$ such that $x\in U_{x,y}$ and $y\in V_{x,y}$. For a fixed $x$, note that $\{V_{x,y} : y\in F\}$ is an open cover of $F$. And since $F$ is compact, there exists a finite subcover $\{V_{x,y_j} : j=1,\dots,m\}$. We now let
    \[U_x = \bigcap_{j=1}^mU_{x,y_j} \isp{and} V_x = \bigcup_{j=1}^mV_{x,y_j},\]
    which gives us open disjoint subsets $U_x,V_x$ such that $x\in U_x$ and $F\subseteq V_x$. Now we have the open cover $\{U_x : x\in E\}$ of $E$, which has a finite subcover $\{U_{x_i} : i=1,\dots n\}$. We now let
    \[U = \bigcup_{i=1}^nU_{x_i} \isp{and} V = \bigcap_{i=1}^nV_{x_i},\]
    which gives us open disjoint subsets $U,V$ such that $E\subseteq U$ and $F\subseteq V$.
    
\end{proof}

\section*{Problem 4}
\begin{problem}
    Prove that a metric space $(X, d)$ is complete if and only if for any nested infinite sequence $E_1 \supset E_2 \supset E_3 \supset \dots$ of closed subsets with $\lim_{i\to\infty} \diam(E_i) = 0$, where $\diam(E_i) := \sup\{d(x, y) : x, y \in E_i\}$, the intersection of all the closed subsets $\cap_iE_i$ is nonempty.
\end{problem}

\begin{proof}
    Suppose $(X,d)$ is a complete metric space. Let $E_1 \supset E_2 \supset E_3 \supset \dots$ be a nested infinite sequence of closed subsets with $\lim_{i\to\infty} \diam(E_i) = 0$. We construct the Cauchy sequence $\seq{x_n}{n}$ in $X$ such that $x_n\in E_n$ for all $n\in\N$. Then for each $k\in\N$, the sequence $\{x_n\}_{n=k}^\infty$ is a Cauchy sequence in $E_k$, which is complete as a closed subset of a complete metric space. Therefore $\seq{x_n}{n}$ converges to some $x\in E_i$ for all $i\in\N$. Therefore $x\in\cap_iE_i$.
    
    Now suppose any nested infinite sequence $E_1 \supset E_2 \supset E_3 \supset \dots$ of closed subsets with $\lim_{i\to\infty} \diam(E_i) = 0$, the intersection of all the closed subsets $\cap_iE_i$ is nonempty. Let $\seq{x_n}{n}$ be a Cauchy sequence in $X$ and let $E_i=\clo{\{x_n : n\geq i\}}$. Then $E_1 \supset E_2 \supset E_3 \supset \dots$ is a nested infinite sequence of closed subsets with $\lim_{i\to\infty} \diam(E_i) = 0$. Therefore $\cap_iE_i\ne\emptyset$ and for some $x\in\cap_iE_i$, the sequence $\seq{x_n}{n}$ converges to $x$. Therefore $(X,d)$ is complete.
    
\end{proof}

\newpage
\section*{Problem 5(a)}
\begin{problem}
    Prove that $\R$ has a countable base for its topology.
\end{problem}

\begin{proof}
    Since $\{(a,b) : a,b\in\R\}$ is a base for $\R$, all open subsets of $\R$ are unions of real open intervals. We claim that the countable set $\{(p,q) : p,q\in\R_0\}$ is also a base for $\R$. Let $(a,b)$ be a real open interval. We claim that
    \[(a,b) = \bigcup_{p,q\in(a,b)\cap\R_0}(p,q).\]
    For any $c\in(a,b)$, there exists some $p,q\in\R_0$ such that $a<p<c<q<b$, that is, $c\in(p,q)\subseteq(a,b)$. So every real open interval is a union of rational open intervals, and therefore $\{(p,q) : p,q\in\R_0\}$ is a countable base for $\R$.
    
\end{proof}

\section*{Problem 5(b)}
\begin{problem}
    Prove that every open cover of $\R$ has a countable subcover.
\end{problem}

\begin{proof}
    Suppose $\{U_\alpha : \alpha \in I\}$ is an open cover of $\R$. Note that
    \[\R= \bigcup_{n=1}^\infty[-n,n],\]
    where each $[-n,n]$ is a compact subset of $\R$. Since $\{U_\alpha : \alpha\in I\}$ is also an open cover for $[-n,n]$, there exists a finite subcover $\{U^{(n)}_{\alpha_i} : i=1,\dots, k_n\}$. Then, since a union of countable many finite sets is countable,
    \[\bigcup_{n=1}^\infty\{U^{(n)}_{\alpha_i} : i=1,\dots, k_n\},\]
    is a countable subcover for $\R$.
    
\end{proof}




\end{document}
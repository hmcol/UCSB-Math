\documentclass[12pt]{article}

% Packages
\usepackage[margin=1in]{geometry}
\usepackage{fancyhdr, parskip}
\usepackage{amsmath, amsthm, amssymb}
\usepackage{tikz, tikz-cd}
\usepackage[shortlabels]{enumitem}

% Page Style
\makeatletter
\fancypagestyle{title}{
    \renewcommand{\headrulewidth}{0.4pt}
    \setlength{\headheight}{15pt}
    \fancyhead[R]{\@author}
    \fancyhead[L]{\@title}
    \fancyhead[C]{\@date}
}
\makeatother
\renewcommand{\maketitle}{\thispagestyle{title}}
\fancypagestyle{plain}{
    \fancyhf{}
    \renewcommand{\headrulewidth}{0pt}
    \renewcommand{\footrulewidth}{0pt}
    \fancyfoot[R]{\thepage}
}
\pagestyle{plain}

% Problem Box
\setlength{\fboxsep}{4pt}
\newlength{\myparskip}
\setlength{\myparskip}{\parskip}
\newsavebox{\savefullbox}
\newenvironment{fullbox}{\begin{lrbox}{\savefullbox}\begin{minipage}{\dimexpr\textwidth-2\fboxsep\relax}\setlength{\parskip}{\myparskip}}{\end{minipage}\end{lrbox}\framebox[\textwidth]{\usebox{\savefullbox}}}
\newenvironment{pbox}[1][]{\begin{fullbox}\ifx#1\empty\else\paragraph{#1}\phantom{}\fi}{\end{fullbox}}

% Theorem Environments
\theoremstyle{definition}
\newtheorem{lemma}{Lemma}

% Tikz Environments
\newenvironment{drawing}{\begin{center}\begin{tikzpicture}}{\end{tikzpicture}\end{center}}
% \tikzcdset{row sep/normal=0pt}
\newenvironment{cd}{\begin{center}\begin{tikzcd}}{\end{tikzcd}\end{center}}

% Default Commands
\newcommand{\isp}[1]{\quad\text{#1}\quad}
\newcommand{\N}{\mathbb{N}} 
\newcommand{\Z}{\mathbb{Z}}
\newcommand{\Q}{\mathbb{Q}}
\newcommand{\R}{\mathbb{R}}
\newcommand{\C}{\mathbb{C}}
\newcommand{\A}{\mathbb{A}}
\renewcommand{\P}{\mathbb{P}}
\newcommand{\eps}{\varepsilon}
\renewcommand{\phi}{\varphi}
\renewcommand{\emptyset}{\varnothing}
\newcommand{\<}{\langle}
\renewcommand{\>}{\rangle}
\newcommand{\isom}{\cong}
\newcommand{\seq}{\subseteq}
\newcommand{\teq}{\trianglelefteq}
\newcommand{\eqc}{\overline}
\newcommand{\clo}{\overline}
\newcommand{\inc}{\hookrightarrow}
\DeclareMathOperator{\id}{id}
\DeclareMathOperator{\im}{im}

% Extra Commands
\newcommand{\mm}{\mathfrak{m}}
\newcommand{\pp}{\mathfrak{p}}
\renewcommand{\aa}{\mathfrak{a}}


\renewcommand{\AA}{\mathcal{A}}
\newcommand{\CC}{\mathcal{C}}

\newcommand{\tensor}{\otimes}
\newcommand{\To}{\longrightarrow}
\newcommand{\Mapsto}{\longmapsto}

\newcommand{\eval}{\mathrm{eval}}
\newcommand{\tors}{\mathrm{tors}}

\DeclareMathOperator{\Supp}{Supp}
\DeclareMathOperator{\Ann}{Ann}
\DeclareMathOperator{\Frac}{Frac}
\DeclareMathOperator{\coker}{coker}




% Document
\begin{document}
\title{MATH 220B Final}
\author{Harry Coleman}
\date{March 15, 2022}
\maketitle

\begin{pbox}[1]
    Let $f=a_0+a_1x+\cdots+a_nx^n\in R[x]$ be a polynomial.
\end{pbox}

\textit{Note.} I use (b) in the proof of (a), but the proof of (b) does not rely on (a).

\begin{pbox}[(a)]
    Show that if $f$ is a unit in $R[x]$, then $a_0$ is a unit in $R$ and $a_1,\dots,a_n$ are nilpotent.
\end{pbox}

\begin{proof}
    Let $g = b_0 + b_1x + \cdots + b_m x^m \in R[x]$ be any polynomial, then we have the product
    \[
        fg = \sum_{d=0}^{n+m} c_dx^d \isp{where} c_d = \sum_{i + j = d} a_i b_j.
    \]
    Then $fg = 1$ if and only if $c_0 = 1$ and $c_d = 0$ for all $d > 0$.
    Assuming $g$ is the inverse of $f$, this means that $1 = c_0 = a_0 b_0$, so indeed $a_0$ is a unit in $R$.

    The next step is to show that the leading coefficient of $f$ is nilpotent when $n \geq 1$.
    To do this, we first claim that $a_n^{k+1}b_{m-k} = 0$ for $k = 0, 1, \dots, m$---we will prove this by induction on $k$.
    For the base case of $k = 0$, we immediately have
    \[
        0 = c_{n + m} = a_nb_m.
    \]
    Assuming the result holds for all indices less than some $k \geq 1$.
    Then
    \[
        0 = c_{n + m - k} = a_nb_{m - k} + a_{n-1}b_{m-(k-1)} + \cdots + a_{n-k}b_m,
    \]
    and multiplying by $a_n^k$ gives
    \[
        0
            = a_n^{k+1}b_{m-k} + a_{n-1}(a_n^{k+1}b_{m-(k-1)}) + \cdots + a_{n-k}(a_n^{k+1}b_m)
            = a_n^{k+1}b_{m-k}.
    \]
    This completes the induction.
    In particular, $k = m$ tells us that $a_n^{m+1}b_0 = 0$, and multiplying by $a_0$ gives $a_n^{m+1} = 0$, i.e., $a_n$ is nilpotent in $R$.

    Lastly, we perform induction on $n = \deg f$ to show that the coefficients of the remaining nonconstant terms are nonzero.
    For the base case of $n = 0$, $f = a_0$ has no nonconstant terms, so the result is vacuously true.
    Assume that the result holds for all unit polynomials of degree less than $n \geq 1$.
    Since $a_n$ is nilpotent in $R$, we know that $a_nx^n$ is nilpotent in $R[x]$, so part (b) tells us that $f - a_nx^n$ is a unit in $R[x]$.
    But the degree of $f - a_nx^n$ is strictly less than $n$, so the inductive hypothesis tells us that all the coefficients of all of its nonconstant terms are nilpotent---these are precisely $a_1, \dots, a_{n-1}$, so the induction is complete.
\end{proof}

\begin{pbox}[(b)]
    Show that the sum of a unit and a nilpotent element is a unit.
\end{pbox}

\begin{proof}
    We consider elements of an arbitrary ring.

    If $a$ is nilpotent with $a^n = 0$ then
    \[
        (1 - a)(1 + a + \cdots + a^{n-1}) = 1 - a^n = 1.
    \]
    In particular, $1 - a$ is a unit whenever $a$ is nilpotent.

    Let $u$ be a unit and $a$ be nilpotent.
    Then $-u^{-1}a$ is also nilpotent and the above result tells us that $1 + u^{-1}a$ is a unit.
    Multiplying by the unit $u$, we conclude that $u + a$ is a unit.
\end{proof}



\begin{pbox}[(c)]
    Show that the converse of (a) also holds.
\end{pbox}

\begin{proof}
    Since the set of nilpotents of $R[x]$ is an ideal (the nilradical), then any $R[x]$-linear combination of nilpotents $a_1, \dots, a_n \in R \seq R[x]$ is also nilpotent.
    In particular,
    \[
        a_1x + \cdots + a_nx^n
    \]
    is nilpotent.
    Then if $a_0$ is a unit in $R$, it is still a unit in $R[x]$, so part (b) tells us that
    \[
        f = a_0 + a_1x + \cdots + a_nx^n
    \]
    is a unit in $R[x]$.
\end{proof}


\newpage
\begin{pbox}[2]
    Let $M$ be an $R$-module, and $I\subset R$ an ideal. Show that if $M_{\mathfrak{m}}=\{0\}$ for all maximal ideals $\mathfrak{m}\subset R$ containing $I$, then $M=IM$.
\end{pbox}

\begin{proof}
    Note that $M = IM$ if and only if $M/IM = 0$.
    Moreover, $M/IM = 0$ if and only if the localization $(M/IM)_\mm = 0$ at every maximal ideal $\mm \seq R$.

    If $\mm \seq R$ is maximal ideal containing $I$, the quotient map followed by a canonical isomorphism gives us a surjection
    \[
        0 = M_\mm \To M_\mm/(IM)_\mm \isom (M/IM)_\mm,
    \]
    hence $(M/IM)_\mm = 0$.

    If $\mm \seq R$ is a maximal ideal not containing $I$, we can find a scalar $t \in I \setminus \mm$.
    Then for any element in the localization $\frac{m + IM}{s} \in (M/IM)_\mm$, we have $tm \in IM$ so
    \[
        t(m + IM) = 0 \in M/IM \implies \tfrac{m + IM}{s} = 0 \in (M/IM)_\mm.
    \]
    So again we deduce that $(M/IM)_\mm = 0$.
\end{proof}

% \begin{proof}
%     Note that $M = IM$ if and only if $M/IM = 0$.
%     As an $R/I$-module, $M/IM = 0$ if and only if its localization at every maximal ideal of $R/I$ is zero.
%     Also note that the maximal ideals of $R/I$ are precisely the quotients $\mm/I$ of maximal ideals $\mm \seq R$ containing $I$.
%     Altogether, we have $M = IM$ if and only if $(M/IM)_{\mm/I} = 0$ for all maximal ideals $\mm \seq R$ containing $I$.

%     Suppose $\mm \seq R$ is a maximal ideal containing $I$.
%     There are canonical module isomorphisms
%     \begin{align*}
%         (M/IM)_{\mm/I}
%             &\isom_{R/I} M/IM \tensor_{R/I} (R/I)_{\mm/I} \\
%             &\isom_{R/I} (M \tensor_R R/I) \tensor_{R/I} (R/I)_{\mm/I} \\
%             &\isom_R M \tensor_R (R/I \tensor_{R/I} (R/I)_{\mm/I}) \\
%             &\isom_R M \tensor_R (R/I)_{\mm/I}.
%     \end{align*}
%     The $R$-module structure of $(M/IM)_{\mm/I}$ induced by these isomorphisms is the natural structure we would obtain by expanding the definitions of quotients and localizations.
%     We have the isomorphism $(R/I)_{\mm/I} \isom R_\mm/I_\mm$ as rings, so the same is true as $R$-modules, hence
%     \[
%         (M/IM)_{\mm/I} \isom_R M \tensor_R (R_\mm/I_\mm).
%     \]
%     The quotient $R_\mm \to R_\mm/I_\mm$ is a surjective $R$-module homomorphism, so tensoring with $M$ induces a surjective $R$-module homomorphism
%     \begin{cd}
%         M \tensor_R R_\mm \rar[twoheadrightarrow] & M \tensor_R R_\mm/I_\mm.
%     \end{cd}
%     However,
%     \[
%         M \tensor_R R_\mm \isom_R M_\mm = 0,
%     \]
%     which means there is a surjective $R$-module homomorphism
%     \begin{cd}
%         0 \rar[twoheadrightarrow] & (M/IM)_{\mm/I}.
%     \end{cd}
%     It follows that $(M/IM)_{\mm/I} = 0$, giving us $M = IM$ by the initial discussion.
% \end{proof}


\newpage
\begin{pbox}[3]
    Let $M$ be an $R$-module. Define the \emph{support} of $M$ to be  
    \[
    {\rm Supp}(M):=\{\textrm{all prime ideals}\;\mathfrak{p}\subset R\;\textrm{such that}\;M_{\mathfrak{p}}\neq 0\},
    \]
    and the \emph{annihilator} of $M$ to be
    \[
    {\rm Ann}_R(M):=\{r\in R\;\textrm{such that}\;rm=0\;\textrm{for all}\;m\in M\}.
    \] 
    Show that if $M$ is finitely generated over $R$, then ${\rm Supp}(M)$ is the same as the set of all prime ideals $\mathfrak{p}\subset R$ containing  ${\rm Ann}_R(M)$.
\end{pbox}

\begin{proof}
    Suppose $\pp \in \Supp(M)$.
    The fact that $M_\pp \ne 0$ means there is some $m \in M$ such that $tm \ne 0$ for all $t \in R \setminus \pp$.
    In particular, for all $a \in \Ann_R(M)$ we have $am = 0$, which implies $a \in \pp$.
    Hence, $\Ann_R(M) \seq \pp$.

    Suppose $\pp \seq R$ is a prime ideal with $M_\pp = 0$, i.e., $\pp \notin \Supp(M)$.
    This means that for all $m \in M$ there is some $t \in R \setminus \pp$ such that $tm = 0$.
    If $M$ is generated by $x_1, \dots, x_n \in M$, we can choose scalars $t_i \in R \setminus \pp$ such that $t_ix_i = 0$.
    Since $\pp$ is prime, we know that the product $t = t_1 \cdots t_n$ is not in $\pp$.
    For any $m \in M$, write $m = \sum_{i=1}^{n} a_ix_i$ for some $a_i \in R$, then
    \[
        tm = \sum_{i=1}^{n} a_i(t_1 \cdots t_i \cdots t_n)x_i = 0.
    \]
    That is, $t \in \Ann_R(M)$.
    But since $t \notin \pp$, we conclude that $\Ann_R(M) \nsubseteq \pp$.
\end{proof}

\newpage
\begin{pbox}[4]
    Let $M$ be a nonzero module over a Noetherian ring $R$. We say that a prime ideal $\mathfrak{p}\subset R$ is \emph{associated with $M$} if
    \[
    \mathfrak{p}={\rm Ann}_R(m)
    \]
    for some $m\in M$, where ${\rm Ann}_R(m):=\{r\in R\colon rm=0\}$.

    Show that the set of prime ideals associated with $M$ is nonempty.

    \noindent
    (\emph{Hint:} Consider a maximal element in the set $\{{\rm Ann}_R(m)\colon m\neq 0\in M\}$.)
\end{pbox}

\begin{proof}
    Note that $\Ann_R(m)$ is an ideal of $R$: given $a, b \in \Ann_R(m)$ and $r \in R$ we have
    \[
        (ra + b)m = r(am) + bm = r \cdot 0 + 0 = 0,
    \]
    hence $ra + b \in \Ann_R(m)$.
    Moreover, $\Ann_R(m)$ is a proper ideal if and only if $m \ne 0 \in M$, since both are equivalent to $1 \in \Ann_R(m)$.

    Per the hint, consider the set $\AA = \{\Ann_R(m) \mid m \ne 0 \in M\}$, partially ordered by inclusion.
    This is a set of proper ideals in $R$, which we know to be nonempty because $M$ is nonzero.
    We will use Zorn's lemma to choose a maximal element of $\AA$.

    Suppose $\CC \seq \AA$ is a chain.
    Choose an arbitrary initial element $\aa_0 \in \CC$ and inductively choose $\aa_i \in \CC$ such that $\aa_i \seq \aa_{i+1}$, with strict inclusion whenever $\aa_i$ is not the maximum in $\CC$.
    (It may be worth remarking on the choice of $\aa_i$'s here---to be completely formal, we are using dependent choice.)
    We now have an ascending sequence $\aa_0 \seq \aa_1 \seq \cdots$ of ideals in $R$.
    Since $R$ is noetherian, this sequence eventually stabilizes, i.e., there is an index $n$ such that $\aa_i = \aa_n$ for all $i \geq n$.
    However, by our choice of $\aa_i$'s, this is only possible if $\aa_n$ is the maximum of $\CC$.
    In particular, $\aa_n \in \AA$ is an upper bound for $\CC$, so the condition for Zorn's lemma is satisfied.

    Let $\Ann_R(m) \in \AA$ be a maximal element.
    We already know that $\Ann_R(m)$ is an ideal of $R$, so it remains to prove it is prime.
    Suppose $r, s \in R$ with $rs \in \Ann_R(m)$.
    If $s \notin \Ann_R(m)$, then $sm \ne 0$ and we have
    \[
        \Ann_R(m) \seq \Ann_R(sm) \in \AA.
    \]
    However, since $\Ann_R(m)$ is maximal in $\AA$, we must have equality.
    And since $rsm = 0$, we conclude that
    \[
        r \in \Ann_R(sm) = \Ann_R(m).
    \]
    Hence, $\Ann_R(m)$ is a prime ideal associated with $M$.
\end{proof}

\newpage
\begin{pbox}[5]
    Let $M$ be a flat $R$-module. 
\end{pbox}

\begin{pbox}[(a)]
    Show that if $R$ is an integral domain, then $M_{\rm tors}=\{0\}$.
\end{pbox}

\begin{proof}
    We check that $M_\tors$ is a submodule of $M$.
    Given $m, n\in M_\tors$ and $r \in R$, there are nonzero scalars $s, t \in R$ such that $sm = tn = 0$.
    Then
    \[
        st(rm + n) = rt(sm) + s(tn) = rt \cdot 0 + s \cdot 0 = 0,
    \]
    hence $rm + n \in M_\tors$.
    In particular, the inclusion map $M_\tors \inc M$ is an injective $R$-module homomorphism.

    Let $F = \Frac R$ be the field of fractions of $R$.
    The inclusion $R \inc F$ is an injective ring homomorphism, so it is also an injective $R$-module homomorphism.
    Since $M$ is $R$-flat, the induced $R$-module homomorphism $R \tensor_R M \to F \tensor_R M$ is also injective.
    We now consider the following composition of injective $R$-module homomorphisms:
    \begin{cd}[row sep=tiny]
        M_\tors \rar[hookrightarrow] & M \rar["\sim"] & R \tensor_R M \rar & F \tensor_R M \\
        & m \rar[mapsto] & 1 \tensor m \\
        && r \tensor m \rar[mapsto] & \frac{r}{1} \tensor m
    \end{cd}
    Given $m \in M_\tors$ there is a nonzero scalar $r \in R$ such that $rm = 0 \in M$.
    Under the above map, $m$ is sent to $1 \tensor m \in F \tensor_R M$.
    However, in $F \tensor_R M$, we have
    \[\textstyle
        1 \tensor m
            = \frac{r}{r} \tensor m
            = \frac{1}{r} \tensor rm
            = \frac{1}{r} \tensor 0
            = 0.
    \]
    In other words, $M_\tors \to F \tensor_R M$ is the zero map.
    But because we already know this map to be injective, we must conclude that $M_\tors = 0$.
\end{proof}

\newpage
\begin{pbox}[(b)]
    Show that for any ideal $I\subset R$ we have
    \[
    I\tensor_{R}M\simeq IM
    \]
    as $R$-modules.
\end{pbox}

\begin{proof}
    The natural projection of $R$ onto the quotient $R/I$ can be used to construct the following short exact sequence of $R$-module homomorphisms:
    \begin{cd}
        0 \rar & I \rar[hookrightarrow, "\iota"] & R \rar["\pi"] & R/I \rar & 0.
    \end{cd}
    Since $M$ is $R$-flat, there is an induced short exact sequence of $R$-module homomorphisms
    \begin{cd}[column sep=large]
        0 \rar & I \tensor_R M \rar["\iota \tensor \id_M"] & R \tensor_R M \rar["\pi \tensor \id_M"] & R/I \tensor_R M \rar & 0.
    \end{cd}
    By the universal property of the tensor product, these are in fact the unique maps such that the following diagram commutes with exact rows:
    \begin{cd}[column sep=large]
        0 \rar & I \times M \rar["\iota \times \id_M"] \dar & R \times M \rar["\pi \times \id_M"] \dar & R/I \times M \rar \dar & 0 \\
        0 \rar & I \tensor_R M \rar["\iota \tensor \id_M"] & R \tensor_R M \rar["\pi \tensor \id_M"] & R/I \tensor_R M \rar & 0
    \end{cd}
    (The horizontal maps are $R$-linear while the vertical maps are $R$-bilinear.)

    The natural projection of $M$ onto the quotient $M/IM$ can be used to construct the following short exact sequence of $R$-module homomorphisms:
    \begin{cd}
        0 \rar & IM \rar[hookrightarrow] & M \rar & M/IM \rar & 0.
    \end{cd}
    Consider the following multiplication maps
    \begin{align*}
        I \times M &\To IM & R \times M &\To M & R/I \times M &\To M/IM \\
        (a, m) &\Mapsto am & (r, m) &\Mapsto rm & (r + I, m) &\Mapsto rm + IM
    \end{align*}
    Once again, these $R$-bilinear maps make the following diagram commute with exact rows:
    \begin{cd}[column sep=large]
        0 \rar & I \times M \rar \dar & R \times M \rar \dar & R/I \times M \rar \dar & 0 \\
        0 \rar & IM \rar & M \rar & M/IM \rar & 0
    \end{cd}
    
    The universal property of the tensor product gives us unique $R$-linear maps $\alpha, \beta, \gamma$ such that the following diagram commutes with exact rows:
    \begin{cd}[column sep=tiny]
        0 \ar[rr]
            && I \times M \ar[rr] \dlar 
            && R \times M \ar[rr] \dlar 
            && R/I \times M \rar \dlar
            & 0 \\
        0 \rar 
            & I \tensor_R M \ar[rr] \drar[dashed, "\alpha"]
            && R \tensor_R M \ar[rr] \drar[dashed, "\beta"]
            && R/I \tensor_R M \ar[rr] \drar[dashed, "\gamma"]
            && 0 \\
        0 \ar[rr]
            && IM \ar[from=uu, crossing over] \ar[rr]
            && M \ar[from=uu, crossing over] \ar[rr]
            && M/IM \ar[from=uu, crossing over] \rar
            & 0
    \end{cd}
    Notice that $\beta$ and $\gamma$ are simply the canonical isomorphisms
    \begin{align*}
        R \tensor_R M &\To M & R/I \tensor_R M &\To M/IM \\
        r \tensor m &\Mapsto rm & (r + I) \tensor m &\Mapsto rm + IM
    \end{align*}
    By the 5-lemma (proved below), $\alpha$ is an isomorphism $I \tensor_R M \isom IM$.
    % \begin{cd}
    %     0 \rar & I \tensor_R M \rar & R \tensor_R M \rar \dar["\wr"] & R/I \tensor_R M \rar \dar["\wr"] & 0 \\
    %     0 \rar & IM \rar[hookrightarrow] & M \rar & M/IM \rar & 0
    % \end{cd}
    % \begin{cd}[]
    %     0 \rar & A \ar[rr] \dar && B \ar[rr] \dar && C \rar \dar & 0 \\
    %     0 \rar & A' \ar[rr] \drar && B' \ar[rr] \drar && C' \rar \drar & 0 \\
    %     & 0 \rar & A'' \ar[from=uul, crossing over] \ar[rr] && B'' \ar[from=uul, crossing over] \ar[rr] && C'' \ar[from=uul, crossing over] \rar & 0 \\ 
    % \end{cd}
    % \begin{cd}[column sep=large]
    %     0 \rar 
    %         & I \times M \rar \dar 
    %         & R \times M \rar \dar 
    %         & R/I \times M \rar \dar
    %         & 0 \\
    %     0 \rar 
    %         & I \tensor_R M \rar \drar
    %         & R \tensor_R M \rar \drar
    %         & R/I \tensor_R M \rar \drar
    %         & 0 \\
    %     & 0 \rar
    %         & IM \ar[from=uul, crossing over] \rar
    %         & M \ar[from=uul, crossing over] \rar
    %         & M/IM \ar[from=uul, crossing over] \rar
    %         & 0 \\ 
    % \end{cd}
    % \begin{cd}[column sep=large]
    %     0 \rar 
    %         & I \times M \rar \dar 
    %         & R \times M \rar \dar 
    %         & R/I \times M \rar \dar
    %         & 0 \\
    %     0 \rar 
    %         & I \tensor_R M \rar \dar
    %         & R \tensor_R M \rar \dar
    %         & R/I \tensor_R M \rar \dar
    %         & 0 \\
    %     0 \rar
    %         & IM \ar[from=uu, crossing over, bend left=70] \rar
    %         & M \ar[from=uu, crossing over, bend left=70] \rar
    %         & M/IM \ar[from=uu, crossing over, bend left=70] \rar
    %         & 0 \\ 
    % \end{cd}
    % \begin{cd}[column sep=tiny]
    %     0 \ar[rr]
    %         && I \times M \ar[rr] \dlar 
    %         && R \times M \ar[rr] \dlar 
    %         && R/I \times M \rar \dlar
    %         & 0 \\
    %     0 \rar 
    %         & I \tensor_R M \ar[rr] \drar
    %         && R \tensor_R M \ar[rr] \drar
    %         && R/I \tensor_R M \ar[rr] \drar
    %         && 0 \\
    %     0 \ar[rr]
    %         && IM \ar[from=uu, crossing over] \ar[rr]
    %         && M \ar[from=uu, crossing over] \ar[rr]
    %         && M/IM \ar[from=uu, crossing over] \rar
    %         & 0 \\ 
    % \end{cd}
\end{proof}

\begin{lemma}[5-lemma]
    Suppose the following diagram of $R$-module homomorphisms commutes with exact rows:
    \begin{cd}
        0 \rar & A \rar \dar["\alpha"] & B \rar \dar["\beta"] & C \rar \dar["\gamma"] & 0 \\
        0 \rar & A' \rar & B' \rar & C' \rar & 0
    \end{cd}
    If two of the maps $\alpha, \beta, \gamma$ are isomorphisms then so is the third.
\end{lemma}

\begin{proof}
    Note that a map $\phi : G \to H$ is an isomorphism if and only if the sequence
    \begin{cd}
        0 \rar & G \rar["\phi"] & H \rar & 0
    \end{cd}
    is exact.
    That is, $\phi$ is an isomorphism if and only if $\ker\phi = \coker\phi = 0$.

    By the snake lemma, there is an exact sequence
    \[
        0 \to \ker\alpha \to \ker\beta \to \ker\gamma \to \coker\alpha \to \coker\beta \to \coker\gamma \to 0.
    \]
    If $\alpha$ and $\beta$ are isomorphisms then this becomes
    \[
        0 \To 0 \To 0 \To \ker\gamma \To 0 \To 0 \To \coker\gamma \To 0,
    \]
    which means $\ker\gamma = \coker\gamma = 0$.
    The same argument shows that if $\alpha$ and $\gamma$ are isomorphisms then $\ker\beta = \coker\beta = 0$, and if $\beta$ and $\gamma$ are isomorphisms then $\ker\alpha = \coker\alpha = 0$.
\end{proof}

\newpage
\begin{pbox}[(c)]
    Let $f:R\rightarrow S$ be a ring homomorphism. Show that the map
    \begin{align*}
    M&\rightarrow M\tensor_{R}S\\
    m&\mapsto m\tensor 1
    \end{align*}
    is injective if and only if ${\rm ker}(f)\subset{\rm Ann}_R(M)$.

    \noindent(\emph{Hint:} Consider the $R$-module exact sequence defined by $f$.)
\end{pbox}

\begin{proof}
    Per the hint, there is an exact sequence of $R$-module homomorphisms
    \begin{cd}
        0 \rar & \ker f \rar[hookrightarrow] & R \rar["f"] & S.
    \end{cd}
    Since $M$ is $R$-flat, tensoring with $M$ over $R$ will produce another exact sequence.
    Additionally---similar to part (b)---we use canonical isomorphisms to obtain the following commutative diagram with exact rows:
    \begin{cd}
        0 \rar & \ker f \tensor_R M \dar["\wr"] \rar & R \tensor_R M \dar["\wr"] \rar & S \tensor_R M \dar["\id"] \\
        0 \rar & (\ker f)M \rar[hookrightarrow] & M \rar & S \tensor_R M
    \end{cd}
    As the bottom row is exact, the map $M \to S \tensor_R M$ is injective if and only if $(\ker f)M = 0$.
    This is the case if and only if every element of $\ker f$ annihilates $M$, i.e., $\ker f \seq \Ann_R(M)$.
\end{proof}

\end{document}
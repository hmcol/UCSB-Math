\documentclass[12pt]{article}

% Packages
\usepackage[margin=1in]{geometry}
\usepackage{fancyhdr, parskip}
\usepackage{amsmath, amsthm, amssymb}

% Page Style
\makeatletter
\fancypagestyle{title}{
    \renewcommand{\headrulewidth}{0.4pt}
    \setlength{\headheight}{15pt}
    \fancyhead[R]{\@author}
    \fancyhead[L]{\@title}
    \fancyhead[C]{\@date}
}
\makeatother
\renewcommand{\maketitle}{\thispagestyle{title}}
\fancypagestyle{plain}{
    \fancyhf{}
    \renewcommand{\headrulewidth}{0pt}
    \renewcommand{\footrulewidth}{0pt}
    \fancyfoot[R]{\thepage}
}
\pagestyle{plain}

% Problem Box
\setlength{\fboxsep}{4pt}
\newlength{\myparskip}
\setlength{\myparskip}{\parskip}
\newsavebox{\savefullbox}
\newenvironment{fullbox}{\begin{lrbox}{\savefullbox}\begin{minipage}{\dimexpr\textwidth-2\fboxsep\relax}\setlength{\parskip}{\myparskip}}{\end{minipage}\end{lrbox}\framebox[\textwidth]{\usebox{\savefullbox}}}
\newenvironment{pbox}[1][]{\begin{fullbox}\ifx#1\empty\else\paragraph{#1}\fi}{\end{fullbox}}

% Default Commands
\newcommand{\isp}[1]{\quad\text{#1}\quad}
\newcommand{\N}{\mathbb{N}} 
\newcommand{\Z}{\mathbb{Z}}
\newcommand{\Q}{\mathbb{Q}}
\newcommand{\R}{\mathbb{R}}
\newcommand{\C}{\mathbb{C}}
\newcommand{\eps}{\varepsilon}
\renewcommand{\phi}{\varphi}
\renewcommand{\emptyset}{\varnothing}
\newcommand{\<}{\langle}
\renewcommand{\>}{\rangle}
\newcommand{\isom}{\cong}
\newcommand{\eqc}{\overline}
\newcommand{\clo}{\overline}
\newcommand{\teq}{\trianglelefteq}
\DeclareMathOperator{\id}{id}

% long limits
\newcommand{\longlimit}[2]{
    {
        \def\arraystretch{0}
        \begin{array}[t]{@{}l@{}}
            \displaystyle{#2} \\[8.5pt] \scriptstyle{#1}
        \end{array}
    }
}

\newcommand{\longprod}[2]{\longlimit{#1}{\prod #2}}
\newcommand{\longsum}[2]{\longlimit{#1}{\sum #2}}
\newcommand{\longcup}[2]{\longlimit{#1}{\bigcup #2}}

% Extra Commands


% Document
\begin{document}
\title{MATH 201A Homework 3}
\author{Harry Coleman}
\date{November 9, 2021}
\maketitle

\begin{pbox}[1]
    Let $\lambda$ be the Lebesgue measure and let $\{A_n\}_{n=1}^{\infty}$ be a sequence of Lebesgue-measurable subsets of $[0, 1]$. Assume the set $B$ consists of those points $x \in [0, 1]$ that belong to infinitely many of the $A_n$.
\end{pbox}

\begin{pbox}[(a)]
    Prove that $B$ is Lebesgue-measurable.
\end{pbox}

\begin{proof}
    Note that the set of Lebesgue-measurable subsets of $[0, 1]$ form a $\sigma$-algebra.

    For $n \in \N$, define the Lebesgue-measurable set $B_n = \bigcup_{i=n}^{\infty} A_i$. In other words, $B_n$ is the set of points which appear in some $A_i$ with $i \geq n$. Then the intersection of all the $B_n$'s is the set of points that belong to infinitely many $A_i$'s. That is, $B = \bigcap_{n=1}^{\infty} B_n$, which is also Lebesgue-measurable.
\end{proof}

\begin{pbox}[(b)]
    Prove that if $\lambda(A_n) > \delta > 0$ for every $n \in \N$, then $\lambda(B) \geq \delta$.
\end{pbox}

\begin{proof}
    Note that we have a decreasing sequence $B_n \supseteq B_{n+1}$. Then a theorem that tells us
    \[
        \lambda(B)
            = \lambda\left(\bigcap_{n=1}^{\infty} B_n\right)
            = \lim_{n \to \infty} \lambda(B_n).
    \]
    Additionally, $B_n \supseteq A_n$, so monotonicity implies
    \[
        \lambda(B)
            \geq \lim_{n \to \infty} \lambda(A_n)
            \geq \lim_{n \to \infty} \delta
            = \delta.
    \]
\end{proof}

\begin{pbox}[(c)]
    Prove that if $\sum_{n=1}^{\infty} \lambda(A_n) < \infty$, then $\lambda(B) = 0$.
\end{pbox}

\begin{proof}
    From part (a), we have $B \subseteq B_n$, so $\lambda(B) \leq \lambda(B_n)$. Assuming $\sum_{n=1}^{\infty} \lambda(A_n) < \infty$, then for all $\eps > 0$, there exists $N \in \N$ such that $\sum_{i=N}^{\infty} \lambda(A_i) < \eps$. Then
    \[
        \lambda(B)
            \leq \lambda(B_N)
            \leq \sum_{i=N}^{\infty} \lambda(A_i)
            < \eps.
    \]
    Letting $\eps \to 0$, we obtain $\lambda(B) = 0$.
\end{proof}

\begin{pbox}[(d)]
    Give an example where $\sum_{n=1}^{\infty} \lambda(A_n) = \infty$, but $\lambda(B) = 0$.
\end{pbox}

\begin{proof}
    Define $A_n = [0, 1/n]$ for all $n \in \N$. Then $B_n = A_n$, so $B = \{0\}$. Then
    \[
        \sum_{n=1}^{\infty} \lambda(A_n)
            = \sum_{n=1}^{\infty} \frac{1}{n}
            = \infty,
    \]
    but $\lambda(B) = \lambda(\{0\}) = 0$.
\end{proof}




\newpage
\begin{pbox}[2]
    Prove that if the set $A \subseteq \R$ is Lebesgue-measurable, with $\lambda(A) > 0$, then there is a subset of $A$ that is not Lebesgue-measurable.
\end{pbox}

\begin{proof}
    We perform a construction similar to a Vitali set.

    Consider the additive quotient $A/\Q$, where $\eqc{x} = \eqc{y}$ if $x - y \in \Q$. Using the axiom of choice, construct a set $V \subseteq A$ with one representative from each equivalence class $\eqc{x} \in A/\Q$.

    First, assume $A$ is bounded by the interval $[-M, M]$. Then
    \[
        A \subseteq \longcup{x \in [-2M, 2M] \cap \Q}{\;x + V} \subseteq [-3M, 3M].
    \]
    Assume, for contradiction, that $V$ is measurable, then $\lambda(V) = \lambda(x + V)$ for all $x \in \R$. And since the elements in $V$ are in different equivalence classes in $A/\Q$, then shifted sets $x + V$ are mutually disjoint over $x \in \Q$. So we have
    \[
        \lambda\left(\longcup{x \in [-2M, 2M] \cap \Q}{\;x + V}\right)
            = \longsum{x \in [-2M, 2M] \cap \Q}{\lambda(x + V)}
            = \longsum{x \in [-2M, 2M] \cap \Q}{\lambda(V)}.
    \]
    Hence,
    \[
        0 < \lambda(A) \leq \longsum{x \in [-2M, 2M] \cap \Q}{\lambda(V)} \leq \lambda([-3M, 3M]) = 6M.
    \]
    But, what is $\lambda(V)$? We must have $\lambda(V) > 0$, since the above sum is positive. But if this is the case, the sum of countably many copies of this positive number would be infinite. This is a contradiction, so $V$ is not Lebesgue-measurable.

    This proves that every bounded Lebesgue-measurable set of positive measure contains a non-measurable subset. For an unbounded measurable set $A \subseteq \R$, we can find some interval $[-M, M]$ for which $A \cap [-M, M]$ has positive measure and repeat the process to construct a non-measurable subset of $A$.

\end{proof}


\begin{pbox}[3]
    Let $\lambda$ be the Lebesgue measure on $\R$.
\end{pbox}

\begin{pbox}[(a)]
    Let $A \subseteq \R$ be a set such that $\lambda(A) > 0$. Prove that for any $\eps > 0$, there exists an interval $(a, b) \subseteq \R$ such that $\lambda(A \cap (a, b)) > (1 - \eps)(b - a)$.
\end{pbox}

\begin{proof}
    Let $\eps > 0$ be given. Assume, for contradiction, that $\lambda(A \cap (a, b)) < (1 - \eps)(b - a)$ for all intervals $(a, b) \subseteq \R$. By definition of the Lebesgue measure, there is a cover of $A$ by disjoint open intervals $(a_i, b_i) \subseteq \R$ such that
    \[
        \lambda(A) \leq \sum_{i=1}^{\infty} (b_i - a_i) < \lambda(A) + \eps\lambda(A). 
    \]
    Denote $U = \bigcup_{i=1}^{\infty} (a_i, b_i)$, so
    \begin{align*}
        \lambda(A)
            &= \lambda(A \cap U) + \lambda(A \setminus U) \\
            &= \lambda\left(\bigcup_{i=1}^{\infty} A \cap (a_i, b_i) \right) + \lambda(\emptyset) \\
            &\leq \sum_{i=1}^{\infty} \lambda(A \cap (a_i, b_i)) \\
            &\leq \sum_{i=1}^{\infty} (1 - \eps)(b_i - a_i) \\
            &= (1 - \eps)\lambda(U) \\
            &< (1 - \eps)(\lambda(A) + \eps\lambda(A)) \\
            &= (1 - \eps^2)\lambda(A).
    \end{align*}
    This is a contradiction.

\end{proof}

\begin{pbox}[(b)]
    Construct a Borel set $B \subseteq \R$ such that $\lambda(B) > 0$ and $\lambda(B \cap I) < \lambda(I)$ for every non-degenerate interval $I \subseteq \R$.
\end{pbox}

Let $\{a_k\}_{k\in\N}$ be a sequence indexing all the rationals $\Q$. For each $k \in \N$, define the interval
\[
    A_k = (a_k - 2^{-k},\; a_k + 2^{-k}) \subseteq \R,
\]
and $A = \bigcup_{k=1}^{\infty} A_k$. As the union of countable many intervals, $A$ is a Borel set. Then its complement $B = \R \setminus A$ is also Borel. Since the measure of $A$ is bounded above by
\[
    \lambda(A) \leq \sum_{k=1}^{\infty} \frac{2}{2^k} = 2,
\]
its complement must have infinite---in particular positive---measure, i.e., $\lambda(B) > 0$.

On the other hand, we now consider the intersection of $B$ with a non-degenerate interval $I \subseteq \R$. Since $\Q$ is dense in $\R$, $I$ must contain some rational $a_k \in I$. Then $J = I \cap A_k$ is some non-degenerate interval. Then
\[
    B \cap I \subseteq I \setminus A_k = I \setminus J,
\]
so $\lambda(B \cap I) = \lambda(I) - \lambda(J) < \lambda(I)$.



\newpage
\begin{pbox}[4]
    Prove that if a Lebesgue-measurable set $A \subseteq \R$ has positive Lebesgue measure, then the set
    \[
        A - A = \{a - b : a, b \in A\}
    \]
    contains a neighborhood of the origin.
\end{pbox}

\begin{proof}
    Since $\lambda$ is radon, there is a compact set $K \subseteq A$ such that $0 < \lambda(K) \leq \lambda(A)$. Then $K - K \subseteq A - A$, so it suffices to show $K - K$ contains a neighborhood of the origin. 

    Again, since $\lambda$ is radon, there is an open set $U \supseteq K$ such that $\lambda(K) \leq \lambda(U) < 2\lambda(K)$. Since $U$ is an open neighborhood of the compact set $K$, there is some $\eps > 0$ such that the $\eps$-neighborhood of $K$
    \[
        B_\eps(K) = \bigcup_{x \in K} B_\eps(x) = \bigcup_{r \in (-\eps, \eps)} (r + K)
    \]
    is still contained in $U$. For any $r \in (-\eps, \eps)$, we know that $\lambda(r + K) = \lambda(K)$, so
    \[
        \lambda(K \cup (r + K)) \leq \lambda(U) < 2\lambda(K) = \lambda(K) + \lambda(r + K).
    \]
    This implies $K \cap (r + K)$ must have positive measure. In particular, $K \cap (r + K)$ is nonempty, so there is some $x, y \in K$ such that $x = r + y$, i.e., $r = x - y \in K - K$. In other words, we have shown that $(-\eps, \eps) \subseteq K - K$.
\end{proof}

\begin{pbox}
    Is the statement true if one only assumes $\lambda(A) > 0$ (i.e., $A$ is not Lebesgue-measurable)?
\end{pbox}

No.

If $V$ is a Vitali set, then $\lambda(V) > 0$, but by construction $(V - V) \cap \Q = \emptyset$. Since every neighborhood of the origin contains rational numbers, $V - V$ cannot contain a neighborhood of the origin.



\newpage
\begin{pbox}[5]
    Let $A \subseteq \R$ be any set. Prove that the set
    \[
        B = \bigcup_{x \in A} [x - 1, x + 1]
    \]
    is Lebesgue-measurable.
\end{pbox}


\begin{proof}
    For each $x \in B$, there is an interval containing $x$ and contained in $B$; let $I_x$ be the maximal such interval. We say that $I_x$ is a connected component of $B$, in the topological sense. We claim that $B$ has countably many connected components. Each connected component of $B$ contains an interval $[x-1, x+1]$ for some $x \in A$. That means we can choose a representative from the interior of each connected component. Let $C$ be a set of representatives from each connected component of $B$ such that for each $x \in C$ we have $x \in \operatorname{int} I_x$. Then $|C|$ is the number of connected components of $B$. Additionally, note that $C$ is a discrete subset of $\R$, so it must be countable. Hence, $B$ has countably many connected components. And since
    \[
        B = \bigcup_{x \in C} I_x,
    \]
    where each $I_x$ is an interval and $C$ is countable, then in fact $B$ is a Borel set. Since the Lebesgue measure is Borel, we conclude that $B$ is Lebesgue-measurable.

\end{proof}

\end{document}
\documentclass[12pt]{article}

% Packages
\usepackage[margin=1in]{geometry}
\usepackage{fancyhdr}
\usepackage{amsmath, amsthm, amssymb, physics}

% Page Style
\fancypagestyle{plain}{
    \fancyhf{}
    \renewcommand{\headrulewidth}{0pt}
    \renewcommand{\footrulewidth}{0pt}
    \fancyfoot[R]{\thepage}
}
\pagestyle{plain}

% Problem Box
\setlength{\fboxsep}{4pt}
\newsavebox{\savefullbox}
\newenvironment{fullbox}{\begin{lrbox}{\savefullbox}\begin{minipage}{\dimexpr\textwidth-2\fboxsep\relax}}{\end{minipage}\end{lrbox}\begin{center}\framebox[\textwidth]{\usebox{\savefullbox}}\end{center}}
\newenvironment{pbox}[1][]{\begin{fullbox}\ifx#1\empty\else\paragraph{#1}\fi}{\end{fullbox}}

% Options
\renewcommand{\thesubsection}{\thesection(\alph{subsection})}
\allowdisplaybreaks
\addtolength{\jot}{4pt}
\theoremstyle{definition}

% Default Commands
\newtheorem{proposition}{Proposition}
\newtheorem{lemma}{Lemma}
\newcommand{\ds}{\displaystyle}
\newcommand{\isp}[1]{\quad\text{#1}\quad}
\newcommand{\N}{\mathbb{N}}
\newcommand{\Z}{\mathbb{Z}}
\newcommand{\Q}{\mathbb{Q}}
\newcommand{\R}{\mathbb{R}}
\newcommand{\C}{\mathbb{C}}
\newcommand{\eps}{\varepsilon}
\renewcommand{\phi}{\varphi}
\renewcommand{\emptyset}{\varnothing}
\newcommand{\pfrac}[2]{\left(\frac{#1}{#2}\right)}

% Extra Commands


% Document Info
\fancypagestyle{title}{
    \renewcommand{\headrulewidth}{0.4pt}
    \setlength{\headheight}{15pt}
    \fancyhead[R]{Harry Coleman}
    \fancyhead[L]{MATH 118B Homework 6}
    \fancyhead[C]{February 23, 2021}
}

% Begin Document
\begin{document}
\thispagestyle{title}


\begin{pbox}[1]
    Suppose $\{f_n\}_{n\in\N}$ and $f$ are continuous on $[a,b]$ and $f_n \to f$ uniformly. Define $M_n = \max_{[a,b]} f_n$ and $M = \max_{[a,b]} f$. Show that $M_n\to M$. What can you say about the sequence of minima?
\end{pbox}

\begin{proof}
    Let $\eps > 0$ be given and choose $N \in \N$ such that $|f_n(x) - f(x)| < \eps$ for all $n \geq N$ and $x \in [a, b]$. In particular for a fixed $n \geq N$ we have
    \[
        f_n(x) < f(x) + \eps \leq M + \eps
    \]
    for all $x \in [a, b]$, which implies that $M_n \leq M + \eps$, or equivalently, $M_n - M < \eps$. Similarly,
    \[
        f(x) < f_n(x) + \eps \leq M_n + \eps.
    \]
    for all $x \in [a, b]$, which implies that $M \leq M_n + \eps$, or equivalently, $M - M_n < \eps$. Combining these two results, we find $|M_n - M| < \eps$ for all $n \geq N$, implying that $M_n \to M$.
    
    
    
    
    
\end{proof}


\newpage
\begin{pbox}[2]
    Determine the limit of the following sequences and whether or not the convergence is uniform on $I$:
\end{pbox}

\begin{pbox}[(a)]
    \begin{equation}
    f_n(x) = \frac{\cos (nx)}{\sqrt{n}},\quad x\in I = \R.
    \end{equation}
\end{pbox}

For a fixed $x \in \R$ and any $n \in \N$, we have
\[
    \left|f_n(x)\right|
        = \left|\frac{\cos (nx)}{\sqrt{n}}\right|
        \leq \frac{1}{\sqrt{n}}.
\]
That is, $|f_n(x)| \to 0$ as $n \to \infty$ for all $x \in \R$. So the limit of this sequence of functions is the zero function. Moreover, it can be seen that this convergence is independent of $x$. Given $\eps > 0$, take $N \in \N$ such that $N > 1/\eps^2$. Then if $n \geq N$ and $x \in \R$, we have
\[
    |f_n(x)|
        \leq \frac{1}{\sqrt{n}}
        < \frac{1}{\sqrt{1/\eps^2}}
        = \eps.
\]
Hence, $f_n \to 0$ uniformly on $I$.




\begin{pbox}[(b)]
    \begin{equation}
    f_n(x) = nxe^{-nx^2},\quad x\in I=[0,1].
    \end{equation}
\end{pbox}

Note that $f_n(0) = 0$ for all $n \in \N$ so $f_n(0) \to 0$. For a fixed $x \in (0, 1]$ and any $n \in \N$ we have
\[
    f_n(x)
        = xe^{\log n}e^{- nx^2}
        = xe^{n\left(\frac{\log n}{n} - x^2\right)}
\]
Taking the limit to the continuous case and applying L'H\^opital's, we find
\[
    \lim_{n\to \infty} \frac{\log n}{n}
        = \lim_{y \to \infty} \frac{\log y}{y}
        = \lim_{y \to \infty} \frac{1}{y}
        = 0,
\]
which tells us
\[
    \lim_{n\to \infty} n\left(\frac{\log n}{n} - x^2\right) = -\infty.
\]
Since $e^y \to 0$ as $y \to -\infty$ and $x > 0$, we conclude that
\[
    \lim_{n\to \infty} f_n(x)
        = xe^{n\left(\frac{\log n}{n} - x^2\right)}
        = 0.
\]
However, this convergence is not uniform on $I$. For each $n \in \N$, consider the point $x_n = 1/\sqrt{2n} \in [0,1]$. Then
\[
    f_n(x_n)
        = n\frac{1}{\sqrt{2n}}e^{-n(1/\sqrt{2n})^2}
        = \frac{\sqrt{n}}{\sqrt{2} e^{1/2}}.
\]
Since $f_n(x_n) \leq \max_I f_n$, then $\max_I f_n \to \infty$ as $n \to \infty$. From Problem 1, this means that the convergence $f_n \to f$ is not uniform, as otherwise we would have $\max_I f_n \to \max_I f < \infty$. Hence $f_n \to 0$ pointwise, but not uniformly, on $I$.




\newpage
\begin{pbox}[(c)]
    \begin{equation}
    f_n(x) = nxe^{-nx^2},\quad x\in I=[1,+\infty).
    \end{equation}
\end{pbox}

As in (b), we have $f_n \to 0$ pointwise on $I$, since the argument only requires that $x > 0$. To show that this convergence is uniform on $I$, we write
\[
    f_n(x)
        = ne^{\log x}e^{-nx^2}
        = ne^{x^2\left(\frac{\log x}{x^2} - n\right)}
\]
Using L'H\^opital's rule, we have
\[
    \lim_{x \to \infty} \frac{\log x}{x^2}
        = \lim_{x \to \infty} \frac{1/x}{2x}
        = \lim_{x \to \infty} \frac{1}{2x^2}
        = 0.
\]
Let $R \in \R$ such that
\[
    \frac{\log x}{x^2} < 1
\]
for all $x \geq R$. If $R > 1$, we consider the function $g(x) = x^{-2} \log x$, which is differentiable on the compact interval $[1, R]$. Therefore, $g$ attains its maximum at a point $x_0 \in [1, R]$. If $x_0 = 1$, then $g(x_0) = 0$. If $x_0 = R$, then the above implies $g(x_0) < 1$. Otherwise, $x_0 \in (1, R)$ and we know that $g'(x_0) = 0$. Applying the quotient rule, we find
\[
    g'(x)
        = \frac{x^2 \dv{x} \log x - \log x \dv{x} x^2}{x^4}
        = \frac{1 - 2\log x}{x^3}.
\]
Then $g'(x_0) = 0$ implies that $x_0 = e^{1/2}$. In which case,
\[
    g(x_0)
        = \frac{\log e^{1/2}}{(e^{1/2})^2}
        = \frac{1}{2e}
        < 1.
\]
So $g(x) < 1$ for all $x \in [1, R]$, and we conclude that
\[
    \frac{\log x}{x^2} = g(x) < 1
\]
for all $x \geq 1$. So for all $n \in \N$ and $x \in I$, because the exponential function is increasing on $\R$, we have
\[
    f_n(x)
        = ne^{x^2\left(g(x) - n\right)}
        \leq ne^{x^2(1 - n)}
        = \frac{n}{e^{x^2(n-1)}}.
\]
Since $x \geq 1$, then $x^2(n-1) \geq n-1$, so
\[
    f_n(x)
        \leq \frac{n}{e^{n-1}}
        = \frac{n}{e^n}e.
\]
Taking the limit to the continuous case and applying L'H\^opital's, we find
\[
    \lim_{n \to \infty} \frac{n}{e^n}
        = \lim_{y \to \infty} \frac{y}{e^y}
        = \lim_{y \to \infty} \frac{1}{e^y}
        = 0.
\]
Then for a given $\eps > 0$, we choose $N \in \N$ such that $n/e^n < \eps/e$ for all $n \geq N$. This gives us
\[
    f_n(x) < \frac{\eps}{e} e = \eps
\]
for all $n \geq N$ and $x \in I$. Hence, $f_n \to 0$ uniformly on $I$.



\begin{pbox}[(d)]
    \begin{equation}
    f_n(x) = \frac{nx^2}{1+nx},\quad x\in I=[0,2].
    \end{equation}
\end{pbox}

Note that $f_n(0) = 0$. For a fixed $x \in (0, 2]$, we have
\[
    \lim_{n \to \infty} f_n(x)
        = \lim_{n \to \infty} \frac{nx^2}{1+nx}
        = \lim_{n \to \infty} \frac{x^2}{\frac{1}{n}+x}
        = \frac{x^2}{0 + x}
        = x.
\]
Hence, $f_n(x) \to x$ pointwise on $I$. Now for a fixed $n \in \N$, we have $|f_n(0) - 0| = 0$, so it remains to show that $f_n(x) \to x$ uniformly on $(0, 2]$. Assuming $x \in (0, 2]$, we find that
\begin{align*}
    |f_n(x) - x|
        &= \left|\frac{nx^2}{1 + nx} - x\right| \\
        &= \left|\frac{nx^2 - x - nx^2}{1 + nx}\right| \\
        &= \frac{1}{\frac{1}{x} + n} \\
        &\leq \frac{1}{n}.
\end{align*}
So for a given $\eps > 0$, we choose $N \in \N$ such that $N > 1/\eps$. Then for all $n \geq N$ and $x \in (0, 2]$, we have
\[
    |f_n(x) - x| \leq \frac{1}{n} < \frac{1}{1/\eps} = \eps.
\]
Hence, $f_n(x) \to x$ uniformly on $(0, 2]$, implying that the convergence is uniform on $I$.



\newpage
\begin{pbox}[3]
    Give an example of sequences of function $\{f_n\}_{n\in\N}$ and $\{g_n\}_{n\in\N}$ that converge uniformly but such that $\{f_ng_n\}_{n\in\N}$ does not converge uniformly.
\end{pbox}

For each $n \in \N$, define the function
\[
    f_n(x) = x + \frac{1}{n}, \quad x \in \R.
\]
Then for all $x \in \R$, we have
\[
    |f_n(x) - x| = \frac{1}{n},
\]
so $f_n(x) \to x$ uniformly on $\R$. However, consider the product sequence
\[
    f_n^2(x)
        = \left(x + \frac{1}{n}\right)^2
        = x^2 + \frac{2x}{n} + \frac{1}{n^2}.
\]
For a fixed $x \in \R$, we have
\[
    |f_n^2(x) - x^2|
        = \left|\frac{2x}{n} + \frac{1}{n^2}\right|,
\]
so $f_n^2(x) \to x^2$ pointwise on $\R$. However, for any $n \in \N$, we can pick some point $x \geq n$, then
\[
    |f_n^2(x) - x^2|
        = \frac{2x}{n} + \frac{1}{n^2}
        \geq 2.
\]
Hence, this convergence is not uniform on $\R$.




\newpage
\begin{pbox}[4]
    Prove that the series 
    \begin{equation}
    \sum_{n=1}^\infty (-1)^n \frac{x^2+n}{n^2}
    \end{equation}
    converges uniformly in every bounded interval, but does not converge absolutely for any values of $x$. 
\end{pbox}

\begin{proof}
    Consider the sequence of partial sum functions, defined on $\R$ by
    \begin{align*}
        f_n(x)
            &= \sum_{k=1}^{n} (-1)^k \frac{x^2+k}{k^2} \\
            &= \sum_{k=1}^{n} (-1)^k \left(\frac{x^2}{k^2} + \frac{1}{k}\right) \\
            &= x^2\sum_{k=1}^{n} \frac{(-1)^k}{k^2} + \sum_{k=1}^{n}\frac{(-1)^k}{k}.
    \end{align*}
    Since $1/k^2 \to 0$ and $1/k \to 0$ as $k \to \infty$, then we have convergence of the alternating series. Let
    \[
        A_n = \sum_{k=1}^{n} \frac{(-1)^k}{k^2}
        \isp{and}
        B_n = \sum_{k=1}^{n}\frac{(-1)^k}{k},
    \]
    and let their infinite sums be $A_n \to A$ and $B_n \to B$. Then we have $f_n(x) = A_n x^2 + B_n$, which converges $f(x) = Ax^2 + B$ pointwise on $\R$. For any point $x$ in a bounded interval $[-M, M]$, we find
    \begin{align*}
        |f_n(x) - f(x)|
            &= |A_nx^2 + B_n - (Ax^2 + B)| \\
            &\leq |A_nx^2 - Ax^2| + |B_n - B| \\
            &= |x|^2|A_n - A| + |B_n - B| \\
            &\leq M^2|A_n - A| + |B_n - B|.
    \end{align*}
    For a given $\eps > 0$, we choose $N \in \N$ such that $n \geq N$ implies $|B_n - B| < \eps$ and $|A_n - A| < \eps/M^2$. Then for all $n \geq N$ and $x \in [-M, M]$, we have
    \[
        |f_n(x) - f(x)|
            < M^2\frac{\eps}{M^2} + \eps
            = 2\eps.
    \]
    Hence, $f_n \to f$ uniformly on $[-M, M]$. However, for a fixed $x \in \R$, we have
    \begin{align*}
        \sum_{n=1}^\infty \left|(-1)^n \frac{x^2+n}{n^2}\right|
            = \sum_{n=1}^\infty \frac{x^2+n}{n^2}
            \geq \sum_{n=1}^{\infty} \frac{1}{n}.
    \end{align*}
    And since the harmonic series diverges, then the above series does not converge absolutely.
    
\end{proof}


\begin{pbox}[5]
    Consider the power series 
    \begin{equation}
    f(x) = \sum_{n=0}^\infty x^n,\quad x\in (-1,1).
    \end{equation}
    Show that it converges uniformly on $[-a,a]$ for any $0\le a <1$. {\it Hint: What are the partial sums?}
\end{pbox}

\begin{proof}
    For each $n \in \N$ and $x \in [-a, a]$, we have $|x^n| = |x|^n \leq a^n$. And since $a \in [0, 1)$, then we have convergence of the geometric series
    \[
        \sum_{n=1}^{\infty} a^n = \frac{a}{1 - a}.
    \]
    Therefore, by the Weierstrass $M$-test, the power series
    \[
        f(x) = \sum_{n=1}^\infty x^n
    \]
    converges uniformly on $[-a, a]$.
    
\end{proof}

\end{document}
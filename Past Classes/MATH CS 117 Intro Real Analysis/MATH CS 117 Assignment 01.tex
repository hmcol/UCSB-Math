\documentclass[12pt]{article}

% packages
\usepackage{kantlipsum}
\usepackage[margin=1in]{geometry}
\usepackage[labelfont=it]{caption}
\usepackage[table]{xcolor}
\usepackage{subcaption,framed,colortbl,multirow}
\usepackage{amsmath,amsthm,amssymb,wasysym,mathrsfs,mathtools}
\usepackage{tikz,graphicx,pgf,pgfplots}
\usetikzlibrary{arrows, angles, quotes, decorations.pathreplacing, math, patterns, calc}
\pgfplotsset{compat=1.16}

% custom commands
\newcommand{\N}{\mathbb{N}}
\newcommand{\Z}{\mathbb{Z}}
\newcommand{\I}{\mathbb{I}}
\newcommand{\R}{\mathbb{R}}
\newcommand{\Q}{\mathbb{Q}}
\newcommand{\C}{\mathbb{C}}
\newcommand{\F}{\mathbb{F}}
\newcommand{\p}{^{\prime}}
\newcommand{\powerset}{\raisebox{.15\baselineskip}{\Large\ensuremath{\wp}}}
\DeclarePairedDelimiter{\ceil}{\lceil}{\rceil}
\DeclarePairedDelimiter\floor{\lfloor}{\rfloor}

\setlength{\fboxsep}{4pt}
\newcommand{\generic}[2]{\section*{#1}\begin{center}\framebox{\begin{minipage}{\textwidth-10pt}#2\end{minipage}}\end{center}}
\newcommand{\ex}[2]{\generic{Exercise #1}{#2}}
\newcommand{\prob}[2]{\generic{Problem #1}{#2}}
\newcommand{\ques}[2]{\generic{Question #1}{#2}}

\newenvironment{drawing}{\begin{center}\begin{tikzpicture}}{\end{tikzpicture}\end{center}}

\newcommand{\ds}{\displaystyle}
\newcommand{\seq}[1]{\{#1\}_{n=1}^\infty}
\newcommand{\eps}{\varepsilon}

 
\begin{document}
 
\title{Assignment 1\\
    \large MATH CS 117 Intro to Real Analysis}
\author{Harry Coleman}
\date{April 4, 2020}
\maketitle

\ques{1a}{
    Prove that $\N$ is not bounded above.
}

Suppose, to the contrary, that $\N$ is bounded above. Then by the least upper bound principle, there exists $x = \sup\N$. This means that for all $n\in\N$, $n<x$. Since $n+1\in\N$ for all $n\in\N$, it is also the case that $n+1<x$ for all $n\in\N$, which gives us $n<x-1$ for all $n\in\N$. Thus $x-1$ is a lesser upper bound of $\N$ than $x$ which is a contradiction. Therefore, $\N$ is not bounded above.

\ques{1b}{
    Let $x>0$.  Prove that there exists $n\in\N$ such that $\ds\frac1n<x$.
}

Since $x,1\in\R$ and $x>0$, then by the Archimedean property, there exists $n\in\N$ such that $nx > 1$. We know $n\ne0$, since otherwise we would have $0>1$. Therefore, we have $\ds x>\frac1n$.

\ques{1c}{
    Prove that $\ds\inf\left\{\frac1n:n\in\N\right\}=0$.
}

Since $\ds\frac1n$ is always positive, 0 is a lower bound of the set. To show 0 is the greatest upper bound, we suppose that there is some $\eps>0$ that is also a lower bound. Since $\N$ is not bounded above, there exists some $m\in\N$ such that $\ds\frac1\eps<m$. So $\ds\frac1m<\eps$, which contradicts $\eps$ being a lower bound. Therefore, 0 is the infimum.

\ques{1d}{
    Prove that $\ds\lim_{n\to\infty}\frac1n=0$.
}

Let $\eps>0$ and similar to question 1c, let $N\in\N$ such that $\ds\frac1\eps<N$. So $\ds\frac1N<\eps$. Since $n\geq N$ implies $\ds\frac1n \leq \frac1N$ for all $n\in\N$, then $n\geq N$ implies $\ds\frac1n <\eps$ for all $n\in\N$. Therefore, $0$ is the limit of the sequence.

\ques{2}{
    Let $S$ be a nonempty subset of $\R$ which is bounded above.  Prove that for every $\eps>0$, there exists $x\in S$ such that $\sup S-\eps<x\leq\sup S$.
}

Let $\eps>0$. Since $\sup S - \eps < \sup S$, then $\sup S - \eps$ is not an upper bound of $S$. So there exists some $x\in S$ (so necessarily $x\leq\sup S$) such that $\sup S - \eps < x \leq \sup S$.

\ques{3}{
    Let $x\in\R$.  Prove that if $|x|<\eps$, for every $\eps>0$, then $x=0$.
}

Suppose $x\ne0$, so $|x|>0$ and $0<|x/2|<|x|$. However, since $|x/2|>0$, then $|x|<|x/2|$. This is a contradiction, so $x=0$.

\ques{4}{
    Prove that every Cauchy sequence in $\R$ is bounded.
}

Let $\seq{a_n}$ be a Cauchy sequence. Let $\eps>0$ and $N\in\N$ such that for all $n,m\in\N$,
\[n,m\geq N \implies |a_n-a_m| < \eps.\]
Since for all $n,m \geq N$,
\[|a_n| = |a_n - a_m + a_m| \leq |a_n - a_m| + |a_m| < \eps + |a_m|,\]
take $m=N$ to obtain $|a_n| < \eps+|a_N|$ for all $n>N$. So $\seq{a_n}$ is bounded.


\ques{5}{
    Let $\seq{a_n}$  be a sequence in $\R$ which converges to $A$. Define
    \[ b_n=\frac{a_n+a_{n+1}}2,\quad n\in\N.\]
    Using the definition of the limit, prove that the sequence $\seq{b_n}$ converges to $A$.
}

Let $\eps>0$ and $N\in\N$ such that $n\geq N$ implies $|a_n-A|<\eps$. We want to show now that $|b_n - A|<\eps$. So
\begin{align*}
    |b_n - A|   &= \left| \frac{a_n+a_{n+1}}{2} - A \right| \\
                &= \frac12 | (a_n-A) + (a_{n+1} - A) | \\
                &\leq \frac12 (|a_n-A| + |a_{n+1} - A|) \\
                &< \frac12 (\eps + \eps) \\
                &= \eps.
\end{align*}
Therefore $|b_n-A| < \eps$, so $\seq{b_n}$ converges to $A$.




\end{document}
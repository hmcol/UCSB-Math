\documentclass[12pt]{article}

% packages
\usepackage{kantlipsum}
\usepackage[margin=1in]{geometry}
\usepackage[labelfont=it]{caption}
\usepackage[table]{xcolor}
\usepackage{subcaption,framed,colortbl,multirow}
\usepackage{amsmath,amsthm,amssymb,wasysym,mathrsfs,mathtools}
\usepackage{tikz,graphicx,pgf,pgfplots}
\usetikzlibrary{arrows, angles, quotes, decorations.pathreplacing, math, patterns, calc}
\pgfplotsset{compat=1.16}

% Set Names
\newcommand{\N}{\mathbb{N}}
\newcommand{\Z}{\mathbb{Z}}
\newcommand{\I}{\mathbb{I}}
\newcommand{\R}{\mathbb{R}}
\newcommand{\Q}{\mathbb{Q}}
\newcommand{\C}{\mathbb{C}}

% Misc Characters
\newcommand{\F}{\mathbb{F}}
\newcommand{\powerset}{\raisebox{.15\baselineskip}{\Large\ensuremath{\wp}}}
\newcommand{\eps}{\varepsilon}

% Paired Delimiters
\DeclarePairedDelimiter{\ceil}{\lceil}{\rceil}
\DeclarePairedDelimiter\floor{\lfloor}{\rfloor}

% Homework Sections
\setlength{\fboxsep}{4pt}
\newcommand{\generic}[2]{\section*{#1}\begin{center}\framebox{\begin{minipage}{\textwidth-10pt}#2\end{minipage}}\end{center}}
\newcommand{\ex}[2]{\generic{Exercise #1}{#2}}
\newcommand{\prob}[2]{\generic{Problem #1}{#2}}
\newcommand{\ques}[2]{\generic{Question #1}{#2}}

% Environments
\newenvironment{drawing}{\begin{center}\begin{tikzpicture}}{\end{tikzpicture}\end{center}}

% MATH CS 117 Intro to Real Analysis
\newcommand{\ds}{\displaystyle}
\newcommand{\seq}[1]{\left\{#1\right\}_{n=1}^\infty}
\newcommand{\isp}[1]{\quad\text{#1}\quad}

 
\begin{document}
 
\title{Assignment 2\\
    \large MATH CS 117 Intro to Real Analysis}
\author{Harry Coleman}
\date{April 13, 2020}
\maketitle


\ques{1}{
    Let $\seq{a_n}$ be a sequence in $\R$ which converges to $L$.  Prove that the sequence $\seq{|a_n|}$ converges to $|L|$.
}

Let $\eps > 0$ and $N\in\N$ such that
\[n\geq N \implies |a_n - L| < \eps.\]
By the triangle inequality, if $n\geq N$, then
\[||a_n| - |L|| \leq |a_n - L| < \eps.\]
Therefore, $|a_n| \to |L|$.

\ques{2}{
    Suppose that $\seq{a_n}$, $\seq{b_n}$, $\seq{c_n}$ are sequences in $\R$ such that
    \[a_n\le b_n\le c_n,\isp{for all} n\in\N.\]
    Prove that if $\seq{a_n}$ and $\seq{c_n}$ converge to the same limit $L$, then $\seq{b_n}$ also converges to $L$.
}

Let $\eps>0$ and $N_1,N_2\in\N$ such that
\begin{align*}
    n \geq N_1 &\implies |a_n-L| < \eps, \\
    n \geq N_2 &\implies |c_n-L| < \eps. 
\end{align*}
Let $N=\max\{N_1,N_2\}$. So if $n\geq N$, then
\[-\eps < a_n-L \leq b_n-L \leq c_n-L < \eps,\]
or equivalently,
\[|b_n-L| < \eps.\]
Thus, $b_n\to L$.

\ques{3}{
    Prove that $\ds\lim_{n\to\infty}2^{-n}=0$.  (Suggestion:  Use problem (2).)
}

We first show that $2^{-n}\leq \ds\frac1n$ for all $n\in\N$ by induction on $n$. As the base case, when $n=1$,
\[2^{-1} = \frac12 \leq \frac11.\]
For the inductive step, we assume for some $n>1$ that $2^{-n} \leq \ds\frac1n$. Consider
\begin{align*}
    2^{-(n+1)}  &= \frac1{2^n}\cdot\frac12 \\
                &\leq \frac1n\cdot\frac12 \\
                &= \frac1{n+n} \\
                &\leq \frac1{n+1}.
\end{align*}
Thus, $2^{-n}\leq \ds\frac1n$ for all $n\in\N$. Since $0\to0$, $\ds\frac1n \to 0$, and
\[0 \leq 2^{-n} \leq \frac1n \quad \forall n\in\N,\]
then $2^{-n} \to 0$.


\ques{4}{
    Prove that \[\lim_{n\to\infty}\frac{\cos n}{n}\] exists, and find the limit. (You may pretend that we have rigorously defined the cosine function and derived its properties. Suggestion:  Use problem (2).)
}

Since $-1\leq \cos n \leq 1$ for all $n\in\N$, then
\[\frac{-1}n \leq \frac{\cos n}n \leq \frac1n \quad \forall n\in\N.\]
And since $\ds\frac{-1}n \to 0$ and $\ds\frac1n \to 0$, then $\ds\frac{\cos n}n \to 0$.

\newpage
\ques{5a}{
    Let $\seq{a_n}$ and $\seq{b_n}$ be sequences in $\R$ such that $\seq{a_n}$ converges to $L\ne0$ and $\seq{b_n}$ is unbounded.  Prove the the sequence $\seq{a_nb_n}$ is unbounded.
}

Suppose, to the contrary that $\seq{a_nb_n}$ is bounded, and let $M\in\N$ such that
\[|a_nb_n| < M \quad \forall n\in\N.\]
Let $\eps>0$ such that $\eps < |L|$. Now let $N\in\N$ such that
\[n\geq N \implies |a_n-L| < \eps \quad \forall n\in\N.\]
By the triangle inequality, if $n\geq N$, then
\[||a_n| - |L|| \leq |a_n-L| < \eps,\]
or equivalently,
\[|L|-\eps < |a_n| < |L| + \eps.\]
Since $|L|-\eps > 0$, by the Archimedean principle we let $K\in\N$ such that
\[(|L|-\eps)K > M.\]
Since $\seq{b_n}$ is unbounded, then for all $x\in\R$, there are an infinite number of $n\in\N$ such that $b_n > x$. Let $k\in\N$ such that $k\geq N$ and $b_k>K$. Consider
\begin{align*}
    |a_kb_k|    &= |a_k||b_k| \\
                &> (|L|-\eps)K \\
                &> M.
\end{align*}
This is a contradiction with the fact that
\[|a_nb_n| < M \quad \forall n\in\N.\]
Therefore, $\seq{a_nb_n}$ is unbounded.

\ques{5b}{
    Is the statement true if $L=0$?
}

Not necessarily. If we consider the sequences $\seq{n^{-2}}$ which converges to 0 and $\seq{n}$ which is unbounded, then their product sequence $\seq{n^{-1}}$ converges to 0 and is thus bounded. However, we might also consider the sequences $\seq{n^{-1}}$ which converges to 0 and $\seq{n^2}$ which is unbounded, but whose product sequence $\seq{n}$ is unbounded. So if $L=0$, then $\seq{a_nb_n}$ could be bounded or unbounded depending on $a_n$ and $b_n$.

\ques{6}{
    Prove that if $A\subset B\subset\R$, then $A'\subset B'$.
}

Let $L\in A'$. In order to show $A'\subset B'$, we must now show that $L\in B'$. Let $\seq{a_n}$ be a sequence in $A$ that converges to $L$, and let $\seq{b_n}$ be a sequence in $B$ defined by
\[b_n = a_n \quad \forall n\in\N.\]
Note that $\seq{b_n}$ is in fact a sequence in $B$ since $a_n\in B$ for all $n\in\N$. Now since
\[a_n \leq b_n \leq a_n \quad \forall n\in\N,\]
and $a_n\to L$, then $b_n\to L$. Now since $\ds\lim_{n\to\infty}b_n = L$, then $b_n$ belongs to every neighborhood of $L$ for infinitely many $n\in\N$. That is, every neighborhood of $L$ contains infinitely many points in $B$, thus $L$ is an accumulation point of $B$, so $L\in B'$. Therefore, $A'\subset B'$.

\ques{7}{
    Suppose that $S\subset \R$ is bounded above.  Prove that $\sup S\in S\cup S'$.
}

Let $L=\sup S$, If $L\in S$, then clearly $L \in S \cup S'$, so without loss of generality, we assume $L \notin S$ and seek to prove $L\in S'$. Consider the sequence $\seq{a_n}$ in $\R$ defined by
\[a_n = L-\frac1n.\]
Since $L\to L$ and $\ds\frac1n \to 0$, then $a_n\to L$. We now let $\seq{b_n}$ be a sequence in $S$ such that
\[a_n \leq b_n \leq L \quad \forall n\in\N.\]
This is possible since for all $n\in\N$, $a_n < L$, there must be some $b_n\in S$ such that $a_n\leq b_n$, otherwise $a_n$ would be the supremum of $S$. Now since $a_n\to L$ and $L\to L$, it is also the case that $b_n\to L$. Therefore, every neighborhood of $L$ contains $b_n$ for all but finitely many $n\in\N$. That is, every neighborhood of $L$ contains infinitely many points of $S$, therefore $L\in S'$. 


\newpage
\ques{8}{
    Decide for which values of $k\in\N$ the sequence 
    \[\seq{\frac{2n^k-n+1}{5n^3+1}}\] is convergent or divergent, and prove your answer.
}

Since
\begin{align*}
    \frac{2n^k-n+1}{5n^3+1} &\leq  \frac{2n^k-n+1}{5n^3} \\
                            &=  \frac15\left(2n^{k-3} - \frac1{n^2} + \frac1{n^3}\right),
\end{align*}
and $\frac1{n^3}\to 0$ and $\frac1{n^2}\to 0$, then 
\[\lim_{n\to \infty} \frac{2n^k-n+1}{5n^3+1} = \frac25\lim_{n\to \infty} n^{k-3}.\]
Now if $k\geq 4$, then $n^{k-3}$ diverges and if $k\leq 3$, then $n^{k-3}$ converges. Therefore, the same can be said of
\[\seq{\frac{2n^k-n+1}{5n^3+1}}.\]



\ques{9}{
    Prove true or give a counterexample:  An uncountable set $S\subset\R$ has an accumulation point.
    %(Suggestion:  First show that $S'\ne\emptyset$, then show that $S'$ cannot be finite or countable.)
}

I was unable to finish this problem in time.






\end{document}
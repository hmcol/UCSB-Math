\documentclass[12pt]{article}

% Packages
\usepackage[margin=1in]{geometry}
\usepackage{fancyhdr, parskip}
\usepackage{amsmath, amsthm, amssymb}
\usepackage{tikz, tikz-cd}

% \usepackage[mathscr]{euscript}
\usepackage{mathrsfs}

% Page Style
\makeatletter
\fancypagestyle{title}{
    \renewcommand{\headrulewidth}{0.4pt}
    \setlength{\headheight}{15pt}
    \fancyhead[R]{\@author}
    \fancyhead[L]{\@title}
    \fancyhead[C]{\@date}
}
\makeatother
\renewcommand{\maketitle}{\thispagestyle{title}}
\fancypagestyle{plain}{
    \fancyhf{}
    \renewcommand{\headrulewidth}{0pt}
    \renewcommand{\footrulewidth}{0pt}
    \fancyfoot[R]{\thepage}
}
\pagestyle{plain}

% Problem Box
\setlength{\fboxsep}{4pt}
\newlength{\myparskip}
\setlength{\myparskip}{\parskip}
\newsavebox{\savefullbox}
\newenvironment{fullbox}{\begin{lrbox}{\savefullbox}\begin{minipage}{\dimexpr\textwidth-2\fboxsep\relax}\setlength{\parskip}{\myparskip}}{\end{minipage}\end{lrbox}\framebox[\textwidth]{\usebox{\savefullbox}}}
\newenvironment{pbox}[1][]{\begin{fullbox}\ifx#1\empty\else\paragraph{#1}\phantom{}\fi}{\end{fullbox}}

% Theorem Environments
\theoremstyle{definition}
\newtheorem{lemma}{Lemma}

% Tikz Environments
\newenvironment{drawing}{\begin{center}\begin{tikzpicture}}{\end{tikzpicture}\end{center}}
% \tikzcdset{row sep/normal=0pt}
\newenvironment{cd}{\begin{center}\begin{tikzcd}}{\end{tikzcd}\end{center}}

% Default Commands
\newcommand{\isp}[1]{\quad\text{#1}\quad}
\newcommand{\N}{\mathbb{N}} 
\newcommand{\Z}{\mathbb{Z}}
\newcommand{\Q}{\mathbb{Q}}
\newcommand{\R}{\mathbb{R}}
\newcommand{\C}{\mathbb{C}}
\newcommand{\A}{\mathbb{A}}
\renewcommand{\P}{\mathbb{P}}
\newcommand{\eps}{\varepsilon}
\renewcommand{\phi}{\varphi}
\renewcommand{\emptyset}{\varnothing}
\newcommand{\<}{\langle}
\renewcommand{\>}{\rangle}
\newcommand{\isom}{\cong}
\newcommand{\eqc}{\overline}
\newcommand{\clo}{\overline}
\newcommand{\teq}{\trianglelefteq}
\DeclareMathOperator{\id}{id}
\DeclareMathOperator{\im}{im}

% Extra Commands
\newcommand{\Set}{\mathsf{Set}}
\newcommand{\Ring}{\mathsf{Ring}}
\newcommand{\Top}{\mathsf{Top}}
\newcommand{\FinVec}{\mathsf{FinVec}}
\newcommand{\Mod}{\mathsf{Mod}}

\DeclareMathOperator{\Mor}{Mor}
\DeclareMathOperator{\Hom}{Hom}

\newcommand{\VV}{\mathcal{V}}
\newcommand{\OO}{\mathcal{O}}

\DeclareMathOperator{\tensor}{\otimes}
\DeclareMathOperator{\eval}{eval}

\newcommand{\vv}{{\vee\vee}}



% Document
\begin{document}
\title{MATH 237A Homework 7}
\author{Harry Coleman\makebox[0pt][r]{\raisebox{-0.25in}[0pt][0pt]{(worked with Joseph Sullivan)}}}
\date{November 19, 2021}
\maketitle

\begin{pbox}[1 Vakil Exercise 1.2.C]
    Let $(-)^\vv : \FinVec_k \to \FinVec_k$ be the double dual functor from the category of finite-dimensional vector spaces over $k$ to itself.
    Show that $(-)^\vv$ is naturally isomorphic to the identity functor on $\FinVec_k$. (Without the finite-dimensional hypothesis, we only get a natural transformation of functions from $\id$ to $(-)^\vv$.)
\end{pbox}

\begin{proof}
    We will construct a natural isomorphism:
    \begin{cd}
        \FinVec_k
        \rar[bend left=50, "\id" name=U]
        \rar[bend right=50, "(-)^\vv"' name=D]
        &
        \FinVec_k
        \ar[from=U, to=D, shorten=4mm, Rightarrow, "\alpha"]
    \end{cd}
    For $V \in \FinVec_k$ define $\alpha_V : V \to V^\vv$ by $\alpha_V(v) = \operatorname{eval}_v : V^\vee \to k$.
    
    We check the linearity of $\alpha_V$.
    We evaluate at elements $c \in k$; $u, v \in V$; and $\phi \in V^\vee$:
    \begin{align*}
        \alpha_v(cu + v)(\phi)
            &= \phi(cu + v) \\
            &= c\phi(u) + \phi(v) \\
            &= c\alpha_v(u)(\phi) + \alpha_v(v)(\phi) \\
            &= (c\alpha_v(u) + \alpha_v(v))(\phi).
    \end{align*}
    hence $\alpha_V(cu + v) = a\alpha_V(u) + \alpha_V(v)$.

    We check the injectivity of $\alpha_V$.
    Suppose $v \in \ker \alpha_V$, i.e., $\alpha_V(v) : V^\vee \to k$ is the zero map.
    Assume---for contradiction---that $v \ne 0$. 
    Then $v$ can be extended to a basis $\{v, v_2, \dots, v_n\}$ for $V$ (where $n = \dim_k V$).
    Then there is a linear functional $\phi : K \to k$ which maps $c_1v + c_2v_2 + \dots + c_nv_n \mapsto c_1$ for all coefficients $c_i \in k$.
    Evaluating $\alpha_V(v)$ at $\phi$, we find
    \[
        \alpha_V(v)(\phi) = \eval_v(\phi) = \phi(v) = 1 \ne 0.
    \]
    This contradicts the assumption that $\alpha_V(v)$ is the zero map.
    Therefore, $v = 0$ and we conclude that $\alpha_V$ is injective.

    Since $\alpha_V$ is an injective linear transformation and $V$ is finite-dimensional, we deduce
    \[
        \dim_k \im \alpha_V = \dim_k V = \dim_k V^\vee = \dim_k V^\vv.
    \]
    It follows that $\im \alpha_V = V^\vv$, i.e., $\alpha_V$ is surjective---therefore an isomorphism.

    We check the naturality of $\alpha$, i.e., that the following diagram commutes for all $T \in \Mor(V, W)$:
    \begin{cd}
        V \rar["T"] \dar["\alpha_V"'] & W \dar["\alpha_W"] \\
        V^\vv \rar["T^\vv"'] & W^\vv
    \end{cd}
    Given $v \in V$ consider the following elements of $W^\vv$:
    \[
        (T^\vv \circ \alpha_V)(v) = {\eval_v} \circ T^\vee,
        \qquad
        (\alpha_W \circ T)(v) = \eval_{T(v)}.
    \]
    For $\psi \in W^\vv$ we evaluate
    \[
        \eval_v(T^\vee(\psi))
            = \eval_v(\psi \circ T)
            = \psi(T(v)) 
            = \eval_{T(v)}(\psi),
    \]
    hence $\alpha$ is a natural transformation.
    Since each $\alpha_V$ is an isomorphism, $\alpha$ is in fact a natural isomorphism.
\end{proof}


\newpage
\begin{pbox}[2 Vakil Exercise 1.2.D]
    Let $\VV$ be the category whose objects are the $k$-vectors spaces $k^n$ for each $n \geq 0$, and whose morphisms are linear transformations.
    The objects of $\VV$ can be thought of as vector spaces with bases, and the morphisms as matrices.
    There is an obvious functor $\VV \to \FinVec_k$, as each $k^n$ \textit{is} a finite-dimensional vector space.

    Show that $\VV \to \FinVec_k$ gives an equivalence of categories, by describing an ``inverse'' functor. 
\end{pbox}

\begin{proof}
    Let $J : \VV \to \FinVec_k$ denote the described ``inclusion'' functor.
    
    We will construct a functor $U : \FinVec_k \to \VV$ which sends each $n$-dimensional vector space to the space $k^n \in \VV$ by choosing some basis.
    (The simultaneous choice of basis for every object in the category is where we require a generalized set theory to be completely formal.)
    A linear map $T : V \to W$ in $\FinVec_k$ is sent to the linear map $U(T) : k^n \to k^m$ in $\VV$, where the matrix of $U(T)$ in the standard bases of $k^n$ and $k^m$ is the same as the matrix of $T$ in the bases of $V$ and $W$ chosen by $U$.
    The functorality of $U$ follows immediately from its construction.

    Assuming that the standard basis is chosen for each $k^n \in \FinVec_k$, $U$ is in fact a left inverse of $J$ in the sense that $U \circ J = \id_\VV$.
    This means the identity on $\id_\VV$ (i.e., the natural transformation $\id_{\id_\VV} : \id_\VV \Rightarrow \id_\VV$) gives us a trivial natural isomorphism $U \circ J \isom \id_\VV$.

    We will construct a natural transformation
    \begin{cd}
        \FinVec_k
        \rar[bend left=50, "J \circ U" name=U]
        \rar[bend right=50, "\id"' name=D]
        &
        \FinVec_k
        \ar[from=U, to=D, shorten=4mm, Rightarrow, "\alpha"]
    \end{cd}
    For each $n$-dimensional $V \in \FinVec_k$ note that $J(U(V)) = k^n$ as an object of $\FinVec_k$.
    We take the linear map $\alpha_V : k^n \to V$ with $e_i \mapsto v_i$, where $\{v_1, \dots, v_n\}$ is the fixed basis of $V$.
    Clearly each $\alpha_V$ is an isomorphism, so $\alpha$ is the desired natural isomorphism.

    Hence, $U$ and $J$ describe an equivalence of categories.
\end{proof}

\newpage
\begin{pbox}[3 Vakil Exercise 1.3.B]
    What are the initial and final objects (if they exist)?
\end{pbox}

\begin{pbox}
    $\Set$
\end{pbox}

The empty set $\emptyset \in \Set$ is initial.
For $S \in \Set$, a morphism $\emptyset \to S$ contains no information about mapping elements, so only the empty function is possible.

Any singleton $\{*\} \in \Set$ is terminal.
For $S \in \Set$, a morphism $S \to \{*\}$ must map every element of $S$ to the element $*$.
Since functions are characterized by their behavior on elements, only the constant function ($x \mapsto *$ for all $x \in S$) is possible.

The category of sets has no zero object, since there are no functions to the empty set from any nonempty set, and there are multiple functions from a singleton to any set with at least two elements.

\begin{pbox}
    $\Ring$
\end{pbox}

The ring of integers $\Z \in \Ring$ is initial (taking rings to have $1$).
For $R \in \Ring$, a morphism $\Z \to R$ must be a ring homomorphism sending $0_\Z \to 0_R$ and $1_\Z \to 1_R$, which completely characterizes the map for all elements of $\Z$.

The zero ring $0 \in \Ring$ is terminal (allowing $0 = 1$).
A ring homomorphism is characterized by its corresponding function on the underlying sets of its domain and codomain.
It follows that the there is at most one ring homomorphism $R \to 0$ for $R \in \Ring$, characterized by the zero function on the underlying sets.
Since this is in fact a ring homomorphism, it is the unique morphism $R \to 0$.

Since ring homomorphisms must preserve $1$, the zero ring---despite the name---is not a zero object of $\Ring$.
It is semistandard to give the name \textit{rngs} to those things which are similar to rings but may lack $1$.
Then a \textit{rng homomorphism} is similar to a ring homomorphism but need not preserve $1$ when it happens to be present.
In the category $\mathsf{Rng}$ of rngs, the zero rng is in fact a zero object.

\begin{pbox}
    $\Top$
\end{pbox}

Since morphisms in $\Top$ are characterized by functions between the underlying sets, then the underlying set of any initial/terminal object in $\Top$ must itself be such an object in $\Set$.

The empty space $\emptyset \in \Top$ is initial.
For $X \in \Top$, the only possible morphism $\emptyset \to X$ is characterized by the empty function.
Since the empty function is continuous, the morphism does exist and is therefore unique.

Any one-point space $\{*\} \in \Top$ is terminal.
For $X \in \Top$, the only possible morphism $X \to \{*\}$ is characterized by the constant function.
Since the constant function is continuous, the morphism does exist and is therefore unique.

For the same reason as $\Set$, $\Top$ has no zero object.

\begin{pbox}
    If $X$ is a set, then the subsets of $X$ form a partially ordered set, where the order is given by inclusions.
    Informally, if $U \subseteq V$, then we have exactly one morphism $U \to V$ in the category (and otherwise none).
\end{pbox}

Let $2^X$ denote the poset of subsets of $X$ considered as a category. This is a subcategory of $\Set$ with only the inclusion functions.

The empty set $\emptyset \in 2^X$ is initial.
For $U \in 2^X$, we have $\emptyset \subseteq U$ so there is a morphism $\emptyset \to U$, which is unique by construction.

The whole set $X \in 2^X$ is terminal.
For $U \in 2^X$, we have $U \subseteq X$ so there is a morphism $U \to X$, which is unique by construction.

For the same reason as $\Set$, $2^X$ has no zero object.

\begin{pbox}
    If $X$ is a topological space, then the open sets form a partially ordered set, where the order is given by inclusion.
\end{pbox}

Let $O(X)$ denote the poset of open subsets of $X$ considered as a category. This is a subcategory of $2^X$.

The empty set $\emptyset \in O(X)$ is initial for the same reason it is initial in $2^X$.

The whole space $X \in O(X)$ is terminal for the same reason it is terminal in $2^X$.

For the same reason as $\Set$ and $2^X$, $O(X)$ has no zero object.

\newpage
\begin{pbox}[4 Vakil Exercise 1.3.G]
    Show that $\Z/10\Z \tensor_\Z \Z/12\Z \isom \Z/2\Z$. 
\end{pbox}

\begin{proof}
    Denote $M = \Z/10\Z \tensor_\Z \Z/12\Z$.

    Since $\gcd(10, 12) = 2$, there exist $a, b \in \Z$ such that $10a + 12b = 2$.
    Then in $M$ we have
    \begin{align*}
        \eqc{2} \tensor \eqc{1}
            &= \eqc{1} \tensor \eqc{2} \\
            &= \eqc{1} \tensor \eqc{10a + 12b} \\
            &= \eqc{1} \tensor \eqc{10a} \\
            &= \eqc{10a} \tensor \eqc{1} \\
            &= \eqc{0} \tensor \eqc{1} \\
            &= 0_M.
    \end{align*}
    So all monomial generators of $A$ can be written as $\eqc{a} \tensor \eqc{b}$ for some $a, b \in \{0, 1\}$.
    Note that
    \[
        \eqc{1} \tensor \eqc{0}
            = \eqc{0} \tensor \eqc{1}
            = \eqc{1} \tensor \eqc{0}
            = 0_M
    \]
    and
    \[
        \eqc{1} \tensor \eqc{1}
            = 1_M.
    \]
    So $M$ consists only of the elements $0_M, 1_M$.

    The multiplication map $\Z/10\Z \times \Z/12\Z \to \Z/2\Z$ defined by $(\eqc{a}, \eqc{b}) \mapsto (\eqc{ab})$ is surjective and $\Z$-bilinear (well-defined since $10$ and $12$ are both even). By the universal property of tensor products, the multiplication map factors through a surjective $\Z$-module homomorphism $M \to \Z/2\Z$ where $\eqc{a} \tensor \eqc{b} \mapsto \eqc{ab}$. Since $M$ has only two elements, this map is also injective---therefore an isomorphism.
\end{proof}

\newpage
\begin{pbox}[5 Vakil Exercise 1.3.H]
    Show that $- \tensor_R N$ gives a covariant functor $\Mod_R \to \Mod_R$. Show that $- \tensor_R N$ is a \textbf{right-exact functor}, i.e., if
    \begin{cd}
        A \rar["f"] & B \rar["g"] & C \rar & 0
    \end{cd}
    is an exact sequence of $R$-modules, then the sequence
    \begin{cd}
        A \tensor_R N \rar["f \tensor \id"] & B \tensor_R N \rar["g \tensor \id"] & C \tensor_R N \rar & 0
    \end{cd}
    is also exact.
\end{pbox}

\begin{lemma}
    The sequence of $R$-modules
    \begin{cd}
        A \rar["f"] & B \rar["g"] & C \rar & 0
    \end{cd}
    is exact if and only if the sequence
    \begin{cd}
        0 \rar & \Hom(C, N) \rar["g^*"] & \Hom(B, N) \rar["f^*"] & \Hom(A, N)
    \end{cd}
    is exact for every $R$-module $N$.
\end{lemma}

\begin{proof}
    Suppose the first sequence is exact and let $N$ be an $R$-module.

    Let $\phi \in \ker g^*$, which means $\phi \circ g = 0$.
    In other words $\phi|_{\im g} = 0$.
    Since $\im g = C$, we in fact have $\phi = 0$, hence $\ker g^* = 0$.

    If $\phi \in \Hom(C, N)$ then $f^*g^*\phi = \phi \circ (g \circ f) = \phi \circ 0 = 0$, so $\im g^* \subseteq \ker f^*$.

    Let $\phi \in \ker f^*$.
    There is an isomorphism $\gamma$ such that the following diagram commutes:
    \begin{cd}
        B \rar["g"] \dar["\pi"'] & \im g = C \\
        B/\ker g \urar[dashed, "\gamma"']
    \end{cd}
    The fact that $\phi \circ f = 0$ means $\im f \subseteq \ker \phi$.
    Therefore, there is a homomorphism $\eta$ such that the following diagram commutes:
    \begin{cd}
        B \rar["\phi"] \dar["\pi'"'] & N \\
        B/\im f \urar[dashed, "\eta"']
    \end{cd}
    Since $\im f = \ker g$, the projections $\pi$ and $\pi'$ in each diagram are the same.
    Let $K = \ker \pi$, then putting the two diagrams together gives us the following commutative diagram:
    \begin{cd}
        & B \dlar["g"'] \dar["\pi"] \drar["\phi"] \\
        C \rar["\gamma^{-1}"'] & B/K \rar["\eta"'] & N
    \end{cd}
    In other words $g^*(\eta \circ \gamma^{-1}) = \phi$, hence $\phi \in \im g^*$.
    We conclude that $\im g^* = \ker f^*$.

    Suppose the second sequence is exact for every $R$-module $N$.

    The exactness at $\Hom(B, N)$ implies $f^*g^*\phi = 0$ for all $\phi \in \Hom(C, N)$.
    For $N = C$ this tells us $g \circ f = f^*g^*\id_C = 0$, i.e., $\im f \subseteq \ker g$.

    If $\pi : B \to B/\im f$ is the natural projection, then $\pi \circ f = 0$. For $N = B/\im f$ the exactness at $\Hom(B, N)$ tells us $\pi \in \ker f^* = \im g^*$, implying $\pi = \phi \circ g$ for some $\phi \in \Hom(C, N)$. In particular, we have $\ker g \subseteq \ker \pi = \im f$. We conclude that $\im f = \ker g$.

    If $\pi : C \to C/\im g$ is the natural projection, then $\pi \circ g = 0$.
    For $N = C/\im g$ the exactness at $\Hom(C, N)$ tells us $\pi \in \ker g^* = 0$, implying $\pi = 0$.
    In other words $C/\im g = \im \pi = 0$, so in fact $\im g = C$.
\end{proof}

\begin{lemma}
    For any $R$-modules $A$, $B$, and $C$ there is a natural isomorphism
    \[
        \Hom(A, \Hom(B, C)) \isom \Hom(A \tensor_R B, C).
    \]
\end{lemma}

\begin{proof}
    Given $f : A \to \Hom(B, C)$ there is a map $A \times B \to C$ defined by $(a, b) \mapsto f(a)(b)$. We check that this map is bilinear. Since $f$ is an $R$-module homomorphism, we have
    \[
        f(ra_1 + a_2)(b) = (rf(a_1) + f(a_2))(b) = rf(a_1)(b) + f(a_2)(b).
    \]
    Since $f(a)$ is an $R$-module homomorphism, we have
    \[
        f(a)(rb_1 + b_2) = rf(a)(b_1) + f(a)(b_2).
    \]
    Hence, the described map is $R$-bilinear and therefore factors through an $R$-module homomorphism $F(f) : A \tensor_R B \to C$, i.e., $F(f)(a \tensor b) = f(a)(b)$.

    This gives us a map
    \[
        F : \Hom(A, \Hom(B, C)) \longrightarrow \Hom(A \tensor_R B, C).
    \]
    For $f, g \in \Hom(A, \Hom(B, C))$ and $x \in A \tensor_R B$ we have
    \[
        F(rf + g)(x)
            = (rf + g)(x) = rf(x) + g(x)
            = rF(f)(x) + F(g)(x).
    \]
    That is, $F$ is an $R$-module homomorphism.

    Given $f : A \tensor_R B \to C$ we define a map $G(f) : A \to \Hom(B, C)$ by $a \mapsto f(a \tensor -)$, where $f(a \tensor -)(b) = f(a \tensor b)$.
    
    (I skip some details here because it is just more tedious linearity checking.)

    Hence, we have an $R$-module homomorphism
    \[
        G : \Hom(A \tensor_R B, C) \longrightarrow \Hom(A, \Hom(B, C)).
    \]
    Then $F$ and $G$ are inverses, describing the isomorphism in question.
\end{proof}

We now prove the main result.

\begin{proof}
    By Lemma 1, the exactness of the sequence
    \begin{cd}
        A \rar["f"] & B \rar["g"] & C \rar & 0
    \end{cd}
    is equivalent to the exactness of the sequence
    \begin{cd}
        0 \rar & \Hom(A, \Hom(N, M)) \rar & \Hom(B, \Hom(N, M)) \rar & \Hom(C, \Hom(N, M))
    \end{cd}
    for all $R$-modules $N$ and $M$. By Lemma 2, this gives us the exact sequence
    \begin{cd}
        0 \rar & \Hom(A \tensor_R N, M) \rar & \Hom(B \tensor_R N, M) \rar & \Hom(C \tensor_R N, M).
    \end{cd}
    Again applying Lemma 1, we obtain the exact sequence
    \begin{cd}
        A \tensor_R N \rar & B \tensor_R N \rar & C \tensor_R N \rar & 0.
    \end{cd}
\end{proof}

\end{document}
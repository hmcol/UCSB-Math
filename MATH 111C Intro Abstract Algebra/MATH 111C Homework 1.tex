\documentclass[12pt]{article}

% Packages
\usepackage[margin=1in]{geometry}
\usepackage{fancyhdr}
\usepackage{amsmath, amsthm, amssymb, physics}

% Page Style
\fancypagestyle{plain}{
    \fancyhf{}
    \renewcommand{\headrulewidth}{0pt}
    \renewcommand{\footrulewidth}{0pt}
    \fancyfoot[R]{\thepage}
}
\pagestyle{plain}

% Problem Box
\setlength{\fboxsep}{4pt}
\newsavebox{\savefullbox}
\newenvironment{fullbox}{\begin{lrbox}{\savefullbox}\begin{minipage}{\dimexpr\textwidth-2\fboxsep\relax}}{\end{minipage}\end{lrbox}\begin{center}\framebox[\textwidth]{\usebox{\savefullbox}}\end{center}}
\newenvironment{pbox}[1][]{\begin{fullbox}\ifx#1\empty\else\paragraph{#1}\fi}{\end{fullbox}}

% Options
\renewcommand{\thesubsection}{\thesection(\alph{subsection})}
\allowdisplaybreaks
\addtolength{\jot}{4pt}
\theoremstyle{definition}

% Default Commands
\newtheorem{proposition}{Proposition}
\newtheorem{lemma}{Lemma}
\newcommand{\ds}{\displaystyle}
\newcommand{\isp}[1]{\quad\text{#1}\quad}
\newcommand{\N}{\mathbb{N}}
\newcommand{\Z}{\mathbb{Z}}
\newcommand{\Q}{\mathbb{Q}}
\newcommand{\R}{\mathbb{R}}
\newcommand{\C}{\mathbb{C}}
\newcommand{\eps}{\varepsilon}
\renewcommand{\phi}{\varphi}
\renewcommand{\emptyset}{\varnothing}
\newcommand{\pfrac}[2]{\left(\frac{#1}{#2}\right)}

% Extra Commands
\newcommand{\isom}{\cong}

% Document Info
\fancypagestyle{title}{
    \renewcommand{\headrulewidth}{0.4pt}
    \setlength{\headheight}{15pt}
    \fancyhead[R]{Harry Coleman}
    \fancyhead[L]{MATH 111C Homework 1}
    \fancyhead[C]{April 9, 2021}
}

% Begin Document
\begin{document}
\thispagestyle{title}


\begin{pbox}[Q1 Problem 13.1.1]
    Show that $p(x) = x^3 + 9x + 6$ is irreducible in $\Q[x]$. Let $\theta$ be a root of $p(x)$. Find the inverse of $1 + \theta$ in $\Q(\theta)$.
\end{pbox}

Since $p(x) \in \Z[x]$ is monic with $3 \mid 9$ and $3 \mid 6$, but $3^2 = 9 \nmid 6$, Eisenstein's Criterion for $\Z[x]$ tells us that $p(x)$ is irreducible in $\Q[x]$. Applying the Euclidean algorithm to $p(x)$ and $x + 1$, we find
\[
    p(x) = x^3 + 9x + 6 = (x + 1)(x^2 - x + 10) - 4.
\]
Then in $\Q(\theta) \isom \Q[x]/(p(x))$, we have $p(\theta) = 0$, so
\[
    0 = (\theta + 1)(\theta^2 - \theta + 10) - 4.
\]
Hence, $(1 + \theta)^{-1} = (\theta^2 - \theta + 10)/4$.



\begin{pbox}[Q2 Problem 13.2.7]
    Prove that $\Q(\sqrt{2} + \sqrt{3}) = \Q(\sqrt{2}, \sqrt{3})$ [one is obvious, for the other consider $(\sqrt{2} + \sqrt{3})^2$, etc.]. Conclude that $[\Q(\sqrt{2} + \sqrt{3}) : \Q] = 4$. Find an irreducible polynomial satisfied by $\sqrt{2} + \sqrt{3}$.
\end{pbox}

Clearly, $\Q(\sqrt{2} + \sqrt{3}) \subseteq \Q(\sqrt{2}, \sqrt{3})$. To show the opposite inclusion, it suffices to show that $\sqrt{2}, \sqrt{3} \in \Q(\sqrt{2} + \sqrt{3})$. Let $a =\sqrt{2} + \sqrt{3}$, then
\begin{align*}
    \frac12(a^2 - 5)a - 2a
        &= \frac12(2 + 2\sqrt{2}\sqrt{3} + 3 - 5)a - 2a \\
        &= \sqrt{2}\sqrt{3}(\sqrt{2} + \sqrt{3}) - 2a \\
        &= 2\sqrt{3} + 3\sqrt{2} - 2\sqrt{2} - 2\sqrt{3} \\
        &= \sqrt{2}.
\end{align*}
Hence, $\sqrt{2} \in \Q(\sqrt{2} + \sqrt{3})$, which also gives us $\sqrt{3} = a - \sqrt{2} \in \Q(\sqrt{2} + \sqrt{3})$. Therefore, $\Q(\sqrt{2} + \sqrt{3}) = \Q(\sqrt{2}, \sqrt{3})$.

We now have that
\[
    \Q(\sqrt{2} + \sqrt{3}) = \Q(\sqrt{2}, \sqrt{3}) = (\Q(\sqrt{2}))(\sqrt{3}),
\]
so
\[
    [\Q(\sqrt{2} + \sqrt{3}) : \Q] = [(\Q(\sqrt{2}))(\sqrt{3}) : \Q(\sqrt{2})][\Q(\sqrt{2}) : \Q] = 2[(\Q(\sqrt{2}))(\sqrt{3}) : \Q(\sqrt{2})].
\]
The degree of $\sqrt{3}$ in the field $\Q(\sqrt{2})$ is the degree of a minimal polynomial of $\sqrt{3}$ in the same field. We will show that $x^2 - 3$ is a minimal polynomial for $\sqrt{3}$. Since $x^2 - 3$ is monic and has $\sqrt{3}$ as a root, then it remains to show that it is irreducible in $\Q(\sqrt{2})$. Since its degree is $2$, then it is irreducible in $\Q(\sqrt{2})$ if and only if it has a root in $\Q(\sqrt{2})$. Suppose for contradiction that $a + b\sqrt{2}$ is such a root, i.e., $a, b \in \Q$. Then
\[
    3 = (a + b\sqrt{2})^2 = a^2 + 2ab\sqrt{2} + 2b^2.
\]
If $a = 0$, then $\sqrt{3} = b\sqrt{2}$. In which case we would have $\sqrt{6} = 2b$, implying that $\sqrt{6}$ is rational, which is not the case. If $b = 0$, then  $\sqrt{3} = a$ is a rational number, which is also not the case. So both $a$ and $b$ are nonzero, implying that
\[
    \frac{3 - a^2 - 2b^2}{2ab} = \sqrt{2}
\]
is a rational number, which is not the case. Therefore, $x^2 - 3$ has no roots, and is therefore irreducible, in $\Q(\sqrt{2})$. Hence,
\[
    [\Q(\sqrt{2} + \sqrt{3}) : \Q] = 2[(\Q(\sqrt{2}))(\sqrt{3}) : \Q(\sqrt{2})] = 4.
\]

We know that such a polynomial in $\Q[x]$ must be of degree $4$, so we consider
\[
    (\sqrt{2} + \sqrt{3})^4 = 49 + 20\sqrt{6}.
\]
Also knowing that $(\sqrt{2} + \sqrt{3})^2 = 5 + 2\sqrt{6}$, then we see that
\[
    (\sqrt{2} + \sqrt{3})^4 - 10(\sqrt{2} + \sqrt{3})^2 = -1.
\]
So the polynomial $x^4 - 10x^2 + 1 \in \Q[x]$ is monic, irreducible, and has $\sqrt{2} + \sqrt{3}$ as a root.

\newpage
\begin{pbox}[Q3]
    Let $K/F$ be a field extension and $\alpha_1, \dots, \alpha_n \in K$. Show that
    \[
        F(\alpha_1, \dots, \alpha_n) = (F(\alpha_1, \dots, \alpha_{n-1}))(\alpha_n).
    \]
    (The LHS is the subfield generated by $\alpha_1, \dots, \alpha_n$ over $F$. The RHS is the subfield generated by $\alpha_n$ over the field $F(\alpha_1, \dots, \alpha_{n-1})$.
\end{pbox}

\begin{proof}
    By definition, $F(S)$ is the intersection of all subfields of $K$ containing $F \cup S$. So
    \[
        F \cup \{\alpha_1, \dots, \alpha_{n-1}\} \subseteq F(\alpha_1, \dots, \alpha_{n-1}) \subseteq (F(\alpha_1, \dots, \alpha_{n-1}))(\alpha_n).
    \]
    And since $\alpha_n \in (F(\alpha_1, \dots, \alpha_{n-1}))(\alpha_n)$, we conclude that
    \[
        F \cup \{\alpha_1, \dots, \alpha_n\} \subseteq (F(\alpha_1, \dots, \alpha_{n-1}))(\alpha_n).
    \]
    And since $(F(\alpha_1, \dots, \alpha_{n-1}))(\alpha_n)$ is a subfield of $K$, then this tells us that
    \[
        F(\alpha_1, \dots, \alpha_n) \subseteq (F(\alpha_1, \dots, \alpha_{n-1}))(\alpha_n).
    \]

    Now since $F(\alpha_1, \dots, \alpha_n)$ is a subfield of $K$ containing $F$ and the elements $\alpha_1, \dots, \alpha_{n-1}$, then we have the inclusion
    \[
        F(\alpha_1, \dots, \alpha_{n-1}) \subseteq F(\alpha_1, \dots, \alpha_n).
    \]
    And since $F(\alpha_1, \dots, \alpha_n)$ also contains $\alpha_n$, then in fact
    \[
        (F(\alpha_1, \dots, \alpha_{n-1}))(\alpha_n) \subseteq F(\alpha_1, \dots, \alpha_n),
    \]
    giving us equality.
    
\end{proof}



\newpage
\begin{pbox}[Q4]
    Let $K/F$ be a field extension and $\alpha, \beta \in K$. Suppose that $[F(\alpha) : F]$ and $[F(\beta) : F]$ are both finite.
\end{pbox}

\begin{pbox}[(a)]
    Show that $[F(\alpha) : F] \geq [F(\alpha, \beta) : F(\beta)]$.
\end{pbox}

\begin{proof}
    Since $[F(\alpha) : F] < \infty$, then $\alpha$ is algebraic over $F$ and a minimal polynomial $m_{\alpha, F}(x) \in F[x]$. Since $F \subseteq F(\beta)$, then we also have $m_{\alpha, F}(x) \in (F(\beta))[x]$, so $m_{\alpha, F(\beta)}(x) \mid m_{\alpha, F}(x)$, giving us
    \[
        [F(\alpha, \beta) : F(\beta)] = [(F(\beta))(\alpha) : F(\beta)] = \deg m_{\alpha, F(\beta)}(x) \leq \deg m_{\alpha, F}(x) = [F(\alpha) : F].
    \]

\end{proof}

\begin{pbox}[(b)]
    Show that $[F(\alpha, \beta) : F] \leq [F(\alpha) : F][F(\beta) : F]$, and the equality holds if $[F(\alpha) : F]$ and $[F(\beta) : F]$ are coprime.
\end{pbox}

\begin{proof}
    Since the result of (a) holds for both $\alpha$ and $\beta$ and the degree of a field extension is always at least $1$, then we have
    \[
        [F(\alpha, \beta) : F] \leq [F(\alpha) : F][F(\beta) : F].
    \]

    Suppose $n = [F(\alpha) : F]$ and $m = [F(\beta) : F]$ are coprime. Then since $F(\alpha, \beta)/F$ is a finite extension and $F(\alpha)$ and $F(\beta)$ are subfields, then both $n$ and $m$ divide $k = [F(\alpha, \beta) : F]$. And since they are coprime, $nm \mid k$. And since $k \leq nm$, then we must have $k = nm$.

\end{proof}

\begin{pbox}[(c)]
    Given $\alpha_1, \dots, \alpha_n \in K$ with $[F(\alpha_j) : F]$, $1 \leq j \leq n$, all finite, show that
    \[
        [F(\alpha_1, \dots, \alpha_n) : F] \leq [F(\alpha_1) : F][F(\alpha_2) : F] \cdots [F(\alpha_n) : F].
    \]
\end{pbox}

\begin{proof}
    For induction on $n$, (b) gives us the base case. Now suppose the inequality holds for $n - 1$. We first see that
    \[
        [(F(\alpha_1, \dots, \alpha_{n-1})(\alpha_n) : F(\alpha_1, \dots, \alpha_{n-1})] \leq [F(\alpha_n) : F],
    \]
    since the minimal polynomial of $\alpha_n$ in $F$ is also a polynomial in $F(\alpha_1, \dots, \alpha_{n-1})$ with $\alpha_n$ as a root. Hence, the minimal polynomial of $\alpha_n$ in the latter field must have degree at most $[F(\alpha_n) : F]$, which is to say that the above inequality holds. With this and the inductive hypothesis, we find
    \begin{align*}
        [F(\alpha_1, \dots, \alpha_n) : F]
            &= [(F(\alpha_1, \dots, \alpha_{n-1})(\alpha_n) : F] \\
            &= [F(\alpha_1, \dots, \alpha_{n-1}) : F][(F(\alpha_1, \dots, \alpha_{n-1})(\alpha_n) : F(\alpha_1, \dots, \alpha_{n-1})] \\
            &\leq [F(\alpha_1) : F] \cdots [F(\alpha_{n-1}) : F][F(\alpha_n) : F].
    \end{align*}
    
\end{proof}




\end{document}
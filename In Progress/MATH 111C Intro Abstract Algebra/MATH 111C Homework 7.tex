\documentclass[12pt]{article}

% Packages
\usepackage[margin=1in]{geometry}
\usepackage{fancyhdr, parskip}
\usepackage{amsmath, amsthm, amssymb}

% Page Style
\fancypagestyle{plain}{
    \fancyhf{}
    \renewcommand{\headrulewidth}{0pt}
    \renewcommand{\footrulewidth}{0pt}
    \fancyfoot[R]{\thepage}
}
\pagestyle{plain}

% Problem Box
\setlength{\fboxsep}{4pt}
\newsavebox{\savefullbox}
\newenvironment{fullbox}{\begin{lrbox}{\savefullbox}\begin{minipage}{\dimexpr\textwidth-2\fboxsep\relax}}{\end{minipage}\end{lrbox}\begin{center}\framebox[\textwidth]{\usebox{\savefullbox}}\end{center}}
\newenvironment{pbox}[1][]{\begin{fullbox}\ifx#1\empty\else\paragraph{#1}\fi}{\end{fullbox}}

% Theorem Environments
%\theoremstyle{definition}
%\newtheorem{proposition}{Proposition}
%\newtheorem{lemma}{Lemma}

% Options
%\allowdisplaybreaks
%\addtolength{\jot}{4pt}

% Default Commands
\newcommand{\isp}[1]{\quad\text{#1}\quad}
\newcommand{\N}{\mathbb{N}} 
\newcommand{\Z}{\mathbb{Z}}
\newcommand{\Q}{\mathbb{Q}}
\newcommand{\R}{\mathbb{R}}
\newcommand{\C}{\mathbb{C}}
\newcommand{\eps}{\varepsilon}
\renewcommand{\phi}{\varphi}
\renewcommand{\emptyset}{\varnothing}
\newcommand{\<}{\langle}
\renewcommand{\>}{\rangle}
\newcommand{\isom}{\cong}
\newcommand{\eqc}{\overline}
\newcommand{\clo}{\overline}

% Extra Commands
\DeclareMathOperator{\id}{id}
\DeclareMathOperator{\Gal}{Gal}
\newcommand{\teq}{\trianglelefteq}

% Document Info
\fancypagestyle{title}{
    \renewcommand{\headrulewidth}{0.4pt}
    \setlength{\headheight}{15pt}
    \fancyhead[R]{Harry Coleman}
    \fancyhead[L]{MATH 111C Homework 7}
    \fancyhead[C]{May 28, 2021}
}

% Begin Document
\begin{document}
\thispagestyle{title}

\begin{pbox}[Q1]
    Let $K_1, K_2, \dots, K_n$ be subfields of $K$. The composite field of $K_1, K_2, \dots, K_n$, denoted $K_1 K_2 \cdots K_n$, is defined to be the smallest subfield of $K$ containing $K_1, K_2,\dots, K_n$.
\end{pbox}

\begin{pbox}[(a)]
    Suppose that $K_j = F(S_j)$ for some $S_j \subseteq K$, $1 \leq j \leq n$. Show that $K_1 K_2 \cdots K_n = F(S_1 \cup S_2 \cup \cdots \cup S_n)$.
\end{pbox}

\begin{proof}
    Denote $S = S_1 \cup \cdots \cup S_n \subseteq K$. For $j = 1, \dots, n$, we have
    \[
        K_j = F(S_j) \subseteq F(S) \subseteq K,
    \]
    so $K_1 \cdots K_n \subseteq F(S)$.

    On the other hand, for $j = 1, \dots, n$, we have
    \[
        S_j \subseteq F(S_j) = K_j \subseteq K_1 \cdots K_n,
    \]
    so $S \subseteq K_1 \cdots K_n$.
    And, in particular,
    \[
        F \subseteq F(S_1) = K_1 \subseteq K_1 \cdots K_n.
    \]
    By definition, $F(S)$ is the smallest subfield of $K$ containing $F$ and $S$, so $F(S) \subseteq K_1 \cdots K_n$.

    Hence, $K_1 \cdots K_n = F(S)$.

\end{proof}

\begin{pbox}[(b)]
    Let $K \subseteq \overline{F}$ be a finite separable field extension of $F$ and $L \subseteq \overline{F}$ be the Galois closure of $K$ over $F$. Suppose $\Gal(L/F) = \{\sigma_1, \dots, \sigma_n\}$. Show that $L = \sigma_1(K) \sigma_2(K) \cdots \sigma_n(K)$.
\end{pbox}

\begin{proof}
    

    By the primitive element theorem, $K/F$ being a finite separable extension  implies that $K = F(\alpha)$, for some $\alpha \in K$. Then, for any $F$-embedding $\phi : K \to \clo{F}$, we have
    \[
        \phi(K) = \phi(F(\alpha)) = F(\phi(\alpha)).
    \]
    Each $\sigma \in \Gal(L/F)$ can be restricted to an $F$-embedding $\sigma|_{K} : K \to \clo{F}$, so $\sigma(K) = F(\sigma(\alpha))$. Let $E = \sigma_1(K) \cdots \sigma_n(K)$, then applying part (a) to $S_j = \sigma_j(\alpha)$, we find
    \[
        E
            = F(\sigma_1(\alpha)) \cdots F(\sigma_n(\alpha))
            = F(\sigma_1(\alpha), \dots, \sigma_n(\alpha)).
    \]
    We now claim that
    \[
        \Gal(L/E) \teq \Gal(L/F).
    \]
    Let $\tau \in \Gal(L/E)$ and $\sigma_j \in \Gal(L/F)$, then we immediately know $\sigma_j^{-1}\tau\sigma_j$ is an automorphism of $L$ fixing $F$. To see that $\sigma_j^{-1}\tau\sigma_j$ also fixes $E$, it suffices to show that it fixes each $\sigma_i(\alpha)$, as they are the generators of $E$ over $F$. Since both $\sigma_i$ are both $\sigma_j$ are elements in $\Gal(L/F)$, then so is $\sigma_j\sigma_i$, i.e., $\sigma_j\sigma_i = \sigma_k$ for some $1 \leq k \leq n$. Since $\tau$ fixes $E$ and
    \[
        \sigma_k(\alpha) \in \sigma_k(F(\alpha)) = \sigma_k(K) \subseteq E,
    \]
    then in particular, $\tau$ fixes $\sigma_k(\alpha)$. We now derive
    \[
        \sigma_j^{-1}\tau\sigma_j(\sigma_i(\alpha))
            = \sigma_j^{-1}(\tau(\sigma_k(\alpha)))
            = \sigma_j^{-1}(\sigma_k(\alpha))
            = \sigma_j^{-1}\sigma_j(\sigma_i(\alpha))
            = \sigma_i(\alpha).
    \]
    Hence, $\sigma_j^{-1}\tau\sigma_j$ fixes $F$ and the generators of $E$ over $F$, implying that it fixes $E$. That is, $\sigma_j^{-1}\tau\sigma_j \in \Gal(L/E)$, which tells us that $\Gal(L/E)$ is in fact a normal subgroup of $\Gal(L/F)$.
    
    By the fundamental theorem, we conclude that $E/F$ is a Galois subextension of $L/F$. Since $\id_L \in \Gal(L/F)$, then in particular, we know
    \[
        K = \id_L(K) \subseteq \sigma_1(K) \cdots \sigma_n(K) = E.
    \]
    That is, $K \subseteq E \subseteq L$ with $E/F$ Galois. Since $L/F$ is the Galois closure of $K/F$, then we must have
    \[
        L = E = \sigma_1(K) \cdots \sigma_n(K).
    \]


\end{proof}


\newpage
\begin{pbox}[Q2 Problem 14.4.5]
    Let $p$ be a prime and let $F$ be a field. Let $K$ be a Galois extension of $F$ whose Galois group is a $p$-group (i.e., the degree $[K : F]$ is a power of $p$). Such an extension is called a $p$-\textit{extension} (note that $p$-extensions are Galois by definition).
\end{pbox}

\begin{pbox}[(a)]
    Let $L$ be a $p$-extension of $K$. Prove that the Galois closure of $L$ over $F$ is a $p$-extension of $F$.
\end{pbox}

\begin{proof}
    Let $k, \ell \in \Z_{\geq0}$ such that $[K : F] = p^k$ and $[L : K] = p^\ell$. In particular, $L/F$ is a finite extension with $[L : F] = p^{k + \ell}$. Since $L/K$ and $K/F$ are both separable, then so is $L/F$. 

    Let $E$ be the Galois closure of the finite separable extension $L/F$, and write
    \[
        \Gal(E/F) = \{\sigma_1, \dots, \sigma_n\}.
    \]
    Applying Q1(b), we have
    \[
        E = \sigma_1(L) \cdots \sigma_n(L).
    \]

    Any $\sigma \in \Gal(E/F)$ restricts to an $F$-embedding $\sigma|_K : K \to \clo{F}$. Since $K/F$ is Galois, it is normal, implying $\sigma(K) = K$. Then the field extension $\sigma(L)/\sigma(K) = \sigma(L)/K$ is isomorphic to the finite Galois extension $L/K$, so
    \[
        \Gal(\sigma(L)/K) \isom \Gal(L/K).
    \]

    Then $E/K = \sigma_1(L) \cdots \sigma_n(L)/K$ is a Galois extension with
    \[
        \Gal(E/K) = \Gal(\sigma_1(L) \cdots \sigma_n(L)/K)
    \]
    isomorphic to a subgroup of
    \[
        \Gal(\sigma_1(L)/K) \times \cdots \times \Gal(\sigma_n(L)/K) \isom \Gal(L/K)^n.
    \]
    (We have proven this result for composites of pairs of fields, and it easily generalizes to composites of finitely many fields.) In particular, $|\Gal(E/K)|$ divides $|\Gal(L/K)|^n = p^{\ell n}$, so $|\Gal(E/K)| = p^m$ for some nonnegative integer $m$. Therefore,
    \[
        [E : F] = [E : K][K : F] = |\Gal(E/K)|p^k = p^{m + k},
    \]
    meaning $E/F$ is a $p$-extension of $F$.

\end{proof}

\begin{pbox}[(b)]
    Give an example to show that (a) need not hold if $[K : F]$ is a power of $p$ but $K/F$ is not Galois.
\end{pbox}

Take $F = \Q$ and $K = L = \Q(\sqrt[3]{2})$. Then $[K : F] = 3$ and $[L : K] = 1 = 3^0$. And since $K$ is Galois over itself, then $L$ is trivially a $3$-extension of $K$. However, the Galois closure of $L$ over $F$ is the splitting field of $x^3 - 2$, whose Galois group over $F$ is isomorphic to $S_3$. Since $|S_3| = 6$, this could not be a $3$-extension of $F$.



\newpage
\begin{pbox}[Q3 Problem 14.4.9]
    Suppose $K/F$ is Galois with Galois group $G$ and $\theta$ is a primitive element for $K$, i.e., $K = F(\theta)$. For any subgroup $H$ of $G$, let $f(x) = \prod_{\sigma \in H}(x - \sigma(\theta))$. Show $f(x) \in E[x]$ where $E$ is the fixed field of $H$ in $K$, and that $f(x)$ is the minimal polynomial for $\theta$ over $E$. Prove that the coefficients of $f(x)$ generate $E$ over $F$ (these coefficients are the `elementary symmetric functions' of the conjugates $\sigma(\theta)$ of $\theta$ for $\sigma \in H$, cf. Section 6).
\end{pbox}

\begin{proof}
    Any automorphism of $K = F(\theta)$ fixing $F$ is completely determined by the image of $\theta$. Moreover, for any $\sigma \in \Gal(K/F)$,
    \[
        K = \sigma(K) = \sigma(F(\theta)) = F(\sigma(\theta)).
    \]
    This means that any automorphism of $K$ fixing $F$ is also completely determined by the image of $\sigma(\theta)$, for any $\sigma \in \Gal(K/F)$. In particular, for any $\sigma_1, \sigma_2, \tau \in \Gal(K/F)$,
    \[
        \tau(\sigma_1(\theta)) = \tau(\sigma_2(\theta)) \implies \sigma_1(\theta) = \sigma_2(\theta) \implies \sigma_1 = \sigma_2.
    \]
    In other words, each $\tau \in \Gal(K/F)$ is injective on the set $\{\sigma(\theta) \mid \sigma \in \Gal(K/F)\}$.

    For any $\tau \in H$, we can extend $\tau$ to an automorphism of $K[x]$, acting on coefficients. Then
    \[
        \tau(f(x)) = \prod_{\sigma \in H} (x - \tau(\sigma(\theta))) = \prod_{\sigma \in H} (x - \sigma(\theta)) = f(x),
    \]
    where the second equality follows from the injectivity of $\tau$, mentioned above, and the fact that $\tau\sigma \in H$ for all $\sigma \in H$, meaning $\tau$ is a bijection on the set $\{\sigma(\theta) : \sigma \in H\}$. This tells us that the coefficients of $f(x)$ are fixed under every $\tau \in H$, implying $f(x) \in K^H[x] = E[x]$.

    Since $\id_K \in H$, then $(x - \theta) \mid f(x)$, implying $f(\theta) = 0$, so $m_{\theta, E}(x) \mid f(x)$. Clearly, $f(x)$ is monic, so it remains to show $f(x)$ is irreducible in $E[x]$. Suppose, for contradiction, that $f(x) = g(x)h(x)$ for some nonconstant $g(x), h(x) \in E[x]$. By the construction of $f(x)$, we can assume
    \[
        g(x) = \prod_{j=1}^{k} (x - \sigma_j(\theta)) \isp{and} h(x) = \prod_{j=k+1}^{n} (x - \sigma_j(\theta)),
    \]
    for some $1 \leq k < n$, and where $H = \{\sigma_1, \dots, \sigma_n\}$. Assume $\sigma_1 = \id_K$, so that $\theta$ is a root of $g(x)$, but not of $h(x)$. Since $h(x) \in E[x] = K^H[x]$ and $\sigma_{k+1}^{-1} \in H$, then we must have
    \[
        h(x) = \sigma_{k+1}^{-1}(h(x)) = (x - \theta)\prod_{j=k+2}^{n} (x - \sigma_{k+1}^{-1}\sigma_j(\theta)).
    \]
    However, this would imply that $f(x)$ has $\theta$ as a double root, which is contradiction. Hence, $f(x)$ is irreducible in $E[x]$, and we conclude that $f(x) = m_{\theta, E}(x)$.

    Since $f(x) \in E[x]$, then the field generated by the coefficients of $f(x)$ over $F$ is a subfield of $E$. Moreover, the minimal polynomial of $\theta$ over this field would still be $f(x)$, as $\theta$ is a root and it is irreducible over the possibly larger field of $E$. Therefore, the degree of $K$ over this field would equal $\deg f(x) = [K : E]$, implying that the field generated by the coefficients of $f(x)$ over $F$ is precisely $E$.

\end{proof}

\newpage
\begin{pbox}[Q4 Problem 14.7.12]
    Let $L$ be the Galois closure of the finite extension $\Q(\alpha)$ of $\Q$. For any prime $p$ dividing the order of $\Gal(L/\Q)$ prove there is a subfield $F$ of $L$ with $[L : F] = p$ and $L = F(\alpha)$.
    
    (Hint: One can use Cauchy's Theorem: If $G$ is a finite group, $p$ is a prime number and $p \mid |G|$, then $G$ has a subgroup of order $p$.)
\end{pbox}

\begin{proof}
    By Cauchy's theorem, there exists a subgroup $H \leq \Gal(L/\Q)$ of order $p$. Suppose $\Gal(L/\Q) = \{\sigma_1, \dots, \sigma_n\}$, then applying Q1(b), we find
    \[
        L
            = \sigma_1(\Q(\alpha)) \cdots \sigma_n(\Q(\alpha))
            = \Q(\sigma_1(\alpha)) \cdots \Q(\sigma_n(\alpha))
            = \Q(\sigma_1(\alpha), \dots, \sigma_n(\alpha)).
    \]
    This means that any automorphism of $L$ fixing $\Q$ is completely determined by its image of the generators $\sigma_1(\alpha), \dots, \sigma_n(\alpha)$. Therefore, choosing $\tau \in H \setminus\{\id_L\}$ (which exists since $|H| = p \geq 2$), we know there must be some $\sigma \in \Gal(L/F)$ such that $\tau\sigma(\alpha) \ne \sigma(\alpha)$. In which case, $\sigma^{-1}\tau\sigma(\alpha) \ne \alpha$, meaning $\alpha$ is not fixed by the conjugate subgroup $\sigma^{-1}H\sigma \leq \Gal(L/\Q)$.
    
    Define $F = L^{\sigma^{-1}H\sigma} \subseteq L$, then by construction, $\alpha \notin F$. Since conjugation by $\sigma$ is an injective endomorphism on $\Gal(L/\Q)$, we deduce
    \[
        [L : F] = [L : L^{\sigma^{-1}H\sigma}] = |\sigma^{-1}H\sigma| = |H| = p.
    \]
    Since $p$ is prime, then $L$ and $F$ are the only subfields of $L$ containing $F$. Since $\alpha \notin F$, then $F(\alpha)$ is a nontrivial field extension of $F$ contained in $L$, implying $F(\alpha) = L$.
    

    

\end{proof}



\newpage
\begin{pbox}[Q5]
    Let $F$ be a field and $n$ be a positive integer. Suppose that $\operatorname{ch}(F) = 0$ or $\operatorname{ch}(F)\nmid n$ and $x^n-1$ splits completely over $F$. Denote by $\sqrt[n]{a}$ a root in $\clo{F}$ of $x^n - a \in F[x]$. Let $m = [F(\sqrt[n]{a}) : F]$. Show that $m$ is the smallest positive integer such that $(\sqrt[n]{a})^m \in F$.
\end{pbox}


\begin{proof}
    The hypothesis on $F$ is precisely the conditions for $\Gal(F(\sqrt[n]{a})/F) \isom \Z/m\Z$. Suppose the Galois group is generated by some $\sigma$, so that
    \[
        \Gal(F(\sqrt[n]{a})/F) = \<\sigma\> = \{\id_{F(\sqrt[n]{a})}, \sigma, \sigma^2, \dots, \sigma^{m-1}\},
    \]
    where $\id_{F(\sqrt[n]{a})} = \sigma^m$, since $|\sigma| = m$. Any automorphism of $F(\sqrt[n]{a})$ fixing $F$ is completely determined by the image of $\sqrt[n]{a}$, which must be mapped to some other root of $x^n - a$. Suppose $\sigma(\sqrt[n]{a}) = \sqrt[n]{a}\zeta_n^r$, where $\zeta_n \in F$ is a primitive $n$-th root of unity and $r$ is a nonnegative integer, so for all integers $k$, 
    \[
        \sigma^k(\sqrt[n]{a}) = \sqrt[n]{a}\zeta_n^{rk}.
    \]
    In particular, $\sigma^m = \id_{F(\sqrt[n]{a})}$ implies $\zeta_n^{rm} = 1$, i.e., that $\zeta_n^r$ is an $m$-th root of unity. Moreover, for $1 \leq k < m$, the fact that $\sigma^k \ne \id_{F(\sqrt[n]{a})}$ means $\zeta_n^{rk} \ne 1$. From this, we deduce that $\zeta_n^r$ is in fact a primitive $m$-th root of unity, and denote it by $\zeta_m$. 
    
    Applying Q3 to $\<\sigma\>$ as a subgroup (of course, itself being the entire Galois group), the fixed field is
    \[
        F(\sqrt[n]{a})^{\<\sigma\>} = F(\sqrt[n]{a})^{\Gal(F(\sqrt[n]{a})/F)} = F,
    \]
    and the minimal polynomial of $\sqrt[n]{a}$ over this fixed field is given by
    \[
        m_{\sqrt[n]{a}, F}(x)
            = \prod_{\tau \in \<\sigma\>} (x - \sigma^k(\sqrt[n]{a}))
            = \prod_{k=0}^{m-1} (x - \sqrt[n]{a}\zeta_m^k) = x^m - \sqrt[n]{a}^m.
    \]
    In particular, this implies $\sqrt[n]{a}^m \in F$. Moreover, $\sqrt[n]{a}^k \notin F$ for any positive integer $k < m$. Otherwise, $x^k - \sqrt[n]{a}^k$ would be a polynomial in $F[x]$ with $\sqrt[n]{a}$ as a root, but having a strictly smaller degree than the minimal polynomial of $\sqrt[n]{a}$ over $F$.

\end{proof}

\end{document}
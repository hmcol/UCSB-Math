\documentclass[12pt]{article}

% Packages
\usepackage[margin=1in]{geometry}
\usepackage{fancyhdr, parskip}
\usepackage{amsmath, amsthm, amssymb}
\usepackage{tikz, tikz-cd}
\usepackage[shortlabels]{enumitem}

% Font
\usepackage{tgpagella} % Palatino clone
\usepackage[euler-digits]{eulervm}
\DeclareMathAlphabet{\mathcal}{OMS}{cmsy}{m}{n} % Computer Modern calligraphic font

% Page Style
\makeatletter
\fancypagestyle{title}{
    \renewcommand{\headrulewidth}{0.4pt}
    \setlength{\headheight}{15pt}
    \fancyhead[R]{\@author}
    \fancyhead[L]{\@title}
    \fancyhead[C]{\@date}
}
\makeatother
\renewcommand{\maketitle}{\thispagestyle{title}}
\fancypagestyle{plain}{
    \fancyhf{}
    \renewcommand{\headrulewidth}{0pt}
    \renewcommand{\footrulewidth}{0pt}
    \fancyfoot[R]{\thepage}
}
\pagestyle{plain}

% Problem Box
\setlength{\fboxsep}{4pt}
\newlength{\myparskip}
\setlength{\myparskip}{\parskip}
\newsavebox{\savefullbox}
\newenvironment{fullbox}{\begin{lrbox}{\savefullbox}\begin{minipage}{\dimexpr\textwidth-2\fboxsep\relax}\setlength{\parskip}{\myparskip}}{\end{minipage}\end{lrbox}\framebox[\textwidth]{\usebox{\savefullbox}}}
\newenvironment{pbox}[1][]{\begin{fullbox}\def\temp{#1}\ifx\temp\empty\else\paragraph{#1}\phantom{}\fi}{\end{fullbox}}

% Theorem Environments
\theoremstyle{definition}
\newtheorem{lemma}{Lemma}
\newtheorem{proposition}{Proposition}

% Tikz Environments
\newenvironment{drawing}{\begin{center}\begin{tikzpicture}}{\end{tikzpicture}\end{center}}
% \tikzcdset{row sep/normal=0pt}
\newenvironment{cd}{\begin{center}\begin{tikzcd}}{\end{tikzcd}\end{center}}



% Default Commands
\newcommand{\isp}[1]{\quad\text{#1}\quad}
\newcommand{\eps}{\varepsilon}
\renewcommand{\phi}{\varphi}
\renewcommand{\emptyset}{\varnothing}
\newcommand{\<}{\langle}
\renewcommand{\>}{\rangle}
\newcommand{\eqc}{\overline}
\newcommand{\clo}{\overline}
\renewcommand{\tilde}{\widetilde}
\renewcommand{\hat}{\widehat}
\newcommand{\dd}{\mathrm{d}}
\newcommand{\mat}[1]{\begin{bmatrix}#1\end{bmatrix}}

% infix symbols
\newcommand{\iso}{\cong}
\newcommand{\seq}{\subseteq}
% \newcommand{\teq}{\trianglelefteq} % bro what
\newcommand{\inc}{\hookrightarrow}
\newcommand{\tensor}{\otimes}

% math roman and operator names
\newcommand{\id}{\mathrm{id}} % yes (:
\newcommand{\pr}{\mathrm{pr}}

\DeclareMathOperator{\im}{im}
\DeclareMathOperator{\Mor}{Mor}

% mathbb
\newcommand{\N}{\mathbb{N}} 
\newcommand{\Z}{\mathbb{Z}}
\newcommand{\Q}{\mathbb{Q}}
\newcommand{\R}{\mathbb{R}}
\newcommand{\C}{\mathbb{C}}
\newcommand{\A}{\mathbb{A}}
\renewcommand{\P}{\mathbb{P}}

% mathcal
\newcommand{\CC}{\mathcal{C}}
\newcommand{\GG}{\mathcal{G}}
\newcommand{\NN}{\mathcal{N}}

% category names
\newcommand{\mathcat}{\mathsf}
\newcommand{\Set}{\mathcat{Set}}
\newcommand{\Mod}{\mathcat{Mod}}

% Extra Commands
\newcommand{\obj}{\bullet}
\newcommand{\keyword}{\textbf}


% Document
\begin{document}
\title{TRS Exercises 1}
\author{Harry Coleman}
\date{2024}
\maketitle

\begin{pbox}[1.1.A.]
    A category in which each morphism is an isomorphism is called a \textbf{groupoid}.
\end{pbox}

\begin{pbox}[(a)]
    A perverse definition of a \textbf{group} is: a groupoid with one object. 
    Make sense of this.
\end{pbox}

Let $\GG$ be a (locally-small) category with one object.
Call the object $\obj$ and assume all morphisms are isomorphisms, i.e.,
\[
    \Mor_\GG(\obj, \obj) = \Mor_\GG(\obj) = \operatorname{Aut}_\GG(\obj) = \operatorname{Iso}_\GG(\obj).
\]

We claim there is a group $G$ whose underlying set is $\underline{G} = \Mor_G(\obj)$ and whose operation is given by composition of morphisms.
For any two elements $f, g \in G$, put $f \cdot g = f \circ g$, where composition is performed in $\GG$. The distinguished identity element of $G$ will be the identity morphism of $\obj$; put $1_G = \id_\obj$.

To see that this element satisfied the identity axioms of a group, we appeal to the axioms of a category.
The associativity of the operation in $G$ follows from the associativity of composition in $\GG$:
\[
    (f \cdot g) \cdot h
        = (f \circ g) \circ h
        = f \circ (g \circ h)
        = f \cdot (g \cdot h).
\]
The properties of $1_G$ in $G$ ``are the same as'' the properties of $\id_\obj$ in $\GG$:
\[
    f \cdot 1_G = f \circ \id_\obj = f
    \isp{and}
    1_G \cdot f = \id_\obj \circ f = f.
\]
The existence of inverses in $G$ is guaranteed by the fact that all morphisms in $\GG$ are isomorphisms: if $f \in \Mor_\GG(\obj)$ then there exists $f^{-1} \in \Mor_\GG(\obj)$ giving us
\[
    f \cdot f^{-1} = f \circ f^{-1} = \id_\obj = 1_G
    \isp{and}
    f^{-1} \cdot f = f^{-1} \circ f = \id_\obj = 1_G.
\]

Hence, $G$ is a group.

This construction may be followed backwards from a group to obtain a category with one object, where all morphisms are isomorphisms.

\begin{pbox}[(b)]
    Describe a groupoid that is not a group.
\end{pbox}

We give the data of a category $\NN$ as follows:
\begin{enumerate}[1.]
    \item a single object $\obj$,
    \item a set of morphisms $\Mor_\NN(\obj) = \{\id_\obj, s, s^2, \dots, s^n, \dots\}$, for which $s^n = s \circ \cdots \circ s$ is the $n$-times composition of $s$ with itself, and all such compositions are distinct.
\end{enumerate}

Putting $s^0 = \id_\obj$ and $s^1 = s$ suggests an obvious bijection between the set of morphisms $\Mor_\NN(\obj)$ and the set of natural numbers $\N = \{0, 1, 2, \dots,\}$.

Indeed, this construction of $\NN$ mirrors the inductive construction of the natural numbers, in which the morphism $s$ is the successor function.

Indeed, this is a monoid, where composition is the addition of natural numbers.

\begin{pbox}[1.1.C, 1.1.D]
    omitted
\end{pbox}

Did in MATH 237A Homework 7.

Didn't check naturality for D?


\begin{pbox}[1.2.A.]
    Show that any two initial objects are uniquely isomorphic.
    Show that any two final objects are uniquely isomorphic.
\end{pbox}

Let $I$ and $I'$ be two initial objects.

Let $f : I \to I'$ be the unique map from $I$ to $I'$, which exists since $I$ is initial.

Let $g : I' \to I$ be the unique map from $I'$ to $I$, which exists since $I'$ is initial.

Then $g \circ f : I \to I$ is an endomorphism of $I$.
But $\id_I$ is also an endomorphism of $I$, so by the universal property of the initial object $I$, we have $g \circ f = \id_I$.

By an analogous argument, $f \circ g = \id_{I'}$.

Hence, $f$ and $g$ are inverse morphisms, so $I$ and $I'$ are isomorphic.

The uniqueness of this isomorphism follows from the universal properties of the initial objects.

By duality, a final object in a category is an initial object in the opposite category.
Initial objects in (opposite) categories are uniquely isomorphic, so final objects in the original category are uniquely isomorphic.

\begin{pbox}[1.2.B]
    omitted
\end{pbox}

Did in MATH 237A Homework 7.

\begin{pbox}[1.2.C.]
    Show that $\iota : A \to S^{-1}A$ ($a \mapsto a/1$) is injective iff $S$ contains no zero divisors.
\end{pbox}

In a stupid predicate logic way for fun.
\begin{align*}
    \iota \text{ is injective}
        &\iff \ker\iota = 0 && \text{classic algebra result} \\
        &\iff \forall a \in A \text{ nonzero}, \iota(a) \ne 0 && \text{def of kernel} \\
        &\iff \forall a \in A \text{ nonzero, } a/1 \ne 0 && \text{def of } \iota \\
\end{align*}
Then
\begin{align*}
    a/1 = 0
        &\iff \exists u \in S \text{ s.t. } ua = 0 && \text{def of `$=$' in } S^{-1}A \\
        &\iff \lnot\forall u \in S, ua \ne 0 && \text{quantifier duality} \\
\end{align*}
Then
\begin{align*}
    \iota \text{ is injective} 
        &\iff \forall a \in A \text{ nonzero, } \forall u \in S, ua \ne 0 
            && \text{substitution} \\
        &\iff \forall u \in S, \forall a \in A \text{ nonzero, } ua \ne 0 
            && \text{$\forall$-commutativity}\\
        &\iff \forall u \in S, \lnot\exists a \in A \text{ nonzero s.t. } ua = 0 
            && \text{quantifier duality} \\
        &\iff \forall u \in S, u \text{ is not a zero divisor} 
            && \text{def of zero divisor} \\
        &\iff \lnot\exists u \in S \text{ s.t. } u \text{ is a zero divisor} 
            && \text{quantifier duality} \\
        &\iff S \text{ contains no zero divisors}
            && \text{rewriting} \\
\end{align*}


\begin{pbox}[1.2.D.]
    Verify that $A \to S^{-1}A$ satisfies the following universal property: $S^{-1}A$ is initial among $A$-algebras $B$ where every element of $S$ is sent to an invertible element in $B$.
\end{pbox}

Define the map $f : S^{-1}A \to B$ by $a/s \mapsto s^{-1}a$ where $s^{-1} \in B$ is the inverse of $s \in S$.

Check well-definedness: show $a/s = b/t$ implies  $s^{-1}a = t^{-1}b$. (falls out of equality in $S^{-1}A$).

Check that $f$ is an $A$-Homomorphism.
Show
\begin{itemize}
    \item $f(a/s + b/t) = f(a/s) + f(b/t)$
    \item $f((a/s)(b/t)) = f(a/s)f(b/t)$
    \item $f(a(b/s)) = af(b/s)$
\end{itemize}

For another $A$-Hom $g : S^{-1}A \to B$

Check uniqueness: show that if $g : S^{-1}A \to B$ is an $A$-Hom which sends elements of $S$ to invertible elements of $B$, then $g = f$.
Use def of $A$-Hom and def of $f$.

\begin{pbox}[1.2.I.]
    Show that the tensor product $(T, t : M \times N \to T)$ defined by the universal property is unique up to unique isomorphism.
\end{pbox}

Consider the category $\CC$ consisting of
\begin{itemize}
    \item objects: pairs $(T, t)$ where $t : M \times N \to T$ is $A$-bilinear,
    \item morphisms: a morphism $(T, t) \to (T, t')$ consists of an $A$-linear map $f : T \to T'$ such that the following diagram commutes:
    \begin{cd}
        & M \times N \dlar["t"'] \drar["t'"] \\
        T \ar[rr, "f"] && T'
    \end{cd}
\end{itemize}
Moreover, the identity morphism on $(T, t)$ is the identity map on $T$, and composition of morphisms is composition of the linear maps.

If $(T, t)$ and $(T', t')$ are two tensor products of $M$ and $N$, then there are morphisms $f : T \to T'$ and $g : T' \to T$, which both commute with $t$ and $t'$.
Then their compositions must be the endomorphisms of $T$ and $T'$ which make the relevant diagrams commute.
But the universal property requires that these be the identity morphisms on $T$ and $T'$, respectively.
Hence, $(T, t)$ and $(T', t')$ are isomorphic in $\CC$.
Additionally, the universal property of the tensor product guarantees that this isomorphism is unique.

\begin{pbox}[1.2.J.]
    no
\end{pbox}

\begin{pbox}[Tensor Product Construction]
    Let $A$ be a commutative ring and let $M$ and $N$ be $A$-modules.
    For elements $m \in M$ and $n \in N$, define the \keyword{pure tensor} \textit{of $m$ and $n$} to be the symbol
    \[
        m \tensor n.
    \]
    We now consider the free $A$-module $F$ generated by the set of all pure tensors:
    \[
        F 
            := \bigoplus_{\substack{m \in M \\ n \in N}} A\<m \tensor n\>
            \iso A^{(M \times N)},
    \]
    where each $A\<m \tensor n\>$ is a copy of $A$ with basis element $m \tensor n$.
    Now define the submodule $K$ of $F$, generated by the following elements:
    \[
        K := \left\<\begin{array}{l}
            (m_1 + m_2) \tensor n - m_1 \tensor n - m_2 \tensor n, \\
            m \tensor (n_1 + n_2) - m \tensor n_1 - m \tensor n_2, \\
            a(m \tensor n) - am \tensor n, \\
            a(m \tensor n) - m \tensor an
        \end{array}
        \right\>.
    \]
    Now we define the \keyword{tensor product} of $M$ and $N$ over $A$ to be the quotient module
    \[
        M \tensor_A N := F/K.
    \]

    In other words, $M \tensor_A N$ is the module whose elements are finite $A$-linear combinations of pure tensors, subject to the relations generating $K$.
\end{pbox}

\begin{pbox}[1.2.K. (a)]
    If $M$ is an $A$-module and $A \to B$ is a morphism of rings, give $B \tensor_A M$ the structure of a $B$-module (this is part of the exercise).
    Show that this describes a functor $\Mod_A \to \Mod_B$.
\end{pbox}

With $A \to B$ a morphism of rings, we consider $B$ as an $A$-algebra and, in particular, as an $A$-module.

Fix a scalar $\beta \in B$.
We now define the map
\begin{align*}
    g_\beta : B \times M &\longrightarrow B \tensor_A M, \\
    (b, m) &\longmapsto (\beta b) \tensor m.
\end{align*}
We claim that this map is $A$-bilinear, i.e., it is $A$-linear in each variable.
\begin{align*}
    g_\beta(ab_1 + b_2, m)
        &= (\beta(ab_1 + b_2)) \tensor m \\
        &= (\beta ab_1 + \beta b_2) \tensor m \\
        &= (\beta ab_1) \tensor m + (\beta b_2) \tensor m \\
        &= a(\beta b_1 \tensor m) + (\beta b_2 \tensor m) \\
        &= a g_\beta(b_1, m) + g_\beta(b_2, m)
\end{align*}
\begin{align*}
    g_\beta(b, am_1 + m_2)
        &= (\beta b) \tensor (am_1 + m_2) \\
        &= (\beta b) \tensor am_1 + (\beta b) \tensor m_2 \\
        &= a((\beta b) \tensor m_1) + (\beta b) \tensor m_2 \\
        &= a g_\beta(b, m_1) + g_\beta(b, m_2)
\end{align*}
By the universal property of the tensor product, there is an $A$-linear map
\begin{align*}
    \beta\cdot : B \tensor_A M &\longrightarrow B \tensor_A M, \\
        b \tensor m &\longmapsto (\beta b) \tensor m.
\end{align*}
We call this the $B$-scalar multiplication on $B \tensor_A M$.
We check that this indeed satisfies the requirements of a $B$-module structure:
\begin{enumerate}[(i)]
    \item Because $\beta\cdot$ is $A$-linear on simple tensors, we get this for free:
    \[
        \beta \cdot (b_1 \tensor m_1 + b_2 \tensor m_2)
            = \beta \cdot (b_1 \tensor m_1) + \beta \cdot (b_2 \tensor m_2).
    \]
    \item Using the rules of tensor products, we have
    \begin{align*}
        (\beta_1 + \beta_2) \cdot (b \tensor m)
            &= ((\beta_1 + \beta_2)b) \tensor m \\
            &= (\beta_1 b + \beta_2 b) \tensor m \\
            &= (\beta_1 b) \tensor m + (\beta_2 b) \tensor m \\
            &= \beta_1 \cdot (b \tensor m) + \beta_2 \cdot (b \tensor m).
    \end{align*}
    \item This property comes down to the associativity of multiplication in $B$:
    \begin{align*}
        (\beta_1\beta_2) \cdot (b \tensor m)
            &= ((\beta_1\beta_2)b) \tensor m \\
            &= (\beta_1(\beta_2 b)) \tensor m \\
            &= \beta_1 \cdot ((\beta_2 b) \tensor m) \\
            &= \beta_1 \cdot (\beta_2 \cdot (b \tensor m)).
    \end{align*}
    \item Lastly,
    \[
        1_B \cdot (b \tensor m)
            = (1_B b) \tensor m
            = b \tensor m.
    \]
\end{enumerate}
Thus, we have indeed found a $B$-module structure on $B \tensor_A M$.

(In the case of noncommutative rings, this is a left $B$-module structure.
We do not have a notion of scaling elements of $M$ by elements of $B$, except those coming from $A \to B$.
In other words, we have $\alpha b \tensor m = b \tensor \alpha m$, but $\beta b \tensor m \ne b \tensor \beta m$ (what would $\beta m$ even mean?).
We can think of the $B$ in $B \tensor_A M$ as holding all the $B$-scalar data that cannot be directly applied to $M$.
It is ``the most general'' way to extend the $A$-module structure of $M$ to a $B$-module structure, while respecting the $A$-module structure of $B$.)

We now have a rule between the categories
\begin{cd}[row sep=small, column sep=large]
    \Mod_A \rar["B \tensor_A -"] & \Mod_B \\
    M \rar[|->] & B \tensor_A M
\end{cd}
We claim that this is a functor.

To define $B \tensor_A -$ on morphisms, consider $f : M \to N$ an $A$-homomorphism and the $A$-bilinear maps characterizing the tensor products in following diagram:
\begin{cd}[row sep=large]
    B \times M \rar["\id_B \times f"] \dar & B \times N \dar \\
    B \tensor_A M & B \tensor_A N
\end{cd}
Composing the top and right arrows gives us an $A$-bilinear map as follows:
\begin{cd}[row sep=large]
    B \times M \rar["\id_B \times f"] \dar \drar[dashed] & B \times N \dar \\
    B \tensor_A M & B \tensor_A N
\end{cd}
Now, the universal property of the tensor product gives us a unique $A$-linear map which makes the following diagram commute:
\begin{cd}[row sep=large]
    B \times M \rar["\id_B \times f"] \dar & B \times N \dar \\
    B \tensor_A M \rar[dashed, "\exists!"'] & B \tensor_A N
\end{cd}
This map is what we will take as $B \tensor_A f$.
On pure tensors, this is the map $b \tensor m \mapsto b \tensor f(m)$.
However, as we want this to be a morphism in $\Mod_B$, we need to check that it is $B$-linear.
We get additivity for free as we already know the map to be $A$-linear, so we need only check that it commutes with arbitrary scalars in $B$.
Let $\beta \in B$, then
\begin{align*}    
    (B \tensor_A f)(\beta(b \tensor m))
        &= (B \tensor_A f)(\beta b \tensor m) \\
        &= \beta b \tensor f(m) \\
        &= \beta(b \tensor f(m)) \\
        &= \beta(B \tensor_A f)(b \tensor m).  
\end{align*}

We now have the data of a functor $B \tensor_A -$ from $\Mod_A$ to $\Mod_B$:
\begin{itemize}
    \item For each $A$-module $M$, a $B$-module $B \tensor_A M$.
    \item For each $A$-homomorphism $f : M \to N$, a $B$-homomorphism $B \tensor_A f$ from $B \tensor_A M$ to $B \tensor_A N$.
\end{itemize}
We now check the functorial properties.
Temporarily denote the functor by $T = B \tensor_A -$.
\begin{enumerate}[(i)]
    \item \textit{Preserves composition}.
    Let $f : M \to N$ and $g : N \to P$ be $A$-homomorphisms.
    We check how $T(g \circ f)$ acts on pure tensors:
    \begin{align*}
        (T(g \circ f))(b \tensor m)
            &= b \tensor (g(f(m))) \\
            &= (Tg)(b \tensor f(m)) \\
            &= (Tg)((Tf)(b \tensor m)) \\
            &= (Tg \circ Tf)(b \tensor m).
    \end{align*}
    So in fact, $T(g \circ f) = Tg \circ Tf$.
    \item \textit{Preserves identities}.
    Let $M$ be an $A$-module, then
    \[ 
        (T\id_M)(b \tensor m)
            = b \tensor \id_M(m)
            = b \tensor m
            = \id_{TM}(b \tensor m).
    \]
\end{enumerate}

Hence, $T = B \tensor_A -$ is a functor from $\Mod_A$ to $\Mod_B$.

\begin{pbox}[1.2.K. (b)]
    If further $A \to C$ is another morphism of rings, show that $B \tensor_A C$ has a natural structure of a ring.
    Hint: multiplication will be given by $(b_1 \tensor c_1)(b_2 \tensor c_2) = (b_1b_2) \tensor (c_1c_2)$.
\end{pbox}

Sketch:

map $(b_1, c_1, b_2, c_2) \mapsto b_1 b_2 \tensor c_1 c_2$ is multilinear so should factor good.
See this by looking at diagram
\begin{cd}
    B \times C \times B \times C \dar["\text{transposition}"] \\
    B \times B \times C \times C \dar["m_B \times m_C"] \\
    B \times C \dar["\tau"] \\
    B \tensor_A C
\end{cd}
Then by universal property has to factor thru multi tensor product
\[
    B \times C \times B \times C \longrightarrow B \tensor_A C \tensor_A B \tensor_A C
\]
Which should factor thru
\[
    B \times C \times B \times C \longrightarrow (B \tensor_A C) \times (B \tensor_A C)
\]

Whole thing can prolly just be shown elementwise easy.

\begin{pbox}[1.2.R.]
    Given morphisms $X_1 \to Y$, $X_2 \to Y$, and $Y \to Z$, show that there is a natural morphism $X_1 \times_Y X_2 \to X_1 \times_Z X_2$, assuming that both the fibered products exist.
\end{pbox}

We consider the fiber product $X_1 \times_Z X_2$ to relative to the morphisms obtained by composition, i.e.,
\[
    X_1 \to Y \to Z
    \isp{and}
    X_2 \to Y \to Z.
\]
The universal property of this fiber product lets us fill in the following commutative diagram:
\begin{cd}
    X_1 \times_Y X_2 \ar[ddr, bend right=20] \ar[drr, bend left=15] \drar[dashed, "\exists!"] \\
    & X_1 \times_Z X_2 \dar \rar & X_2 \dar \\
    & X_1 \rar & Y \drar \\
    & & & Z
\end{cd}


\begin{pbox}[1.2.S.]
    The Diagonal Base Change Diagram.
    Suppose we are given morphism $X_1, X_2 \to Y$ and $Y \to Z$.
    Show that the following diagram is a Cartesian square.
    \begin{cd}
        X_1 \times_Y X_2 \rar \dar & X_1 \times_Z X_2 \dar \\
        Y \rar & Y \times_Z Y
    \end{cd}
\end{pbox}

Give the following names to relevant maps:
\begin{cd}
    X_1 \times_Y X_2 
        \ar[ddr, bend right=20, "p_1"']
        \ar[drr, bend left=15, "p_2"]
        \drar[dashed, "\exists!f"] \\
    & X_1 \times_Z X_2 \dar["q_1"] \rar["q_2"'] & X_2 \dar["\beta"] \\
    & X_1 \rar["\alpha"'] & Y \drar["\gamma"] \\
    & & & Z
\end{cd}
By the universal property of the fiber product $Y \times_Z Y$, we have the following commutative diagram:
\begin{cd}
    Y 
        \ar[ddr, bend right=20, "1_Y"']
        \ar[drr, bend left=15, "1_Y"]
        \drar[dashed, "\exists!i"] \\
    & Y \times_Z Y \dar["r_1"] \rar["r_2"'] & Y \dar["\gamma"] \\
    & Y \rar["\gamma"'] & Z
\end{cd}


\end{document}
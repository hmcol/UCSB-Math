\documentclass[12pt]{article}

% Packages
\usepackage[margin=1in]{geometry}
\usepackage{fancyhdr, parskip}
\usepackage{amsmath, amsthm, amssymb}

% Page Style
\makeatletter
\fancypagestyle{title}{
    \renewcommand{\headrulewidth}{0.4pt}
    \setlength{\headheight}{15pt}
    \fancyhead[R]{\@author}
    \fancyhead[L]{\@title}
    \fancyhead[C]{\@date}
}
\makeatother
\renewcommand{\maketitle}{\thispagestyle{title}}
\fancypagestyle{plain}{
    \fancyhf{}
    \renewcommand{\headrulewidth}{0pt}
    \renewcommand{\footrulewidth}{0pt}
    \fancyfoot[R]{\thepage}
}
\pagestyle{plain}

% Problem Box
\setlength{\fboxsep}{4pt}
\newlength{\myparskip}
\setlength{\myparskip}{\parskip}
\newsavebox{\savefullbox}
\newenvironment{fullbox}{\begin{lrbox}{\savefullbox}\begin{minipage}{\dimexpr\textwidth-2\fboxsep\relax}\setlength{\parskip}{\myparskip}}{\end{minipage}\end{lrbox}\framebox[\textwidth]{\usebox{\savefullbox}}}
\newenvironment{pbox}[1][]{\begin{fullbox}\ifx#1\empty\else\paragraph{#1}\fi}{\end{fullbox}}

% Default Commands
\newcommand{\isp}[1]{\quad\text{#1}\quad}
\newcommand{\N}{\mathbb{N}} 
\newcommand{\Z}{\mathbb{Z}}
\newcommand{\Q}{\mathbb{Q}}
\newcommand{\R}{\mathbb{R}}
\newcommand{\C}{\mathbb{C}}
\newcommand{\eps}{\varepsilon}
\renewcommand{\phi}{\varphi}
\renewcommand{\emptyset}{\varnothing}
\newcommand{\<}{\langle}
\renewcommand{\>}{\rangle}
\newcommand{\isom}{\cong}
\newcommand{\eqc}{\overline}
\newcommand{\clo}{\overline}
\newcommand{\teq}{\trianglelefteq}
\DeclareMathOperator{\id}{id}

% Extra Commands


% Document
\begin{document}
\title{MATH 201A Homework 3}
\author{Harry Coleman}
\date{November 2, 2021}
\maketitle

\begin{pbox}[1]
    Let $\lambda$ be the Lebesgue measure and let $\{A_n\}_{n=1}^{\infty}$ be a sequence of Lebesgue-measurable subsets of $[0, 1]$. Assume the set $B$ consists of those points $x \in [0, 1]$ that belong to infinitely many of the $A_n$.
\end{pbox}

\begin{pbox}[(a)]
    Prove that $B$ is Lebesgue-measurable.
\end{pbox}

\begin{pbox}[(b)]
    Prove that if $\lambda(A_n) > \delta > 0$ for every $n \in \N$, then $\lambda(B) \geq \delta$.
\end{pbox}

\begin{pbox}[(c)]
    Prove that if $\sum_{n=1}^{\infty} \lambda(A_n) < \infty$, then $\lambda(B) = 0$.
\end{pbox}

\begin{pbox}[(d)]
    Give an example where $\sum_{n=1}^{\infty} \lambda(A_n) = \infty$, but $\lambda(B) = 0$.
\end{pbox}



\begin{pbox}[2]
    Prove that if the set $A \subseteq \R$ is Lebesgue-measurable, with $\lambda(A) > 0$, then there is a subset of $A$ that is not Lebesgue-measurable.
\end{pbox}



\begin{pbox}[3]
    Let $\lambda$ be the Lebesgue measure on $\R$.
\end{pbox}

\begin{pbox}[(a)]
    Let $A \subseteq \R$ be a set such that $\lambda(A) > 0$. Prove that for any $\eps > 0$, there exists an interval $(a, b) \subseteq \R$ such that $\lambda(A) \cap (a, b)) > (1 - \eps)(b - a)$.
\end{pbox}

\begin{pbox}[(b)]
    Construct a Borel set $B \subseteq \R$ such that $\lambda B > 0$ and $\lambda(B \cap I) < \lambda(I)$ for every non-degenerate interval $I \subseteq \R$.
\end{pbox}



\begin{pbox}[4]
    Prove that if a Lebesgue-measurable set $A \subseteq \R$ has positive Lebesgue measure, then the set
    \[
        A - A = \{a - b : a, b \in A\}
    \]
    contains a neighborhood of the origin. Is the statement true if one only assumes $\lambda(A) > 0$ (i.e., $A$ is not Lebesgue-measurable)?
\end{pbox}



\begin{pbox}[5]
    Let $A \subseteq R$ be any set. Prove that the set
    \[
        B = \bigcup_{x \in A} [x - 1, x + 1]
    \]
    is Lebesgue-measurable.
\end{pbox}

\end{document}
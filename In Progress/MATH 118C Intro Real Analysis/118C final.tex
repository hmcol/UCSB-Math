% Exam Template for UMTYMP and Math Department courses
%
% Using Philip Hirschhorn's exam.cls: http://www-math.mit.edu/~psh/#ExamCls
%
% run pdflatex on a finished exam at least three times to do the grading table on front page.
%
%%%%%%%%%%%%%%%%%%%%%%%%%%%%%%%%%%%%%%%%%%%%%%%%%%%%%%%%%%%%%%%%%%%%%%%%%%%%%%%%%%%%%%%%%%%%%%

% These lines can probably stay unchanged, although you can remove the last
% two packages if you're not making pictures with tikz.
\documentclass[11pt, addpoints]{exam}
\RequirePackage{amssymb, amsfonts, amsmath, latexsym, verbatim, xspace, setspace,cancel}
\usepackage[shortlabels]{enumitem}
\RequirePackage{tikz, pgfplots, pgflibraryplotmarks,comment}
% By default LaTeX uses large margins.  This doesn't work well on exams; problems
% end up in the "middle" of the page, reducing the amount of space for students
% to work on them.
\usepackage[margin=1in]{geometry}


\pgfplotsset{soldot/.style={color=blue,only marks,mark=*}} \pgfplotsset{holdot/.style={color=blue,fill=white,only marks,mark=*}}


% Here's where you edit the Class, Exam, Date, etc.
\newcommand{\class}{Math 118C}
\newcommand{\term}{Spring 2021}
\newcommand{\examnum}{Final Exam}
%\newcommand{\examdate}{07/17/2021}
\newcommand{\timelimit}{120 Minutes (or per DSP letter)}

% For an exam, single spacing is most appropriate
\singlespacing
% \onehalfspacing
% \doublespacing

% For an exam, we generally want to turn off paragraph indentation
\parindent 0ex

\begin{document} 

% These commands set up the running header on the top of the exam pages
\pagestyle{head}
\firstpageheader{}{}{}
\runningheader{\class}{\examnum}{Page \thepage\ of \numpages}
\runningheadrule

\begin{flushright}
\begin{tabular}{p{2.8in} r l}
\textbf{\class} & \textbf{Name:} & \makebox[2in]{\hrulefill}\\
\textbf{\term} & \textbf{PERM Number:} & \makebox[2in]{\hrulefill}\\
\textbf{\examnum} &&\\
%\textbf{Due date: \examdate} &&\\
\textbf{Time Limit: \timelimit} &&\\
\end{tabular}\\
\end{flushright}
\rule[1ex]{\textwidth}{.1pt}

\begin{itemize}

\item Submit your work on GradeScope within 20 minutes from the end of the exam time indicated above. If you have submitted a DSP letter, this submission window also extends accordingly for you.

\item This is a closed-book exam, you should work independently. In particular, you should not discuss these questions with anyone nor seek help from any internet sources. Violations of academic integrity will be reported. 



\item If you want any clarification during the exam, ask the instructor over Zoom. Zoom ID: 426 530 1130.


\item Organize your work, in a reasonably neat and coherent way. Work scattered all over a page without a clear ordering will receive very little credit.

\item This exam contains \numpages\ pages (including this cover page) and
\numquestions\ problems. 


\end{itemize}












%Do not write in the table to the right.
%\end{minipage}
%\hfill
%\begin{minipage}[t]{2.3in}
%\vspace{0pt}
%\cellwidth{3em}
%\gradetablestretch{2}
%\vqword{Problem}
  % required here by exam.cls, even though questions haven't started yet.	
%\gradetable[v]%[pages]  % Use [pages] to have grading table by page instead of question


\newpage % End of cover page

%%%%%%%%%%%%%%%%%%%%%%%%%%%%%%%%%%%%%%%%%%%%%%%%%%%%%%%%%%%%%%%%%%%%%%%%%%%%%%%%%%%%%
%
% See http://www-math.mit.edu/~psh/#ExamCls for full documentation, but the questions
% below give an idea of how to write questions [with parts] and have the points
% tracked automatically on the cover page.
%
%
%%%%%%%%%%%%%%%%%%%%%%%%%%%%%%%%%%%%%%%%%%%%%%%%%%%%%%%%%%%%%%%%%%%%%%%%%%%%%%%%%%%%%

\begin{questions}

\question[20]
Define $r := \sqrt{x_1^2+x_2^2+x_3^2}$. Let $\omega := (\frac{1}{r})^3 (x_3 dx_1 \wedge dx_2 - x_2 dx_1 \wedge dx_3 + x_1 dx_2 \wedge dx_3)$ be a 2-form on $\mathbb R^3\setminus (0,0,0).$

\begin{parts}
\part Show that $d\omega = 0.$

\vfill

\part Let $B:=\{(x_1, x_2, x_3): (x_1-2)^2+x_2^2+x_3^2=3\}$ be a sphere in $\mathbb R^3,$ find the integral $\int_B \omega.$

\vfill

\end{parts}
\newpage

\question[15] Suppose that 

\newpage

\question[20]
Let $D$ be the closed unit disk in $\mathbb R^2$ and $f$ be a continuous function on $D.$ Show that for any $\epsilon >0,$ there exists a number $n$ and functions $f_1, f_2, \ldots, f_n$ such that $f=f_1+\ldots+f_n$ on $D$ and the support of $f_i$ has Lebesgue measure less than $\epsilon,$ for any $i =1, \ldots, n.$ State any theorem you use.
\newpage





\question[20] 

Prove that a subset $E$ of $\mathbb R^n$ is Lebesgue measurable if and only if for any $\epsilon >0,$ there exists an open set $U\subset \mathbb R^n$ such that $E\subset U$ and $m(U\setminus E)<\epsilon.$


\newpage

\question[20]
Let $\{f_n\}$ be a sequence of measurable functions and define $f:=\liminf_n f_n.$ Is $f$ measurable? If yes, justify your answer. If no, give a counterexample.

\newpage


\question[15]

 Let $\{f_n\}$ be a uniformly convergent and uniformly bounded sequence of Lebesgue integrable functions on $\mathbb R^1$ and let $f:=\lim_n f_n$ be the limit. Is it true that $$\lim_n \int_{\mathbb R^1} f_n dm = \int_{\mathbb R^1} f dm ?$$ If yes, justify your answer. If no, give a counterexample. All integrals are Lebesgue integrals.




\end{questions}


\end{document}

\documentclass[12pt]{article}

% Packages
\usepackage[margin=1in]{geometry}
\usepackage{fancyhdr, parskip}
\usepackage{amsmath, amsthm, amssymb}
\usepackage{tikz, tikz-cd}

\usepackage[absolute,overlay]{textpos}


% Page Style
\fancypagestyle{plain}{
    \fancyhf{}
    \renewcommand{\headrulewidth}{0pt}
    \renewcommand{\footrulewidth}{0pt}
    \fancyfoot[R]{\thepage}
}
\pagestyle{plain}

% Problem Box
\setlength{\fboxsep}{4pt}
\newlength{\myparskip}
\setlength{\myparskip}{\parskip}
\newsavebox{\savefullbox}
\newenvironment{fullbox}{\begin{lrbox}{\savefullbox}\begin{minipage}{\dimexpr\textwidth-2\fboxsep\relax}\setlength{\parskip}{\myparskip}}{\end{minipage}\end{lrbox}\framebox[\textwidth]{\usebox{\savefullbox}}}
\newenvironment{pbox}[1][]{\begin{fullbox}\ifx#1\empty\else\paragraph{#1}\fi}{\end{fullbox}}

% Commutative Diagram
\newenvironment{cd}{\begin{center}\begin{tikzcd}}{\end{tikzcd}\end{center}}

% Theorem Environments
%\theoremstyle{definition}
%\newtheorem{proposition}{Proposition}
%\newtheorem{lemma}{Lemma}

% Options
%\allowdisplaybreaks
%\addtolength{\jot}{4pt}

% Default Commands
\newcommand{\CMDmathbb}[1]{\expandafter\def\csname #1\endcsname{\mathbb{#1}}}

\makeatletter
\newcommand{\FACTORYmathbb}[1]{\@for\chr:=#1\do{\expandafter\CMDmathbb\chr}}
\makeatother

\FACTORYmathbb{N, Z, Q, R, C}


\newcommand{\isp}[1]{\quad\text{#1}\quad}
\newcommand{\eps}{\varepsilon}
\renewcommand{\phi}{\varphi}
\renewcommand{\emptyset}{\varnothing}
\newcommand{\<}{\langle}
\renewcommand{\>}{\rangle}
\newcommand{\isom}{\cong}
\newcommand{\eqc}{\overline}
\newcommand{\clo}{\overline}

% Extra Commands
\DeclareMathOperator{\id}{id}
\DeclareMathOperator{\Aut}{Aut}
\DeclareMathOperator{\Inn}{Inn}
\newcommand{\teq}{\trianglelefteq}


% Document Info
\fancypagestyle{title}{
    \renewcommand{\headrulewidth}{0.4pt}
    \setlength{\headheight}{15pt}
    \fancyhead[R]{Harry Coleman}
    \fancyhead[L]{MATH 220A Homework 1}
    \fancyhead[C]{October 1, 2021}
}

% Begin Document
\begin{document}
\thispagestyle{title}

\begin{textblock*}{0in}(7.5in, 0.75in)
    \makebox[0pt][r]{(worked w/ Joseph Sullivan)}
\end{textblock*}

\begin{pbox}[1 Exercise I.3]
    Let $G$ be a group. A \textbf{commutator} in $G$ is an element of the form $aba^{-1}b^{-1}$ with $a, b \in G$. Let $G^c$ be the subgroup generated by the commutators. Then $G^c$ is called the \textbf{commutator subgroup}. Show that $G^c$ is normal.
\end{pbox}

\begin{proof}
    Let $x \in G$ and $y \in G^c$, then there is a commutator $xyx^{-1}y^{-1} \in G^c$. So, as the product of two elements in $G^c$,
    \[
        xyx^{-1} = (xyx^{-1}y^{-1})y \in G^c.
    \]
    Hence, $xG^cx^{-1} \subseteq G^c$ for all $x \in G$, so in fact $G^c \teq G$.

\end{proof}

\begin{pbox}
    Show that any homomorphism of $G$ into an abelian group factors through $G/G^c$.
\end{pbox}

\begin{proof}
    Let $\phi : G \to A$ be a group homomorphism to an abelian group $A$ and $\pi : G \to G/G^c$ be the natural projection. We want to find a group homomorphism $\psi : G/G^c \to A$ such that $\psi \circ \pi = \phi$. In order for such a $\psi$ to exist, $\phi$ must be constant on the equivalence classes in $G/G^c$. In other words, we must show that $\eqc{x} = \eqc{y} \in G/G^c$ implies $\phi(x) = \phi(y)$. For any $x, y \in G$,
    \[
        \phi(xy) = \phi(x)\phi(y) = \phi(y)\phi(x) = \phi(yx),
    \]
    where the commutation of $\phi(x)$ and $\phi(y)$ is permitted in $A$. Then
    \[
        1 = \phi(xy) \phi(yx)^{-1} = \phi(xyx^{-1}y^{-1}).
    \]
    In other words, all the commutators in $G$ are in the kernel of $\phi$. Since $G^c$ is generated by the commutators, we conclude that $G^c \subseteq \ker \phi$. This means that if $\eqc{x} = \eqc{y} \in G/G^c$, i.e., $xy^{-1} \in G^c$, then $1 = \phi(xy^{-1}) = \phi(x)f(y)^{-1}$, giving us $\phi(x) = \phi(y)$.

    Since $\phi$ is constant on equivalence classes, then we obtain a group homomorphism $\psi$ from $G/G^c$ to $A$, with $\eqc{x} \mapsto \phi(x)$, as desired.

\end{proof}

\begin{pbox}[2 Exercise I.6]
    Prove that the group of inner automorphisms of a group $G$ is normal in $\Aut(G)$.
\end{pbox}

\begin{proof}
    Denote the subgroup of $\Aut(G)$ consisting of the inner automorphisms by
    \[
        \Inn(G) = \{c_x \mid x \in G\}.
    \]
    (Note $c_x$ is the conjugation $y \mapsto xyx^{-1}$.) Let $\sigma \in \Aut(G)$ and $c_x \in \Inn(G)$, then for $y \in G$,
    \begin{align*}
        (\sigma c_x \sigma^{-1})(y)
            &= \sigma(x \sigma^{-1}(y) x^{-1}) \\
            &= \sigma(x) y \sigma(x)^{-1} \\
            &= c_{\sigma(x)}(y).
    \end{align*}
    As $c_{\sigma(x)} \in \Inn(G)$, this proves $\sigma\Inn(G)\sigma^{-1} \subseteq \Inn(G)$, hence $\Inn(G) \teq \Aut(G)$.
    

\end{proof}



\newpage
\begin{pbox}[3 Exercise I.7]
    Let $G$ be a group such that $\Aut(G)$ is cyclic. Prove that $G$ is abelian.
\end{pbox}

\begin{proof}
    Note that $G$ is abelian if and only if $\Inn(G)$ is trivial. Assume, for contradiction, that $\Inn(G)$ is not trivial. Since $\Inn(G)$ is a subgroup of the cyclic group $\Aut(G)$, it is cyclic and generated by some $c_x \ne \id_G$. Then there is some $y \in G$ such that $c_x(y) \ne y$, i.e., $xy \ne yx$. However, as $c_y \in \Inn(G) = \<c_x\>$, then we have $c_y = c_x^{k}$ for some $k \in \Z$. Therefore,
    \begin{align*}
        c_y(x) &= c_x^k(x), \\
        yxy^{-1} &= x^kxx^{-k}, \\
        yx &= xy,
    \end{align*}
    which is a contradiction.

\end{proof}



\newpage
\begin{pbox}[4 Exercise I.12]
    Let $G$ be a group and let $H, N$ be subgroups with $N$ normal. Let $\gamma_x$ be conjugation by an element $x \in G$.
\end{pbox}

\begin{pbox}[(a)]
    Show that $x \mapsto \gamma_x$ induces a homomorphism $f : H \to \Aut(N)$.
\end{pbox}

\begin{proof}
    Since $N$ is normal, then for any $x \in H$, we have $\gamma_x(N) = xNx^{-1}$. So $\gamma_x$ is a bijection, therefore automorphism, on $N$. For $x, y \in H$ and $n \in N$, we find
    \[
        \gamma_{xy}(n)
            = (xy)n(xy)^{-1}
            = x(yny^{-1})x^{-1}
            = (\gamma_x\gamma_y)(n),
    \]
    so $f$ is in fact a group homomorphism.
    
\end{proof}

\begin{pbox}[(b)]
    If $H \cap N = \{e\}$, show that the map $H \times N \to HN$ given by $(x, y) \mapsto xy$ is a bijection, and that this map is an isomorphism if and only if $f$ is trivial, i.e., $f(x) = \id_N$ for all $x \in H$.
\end{pbox}

\begin{proof}
    By definition, $HN$ is the set of points $xy$ such that $x \in H$ and $y \in N$, so the map is surjective. Suppose $x_1, x_2 \in H$ and $y_1, y_2 \in N$ such that $x_1y_1 = x_2y_2$. Then
    \[
        x_1^{-1}x_2
            = y_1^{-1}y_2
            \in H \cap N
            = \{e\},
    \]
    so $x_1^{-1} = x_2^{-1}$ and $y_1^{-1} = y_2^{-1}$. That is, $(x_1, y_1) = (x_2, y_2)$, so the map is injective.

    Denote by $m$ the map $(x, y) \mapsto xy$. If $m$ is an isomorphism, then for $x \in H$ and $y \in N$,
    \[
        \gamma_x(y)
            = xyx^{-1}
            = m(x, y)m(x^{-1}, 1)
            = m(xx^{-1}, y)
            = m(1, y)
            = y.
    \]
    That is, $f(x) = \gamma_x = \id_N$ for all $x \in H$.

\end{proof}

\begin{pbox}[(c)]
    We define $G$ to be the \textbf{semidirect product} of $H$ and $N$ if $G = NH$ and $H \cap N = \{e\}$.

    Conversely, let $N$, $H$ be groups, and let $\psi : H \to \Aut(N)$ be a given homomorphism. Construct a semidirect product as follows. Let $G$ be the set of pairs $(x, h)$ with $x \in N$ and $h \in H$. Define the composition law
    \[
        (x_1, h_1)(x_2, h_2) = (x_1 \psi(h_1) x_2, h_1 h_2).
    \]
    Show that this is a group law, and yields a semidirect product of $N$ and $H$, identifying $N$ with the set of elements $(x, 1)$ and $H$ with the set of elements $(1, h)$.
\end{pbox}

\begin{proof}
    We write $\psi_h = \psi(h)$, so the fact that $\psi$ is a group homomorphism tells us
    \[
        \psi_{h_1h_2} = \psi_{h_1} \circ \psi_{h_2} \in \Aut(N).
    \]
    Let $x, y, z \in N$ and $g, h, k \in H$, then
    \begin{align*}
        \big((x, g)(y, h)\big)(z, k)
            &= \big(x\psi_g(y),\, gh\big)(z, k) \\
            &= \big(x\psi_g(y)\psi_{gh}(z),\, ghk\big) \\
            &= \big(x\psi_g(y)\psi_g\big(\psi_h(z)\big), ghk\big) \\
            &= \big(x\psi_g\big(y\psi_h(z)\big),\, ghk\big) \\
            &= (x, g)\big(y\psi_h(z),\, hk\big) \\
            &= (x, g)\big((y, h)(z, k)\big).
    \end{align*}
    That is, the composition law is associative.

    We claim that $(1, 1)$ is the identity element. Let $x \in N$ and $h \in H$. Since $\psi$ is a group homomorphism, $\psi_1 = \id_N$, so
    \[
        (1, 1)(x, h)
            = (1\psi_1(x), 1h)
            = (x, h).
    \]
    Since $\psi_h \in \Aut(N)$ is a group homomorphism from $N$ to itself, $\psi_h(1) = 1$, so
    \[
        (x, h)(1, 1)
            = (x\psi_h(1), h1)
            = (x, h).
    \]

    Lastly, for $x \in N$ and $h \in H$, we see that $(x, h)^{-1} =(\phi_h^{-1}(x^{-1}), h^{-1})$, since
    \[
        (x, h)(\psi_h^{-1}(x^{-1}), h^{-1})
            = (x\psi_h(\psi_h^{-1}(x^{-1})), hh^{-1})
            = (xx^{-1}, 1)
            = (1, 1).
    \]
    Hence, this is a group law.
    
    We now show that this group is a semidirect product of $N$ and $H$. For any $(x, h)$ in the underlying set, we have $(x, 1) \in N$ and $(1, h) \in H$ under the given identification. Then
    \[
        (x, 1)(1, h)
            = (x\psi_1(1), 1h)
            = (x, h),
    \]
    so $NH$ is the entire set. Moreover, the identification gives us $N \cap H = \{(1, 1)\}$. Lastly, we check that $N$ is normal. For any $(x, h)$ in the group and $y \in N$, we find
    \begin{align*}
        (x, h)(y, 1)(x, h)^{-1}
            &= (x\psi_h(y), h)(\psi_h^{-1}(x^{-1}), h^{-1}) \\
            &= (x\psi_h(y)x^{-1}, 1).
    \end{align*}
    Since $x\psi_h(y)x^{-1} \in N$, this shows $N$ is normal.

\end{proof}


\end{document}
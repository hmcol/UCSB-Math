\documentclass[12pt]{article}

% Packages
\usepackage[margin=1in]{geometry}
\usepackage{fancyhdr, parskip}
\usepackage{amsmath, amsthm, amssymb}

% Page Style
\fancypagestyle{plain}{
    \fancyhf{}
    \renewcommand{\headrulewidth}{0pt}
    \renewcommand{\footrulewidth}{0pt}
    \fancyfoot[R]{\thepage}
}
\pagestyle{plain}

% Problem Box
\setlength{\fboxsep}{4pt}
\newlength{\myparskip}
\setlength{\myparskip}{\parskip}
\newsavebox{\savefullbox}
\newenvironment{fullbox}{\begin{lrbox}{\savefullbox}\begin{minipage}{\dimexpr\textwidth-2\fboxsep\relax}\setlength{\parskip}{\myparskip}}{\end{minipage}\end{lrbox}\framebox[\textwidth]{\usebox{\savefullbox}}}
\newenvironment{pbox}[1][]{\begin{fullbox}\ifx#1\empty\else\paragraph{#1}\fi}{\end{fullbox}}

% Default Commands
\newcommand{\isp}[1]{\quad\text{#1}\quad}
\newcommand{\N}{\mathbb{N}} 
\newcommand{\Z}{\mathbb{Z}}
\newcommand{\Q}{\mathbb{Q}}
\newcommand{\R}{\mathbb{R}}
\newcommand{\C}{\mathbb{C}}
\newcommand{\eps}{\varepsilon}
\renewcommand{\phi}{\varphi}
\renewcommand{\emptyset}{\varnothing}
\newcommand{\<}{\langle}
\renewcommand{\>}{\rangle}
\newcommand{\isom}{\cong}
\newcommand{\eqc}{\overline}
\newcommand{\clo}{\overline}
\DeclareMathOperator{\id}{id}

% Extra Commands
\DeclareMathOperator{\Perm}{Perm}
\DeclareMathOperator{\im}{im}
\DeclareMathOperator{\Aut}{Aut}
\newcommand{\teq}{\trianglelefteq}


\theoremstyle{definition}
\newtheorem{lemma}{Lemma}

% Document Info
\fancypagestyle{title}{
    \renewcommand{\headrulewidth}{0.4pt}
    \setlength{\headheight}{15pt}
    \fancyhead[R]{Harry Coleman\makebox[0pt][r]{\raisebox{-0.25in}[0pt][0pt]{(worked with Joseph Sullivan, Gahl Shemy)}}}
    \fancyhead[L]{MATH 220A Homework 2}
    \fancyhead[C]{October 10, 2021}
}

% Begin Document
\begin{document}
\thispagestyle{title}

\begin{pbox}[1 Exercise I.9]
    \,
\end{pbox}

\begin{pbox}[(a)]
    Let $G$ be a group and $H$ a subgroup of finite index. Show that there exists a normal subgroup $N$ of $G$ contained in $H$ and also of finite index. [\textit{Hint}: If $(G : H) = n$, find a homomorphism of $G$ into $S_n$ whose kernel is contained in $H$.]
\end{pbox}

\begin{proof}
    Consider the map $\phi: G \to \Perm(G/H)$, where $\phi(g) = \phi_g : xH \mapsto gxH$. For $g, h \in G$,
    \[
        \phi_{gh}(xH) = ghxH = g\phi_h(xH) = \phi_g(\phi_h(xH)) = (\phi_g \circ \phi_h)(xH),
    \]
    so $\phi$ is in fact a group homomorphism. Then, $N = \ker\phi$ is a normal subgroup of $G$, with
    \[
        G/N \isom \im\phi \leq \Perm(G/H).
    \]
    Since $|G/H| = [G : H]$ is finite, so is $[G : N] = |G/N| \leq |\Perm(G/H)|$. If $n \in N$, then we have $nH = \phi_n(eH) = H$, implying that $n \in H$. Hence, $N \subseteq H$, so $N$ is as desired.

\end{proof}

\begin{pbox}[(b)]
    Let $G$ be a group and let $H_1, H_2$ be subgroups of finite index. Prove that $H_1 \cap H_2$ has finite index.
\end{pbox}

\begin{proof}
    By part (a), there are normal subgroups of finite index $N_1, N_2 \teq G$ such that $N_1 \subseteq H_1$ and $N_2 \subseteq H_2$, then $N_1 \cap N_2$ is a normal subgroup of $G$ contained in $H_1 \cap H_2$. By [some] isomorphism theorem,
    \[
        N_1/(N_1 \cap N_2) \isom (N_1N_2)/N_2,
    \]
    so we deduce
    \begin{align*}
        [G : N_1 \cap N_2]
            &= [G : N_1][N_1 : N_1 \cap N_2] \\
            &= [G : N_1][N_1N_2 : N_2] \\
            &\leq [G : N_1][G : N_2].
    \end{align*}
    In particular, $N_1 \cap N_2$ is of finite index in $G$. Since $H_1 \cap H_2$ is contained in $N_1 \cap N_2$, we conclude $[G : H_1 \cap H_2] \leq [G : N_1 \cap N_2] < \infty$.

\end{proof}


\newpage
\begin{pbox}[2 Exercise I.14]
    Let $G$ be a finite group and let $N$ be a normal subgroup such that $N$ and $G/N$ have relatively prime orders.
\end{pbox}

\begin{pbox}[(a)]
    Let $H$ be a subgroup of $G$ having the same order as $G/N$. Prove that $G = HN$.
\end{pbox}

\begin{proof}
    Since $H, N \leq HN \leq G$, then $|H|$ and $|N|$ divide $|HN|$, which divides $|G|$. Moreover, the least common multiple of $|H|$ and $|N|$, which is $|H||N|$ since they are coprime, must divide $|HN|$. Then
    \[
        |H||N|
            \leq |HN|
            \leq |G|
            = |G/N||N|
            = |H||N|,
    \]
    which implies $|HN| = |G|$. Since all orders are finite, we conclude that $HN = G$.

\end{proof}

\begin{pbox}[(b)]
    Let $g$ be an automorphism of $G$. Prove that $g(N) = N$.
\end{pbox}

\begin{proof}
    By the diamond isomorphism theorem,
    \[
        g(N)/(g(N) \cap N) \isom g(N)N/N \leq G/N.
    \]
    In particular, $|g(N)N/N|$ divides $|G/N|$. Additionally,
    \[
        |N| = |g(N)| = |g(N)N/N||g(N) \cap N|,
    \]
    which means that $|g(N)N/N|$ also divides $|N|$. Since $|N|$ and $|G/N|$ are coprime, we must have $|g(N)N/N| = 1$. Hence, $g(N)N = N$, implying that $g(N) = N$.

\end{proof}


\newpage
\begin{pbox}[3 Exercise I.15]
    Let $G$ be a finite group operating on a finite set $S$ with $\#(S) \geq 2$. Assume that there is only one orbit. Prove that there exists an element $x \in G$ which has no fixed point, i.e. $xs \ne s$ for all $s \in S$.
\end{pbox}

\begin{proof}
    For all $s \in S$, we have $G \cdot s = S$. Then
    \[
        |S| = |G \cdot s| = |G : G_s| = \frac{|G|}{|G_s|},
    \]
    so
    \[
        |G|
            = |G|\sum_{s \in S} \frac{1}{|S|}
            = \sum_{s \in S} |G_s|.
    \]

    Let $C = \{(x, s) \in G \times S \mid xs = s\}$. Then, applying Exercise I.17 (the next problem) both ways, we obtain
    \[
        \sum_{x \in G} |S^x| = |C| = \sum_{s \in S} |G_s|,
    \]
    where $S^x$ is the subset of $S$ fixed by $g$. So, we have found $|G| = \sum_{x \in G} S^x$. Since $S^e = S$, then $|S^e| = |S| \geq 2$. We deduce that there is some $x \in G$ with $|S^x| = 0$, i.e., $x$ has no fixed points in $S$.

\end{proof}


\newpage
\begin{pbox}[4 Exercise I.17]
    Let $X$, $Y$ be finite sets and let $C$ be a subset of $X \times Y$. For $x \in X$ let $\phi(x) =$ number of elements $y \in Y$ such that $(x, y) \in C$. Verify that
    \[
        \#(C) = \sum_{x \in X} \phi(x). 
    \]
\end{pbox}

\begin{proof}
    For $x \in X$, let $C_x$ be the subset of $C$ with the first component equal to $x$. Then, $C$ can be written as the disjoint union $C = \bigsqcup_{x \in X} C_x$. By construction, $|C_x| = \phi(x)$, so
    \[
        |C|
            = \left| \bigsqcup_{x \in X} C_x \right|
            = \sum_{x \in X} |C_x|
            = \sum_{x \in X} \phi(x).
    \]

\end{proof}

\end{document}
\documentclass[12pt]{article}

% Packages
\usepackage[margin=1in]{geometry}
\usepackage{fancyhdr, parskip}
\usepackage{amsmath, amsthm, amssymb}
\usepackage{tikz, tikz-cd}

% Page Style
\makeatletter
\fancypagestyle{title}{
    \renewcommand{\headrulewidth}{0.4pt}
    \setlength{\headheight}{15pt}
    \fancyhead[R]{\@author}
    \fancyhead[L]{\@title}
    \fancyhead[C]{\@date}
}
\makeatother
\renewcommand{\maketitle}{\thispagestyle{title}}
\fancypagestyle{plain}{
    \fancyhf{}
    \renewcommand{\headrulewidth}{0pt}
    \renewcommand{\footrulewidth}{0pt}
    \fancyfoot[R]{\thepage}
}
\pagestyle{plain}

% Problem Box
\setlength{\fboxsep}{4pt}
\newlength{\myparskip}
\setlength{\myparskip}{\parskip}
\newsavebox{\savefullbox}
\newenvironment{fullbox}{\begin{lrbox}{\savefullbox}\begin{minipage}{\dimexpr\textwidth-2\fboxsep\relax}\setlength{\parskip}{\myparskip}}{\end{minipage}\end{lrbox}\framebox[\textwidth]{\usebox{\savefullbox}}}
\newenvironment{pbox}[1][]{\begin{fullbox}\ifx#1\empty\else\paragraph{#1}\phantom{}\fi}{\end{fullbox}}

% Theorem Environments
\theoremstyle{definition}
\newtheorem{lemma}{Lemma}

% Tikz Environments
\newenvironment{drawing}{\begin{center}\begin{tikzpicture}}{\end{tikzpicture}\end{center}}
% \tikzcdset{row sep/normal=0pt}
\newenvironment{cd}{\begin{center}\begin{tikzcd}}{\end{tikzcd}\end{center}}

% Default Commands
\newcommand{\isp}[1]{\quad\text{#1}\quad}
\newcommand{\N}{\mathbb{N}} 
\newcommand{\Z}{\mathbb{Z}}
\newcommand{\Q}{\mathbb{Q}}
\newcommand{\R}{\mathbb{R}}
\newcommand{\C}{\mathbb{C}}
\newcommand{\A}{\mathbb{A}}
\renewcommand{\P}{\mathbb{P}}
\newcommand{\eps}{\varepsilon}
\renewcommand{\phi}{\varphi}
\renewcommand{\emptyset}{\varnothing}
\newcommand{\<}{\langle}
\renewcommand{\>}{\rangle}
\newcommand{\isom}{\cong}
\newcommand{\eqc}{\overline}
\newcommand{\clo}{\overline}
\newcommand{\seq}{\subseteq}
\newcommand{\teq}{\trianglelefteq}
\DeclareMathOperator{\id}{id}
\DeclareMathOperator{\im}{im}

% Extra Commands
\newcommand{\inc}{\hookrightarrow}
\newcommand{\htpy}{\simeq}
\newcommand{\bd}{\partial}

% Document
\begin{document}
\title{MATH 221B Homework 3}
\author{Harry Coleman}
\date{January 31, 2022}
\maketitle


\begin{pbox}[1 Exercise 0.15]
    Enumerate all the subcomplexes of $S^\infty$, with the cell structure described in this section, having two cells in each dimension.
\end{pbox}

\begin{pbox}[2 Exercise 0.16]
    Show that $S^\infty$ is contractible.
\end{pbox}

\begin{pbox}[3 Exercise 0.18]
    Show that $S^1 * S^1 = S^3$, and more generally $S^n * S^m = S^{n+m+1}$.
\end{pbox}

\begin{pbox}[4 Exercise 0.19]
    Show that the space obtained from $S^2$ by attaching $n$ $2$-cells along any collection of $n$ circles in $S^2$ is homotopy equivalent to the wedge sum of $n + 1$ $2$-spheres.
\end{pbox}

\begin{pbox}[5 Exercise 0.20]
    Show that the subspace $X \seq \R^3$ formed by a klein bottle intersecting itself in a circle is homotopy equivalent to $S^1 \vee S^1 \vee S^2$.
\end{pbox}

\begin{pbox}[6 Exercise 0.23]
    Show that a CW complex is contractible if it is the union of two contractible subcomplexes whose intersection is also contractible.
\end{pbox}

\begin{pbox}[7 Exercise 0.24]
    Let $X$ and $Y$ be CW complexes with $0$-cells $x_0$ and $y_0$.
    Show that the quotient spaces $X * Y / (X * \{y_0\} \cup \{x_0\} * Y)$ and $S(X \wedge Y) / S(\{x_0\} \wedge \{y_0\})$ are homeomorphic, and deduce that $X * Y \htpy S(X \wedge Y)$.
\end{pbox}

\begin{pbox}[8 Exercise 0.25]
    If $X$ is a CW complex with components $X_\alpha$, show that the suspension $SX$ is homotopy equivalent to $Y \bigvee_\alpha SX_\alpha$ for some graph $Y$.
    In the case that $X$ is a finite graph, show that $SX$ is homotopy equivalent to a wedge sum of circles and $2$-spheres.
\end{pbox}


\end{document}
\documentclass[12pt]{article}

% Packages
\usepackage[margin=1in]{geometry}
\usepackage{fancyhdr, parskip}
\usepackage{amsmath, amsthm, amssymb}
\usepackage{tikz, tikz-cd}

% Page Style
\makeatletter
\fancypagestyle{title}{
    \renewcommand{\headrulewidth}{0.4pt}
    \setlength{\headheight}{15pt}
    \fancyhead[R]{\@author}
    \fancyhead[L]{\@title}
    \fancyhead[C]{\@date}
}
\makeatother
\renewcommand{\maketitle}{\thispagestyle{title}}
\fancypagestyle{plain}{
    \fancyhf{}
    \renewcommand{\headrulewidth}{0pt}
    \renewcommand{\footrulewidth}{0pt}
    \fancyfoot[R]{\thepage}
}
\pagestyle{plain}

% Problem Box
\setlength{\fboxsep}{4pt}
\newlength{\myparskip}
\setlength{\myparskip}{\parskip}
\newsavebox{\savefullbox}
\newenvironment{fullbox}{\begin{lrbox}{\savefullbox}\begin{minipage}{\dimexpr\textwidth-2\fboxsep\relax}\setlength{\parskip}{\myparskip}}{\end{minipage}\end{lrbox}\framebox[\textwidth]{\usebox{\savefullbox}}}
\newenvironment{pbox}[1][]{\begin{fullbox}\ifx#1\empty\else\paragraph{#1}\phantom{}\fi}{\end{fullbox}}

% Theorem Environments
\theoremstyle{definition}
\newtheorem{lemma}{Lemma}

% Tikz Environments
\newenvironment{drawing}{\begin{center}\begin{tikzpicture}}{\end{tikzpicture}\end{center}}
% \tikzcdset{row sep/normal=0pt}
\newenvironment{cd}{\begin{center}\begin{tikzcd}}{\end{tikzcd}\end{center}}

% Default Commands
\newcommand{\isp}[1]{\quad\text{#1}\quad}
\newcommand{\N}{\mathbb{N}} 
\newcommand{\Z}{\mathbb{Z}}
\newcommand{\Q}{\mathbb{Q}}
\newcommand{\R}{\mathbb{R}}
\newcommand{\C}{\mathbb{C}}
\newcommand{\A}{\mathbb{A}}
\renewcommand{\P}{\mathbb{P}}
\newcommand{\eps}{\varepsilon}
\renewcommand{\phi}{\varphi}
\renewcommand{\emptyset}{\varnothing}
\newcommand{\<}{\langle}
\renewcommand{\>}{\rangle}
\newcommand{\isom}{\cong}
\newcommand{\eqc}{\overline}
\newcommand{\clo}{\overline}
\newcommand{\seq}{\subseteq}
\newcommand{\teq}{\trianglelefteq}
\DeclareMathOperator{\id}{id}
\DeclareMathOperator{\im}{im}

% Extra Commands
\newcommand{\inc}{\hookrightarrow}
\newcommand{\htpy}{\simeq}
\newcommand{\bd}{\partial}

\DeclareMathOperator{\const}{const}

% Document
\begin{document}
\title{MATH 221B Homework 3}
\author{Harry Coleman\makebox[0pt][r]{\raisebox{-0.25in}[0pt][0pt]{(worked with Joseph Sullivan, Gahl Shemy)}}}
\date{February 7, 2022}
\maketitle


\begin{pbox}[1 Exercise 0.15]
    Enumerate all the subcomplexes of $S^\infty$, with the cell structure described in this section, having two cells in each dimension.
\end{pbox}

For each $n$, the $n$-skeleton of $S^\infty$ is precisely the $n$-sphere $S^n$, which is an $n$-dimensional subcomplex.

Each $S^n$ is constructed out of an equatorial $(n-1)$ sphere $S^{n-1}$ and two $n$-cells, which we will denote $e^n_+$ and $e^n_-$, whose boundaries are glued to the $(n-1)$-skeleton $S^{n-1}$.
In particular, if a subcomplex of $S^\infty$ contains either $n$-cell, say $e^n_+$, then the subcomplex must also contain all of $S^{n-1}$.
However, if the subcomplex does not contain the other $n$-cell $e^n_-$, then it could not contain the $(n+1)$-cells $e^{n+1}_\pm$, since the boundaries of both must be glued to the entire $n$-skeleton.
Hence, we obtain two more $n$-dimensional subcomplexes: $D^n_\pm = e^n_\pm \cup S^{n-1}$.

Thus, for each $n$, there are three $n$-dimensional subcomplexes of $S^\infty$, namely $S^n$ and $D^n_\pm$.

 
\begin{pbox}[2 Exercise 0.16]
    Show that $S^\infty$ is contractible.
\end{pbox}

\begin{proof}
    Consider the space $R^\infty = \bigcup_n \R^n$, whose points are sequences in $\R$ with finite support, and has the final topology with respect to the inclusions $\R^n \inc \R^\infty$.
    Note that $\R^\infty$ is a vector space, and we consider the norm
    \[\textstyle
        \|x\| = \sqrt{\sum_n x_n^2}, \qquad x \in \R^\infty,
    \]
    where the sum is well-defined with only finitely many nonzero terms.

    Then $S^\infty$ is the subspace
    \[\textstyle
        S^\infty = \{x \in \R^\infty \mid \|x\| = 1\}.
    \]
    For each $x \in S^\infty$, we define $\phi(x) \in S^\infty$ to be the point with $\phi(x)_1 = 0$ and $\phi(x)_n = x_{n-1}$ for $n > 1$.
    That is, $\phi$ is the map which shifts the components to the right by one index:
    \begin{align*}
        \phi : S^\infty &\longrightarrow S^\infty \\
            (x_1, x_2, \dots) &\longmapsto (0, x_1, x_2, \dots).
    \end{align*}
    We now define a homotopy $F_t : S^\infty \to S^\infty$ by
    \[
        F_t(x) = \frac{(1-t)x + t\phi(x)}{\|(1-t)x + t\phi(x)\|}.
    \]
    This is the straight-line homotopy between $\id_{S^\infty}$ and $\phi$, inside of $\R^\infty$, composed with the normalization map $\R^\infty \setminus \{0\} \to S^\infty$.
    We conclude that $\id_{\infty} = F_0 \htpy F_1 = \phi$.

    Consider the point $e_1 = (1, 0, \dots) \in S^\infty$.
    We now define the homotopy $G_t : S^\infty \to S^\infty$ by
    \[
        G_t(x) = \sin(\tfrac{\pi}{2}t)e_1 + \cos(\tfrac{\pi}{2}t)\phi(x).
    \]
    It is immediate that this is a continuous map $S^\infty \to \R^\infty$, and to see that the image is in $S^\infty$, note that $e_1 \perp \phi(x)$ so
    \[
        \|G_t(x)\|
            = \sqrt{\sin^2(\tfrac{\pi}{2}t)\|e_1\| + \cos^2(\tfrac{\pi}{2}t)\|\phi(x)\|}
            = \sqrt{\sin^2(\tfrac{\pi}{2}t) + \cos^2(\tfrac{\pi}{2}t)}.
            = 1.
    \]
    Hence, $\phi = G_0 \htpy G_1 = \const_{e_1}$.
    Composing $F$ and $G$, we deduce $\id_{S^\infty} \htpy \phi \htpy \const_{e_1}$, hence $S^\infty$ is contractible.
\end{proof}




\newpage
\begin{pbox}[3 Exercise 0.18]
    Show that $S^1 * S^1 = S^3$, and more generally $S^n * S^m = S^{n+m+1}$.
\end{pbox}

Applying Problem 7 (Exercise 0.24), we have
\[
    S^n * S^m \htpy S(S^n \wedge S^m).
\]
It is stated in Hatcher that $S^n \wedge S^m \isom S^{n + m}$, so
\[
    S^n * S^m \htpy S(S^{n + m}) \isom S^{n + m + 1}.
\]


\begin{pbox}[4 Exercise 0.19]
    Show that the space obtained from $S^2$ by attaching $n$ $2$-cells along any collection of $n$ circles in $S^2$ is homotopy equivalent to the wedge sum of $n + 1$ $2$-spheres.
\end{pbox}

\begin{proof}
    Enumerate the $2$-cells by $e^2_i$ for $i = 1, \dots, n$.
    Then the circles in $S^2$ are described by attaching maps $f_i : S^1 \to S^2$.
    As $S^2$ is contractible, we know that $f_i \htpy \const_1$, where $1 \in S^2$ is some fixed point.
    We now take $X$ to be the cell complex of $n$ disjoint discs with an attaching map $F : X^1 \to S^2$ described by $f_i$ on the $i$th circle, so $F \htpy \const_1$.
    Then by Proposition 0.18, we have
    \[\textstyle
        X \sqcup_F S^2 \htpy X \sqcup_{\const_1} S^2 = (\bigvee_{i=1}^n S^2) \vee S^2 = \bigvee_{i=0}^n S^2.
    \]
\end{proof}

\begin{pbox}[5 Exercise 0.20]
    Show that the subspace $X \seq \R^3$ formed by a klein bottle intersecting itself in a circle is homotopy equivalent to $S^1 \vee S^1 \vee S^2$.
\end{pbox}


\begin{drawing}[scale=0.8]
    \begin{scope}
        \draw (-0.8, 0) to[out=-10, in=190] (0.8, 0);

        \draw (-0.8, 0) to[out=170, in=-90]
            (-1, 0.2) -- 
            (-1, 1) to[out=90, in=-135]
            (-0.5, 2) to[out=45, in=-90]
            (-0.25, 2.5) to[out=90, in=180]
            (0.75, 3.4) to[out=0, in=45, looseness=1.3]
            (0.5, 1);
        
        \draw[dashed] (0.5, 1) to [out=-135, in=170, looseness=1.3] (0.8, 0);

        \draw (0.8, 0) to[out=10, in=-90]
            (1, 0.2) --
            (1, 1) to[out=90, in=-45]
            (0.5, 2) to[out=135, in=-90]
            (0.25, 2.5) to[out=90, in=180]
            (0.75, 3) to[out=0, in=45]
            (0.25, 1.25);

        \draw[dashed] (0.25, 1.25) to [out=-135, in=10, looseness=1.3] (-0.8, 0);

        \draw (0.5, 1) to[out=180, in=-90] (0.25, 1.25);
        \draw[dashed] (0.5, 1) to[out=90, in=0] (0.25, 1.25);
    \end{scope}
    \begin{scope}[xshift=2cm]
        \node at (0, 1.5) {$\htpy$};
    \end{scope}
    \begin{scope}[xshift=4cm]
        \draw (-0.8, 0) to[out=-10, in=190] (0.8, 0);

        \coordinate (A) at (0.325, 1.125);

        \filldraw (A) circle (2pt);

        \draw (-0.8, 0) to[out=170, in=-90]
            (-1, 0.2) -- 
            (-1, 1) to[out=90, in=-135]
            (-0.5, 2) to[out=45, in=-90]
            (-0.25, 2.5) to[out=90, in=180]
            (0.75, 3.4) to[out=0, in=45, looseness=1.3]
            (A);
        
        \draw[dashed] (A) to [out=-135, in=170, looseness=1.3] (0.8, 0);

        \draw (0.8, 0) to[out=10, in=-90]
            (1, 0.2) --
            (1, 1) to[out=90, in=-45]
            (0.5, 2) to[out=135, in=-90]
            (0.25, 2.5) to[out=90, in=180]
            (0.75, 3) to[out=0, in=45]
            (A);

        \draw[dashed] (A) to [out=-135, in=10, looseness=1.3] (-0.8, 0);
    \end{scope}
    \begin{scope}[xshift=6cm]
        \node at (0, 1.5) {$\htpy$};
    \end{scope}
    \begin{scope}[xshift=8cm]
        \draw (-0.8, 0) to[out=-10, in=190] (0.8, 0);

        \coordinate (A) at (0.325, 1.125);

        \filldraw (A) circle (2pt);

        \draw (-0.8, 0) to[out=170, in=-90]
            (-1, 0.2) -- 
            (-1, 1) to[out=90, in=-135]
            (-0.5, 2) to[out=45, in=-90]
            (-0.25, 2.5) to[out=90, in=180]
            (0.75, 3.2);
        
        \filldraw (0.75, 3.2) circle (2pt);
        \draw (0.75, 3.2) to[out=0, in=45] (A);
        

        \draw (0.8, 0) to[out=10, in=-90]
            (1, 0.2) --
            (1, 1) to[out=90, in=-45]
            (0.5, 2) to[out=135, in=-90]
            (0.25, 2.5) to[out=90, in=180]
            (0.75, 3.2);

        \draw[dashed] (A) to [out=-135, in=90] (0, 0);
        \filldraw (0, -0.06) circle (2pt);
    \end{scope}
    \begin{scope}[xshift=10cm]
        \node at (0, 1.5) {$\isom$};
    \end{scope}
    \begin{scope}[xshift=12cm, yshift=1.5cm]
        \draw (0, 0) circle (1cm);
        \draw (-1, 0) to[out=-45, in=-135, looseness=0.9] (1, 0);
        \draw[dashed] (-1, 0) to[out=45, in=135, looseness=0.9] (1, 0);
        
        \filldraw (0, 1) circle (2pt);
        \filldraw (1, 0) circle (2pt);
        \filldraw (0, -1) circle (2pt);

        \draw (0, 1) to[out=45, in=45, looseness=2] (1, 0);
        \draw[dashed] (0, -1) to[out=90, in=180] (1, 0);
    \end{scope}
    \begin{scope}[xshift=14cm]
        \node at (0, 1.5) {$\htpy$};
    \end{scope}
    \begin{scope}[xshift=16cm, yshift=1.5cm]
        \draw (0, 0) circle (1cm);
        \draw (-1, 0) to[out=-45, in=-135, looseness=0.9] (1, 0);
        \draw[dashed] (-1, 0) to[out=45, in=135, looseness=0.9] (1, 0);
        
        \filldraw (1, 0) circle (2pt);

        \draw (1, 0.01) to[out=0, in=90, looseness=400] (1, 0);
        \draw (1, 0.01) to[out=0, in=-90, looseness=400] (1, 0);
    \end{scope}
\end{drawing}


\newpage
\begin{pbox}[6 Exercise 0.23]
    Show that a CW complex is contractible if it is the union of two contractible subcomplexes whose intersection is also contractible.
\end{pbox}

\begin{proof}
    Let $X$ be a CW complex consisting of two contractible subcomplexes $A$ and $B$, whose intersection $Y = A \cap B$ is also contractible.
    As $(X, B)$ is a CW pair with $B$ contractible, Proposition 0.17 tells us $X \htpy X/B$.

    We claim that $X/B \isom A/Y$.
    As sets,
    \[
        X/B = (X \setminus B) \sqcup \{b\} = (A \setminus B) \sqcup \{b\}
    \]
    and
    \[
        A/Y = (A \setminus Y) \sqcup \{y\} = (A \setminus B) \sqcup \{y\}.
    \]
    So there is an obvious bijection $f : X/B \to A/Y$, which is the identity on $A \setminus B$ and maps the point $b \mapsto y$. 
    We claim that $f$ is a homeomorphism.
    
    The topology on $X/B$ is induced by the quotient map $\pi_{X/B} : X \to X/B$ and the topology on $A/Y$ is induced by the quotient map $\pi_{A/Y} : A \to A/Y$.

    Suppose $U \seq A/Y$ is an open set, which means $\pi_{A/Y}^{-1}(U) \seq A$ is open in $A$.

    If $y \notin U$, then $f^{-1}(U) = U = \pi_{A/B}^{-1}(U)$ is an open subset of $A \setminus Y = X \setminus B$.
    Since $B \seq X$ is closed, we know $X \setminus B$ is an open subspace of $X$, which means means $f^{-1}(U)$ is an open subset of $X$.

    If $y \in U$, then $Y$ is contained in the open subset $\pi_{A/Y}^{-1}(U) \seq A$.
    Since $A$ is a subspace of $X$, we know that $\pi_{A/Y}^{-1}(U) = V \cap A$ for some open set $V \seq X$.
    On the other hand, $b \in f^{-1}(U)$, so we can write
    \[
        \pi_{X/B}^{-1}(f^{-1}(U)) 
            = \pi_{A/Y}^{-1}(U) \cup (X \setminus A)
            = (V \cap A) \cup (X \setminus A)
            = V \cup (X \setminus A).
    \]
    With $A \seq X$ closed and $V \seq X$ open, we conclude that $\pi_{X/B}^{-1}(f^{-1}(U)) \seq X$ is open. 
    Therefore, $f^{-1}(U) \seq X/B$ is open, hence $f$ is continuous.

    It is similar to check that $f$ is an open map, so $f$ is a homeomorphism $X/B \isom A/Y$.

    Since $(A, Y)$ is a CW pair with $Y$ contractible, Proposition 0.17 allows us to conclude that
    \[
        X \htpy X/B \isom A/Y \htpy A.
    \]
    With $A$ contractible, it follows that $X$ is also contractible.


\end{proof}


\newpage
\begin{pbox}[7 Exercise 0.24]
    Let $X$ and $Y$ be CW complexes with $0$-cells $x_0$ and $y_0$.
    Show that the quotient spaces $X * Y / (X * \{y_0\} \cup \{x_0\} * Y)$ and $S(X \wedge Y) / S(\{x_0\} \wedge \{y_0\})$ are homeomorphic, and deduce that $X * Y \htpy S(X \wedge Y)$.
\end{pbox}

\begin{proof}
    We can express $X * Y / (X * \{y_0\} \cup \{x_0\} * Y)$ as the quotient $(X \times Y \times I)/{\sim_a}$ for some relation $\sim_a$.
    This relation is generated by the following equivalences:
    \begin{enumerate}
        \item[(a1)] $(x, y_1, 0) \sim_a (x, y_2, 0)$,
        \item[(a2)] $(x_1, y, 1) \sim_a (x_2, y, 1)$,
        \item[(a3)] $(x, y_0, t_1) \sim_a (x_0, y, t_2)$.
    \end{enumerate}
    The equivalences (a1) and (a2) come from the join $X * Y$, and (a3) comes from the quotient by $X * \{y_0\} \cup \{x_0\} * Y$.

    In a similar fashion, we can express $S(X \wedge Y) / S(\{x_0\} \wedge \{y_0\})$ as the quotient $(X \times Y \times I)/{\sim_b}$, where $\sim_b$ is the relation generated by the following equivalences:
    \begin{enumerate}
        \item [(b1)] $(x, y_0, t) \sim_b (x_0, y, t)$,
        \item [(b2)] $(x_1, y_1, 0) \sim_b (x_2, y_2, 0)$,
        \item [(b3)] $(x_1, y_1, 1) \sim_b (x_2, y_2, 1)$,
        \item [(b4)] $(x_0, y_0, t_1) \sim_b (x_0, y_0, t_2)$.
    \end{enumerate}
    The equivalence (b1) comes from the smash product $X \wedge Y$ involving the quotient by the wedge product $X \vee Y$.
    Equivalences (b2) and (b3) come from taking the suspension $S(X \wedge Y)$, and (b4) is the quotient by the suspension $S(\{x_0\} \wedge \{y_0\})$.

    We claim that $\sim_a$ and $\sim_b$ in fact define the same equivalence on $X \times Y \times I$.

    We first show that we can derive the equivalences of (b1-4) in terms of $\sim_a$:
    \begin{enumerate}
        \item[(b1)] $(x, y_0, t) \overset{(a3)}{\sim_a} (x_0, y, t)$,
        \item[(b2)] $(x_1, y_1, 0) \overset{(a1)}{\sim_a} (x_1, y_0, 0) \overset{(a3)}{\sim_a} (x_0, y_0, 0) \overset{(a3)}{\sim_a} (x_2, y_0, 0) \overset{(a1)}{\sim_a} (x_2, y_2, 0)$,
        \item[(b3)] $(x_1, y_1, 1) \overset{(a2)}{\sim_a} (x_0, y_1, 1) \overset{(a3)}{\sim_a} (x_0, y_0, 1) \overset{(a3)}{\sim_a} (x_0, y_2, 1) \overset{(a2)}{\sim_a} (x_2, y_2, 1)$,
        \item[(b4)] $(x_0, y_0, t_1) \overset{(a3)}{\sim_a} (x_0, y_0, t_2)$. 
    \end{enumerate}

    Next, we show that we can derive the equivalences of (a1-3) in terms of $\sim_b$:
    \begin{enumerate}
        \item[(a1)] $(x, y_1, 0) \overset{(b2)}{\sim_b} (x, y_2, 0)$,
        \item[(a2)] $(x_1, y, 1) \overset{(b3)}{\sim_b} (x_2, y, 1)$,
        \item[(a3)] $(x, y_0, t_1) \overset{(b1)}{\sim_b} (x_0, y_0, t_1) \overset{(b4)}{\sim_b} (x_0, y_0, t_2) \overset{(b1)}{\sim_b} (x_0, y, t_2)$.
    \end{enumerate}

    Hence $\sim_a$ and $\sim_b$ generate the same equivalence relation on $X \times Y \times I$, so there is a natural bijection $(X \times Y \times I)/\sim_a \leftrightarrow (X \times Y \times I)/\sim_b$.
    Moreover, this bijection agrees with the quotient maps from $X \times Y \times I$, so the topology on each is the same.
    Hence, we have a homeomorphism between the spaces.

    Note that all of the involved operations, when applied to CW complexes, again give us CW complexes.
    Then $(X * \{y_0\} \cup \{x_0\} * Y)$ is a subcomplex of $X * Y$ and $S(\{x_0\} \wedge \{y_0\})$ is a subcomplex of $S(X \wedge Y)$.
    Moreover, both are contractible, so Proposition 0.17 gives us
    \[
        X * Y
            \htpy X * Y / (X * \{y_0\} \cup \{x_0\} * Y)
            \isom S(X \wedge Y) / S(\{x_0\} \wedge \{y_0\})
            \htpy S(X \wedge Y).
    \]
\end{proof}


\begin{pbox}[8 Exercise 0.25]
    If $X$ is a CW complex with components $X_\alpha$, show that the suspension $SX$ is homotopy equivalent to $Y \bigvee_\alpha SX_\alpha$ for some graph $Y$.
    In the case that $X$ is a finite graph, show that $SX$ is homotopy equivalent to a wedge sum of circles and $2$-spheres.
\end{pbox}

For each pair of components $X_{\alpha_1}$ and $X_{\alpha_2}$, we can perform the following homotopies on the suspension $S(X_{\alpha_1} \sqcup X_{\alpha_2})$:

\begin{drawing}
    \begin{scope}[scale=0.5]
        \draw[fill=gray!20] (0, 0) ellipse (1cm and 0.5cm);
        \draw[fill=gray!20] (3, 0) ellipse (1cm and 0.5cm);

        \filldraw (160: 1 and 0.5) -- (1.5, 3) circle (2pt) -- ([xshift=3cm] 20: 1 and 0.5);
        \draw (1, 0) -- (1.5, 3) -- (2, 0);

        \filldraw (-160: 1 and 0.5) -- (1.5, -3) circle (2pt) -- ([xshift=3cm] -20: 1 and 0.5);
        \draw (1, 0) -- (1.5, -3) -- (2, 0);
    \end{scope} 
    \begin{scope}[xshift=2.5cm]
        \node at (0, 0) {$\htpy$};
    \end{scope}
    \begin{scope}[scale=0.5, xshift=7cm]
        \draw[fill=gray!20] (0, 0) ellipse (1cm and 0.5cm);
        \draw[fill=gray!20] (3, 0) ellipse (1cm and 0.5cm);

        \filldraw (170: 1 and 0.5) -- (0, 3) circle (2pt) -- (10: 1 and 0.5);
        \filldraw[xshift=3cm] (170: 1 and 0.5) -- (0, 3) circle (2pt) -- (10: 1 and 0.5);
        \draw (0, 3) -- (3, 3);

        \filldraw (-160: 1 and 0.5) -- (1.5, -3) circle (2pt) -- ([xshift=3cm] -20: 1 and 0.5);
        \draw (1, 0) -- (1.5, -3) -- (2, 0);

        \filldraw (1.5, 3) circle (2pt) node[above] {$a$};
    \end{scope} 
    \begin{scope}[xshift=6cm]
        \node at (0, 0) {$\htpy$};
    \end{scope}
    \begin{scope}[scale=0.5, xshift=14cm]
        \draw[fill=gray!20] (0, 0) ellipse (1cm and 0.5cm);
        \draw[fill=gray!20] (3, 0) ellipse (1cm and 0.5cm);

        \filldraw (170: 1 and 0.5) -- (0, 3) circle (2pt) -- (10: 1 and 0.5);
        \filldraw[xshift=3cm] (170: 1 and 0.5) -- (0, 3) circle (2pt) -- (10: 1 and 0.5);

        \filldraw (-160: 1 and 0.5) -- (1.5, -3) circle (2pt) -- ([xshift=3cm] -20: 1 and 0.5);
        \draw (1, 0) -- (1.5, -3) -- (2, 0);

        \draw (1.5, -3.5) circle (0.5cm);
        \filldraw (1.5, -4) circle (2pt) node[below] {$a$};
    \end{scope}
\end{drawing}
This gives us $S(X_{\alpha_1} \sqcup X_{\alpha_2}) \htpy SX_{\alpha_1} \vee SX_{\alpha_2} \vee S^1$.
Performing this sort of operation on more than two components, the second step has a segment from the top of each suspension $SX_\alpha$ to $a$, giving us as many segments from bottom point to $a$ in the third step.
Taking the graph $Y$ to have the $0$-skeleton $\{x_0, a\}$, and a segment from $x_0$ to $a$ for each component $X_\alpha$ in $X$, we obtain $SX \htpy Y \bigvee_\alpha SX_\alpha$.


If $X$ is a finite graph, then for a segment connecting a pair of distinct points in $X^0$, we can obtain a homotopy equivalent space from $X$ by contracting the segment to a point.
That is, we have the following:
\begin{drawing}
    \begin{scope}
        \filldraw (0, 0) circle (2pt) -- (1, 0) circle (2pt);
        
        \foreach \a in {90, 112.5, 135, 225, 247.5, 270} {
            \draw (0, 0) -- (\a : 1);
        }
        \node at (-0.5, 0.1) {$\vdots$};

        \foreach \a in {90, 67.5, 45, -45, -67.5, -90} {
            \draw (1, 0) -- +(\a : 1);
        }
        \node at (1.5, 0.1) {$\vdots$};
    \end{scope}
    \begin{scope}[xshift=3cm]
        \node at (0, 0) {$\htpy$};
    \end{scope}
    \begin{scope}[xshift=5cm]
        \filldraw (0.5, 0) circle (2pt);
        
        \foreach \a in {90, 112.5, 135, 225, 247.5, 270} {
            \draw (0.5, 0) -- (\a : 1);
        }
        \node at (-0.5, 0.1) {$\vdots$};

        \foreach \a in {90, 67.5, 45, -45, -67.5, -90} {
            \draw (0.5, 0) -- ([xshift=1cm]\a : 1);
        }
        \node at (1.5, 0.1) {$\vdots$};
    \end{scope}
\end{drawing}
The the number of cycles (in the graph theory sense) remains constant under this operation.
Repeating this operation until there are no more segments to contract, we obtain each component of $X$ homotopy equivalent to a wedge sum of circles, one circle for each distinct cycle in the component.
The suspension of a wedge sum of circles is a wedge sum of circles, so we obtain $SX \htpy Y \bigvee_\alpha SX_\alpha$, where $Y$, as before, is a wedge sum of circles (contracting one of its segments) and each $SX_\alpha$ is homotopy equivalent to a wedge sum of spheres.



\end{document}
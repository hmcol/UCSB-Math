\documentclass[12pt]{article}

% Packages
\usepackage[margin=1in]{geometry}
\usepackage{fancyhdr, parskip}
\usepackage{amsmath, amsthm, amssymb}

% Page Style
\fancypagestyle{plain}{
    \fancyhf{}
    \renewcommand{\headrulewidth}{0pt}
    \renewcommand{\footrulewidth}{0pt}
    \fancyfoot[R]{\thepage}
}
\pagestyle{plain}

% Problem Box
\setlength{\fboxsep}{4pt}
\newlength{\myparskip}
\setlength{\myparskip}{\parskip}
\newsavebox{\savefullbox}
\newenvironment{fullbox}{\begin{lrbox}{\savefullbox}\begin{minipage}{\dimexpr\textwidth-2\fboxsep\relax}\setlength{\parskip}{\myparskip}}{\end{minipage}\end{lrbox}\framebox[\textwidth]{\usebox{\savefullbox}}}
\newenvironment{pbox}[1][]{\begin{fullbox}\ifx#1\empty\else\paragraph{#1}\fi}{\end{fullbox}}

% Default Commands
\newcommand{\isp}[1]{\quad\text{#1}\quad}
\newcommand{\N}{\mathbb{N}} 
\newcommand{\Z}{\mathbb{Z}}
\newcommand{\Q}{\mathbb{Q}}
\newcommand{\R}{\mathbb{R}}
\newcommand{\C}{\mathbb{C}}
\newcommand{\eps}{\varepsilon}
\renewcommand{\phi}{\varphi}
\renewcommand{\emptyset}{\varnothing}
\newcommand{\<}{\langle}
\renewcommand{\>}{\rangle}
\newcommand{\isom}{\cong}
\newcommand{\eqc}{\overline}
\newcommand{\clo}{\overline}
\newcommand{\teq}{\trianglelefteq}
\DeclareMathOperator{\id}{id}

% Extra Commands


% Document Info
\fancypagestyle{title}{
    \renewcommand{\headrulewidth}{0.4pt}
    \setlength{\headheight}{15pt}
    \fancyhead[R]{Harry Coleman}
    \fancyhead[L]{MATH 221A Homework 2}
    \fancyhead[C]{October 13, 2021}
}

% Begin Document
\begin{document}
\thispagestyle{title}


\begin{pbox}[1]
    A space $X$ is \emph{locally (path-)connected at $x$} if for every neighborhood $N$ of $x$, there is a (path-)connected neighborhood $N' \subseteq N$ of $x$.  It is \emph{locally (path-)connected} if it is locally (path-)connected at every point.
\end{pbox}

\begin{pbox}[(a)]
    Show that a connected, locally path-connected space is path-connected.
\end{pbox}

\begin{proof}
    For any $x \in X$, let $P_x \subseteq X$ be its path-connected component, i.e., $P_x$ is the set of points in $X$ with a path to $x$. 
    
    We claim that $P_x$ is an open subset of $X$, and will prove this directly. Let $y \in P_x$, so there is a path between $x$ and $y$. Since $X$ is locally path-connected, there is some path-connected neighborhood $N$ of $y$. For each $z \in N$, there is a path between $y$ and $z$. We can then join the $x$-$y$ path and the $y$-$z$ path to obtain a path between $x$ and $z$. This means that $z \in P_x$, hence $N \subseteq P_x$. This shows that $P_x \subseteq X$ is open for every $x \in X$.

    Let $x \in X$; we claim that $P_x = X$. Assume not, then the set
    \[
        U = \bigcup_{y \in X \setminus P_x} P_y
    \]
    is nonempty. As the union of open sets, $U \subseteq X$ is open. Moreover, if $y \in X \setminus P_x$, then $P_y$ must be disjoint from $P_x$; otherwise, there would be a path between $x$ and $y$, constructed out of paths between each and a point in the intersection. This implies that $U$ is disjoint from $P_x$. We now have $X = P_x \sqcup U$, where $P_x$ and $U$ are open. In other words, $X$ is the disjoint union of nonempty open subsets, which is a contradiction. Hence, $X$ is path-connected.

\end{proof}

\newpage
\begin{pbox}[(b)]
    Construct a subspace of $\mathbb{R}^2$ which is path-connected, but not locally connected.
\end{pbox}

Define $L_0 = [0, 1] \times \{0\} \subseteq \R^2$. For each $n \in \N$, let $L_n \subseteq \R^2$ be the line segment between the origin and the point $(1, 1/n) \in \R^2$. Define $X = \bigcup_{n = 0}^{\infty} L_n$.

Each line segment $L_n$ is trivially path-connected and each contains the origin. Hence, $X$ is path-connected as any two points have paths to the origin and, therefore, to each other.

For any $n \in \N$, we can split $X$ along the line $y = 2x/(2n + 1)$. That is, define the open halfspaces
\[
    H_n = \{(x, y) \in \R^2 : y > 2x/(2n + 1)\}
\]
and
\[
    K_n = \{(x, y) \in \R^n : y < 2x/(2n + 1)\}.
\]
Then $X = \{0, 0\} \sqcup (H_n \cap X) \sqcup (K_n \cap X)$, where $H_n \cap X$ and $K_n \cap X$ are open in the subspace topology on $X \subseteq \R^2$.

We will show that $X$ is not locally connected. Consider the point $x = (1, 0) \in X$, and any neighborhood $N \subseteq B_1(x) \cap X$ of $x$. Note that $N$ does not contain the origin, so $N$ can be written as the disjoint union of open subsets $N = (H_n \cap N) \sqcup (K_n \cap N)$. There is some $\eps > 0$ such that $B_\eps(x) \cap X \subseteq N$, and some $n \in \N$ such that $1/n < \eps$. Then $L_n \cap B_\eps(x) \subseteq H_n \cap N$ is nonempty, as it contains the point $(1, 1/n)$. Moreover, $K_n \cap N$ is nonempty, as it contains $x = (1, 0)$. Hence, $N$ is the disjoint union of nonempty open subsets, i.e., disconnected.



\newpage
\begin{pbox}[2 \\ (a)]
    Show that every countable metric space is totally disconnected.
\end{pbox}

\begin{proof}
    Let $(X, d)$ be a countable metric space.

    Suppose $x, y \in X$ are distinct points, and let $R = d(x, y) > 0$.
    
    We claim that there is some radius $r \in \R$ such that $0 < r < R$ and no $z \in X$ has $d(x, z) = r$. If there were some $z \in X$ with $d(x, z) = r$ for every $r \in (0, R)$, then we could construct an injection $(0, R) \to X$, using the axiom of choice, with $r \mapsto z$. This would imply that $X$ is uncountable, as the real interval $(0, R)$ is uncountable.

    Let $r \in (0, R)$ such that $d(x, z) \ne r$ for all $z \in X$. Define the disjoint sets
    \[
        U = \{z \in X : d(x, z) < r\}
        \isp{and}
        V = \{z \in X : d(x, z) > r\}.
    \]
    By choice of $r$, we know $U$ and $V$ cover $X$. Moreover, $x \in U$ and $y \in U$, so both are nonempty. Lastly, $U = B_r(x)$ is open, implying that $V = X \setminus \clo{U}$ is also open. Hence, $X$ is the disjoint union of the nonempty open subsets $U$ and $V$.

    If $C \subseteq X$ is the connected component containing $x$, then we can write $C = (U \cap C) \sqcup (V \cap C)$, where $U \cap C$ and $V \cap C$ are open in the subspace topology on $C$. Since $U \cap C$ contains $x$, it is nonempty. Since $C$ is connected, we must have $V \cap C$ be empty; in particular, $y \notin C$.

    We have shown that $x, y \in X$ are in the same connected component of $X$ if and only if $x = y$, which means that every connected component is a singleton. Hence, $X$ is totally disconnected.

\end{proof}

\begin{pbox}[(b)]
    Show that if $A \subset \mathbb R^2$ is countable, then $\mathbb R^2 \setminus A$ is path-connected.
\end{pbox}

\begin{proof}
    Denote $X = \R^2 \setminus A$.

    Let $x \in X$ and define $L$ to be the collection of lines in $\R^2$ passing through the point $x$. In other words, the sets in $L$ correspond to the $1$-dimensional subspaces of the real vector space $\R^2$, shifted by the vector $x \in \R^2$. Note that $L$ is at least uncountable, since any choice of slope in $\R$ corresponds to a line through $x$ with that slope. Moreover, any two distinct lines in $L$ intersect only at $x$. 

    We claim that there is some line $\ell \in L$ such that $\ell \cap A = \emptyset$. If there were some $a \in \ell \cap A$ for every $\ell \in L$, the we could construct an injection $L \to A$ by $\ell \mapsto a$, using the axiom of choice. This would imply that $A$ is uncountable, as $L$ is uncountable. Let $\ell$ be a line in $\R^2$ passing through $x$ and disjoint from $A$.

    For any other $y \in X$, we use the same argument to construct a line $\ell'$ in $\R^2$ through $y$ and disjoint from $A$. Without loss of generality, we can assume $\ell'$ is not parallel to $\ell$; otherwise we repeat the argument excluding this line, as there are still uncountably many lines through $y$ not parallel to $\ell$.

    We now have $x, y \in \ell \cup \ell' \subseteq X$. Since $\ell$ and $\ell'$ are not parallel, they intersect at some point $z \in \R^2$. Then $X$ contains a line segment between $x$ and $z$, and a line segment between $z$ and $y$, which join to give a path in $X$ between $x$ and $y$. Hence, $X$ is path-connected.


\end{proof}



\newpage
\begin{pbox}[3]
    Let $X$ be a metric space and $A$ a subset.  For a point $x \in X$, we
      define
      \[d(x,A)=\inf_{a \in A} d(x,a).\]
\end{pbox}

\begin{pbox}[(a)]
    Show that $d(x,A)=0$ if and only if $x \in \bar A$.
\end{pbox}

\begin{proof}
    Suppose $d(x, A) = 0$, so for every $\eps > 0$, there is some $a \in A$ such that $d(x, a) < \eps$. This implies $a \in B_\eps(x)$, i.e., $B_\eps(x) \cap A \ne \emptyset$. Since this holds for all $\eps > 0$ and every neighborhood of $x$ contains a ball of some positive radius, then every neighborhood of $x$ has a nonempty intersection with $A$. Hence, $x \in \clo{A}$.

    Suppose $x \in \clo{A}$. For every $\eps > 0$, the open ball $B_\eps(x)$ is a neighborhood of $x$, so $B_\eps \cap (A) \ne \emptyset$. That is, there is some $a \in A$ such that $d(x, a) < \eps$, implying $d(x, A) \leq \eps$. Since this holds for all $\eps > 0$, we in fact have $d(x, A) = 0$.

\end{proof}

\begin{pbox}[(b)]
    Show that if $A$ is compact, $d(x,A)=d(x,a)$ for some $a \in A$.
\end{pbox}

\begin{proof}
    For each $n \in \N$, there is some $a_n \in A$ such that $d(x, a_n) < d(x, A) + 1/n$, which means that $d(x, a_n) \to d(x, A)$. Then $\{a_n\}$ is a sequence in the compact set $A$, so there is a convergent subsequence $a_{n_k} \to a \in A$. Since $d(x, -)$ is a continuous function, we have convergence $d(x, a_{n_k}) \to d(x, a) = d(x, A)$.
    
\end{proof}

\begin{pbox}[(c)]
    Define the \emph{$\eps$-neighborhood} of $A$ in $X$ to be the set
        \[U(A,\eps)=\{x : d(x,A)<\eps\}.\]
        Show that $U(A,\eps)$ is the union of the open balls $B_d(a,\eps)$ for
        $a \in A$.
\end{pbox}

\begin{proof}
    Given $x \in U(A, \eps)$, we have $d(x, A) < \eps$. That is, there is some $\delta > 0$ such that $d(x, A) = \eps - \delta$. Then, for all $\eta > 0$, there is some $a \in A$ such that $d(x, a) < (\eps - \delta) + \eta$. In particular, take $\eta = \delta$, so there is some $a \in A$ such that $d(x, a) < (\eps - \delta) + \delta = \eps$. In other words, $x \in B_\eps(a)$.

    If $x \in B_\eps(a)$ for some $a \in A$, then $d(x, A) \leq d(x, a) < \eps$, implying that $x \in U(A, \eps)$.e

\end{proof}

\newpage
\begin{pbox}[(d)]
    If $A$ is compact and $U$ is an open set containing $A$, show that some
        $\eps$-neighborhood of $A$ is contained in $U$.
\end{pbox}

\begin{proof}
    Assume, for contradiction, that $U(A, \eps) \cap U^c \ne \emptyset$ for all $\eps > 0$. Then, for every $n \in \N$, there is some $x_n \in U^c$ such that $d(x_n, A) < \frac{1}{n}$. Moreover, since $A$ is compact, there is some $a_n \in A$ such that $d(x_n, a_n) = d(x_n, A)$. Then, $\{a_n\}$ is a sequence in the compact set $A$, so there is a convergence subsequence $a_{n_k} \to a \in A$. The triangle inequality gives us
    \[
        d(x_{n_k}, a)
            \leq d(x_{n_k}, a_{n_k}) + d(a_{n_k}, a),
    \] 
    which converges to zero as $k \to \infty$. This means that for every $\eps > 0$, there is some $k \in \N$ such that $x_{n_k} \in B_{\eps}(a)$, i.e., $B_\eps(a) \cap U^c \ne \emptyset$. However, this contradicts the fact that $a \in A \subseteq U$ and $U$ open implies that there is some $\eps > 0$ such that $B_\eps(x) \subseteq U$. 

\end{proof}

\begin{pbox}[(e)]
    Show that the result of (d) need not hold if $A$ is closed but not
        compact.
\end{pbox}

Consider $X = \R^2$ and let $A$ be the $x$-axis, a closed set. Define
\[
    U = \{(x, y) : x > 0 \text{ and } y < 1/x\}
\]

Then for any $\eps > 0$, $U(A, \eps)$ contains the points $(x, \eps)$ for all $x > 0$. However, $1/\eps \geq 1/\eps$, which implies that the point $(1/\eps, 1/\eps)$ is in $U(A, \eps)$ but not in $U$.

\newpage
\begin{pbox}[4]
    Given a nice enough path-connected subset $X \subseteq \mathbb R^n$, we
      can define the \emph{intrinsic distance} between two points of $X$ by
      \[d(x,y)=\inf \bigl\{{\textstyle\int_0^1} \lvert\dot\gamma(t)\rvert dt : \gamma\text{ is an almost everywhere differentiable path from $x$ to }y\bigr\}.\]
      Give an example of an $X \subset \mathbb R^2$ which is compact and
      path-connected, yet the intrinsic distance takes every positive real value
      (and hence is infinite for some pairs of points).
    
      The moral is that a path being ``infinitely long'' is not a topological
      property and does not prevent you from being path-connected!
    
      \textbf{Optional:} Is there an ambient self-homeomorphism of $\mathbb R^2$ that takes your infinite path to a finite one?  Show that such a homeomorphism can't be differentiable.
\end{pbox}

Take the topologist's sine curve for $x \in (0, 1]$, including the segment from $-1$ to $1$ on the $y$-axis. This is a closed bounded subset of $\R^2$ and, therefore, compact. Add a path from the rightmost point of the curve, looping around the entire curve, and connecting to the origin on the left. This is, again, a compact set.

Moreover, it is path-connected. However, no path can pass from $x > 0$ to $x = 0$ through the sine curve; it must move through the added loop. Then for very small values of $x > 0$, the minimum length of a path from the point on the curve with that value of $x$ to the origin is arbitrarily large, so the intrinsic distance attains every positive real.

(Note: I'm not sure how this would imply infinite intrinsic distance for some pairs of points, because it seems like every pair of points has some finite path, utilizing the added loop.)


\end{document}
\documentclass[12pt]{article}

% Packages
\usepackage[margin=1in]{geometry}
\usepackage{fancyhdr, parskip}
\usepackage{amsmath, amsthm, amssymb}

% Page Style
\fancypagestyle{plain}{
    \fancyhf{}
    \renewcommand{\headrulewidth}{0pt}
    \renewcommand{\footrulewidth}{0pt}
    \fancyfoot[R]{\thepage}
}
\pagestyle{plain}

% Problem Box
\setlength{\fboxsep}{4pt}
\newlength{\myparskip}
\setlength{\myparskip}{\parskip}
\newsavebox{\savefullbox}
\newenvironment{fullbox}{\begin{lrbox}{\savefullbox}\begin{minipage}{\dimexpr\textwidth-2\fboxsep\relax}\setlength{\parskip}{\myparskip}}{\end{minipage}\end{lrbox}\framebox[\textwidth]{\usebox{\savefullbox}}}
\newenvironment{pbox}[1][]{\begin{fullbox}\ifx#1\empty\else\paragraph{#1}\fi}{\end{fullbox}}

% Theorem Environments
\theoremstyle{definition}
%\newtheorem{proposition}{Proposition}
\newtheorem{lemma}{Lemma}

% Options
%\allowdisplaybreaks
%\addtolength{\jot}{4pt}

% Default Commands
\newcommand{\isp}[1]{\quad\text{#1}\quad}
\newcommand{\N}{\mathbb{N}} 
\newcommand{\Z}{\mathbb{Z}}
\newcommand{\Q}{\mathbb{Q}}
\newcommand{\R}{\mathbb{R}}
\newcommand{\C}{\mathbb{C}}
\newcommand{\eps}{\varepsilon}
\renewcommand{\phi}{\varphi}
\renewcommand{\emptyset}{\varnothing}
\newcommand{\<}{\langle}
\renewcommand{\>}{\rangle}
\newcommand{\isom}{\cong}
\newcommand{\eqc}{\overline}
\newcommand{\clo}{\overline}

% Extra Commands
\DeclareMathOperator{\inter}{int}
\DeclareMathOperator{\id}{id}

\newcommand*\strong[1]{\textbf{\textit{#1}}}

\newcommand{\inc}{\hookrightarrow}

% Document Info
\fancypagestyle{title}{
    \renewcommand{\headrulewidth}{0.4pt}
    \setlength{\headheight}{15pt}
    \fancyhead[R]{Harry Coleman}
    \fancyhead[L]{MATH 221A Homework 1}
    \fancyhead[C]{October 6, 2021}
}

% Begin Document
\begin{document}
\thispagestyle{title}

\begin{pbox}[1]
    Let $X$ and $Y$ be topological spaces.
\end{pbox}

\begin{pbox}[(a)]
    Show that for any topological space $T$, a function
    $f: T \to X \times Y$ is continuous if and only if the compositions
    $p_X f: T \to X$, $p_Y f: T \to Y$ are continuous.  Here $p_X$ and $p_Y$ are
    the obvious projection maps.
\end{pbox}

\begin{proof}
    Suppose $f$ is continuous. Let $U \subseteq X$ be open, then $p_X^{-1}(U) = U \times Y \subseteq X \times Y$ is open, as it is in the basis for the product topology. And since $f$ is continuous, 
    \[
        (p_Xf)^{-1}(U) = f^{-1}(U \times Y) \subseteq T
    \]
    is open. This shows $p_Xf$ is continuous, and it is the same to show that $p_Yf$ continuous.

    Suppose both $p_Xf$ and $p_Yf$ are continuous. It suffices to prove that the preimages under $f$ of basis elements are open. Let $U \subseteq X$ and $V \subseteq Y$ be open, so $U \times V \subseteq X \times Y$ is an arbitrary basis element of the product topology. We can rewrite $U \times V$ as
    \[
        U \times V = (U \times Y) \cap (X \times V) = p_X^{-1}(U) \cap p_Y^{-1}(V),
    \]
    then
    \[
        f^{-1}(U \times V) = (p_Xf)^{-1}(U) \cap (p_Yf)^{-1}(V).
    \]
    Since $p_Xf$, $p_Yf$ are continuous and $U$, $V$ are open, $f^{-1}(U \times V) \subseteq T$ is is also open, proving that $f$ is continuous.

\end{proof}

\newpage
\begin{pbox}[(b)]
    Let $Z$ be a topological space with maps $g_X:Z \to X$ and
    $g_Y:Z \to Y$.  Suppose that
    \begin{quote}
        for every space $T$ and pair of continuous functions $f_X:T \to X$ and
        $f_Y:T \to Y$, there is a unique continuous function $f:T \to Z$ such that
        $f_X=g_Xf$ and $f_Y=g_Yf$.
    \end{quote}
    Show that $Z$ must be homeomorphic to $X \times Y$ with the product topology
    and $g_X$ and $g_Y$ are (taken by the homeomorphism to) $p_X$ and $p_Y$.
\end{pbox}

\begin{proof}
    Consider $X \times Y$ with the natural projections $p_X$ and $p_Y$. By the stated universal property, there is a unique continuous function $p : X \times Y \to Z$ such that $p_X = g_Xp$ and $p_Y = g_Yp$. We construct a function $g = (g_X, g_Y) : Z \to X \times Y$, where $z \mapsto (g_X(z), g_Y(z))$. We have $p_Xg = g_X$ and $p_Yg = g_Y$, so 1(a) implies $g$ is continuous. We claim that $p$ and $g$ are inverses and, in which case, describe a homeomorphism between $X \times Y$ and $Z$.
    
    First, it can be seen that
    \[
        gp  = (g_Xp,\, g_Yp) = (p_X,\, p_Y) = \id_{X \times Y}.
    \]
    To prove the opposite composition is the identity on $Z$, we construct the continuous functions
    \[
        f_X : Z \xrightarrow{g} X \times Y \xrightarrow{p} Z \xrightarrow{g_X} X
        \isp{and}
        f_Y : Z \xrightarrow{g} X \times Y \xrightarrow{p} Z \xrightarrow{g_Y} Y.
    \]
    By the universal property, there is a unique continuous function $f : Z \to Z$ such that $f_X = g_Xf$ and $f_Y = g_Yf$. On one hand, the constructions of $f_X$ and $f_Y$ imply that $f = pg$. On the other hand, we have
    \[
        f_X = g_Xpg = p_Xg = g_X
        \isp{and}
        f_Y = g_Ypg = p_Yg = g_Y,
    \]
    which would imply that $f = \id_Z$. Therefore, $pg = f = \id_Z$.

\end{proof}



\newpage
\begin{pbox}[2]
    Similarly to the previous problem, suppose that $Z$ is a space equipped
      with maps $q_X:X \to Z$ and $q_Y:Y \to Z$, and that
      \begin{quote}
        for every space $T$ and pair of continuous functions $f_X:X \to T$ and
        $f_Y:Y \to T$, there is a unique continuous function $f:Z \to T$ such that
        $f_X=fq_X$ and $f_Y=fq_Y$.
      \end{quote}
      Show that $Z$ is homeomorphic to $X \sqcup Y$ and $q_X$ and $q_Y$ are (taken by the homeomorphism to) the obvious inclusions.
      \medskip
    
      \noindent This shows that $X \sqcup Y$ is the coproduct of $X$ and $Y$ in the category of topological spaces.
\end{pbox}

\begin{proof}
    Consider $X \sqcup Y$ with the inclusions $i_X : X \inc X \sqcup Y$ and $i_Y : Y \inc X \sqcup Y$. By the universal property, there is a unique continuous function $i : Z \to X \sqcup Y$ such that $i_X = iq_X$ and $i_Y = iq_Y$. We construct a function $q = q_X \sqcup q_Y : X \sqcup Y \to Z$, where $x \mapsto q_X(x)$ for all $x \in X$ and $y \mapsto q_Y(y)$ for all $y \in Y$. We claim that $i$ and $q$ are inverses and, in which case, describe a homeomorphism between $X \sqcup Y$ and $Z$.

    First, it can be seen that
    \[
        iq = iq_X \sqcup iq_Y = i_X \sqcup i_Y = \id_{X \sqcup Y}.
    \]
    To prove the opposite composition gives the identity on $Z$, we construct the continuous functions
    \[
        f_X : X \xrightarrow{q_X} Z \xrightarrow{i} X \sqcup Y \xrightarrow{q} Z  
        \isp{and}
        f_Y : Y \xrightarrow{q_Y} Z \xrightarrow{i} X \sqcup Y \xrightarrow{q} Z.
    \]
    By the universal property, there is a unique continuous function $f : Z \to Z$ such that $f_X = fq_X$ and $f_Y = fq_Y$. On one hand, the constructions of $f_X$ and $f_Y$ imply that $f = qi$. On the other hand, we have
    \[
        f_X = qiq_X = qi_X = q_X
        \isp{and}
        f_Y = qiq_Y = qi_Y = q_Y,
    \]
    which would imply that $f = \id_Z$. Therefore, $qi = f = \id_Z$.



\end{proof}



\newpage

\begin{lemma}
    The map $f : \R^2 \to \R$, $(x, y) \mapsto x + y$ is continuous.
\end{lemma}

\begin{proof}
    It suffices to prove that the preimages under $f$ of basis elements (open intervals) in $\R$ are open. Suppose $(a, b) \subseteq \R$, where $a \in [-\infty, \infty)$ and $b \in (-\infty, \infty]$. For $(x, y) \in f^{-1}((a, b))$, there is some $\eps > 0$ such that $B_\eps(x + y) \subseteq (a, b)$. Then, for any $(x', y') \in B_{\eps/2}(x) \times B_{\eps/2}(y)$,
    \[
        a
            < (x - \eps/2) + (y - \eps/2) 
            < x' + y'
            < (x + \eps/2) + (y + \eps/2) 
            < b,
    \]
    so $(x', y') \in f^{-1}((a, b))$. That is, we have found an open neighborhood of $(x, y)$ contained in $f^{-1}((a, b))$, namely the basis element $B_{\eps/2}(x) \times B_{\eps/2}(y)$ in the product topology. Hence, $f^{-1}((a, b)) \subseteq \R^2$ is open, so $f$ is continuous.

\end{proof}

\begin{lemma}
    The map $f : \R \to \R$, $x \mapsto -x$ is continuous.
\end{lemma}

\begin{proof}
    Suppose $(a, b) \subseteq \R$ is a possibly unbounded open interval. Then $f^{-1}((a, b))$ is the set of points $x \in \R$ such that $a < -x < b$ or, equivalently, such that $-b < x < -a$. That is, $f^{-1}((a, b)) = (-b, -a)$, which is an open interval in $\R$, so $f$ is continuous.

\end{proof}

\begin{lemma}
    The map $f : \R \to \R$, $x \mapsto |x|$ is continuous.
\end{lemma}

\begin{proof}
    Suppose $(a, b) \subseteq \R$ is a possibly unbounded open interval, then $f^{-1}((a, b)) = (a, b) \cup (-b, -a)$. As the union of open intervals, $f^{-1}((a, b))$ is open, so $f$ is continuous.

\end{proof}

\begin{pbox}[3]
    Let $X$ be a topological space and let $f:X \to \mathbb R$ and
      $g:X \to \mathbb R$ be continuous.  Show that $\min(f,g)$ is a continuous
      function.
\end{pbox}

\begin{proof}
    By lemmas 1, 2, and 3, and the fact that the composition of continuous functions is continuous, we have that $\min(f, g) = \tfrac{1}{2}(f + g - |f - g|)$ is continuous.

\end{proof}



\newpage
\begin{pbox}[4]
    Let $X$ and $Y$ be metric spaces.  Show that the metrics
      \begin{align*}
        d_\infty((x,y),(x',y')) &= \max(d(x,x'),d(y,y')) \\
        d_1((x,y),(x',y')) &= d(x,x')+d(y,y')
      \end{align*}
      both induce the product topology on $X \times Y$.
    
      \medskip
      \noindent(A good exercise is to visualize the balls in
      $\mathbb R \times \mathbb R$ in both of these topologies.)
\end{pbox}

\begin{proof}
    To prove a pair of topologies are the same, we show that every neighborhood, in one topology, around a point contains a neighborhood, in the other topology, of the same point. (If this is the case, then every open set $U$, in one topology, is the union $\bigcup_{x \in U} U_x$, where $U_x$ is an open set, in the other topology, chosen such that $x \in U_x \subseteq U$.)

    We first prove that $d_\infty$ induces the product topology on $X \times Y$. Let $W \subseteq X \times Y$ be open in the product topology and let $(x, y) \in W$. There is an element in the basis for the product topology $U \times V \subseteq X \times Y$, with $U \subseteq X$ open, $V \subseteq Y$ open, and $(x, y) \in U \times V$. Then there is some radius $r_X > 0$ such that $B_{r_X}(x) \subseteq U$, and some radius $r_Y > 0$ such that $B_{r_Y}(y) \subseteq V$. Let $r = \min\{r_X, r_Y\}$, then for any $(x', y') \in B_r((x, y); d_\infty)$, we have
    \[
        d(x, x') \leq d_\infty((x, y), (x', y')) < r \leq r_X
    \]
    and
    \[
        d(y, y') \leq d_\infty((x, y), (x', y')) < r \leq r_Y,
    \]
    so
    \[
        (x', y') \in B_{r_X}(x) \times B_{r_Y}(y) \subseteq U \times V \subseteq W.
    \]
    Thus, $(x, y) \in B_r((x, y); d_\infty) \subseteq W$, which proves that the topology induced by $d_\infty$ is at least as fine as the product topology.

    Let $W \subseteq X \times Y$ be open in the topology induced by $d_\infty$ and let $(x, y) \in W$. There is some radius $r > 0$ such that $(x, y) \in B_r((x, y); d_\infty) \subseteq W$. Then for any $(x', y') \in B_r(x) \times B_r(y)$, we have $d(x, x') < r$ and $d(y, y') < r$, so $d_\infty((x, y), (x', y')) < r$, implying that $(x', y') \in B_r((x, y); d_\infty) \subseteq W$. Hence, we have found a neighborhood of $(x, y)$ in the product topology, namely $B_r(x) \times B_r(Y)$, contained in $W$. This proves that the product topology is at least as fine as the topology induced by $d_\infty$, so we conclude that they are the same topology.

    Next, we will prove that $d_\infty$ and $d_1$ induce the same topology. To do so, we will show that each open ball under one metric contains an open ball in the other metric. Let $B_r((x, y); d_\infty)$ be an open $d_\infty$-ball. Then, for all $(x', y') \in B_{r/2}((x, y); d_1)$, we have
    \[
        \max(d(x, x'), d(y, y'))
            \leq  d(x,x') + d(y,y')
            < r/2 + r/2
            = r.
    \]
    That is, $B_{r/2}((x, y); d_1) \subseteq B_r((x, y); d_\infty)$, so $d_1$ generates a topology at least as fine as $d_\infty$. Now, consider an open $d_1$-ball $B_r((x, y); d_1)$, with $r > 0$. For all $(x', y') \in B_{r/2}((x, y); d_\infty)$,
    \[
        d(x,x') + d(y,y')
            \leq 2\max(d(x, x'), d(y, y'))
            < 2 \cdot r/2
            = r.
    \]
    That is, $B_{r/2}((x, y); d_\infty) \subseteq B_r((x, y); d_1)$, so $d_\infty$ generates a topology at least as fine as $d_1$.

\end{proof}


\newpage
\begin{pbox}[5]
    Give an example of a subset $A \subset \mathbb R$ such that the following
      sets are all different:
      \[A, \quad \overline{A}, \quad \inter A, \quad
      \inter\left(\overline A\right), \quad \overline{\inter A}, \quad
      \overline{\inter\bigl(\overline A\bigr)}, \quad
      \inter\bigl(\overline{\inter A}\bigr).\]
\end{pbox}

Consider the set
\[
    A = [0, 1) \cup (1, 2] \cup \big([2, 3] \cap \Q\big) \cup \{4\} .
\]
We have
\begin{align*}
    \clo{A} &= [0, 3] \cup \{4\}, & \inter A &= (0, 1) \cup (1, 2), \\
    \inter(\clo{A}) &= (0, 3), & \clo{\inter A} &= [0, 2], \\
    \clo{\inter(\clo{A})} &= [0, 3], & \inter(\clo{\inter A}) &= (0, 2).
\end{align*}

\end{document}
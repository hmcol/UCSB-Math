\documentclass[12pt]{article}

% Packages
\usepackage[margin=1in]{geometry}
\usepackage{fancyhdr, parskip}
\usepackage{amsmath, amsthm, amssymb}
\usepackage{tikz, tikz-cd}

% Page Style
\makeatletter
\fancypagestyle{title}{
    \renewcommand{\headrulewidth}{0.4pt}
    \setlength{\headheight}{15pt}
    \fancyhead[R]{\@author}
    \fancyhead[L]{\@title}
    \fancyhead[C]{\@date}
}
\makeatother
\renewcommand{\maketitle}{\thispagestyle{title}}
\fancypagestyle{plain}{
    \fancyhf{}
    \renewcommand{\headrulewidth}{0pt}
    \renewcommand{\footrulewidth}{0pt}
    \fancyfoot[R]{\thepage}
}
\pagestyle{plain}

% Problem Box
\setlength{\fboxsep}{4pt}
\newlength{\myparskip}
\setlength{\myparskip}{\parskip}
\newsavebox{\savefullbox}
\newenvironment{fullbox}{\begin{lrbox}{\savefullbox}\begin{minipage}{\dimexpr\textwidth-2\fboxsep\relax}\setlength{\parskip}{\myparskip}}{\end{minipage}\end{lrbox}\framebox[\textwidth]{\usebox{\savefullbox}}}
\newenvironment{pbox}[1][]{\begin{fullbox}\def\temp{#1}\ifx\temp\empty\else\paragraph{#1}\phantom{}\fi}{\end{fullbox}}

% Theorem Environments
\theoremstyle{definition}
\newtheorem{lemma}{Lemma}

% Tikz Environments
\newenvironment{drawing}{\begin{center}\begin{tikzpicture}}{\end{tikzpicture}\end{center}}
% \tikzcdset{row sep/normal=0pt}
\newenvironment{cd}{\begin{center}\begin{tikzcd}}{\end{tikzcd}\end{center}}

% Default Commands
\newcommand{\isp}[1]{\quad\text{#1}\quad}
\newcommand{\N}{\mathbb{N}} 
\newcommand{\Z}{\mathbb{Z}}
\newcommand{\Q}{\mathbb{Q}}
\newcommand{\R}{\mathbb{R}}
\newcommand{\C}{\mathbb{C}}
\newcommand{\A}{\mathbb{A}}
\renewcommand{\P}{\mathbb{P}}
\newcommand{\eps}{\varepsilon}
\renewcommand{\phi}{\varphi}
\renewcommand{\emptyset}{\varnothing}
\newcommand{\<}{\langle}
\renewcommand{\>}{\rangle}
\newcommand{\iso}{\cong}
\newcommand{\eqc}{\overline}
\newcommand{\clo}{\overline}
\newcommand{\seq}{\subseteq}
\newcommand{\teq}{\trianglelefteq}
\DeclareMathOperator{\id}{id}
\DeclareMathOperator{\im}{im}
\newcommand{\inc}{\hookrightarrow}

% Extra Commands
\newcommand{\dd}{\mathrm{d}}
\newcommand{\DD}{\mathrm{D}}

\newcommand{\pdv}[2]{\frac{\partial #1}{\partial #2}}
\newcommand{\mat}[1]{\begin{bmatrix}#1\end{bmatrix}}

% Document
\begin{document}
\title{MATH 221C Homework 3}
\author{Harry Coleman}
\date{April 18, 2022}
\maketitle

I worked with Joseph Sullivan and Gahl Shemy.

\begin{pbox}[1 Exercise 1.1.9]
    Explicitly exhibit enough parameterizations to cover $S^1 \times S^1 \seq \R^4$.
\end{pbox}

The projection $(x, y) \mapsto x$ is a parameterization on both the open sets $\{(x, y) \in S^1 : y > 0\}$ and $\{(x, y) \in S^1 : y < 0\}$.
Similarly, the projection $(x, y) \mapsto y$ is a parameterization on both the open sets $\{(x, y) \in S^1 : x > 0\}$ and $\{(x, y) \in S^1 : x < 0\}$.
These four parameterizations cover $S^1$, denote them by $\phi_i : U_i \to (-1, 1) \seq \R^1$, for $i = 1, 2, 3, 4$.
Then we get a cover of $S^1 \times S^1$ by sixteen parameterizations
\[
    \phi_i \times \phi_j : U_i \times U_j \to (-1, 1) \times (-1, 1) \seq \R^2.
\]
Indeed, this are smooth by Homework 2 Exercise 1.1.14, and have smooth inverses given by products of inverses, hence diffeomorphisms.

\begin{pbox}[2 Exercise 1.1.15]
    Show that the projection map $X \times Y \to X$, carrying $(x, y)$ to $x$, is smooth.
\end{pbox}

\begin{proof}
    Denote the projection map by $f : X \times Y \to X$.
    Suppose $X \seq \R^N$ and $Y \seq \R^M$, then define $F : \R^{N + M} \to \R^N$ by
    \[
        F(x_1, \dots, x_N, y_1, \dots, y_M) = (x_1, \dots, x_N).
    \]
    This map is linear, hence smooth.
    Moreover, $\R^N$ is an open neighborhood of $X$ and $F|_X = f$.
    In other words, $F$ is a smooth (global) extension of $f$, so $f$ is smooth.
\end{proof}

\begin{pbox}[3 Exercise 1.1.16]
    The \textit{diagonal} $\Delta$ in $X \times X$ is the set of points of the form $(x, x)$.
    Show that $\Delta$ is diffeomorphic to $X$, so $\Delta$ is a manifold if $X$ is.
\end{pbox}

\begin{proof}
    Let $f : X \to \Delta$ be the diagonal map $x \mapsto (x, x)$.
    Suppose $X \seq \R^N$ and define the diagonal map $F : \R^N \to \R^{2N}$ by
    \[
        F(x_1, \dots, x_N) = (x_1, \dots, x_N, x_1, \dots, x_N).
    \]
    This map is linear, hence smooth.
    Moreover, it is a smooth extension of $f$, so $f$ is smooth.

    Note that $\Delta \seq \R^N$ Let $G : \R^{2N} \to \R^N$ be the projection map
    \[
        G(x_1, \dots, x_N, x'_1, \dots, x'_N) = (x_1, \dots, x_N).
    \]
    This map is linear, hence smooth.
    Therefore, the restriction $g = G|_{\Delta} : \Delta \to X$ is smooth.

    Lastly,
    \[
        f(g(x, x)) = f(x) = (x, x) \isp{and} g(f(x)) = g(x, x) = x,
    \]
    so $f$ and $g$ are smooth inverses, hence diffeomorphisms.

    Supposing $X$ is a manifold, let $U \seq X$ be open and $\phi : U \to V \seq \R^k$ be a smooth chart.
    Then $f(U) \seq \Delta$ is open and $\phi \circ g : f(U) \to V$ is a smooth chart.
    An open cover of $X$ is sent to an open cover of $\Delta$ under $f$, so the manifold structure of $X$ corresponds to a manifold structure of $\Delta$ under $f$ and $g$.
\end{proof}

\begin{pbox}[4 Exercise 1.1.17]
    The \textit{graph} of a map $f : X \to Y$ is the subset of $X \times Y$ defined by
    \[
        \mathrm{graph}(f) = \{(x, f(x)) : x \in X\}.
    \]
    Define $F : X \to \mathrm{graph}(f)$ by $F(x) = (x, f(x))$.
    Show that if $f$ is smooth, $F$ is a diffeomorphism; thus $\mathrm{graph}(f)$ is a manifold if $X$ is.
\end{pbox}

\begin{proof}
    Notice that $F$ can be written as the following composition:
    \begin{cd}[column sep=large, row sep=tiny]
        X \rar & \Delta \rar["\id_X \times f"] & \mathrm{graph}(f) \\
        x \rar[mapsto] & (x, x) \rar[mapsto] & (x, f(x))
    \end{cd}
    The first map (the diagonal map) is smooth by the previous problem (Exercise 1.1.16), and the latter is smooth as the product of smooth maps.
    The projection map $X \times Y \to X$ restricted to $\mathrm{graph}(f)$ is inverse to $F$ and smooth by Problem 2 (Exercise 1.1.15), hence $F$ is a diffeomorphism.

    Similar to the previous problem, when $X$ is a manifold, its charts can be turned into charts on $\mathrm{graph}(f)$, giving $\mathrm{graph}(f)$ a manifold structure. 
\end{proof}

\begin{pbox}[5 Exercise 1.1.18 \\ (a)]
    An extremely useful function $f : \R^1 \to \R^1$ is
    \[
        f(x) = \begin{cases}
            e^{-1/x^2} & x > 0, \\
            0 & x \leq 0.
        \end{cases}
    \]
    Prove that $f$ is smooth.
\end{pbox}

\begin{proof}
    We need only check that $f$ is smooth at $0$.
    From Homework 1 Problem 6, we know that the function
    \[
        \hat{f}(x) = \begin{cases}
            e^{-1/x^2} & x \ne 0, \\
            0 & x = 0,
        \end{cases}
    \]
    is smooth.
    Moreover, $\hat{f}^{(n)}(0) = 0$ for all $n$, which implies that derivatives of all orders of $f$ at $0$ from the right are all zero.
    Since the the derivatives of all orders of $f$ at $0$ from the left are also all zero, we conclude that $f$ is smooth at $0$ with $f^{(n)}(0) = 0$ for all $n$.
\end{proof}

\begin{pbox}[(b)]
    Show that $g(x) = f(x - a)f(b - x)$ is a smooth function, positive on $(a, b)$, and zero elsewhere.
    Then
    \[
        h(x) = \frac{\int_{-\infty}^{x} g \,\dd{x}}{\int_{-\infty}^{\infty} g \,\dd{x}}
    \]
    is a smooth function satisfying $h(x) = 0$ for $x \leq a$, $h(x) = 1$ for $x \geq b$ and $0 < h(x) < 1$ for $x \in (a, b)$.
\end{pbox}

\begin{proof}
    We can write $g$ as the composition
    \begin{cd}[row sep=0pt]
        \R^1 \rar & \R^2 \rar & \R^2 \rar & \R^2 \rar & \R^1 \\
        x \rar[mapsto] & (x, x) & (x, y) \rar[mapsto] & (f(x), f(y)) \\
        & (x, y) \rar[mapsto] & (x - a, b - y) & (x, y) \rar[mapsto] & xy \\
    \end{cd}
    The first map is the diagonal, the second map is linear, the third map is $f \times f$, and the last map is multiplication.
    We know all of these to be smooth, so $g$ is smooth as their composition.

    If $x \leq a$, then $x - a \leq 0$ so $f(x - a) = 0$.
    If $x \geq b$, then $b - x \geq 0$ so $f(b - x) = 0$.
    In either case, $g(x) = 0$.
    
    If $a < x < b$ then $x - a > 0$ and $b - x > 0$, so $f(x - a)$ and $f(b - x)$ are positive.
    In which case $g(x)$ is positive.

    By the fundamental theorem of calculus, $G(x) = \int_{-\infty}^x g \,\dd{y}$ is differentiable, with $G' = g$.
    Since $g$ is smooth, this implies $G$ is smooth.
    Multiplying by the constant $1/\int_{-\infty}^{\infty} g \,\dd{y}$ gives us the smooth function $h$.

    Given $x \leq a$, we know $g(y) = 0$ for all $y \leq x$, so
    \[
        h(x)
            = \frac{\int_{-\infty}^{x} g \,\dd{y}}{\int_{-\infty}^{\infty} g \,\dd{y}}
            = \frac{\int_{-\infty}^{x} 0 \,\dd{y}}{\int_{-\infty}^{\infty} g \,\dd{y}}
            = \frac{0}{\int_{-\infty}^{\infty} g \,\dd{y}}
            = 0.
    \]
    Given $x \geq b$, we know $g(y) = 0$ for all $y \geq x$, so
    \[
        h(x)
            = \frac{\int_{-\infty}^{x} g \,\dd{y}}{\int_{-\infty}^{\infty} g \,\dd{y}}
            = \frac{\int_{-\infty}^{b} g \,\dd{y} + \int_{b}^{x} g \,\dd{y}}{\int_{-\infty}^{b} g \,\dd{y} + \int_{b}^{\infty} g \,\dd{y}}
            = \frac{\int_{-\infty}^{b} g \,\dd{y} + 0}{\int_{-\infty}^{b} g \,\dd{y} + 0}
            = 1.
    \]
    Since $g(x) > 0$ for all $x \in (a, b)$, we know that $G$ is strictly increasing on $(a, b)$.
    The same is true of $h$, as a positive scalar multiple of $G$, so
    \[
        0 = h(a) < h(x) < h(b) = 1
    \]
    for all $x \in (a, b)$.
\end{proof}

\begin{pbox}[(c)]
    Now construct a smooth function on $\R^k$ that equals $1$ on the ball of radius $a$, zero outside the ball of radius $b$, and is strictly between $0$ and $1$ at intermediate points.
\end{pbox}

\begin{proof}
    For $x < y$, let $h_x^y : \R^1 \to [0, 1]$ be constructed as $h$ in part (b) with $a = x$ and $b = y$.

    Let $F : \R^k \to \R^1$ be given by the following composition:
    \begin{cd}[row sep=0pt]
        \R^k \rar["\|{-}\|^2"] & \R^1 \rar["h_{a^2}^{b^2}"] & \R^1 \rar[] & \R^1 \\
        (x_1, \dots, x_k) \rar[mapsto] & \sum x_i^2 & x \rar[mapsto] & 1 - x \\
        & x \rar[mapsto] & h_{a^2}^{b^2}(x).
    \end{cd}
    The first map is a polynomial, the second map is $h_{a^2}^{b^2}$, and the last map is linear.
    We know that each of these maps is smooth, therefore $F$ is smooth as their composition.

    For $x \in \R^k$ with $\|x\| \leq a$, we have $\|x\|^2 \leq a^2$.
    This implies $h_{a^2}^{b^2}(\|x\|^2) = 0$, so $F(x) = 1$.

    For $x \in \R^k$ with $\|x\| \geq b$, we have $\|x\|^2 \geq b^2$.
    This implies $h_{a^2}^{b^2}(\|x\|^2) = 1$, so $F(x) = 0$.

    For $a < \|x\| < b^2$, we have $a^2 < \|x\|^2 < b^2$.
    This implies $ 0 < h_{a^2}^{b^2}(\|x\|^2) < 1$, so $0 < F(x) < 1$.

\end{proof}

\begin{pbox}[6 Exercise 1.2.2]
    If $U$ is an open subset of the manifold $X$, check that
    \[
        T_x(U) = T_x(X) \isp{for} x \in U.
    \]
\end{pbox}

\begin{proof}
    Let $V \seq X$ be an open neighborhood of $x$ with smooth parameterization $\phi : W \to V$ with $W \seq \R^k$ open and $\phi(0) = x$.
    Then $U \cap V$ is an open neighborhood of $x$ contained in $U$, and the restriction of $\phi$ to the open set $\phi^{-1}(U \cap V) \seq W$ is a smooth parameterization of $U \cap V$.
    Hence, we have the tangent spaces
    \[
        T_x(X)
            = \dd\phi_0(\R^k)
            = \dd(\phi|_{\phi^{-1}(U \cap V)})_0(\R^k)
            = T_x(U).
    \]
\end{proof}

\begin{pbox}[7 Exercise 1.2.3]
    Let $V$ be a vector subspace of $\R^N$.
    Show that $T_x(V) = V$ if $x \in V$.
\end{pbox}

\begin{proof}
    Suppose $v_1, \dots, v_k \in V$ form a basis for $V$.
    The linear map $L : \R^k \to V$ defined by $e_i \to v_i$ is an isomorphism of vector spaces.
    Fixing $x \in V$, the map $\phi : \R^k \to V$ sending $y \mapsto \L(y) + x$ is a smooth parameterization with $\phi(0) = x$.
    Then the tangent space is given by
    \[
        T_x(V)
            = \dd\phi_0(\R^k)
            = \dd(L + x)_0(\R^k)
            = \dd{L}_0(\R^k)
            = L(\R^k)
            = V.
    \]
\end{proof}

\begin{pbox}[8 Exercise 1.2.4]
    Suppose that $f : X \to Y$ is a diffeomorphism, and prove that at each $x$ its derivative $\dd{f}_x$ is an isomorphism of tangent spaces.
\end{pbox}

\begin{proof}
    Let $g : Y \to X$ be a smooth inverse of $f$, i.e., $f \circ g = \id_Y$ and $g \circ f = \id_X$.
    Their derivatives give linear maps $\dd{f}_x : T_x(X) \to T_{f(x)}(Y)$ and $\dd{g}_{f(x)} : T_{f(x)}(Y) \to T_x(X)$.
    The chain rule lets us compute
    \[
        \dd(f \circ g)_{f(x)}
            = \dd{f}_{g(f(x))} \circ \dd{g}_{f(x)}
            = \dd{f}_x \circ \dd{g}_{f(x)}
    \]
    and
    \[
        \dd(g \circ f)_x = \dd{g}_{f(x)} \circ \dd{f}_x.
    \]
    On the other hand,
    \[
        \dd{f \circ g}_{f(x)} = \dd{\id_y}_{f(x)} = \id_Y
    \]
    and
    \[
        \dd{g \circ f}_x = \dd{\id_X}_x = \id_X.
    \]
    Hence $\dd{f}_x$ and $\dd{g}_{f(x)}$ are linear inverses.
    In particular, $\dd{f}_x$ is an isomorphism.
\end{proof}

\begin{pbox}[9 Exercise 1.2.9 \\ (a)]
    Show that for any manifolds $X$ and $Y$,
    \[
        T_{(x, y)}(X \times Y) = T_x(X) \times T_y(Y).
    \]
\end{pbox}

\begin{proof}
    Suppose we have smooth local parameterizations $\phi : U \to X$ and $\psi : V \to Y$ at the points $x \in X$ and $y \in Y$, respectively, with $U \seq \R^k$ and $V \seq \R^\ell$ open sets.
    Additionally, assume $\phi(0) = x$ and $\psi(0) = y$.
    Then their product $\phi \times \psi : U \times V \to X \times Y$ is a smooth local parameterization at $(x, y) \in X \times Y$, with $(\phi \times \psi)(0) = (\phi(0), \psi(0)) = (x, y)$.
    Thus, we compute the tangent space
    \[
        T_{(x, y)}(X \times Y)
            = \dd(\phi \times \psi)_{(0,0)}(\R^{k + \ell})
            = \dd\phi_0(\R^k) \times \dd\psi_0(\R^\ell)
            = T_x(X) \times T_y(Y).
    \]
\end{proof}

\begin{pbox}[(b)]
    Let $f : X \times Y \to X$ be the projection map $(x, y) \mapsto x$.
    Show that
    \[
        \dd{f}_{(x, y)} : T_x(X) \times T_y(Y) \to T_x(X)
    \]
    is the analogous projection $(v, w) \mapsto v$.
\end{pbox}

\begin{proof}
    Let $f : X \times Y \to X$ be the projection map.
    Let $\phi : U \to X$ and $\psi : V \to Y$ be smooth local parameterizations with $\phi(0) = x$ and $\psi(0) = y$, then there is a commutative diagram
    \begin{cd}
        X \times Y
            \rar["f"]
        & X
        \\
        U \times V
            \uar["\phi \times \psi"]
            \rar[dashed, "h"']
        & U
            \uar["\phi"']
    \end{cd}
    Note that $h : U \times V \to U$ is simply a restriction of the the linear projection $L : \R^k \times \R^\ell \to \R^k$.
    Taking derivatives gives us
    \begin{cd}[column sep=large]
        T_{(x, y)}(X \times Y)
            \rar["\dd{f}_{(x, y)}"]
        & T_x(X)
        \\
        \R^k \times \R^\ell
            \uar["\dd(\phi \times \psi)_0"]
            \rar["\dd{h}_0 = L"']
        & \R^k
            \uar["\dd{\phi}_0"']
    \end{cd}
    For $(v, w) \in T_x(X) \times T_y(Y) = T_{(x, y)}(X \times Y)$, we compute
    \begin{align*}
        \dd{f}_{(x, y)}(v, w)
            &= (\dd\phi_0 \circ \dd{h}_0 \circ \dd(\phi \times \psi)_0^{-1})(v, w) \\
            &= (\dd\phi_0 \circ L \circ (\dd\phi_0^{-1} \times \dd\psi_0^{-1}))(v, w) \\
            &= \dd\phi_0(L(\dd\phi_0^{-1}(v), \dd{\psi}_0^{-1}(w))) \\
            &= \dd\phi_0(\dd\phi_0^{-1}(v)) \\
            &= v.
    \end{align*}
\end{proof}

\begin{pbox}[(c)]
    Fixing any $y \in Y$ gives an injection mapping $f : X \to X \times Y$ by $f(x) = (x, y)$.
    Show that $\dd{f}_x(v) = (v, 0)$.
\end{pbox}

\begin{proof}
    
\end{proof}

\begin{pbox}[(d)]
    Let $f : X \to X'$, $g : Y \to Y'$ be any smooth maps.
    Prove that
    \[
        \dd(f \times g)_{(x, y)} = \dd{f}_x \times \dd{g}_y.
    \]
\end{pbox}

\begin{proof}
    Consider parameterizations described by the following (commutative) diagrams:
    \begin{center}
        \begin{tikzcd}
            X
                \rar["f"]
            & X'
            \\
            U
                \uar["\phi"]
                \rar[dashed, "h"']
            & U'
                \uar["\phi'"']
        \end{tikzcd}
        \hspace{2em}
        \begin{tikzcd}
            Y
                \rar["g"]
            & Y'
            \\
            V
                \uar["\psi"]
                \rar[dashed, "k"']
            & V'
                \uar["\psi'"']
        \end{tikzcd}
    \end{center}
    Taking derivatives gives us the following commutative diagrams:
    \begin{center}
        \begin{tikzcd}
            T_x(X)
                \rar["\dd{f}_x"]
            & T_{f(x)}(X')
            \\
            \R^n
                \uar["\dd{\phi}_0"]
                \rar["\dd{h}_0"']
            & \R^{n'}
                \uar["\dd{\phi'}_0"']
        \end{tikzcd}
        \hspace{2em}
        \begin{tikzcd}
            T_y(Y)
                \rar["\dd{g}_y"]
            & T_{g(y)}(Y')
            \\
            \R^m
                \uar["\dd{\psi}_0"]
                \rar["\dd{k}_0"']
            & \R^{m'}
                \uar["\dd{\psi'}_0"']
        \end{tikzcd}
    \end{center}
    Taking the products of these pairs of diagrams gives us parameterizations
    \begin{center}
        \begin{tikzcd}
            X \times Y
                \rar["f \times g"]
            & X' \times Y'
            \\
            U \times V
                \uar["\phi \times \psi"]
                \rar["h \times k"']
            & U' \times V'
                \uar["\phi' \times \psi'"']
        \end{tikzcd}
        \hspace{2em}
        \begin{tikzcd}[column sep=large]
            T_x(X) \times T_y(Y)
                \rar["\dd{f}_x \times \dd{g}_y"]
            & T_{f(x)}(X') \times T_{g(y)}(Y')
            \\
            \R^n \times \R^m
                \uar["\dd{\phi}_0 \times \dd{\psi}_0"]
                \rar["\dd{h}_0 \times \dd{k}_0"']
            & \R^{n'} \times \R^{m'}
                \uar["\dd{\phi'}_0 \times \dd{\psi'}_0"']
        \end{tikzcd}
    \end{center}
    Applying the definition of the derivative for maps of manifolds and the product result for the usual derivative (i.e., from analysis), we obtain
    \begin{align*}
        \dd(f \times g)_{(x, y)}
            &= \dd(\phi' \times \psi')_0 \circ \dd(h \times k)_0 \circ \dd(\phi' \times \psi')_0^{-1} \\
            &= (\dd{\phi'}_0 \times \dd{\psi'}_0) \circ (\dd{h}_0 \times \dd{k}_0) \circ (\dd{\phi}_0^{-1} \times \dd{\psi}_0^{-1}) \\
            &= (\dd{\phi'}_0 \circ \dd{h}_0 \circ \dd{\phi}_0^{-1}) \times (\dd{\psi'}_0 \circ \dd{k}_0 \circ \dd{\psi}_0^{-1}) \\
            &= \dd{f}_x \times \dd{g}_y.
    \end{align*}
\end{proof}

\begin{pbox}[10 Exercise 1.2.11 \\ (a)]
    Suppose that $f : X \to Y$ is a smooth map, and let $F : X \to X \times Y$ be $F(x) = (x, f(x))$.
    Show that
    \[
        \dd{F}_x(v) = (v, \dd{f}_x(v)).
    \]
\end{pbox}

\begin{proof}
    Let $\Delta : X \to \Delta_X \seq X \times X$ be the diagonal map.
    Then (similar to Exercise 1.1.17 above) $F$ can be written as the composition
    \begin{cd}[column sep=large]
        X \rar["\Delta"] & \Delta_X \rar["\id_X \times f"] & \mathrm{graph}(f) \seq X \times Y.
    \end{cd}
    Then
    \[
        \dd{F}_x
         = \dd((\id_X \times f) \circ \Delta)_x
         = \dd(\id_X \times f)_{(x, x)} \circ \dd{\Delta}_x
         = (\id_{T_x(X)} \times \dd{f}_x) \circ \Delta,
    \]
    so 
    \[
        \dd{F}_x(v)
            = (\id_{T_x(X)} \dd{f}_x)(\Delta(v))
            = (\id_{T_x(X)} \dd{f}_x)(v, v)
            = (v, \dd{f}_x(v)).
    \]
\end{proof}

\begin{pbox}[(b)]
    Prove that the tangent space to $\mathrm{graph}(f)$ at the point $(x, f(x))$ is the graph of $\dd{f}_x : T_x(X) \to T_{f(x)}(Y)$.
\end{pbox}

\begin{proof}
    Since $F : X \to \mathrm{graph}(f)$ is a diffeomorphism, the derivative is an isomorphism.
    In particular, it is surjective, and by part (a) its image is precisely the graph of $\dd{f}_x$.
\end{proof}

\begin{pbox}[11 Exercise 1.3.2 \\ (a)]
    If $X$ is compact and $Y$ connected, show every submersion $f : X \to Y$ is surjective.
\end{pbox}

\begin{pbox}[(b)]
    Show that there exist no submersions of compact manifolds into Euclidean spaces.
\end{pbox}

\begin{pbox}[12 Exercise 1.3.3]
    Show that the curve $t \mapsto (t, t^2. t^3)$ embeds $\R^1$ into $\R^3$.
    Find two independent functions that globally define the image.
    Are your functions independent on all of $\R^3$, or just on an open neighborhood of the image?
\end{pbox}


\begin{pbox}[13 Exercise 1.3.4]
    Prove the following extension of Converse 2.
    Suppose that $Z \seq X \seq Y$ are manifolds, and $z \in Z$.
    Then there exist independent functions $g_1, \dots, g_\ell$ on a neighborhood $W$ of $z$ in $Y$ such that
    \[
        Z \cap W = \{y \in W : g_1(y) = \cdots = g_\ell(y) = 0\}
    \]
    and
    \[
        X \cap W = \{y \in W : g_1(y) = \cdots = g_m(y) = 0\},
    \]
    where $\ell = m$ is the codimension of $Z$ in $X$.
\end{pbox}

\begin{pbox}[14 Exercise 1.3.6]
    More generally, let $p$ be any homogeneous degree $m$ polynomial in $k$ variables.
    Prove that the set of points $x$, where $p(x) = a$, is a $(k - 1)$-dimensional submanifold of $\R^k$, provided that $a \ne 0$.
    Show that the manifolds obtained with $a > 0$ are all diffeomorphic, as are those with $a < 0$.
    [Hint: Use Euler's identity for homogeneous polynomials
    \[
        \sum_{i=1}^{k} x \pdv{p}{x_i} = m \cdot p
    \]
    to prove that $0$ is the only critical value of $p$.]
\end{pbox}

\end{document}
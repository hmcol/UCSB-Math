\documentclass[12pt]{article}

% Packages
\usepackage[margin=1in]{geometry}
\usepackage{fancyhdr, parskip}
\usepackage{amsmath, amsthm, amssymb}
\usepackage{tikz, tikz-cd}

% Page Style
\makeatletter
\fancypagestyle{title}{
    \renewcommand{\headrulewidth}{0.4pt}
    \setlength{\headheight}{15pt}
    \fancyhead[R]{\@author}
    \fancyhead[L]{\@title}
    \fancyhead[C]{\@date}
}
\makeatother
\renewcommand{\maketitle}{\thispagestyle{title}}
\fancypagestyle{plain}{
    \fancyhf{}
    \renewcommand{\headrulewidth}{0pt}
    \renewcommand{\footrulewidth}{0pt}
    \fancyfoot[R]{\thepage}
}
\pagestyle{plain}

% Problem Box
\setlength{\fboxsep}{4pt}
\newlength{\myparskip}
\setlength{\myparskip}{\parskip}
\newsavebox{\savefullbox}
\newenvironment{fullbox}{\begin{lrbox}{\savefullbox}\begin{minipage}{\dimexpr\textwidth-2\fboxsep\relax}\setlength{\parskip}{\myparskip}}{\end{minipage}\end{lrbox}\framebox[\textwidth]{\usebox{\savefullbox}}}
\newenvironment{pbox}[1][]{\begin{fullbox}\def\temp{#1}\ifx\temp\empty\else\paragraph{#1}\phantom{}\fi}{\end{fullbox}}

% Theorem Environments
\theoremstyle{definition}
\newtheorem{lemma}{Lemma}

% Tikz Environments
\newenvironment{drawing}{\begin{center}\begin{tikzpicture}}{\end{tikzpicture}\end{center}}
% \tikzcdset{row sep/normal=0pt}
\newenvironment{cd}{\begin{center}\begin{tikzcd}}{\end{tikzcd}\end{center}}

% Default Commands
\newcommand{\isp}[1]{\quad\text{#1}\quad}
\newcommand{\N}{\mathbb{N}} 
\newcommand{\Z}{\mathbb{Z}}
\newcommand{\Q}{\mathbb{Q}}
\newcommand{\R}{\mathbb{R}}
\newcommand{\C}{\mathbb{C}}
\newcommand{\A}{\mathbb{A}}
\renewcommand{\P}{\mathbb{P}}
\newcommand{\eps}{\varepsilon}
\renewcommand{\phi}{\varphi}
\renewcommand{\emptyset}{\varnothing}
\newcommand{\<}{\langle}
\renewcommand{\>}{\rangle}
\newcommand{\iso}{\cong}
\newcommand{\eqc}{\overline}
\newcommand{\clo}{\overline}
\newcommand{\seq}{\subseteq}
\newcommand{\teq}{\trianglelefteq}
\DeclareMathOperator{\id}{id}
\DeclareMathOperator{\im}{im}
\newcommand{\inc}{\hookrightarrow}
\newcommand{\dd}{\mathrm{d}}

% Extra Commands
\newcommand{\mat}[1]{\begin{bmatrix}#1\end{bmatrix}}

% Document
\begin{document}
\title{MATH 221C Homework 4}
\author{Harry Coleman}
\date{April 25, 2022}
\maketitle

I worked with Joseph Sullivan and Gahl Shemy.

\begin{pbox}[1 Exercise 1.3.6]
\end{pbox}

\begin{pbox}[(a)]
\end{pbox}

\begin{proof}
    If $f$ and $g$ are immersions, their derivatives are injective.
    Then $\dd(f \times g)_(x, y) = \dd{f}_x \times \dd{g}_x$ is also injective, which implies $f \times g$ is an immersion.
\end{proof}

\begin{pbox}[(b)]
\end{pbox}

\begin{proof}
    If $f$ and $g$ are immersions, their derivatives are injective.
    Then $\dd(g \circ f)_x = \dd{g}_{f(x)} \circ \dd{f}_0$ is also injective, which implies $g \circ f$ is an immersion.
\end{proof}

\begin{pbox}[(c)]
\end{pbox}

\begin{proof}
    Suppose $M$ is the domain of $f$ and $\iota : N \inc M$ is an inclusion of manifolds.
    Then the derivative $\dd{\iota}_x : T_xN \inc T_xM$ is the inclusion of tangent spaces.
    In particular, $\dd{\iota}_x$ is injective so $\iota$ is an immersion.
    Then by part (b), we know $f|_N = f \circ \iota$ is an immersion.
\end{proof}

\begin{pbox}[(d)]
\end{pbox}

\begin{proof}
    If $f : X \to Y$ is an immersion, then $\dd{f}_x : T_xX \to T_{f(x)}Y$ is injective.
    Because
    \[
        \dim T_xX = \dim X = \dim Y = \dim T_{f(x)} Y,
    \]
    we know that $\dd{f}_x$ is also surjective, and therefore an isomorphism of tangent spaces.
    By definition, $f$ is a local diffeomorphism.

    If $f : X \to Y$ is a local diffeomorphism, then $\dd{f}_x$ is an isomorphism.
    In particular the derivative is injective, so $f$ is an immersion.
\end{proof}

\begin{pbox}[2 Exercise 1.3.9]
\end{pbox}

\begin{lemma}
    Let $V \leq \R^n$ be a subspace of dimension $k$.
    Then the natural projection of $V$ onto the subspace $\<e_{i_1}, \dots, e_{i_k}\> \leq \R^n$ is an isomorphism for some choice of $e_{i_j}$.
\end{lemma}

\begin{proof}
    Let $\{v_1, \dots, v_k\}$ be a basis of $V$ and consider the $k \times n$ matrix with the $v_j$'s as rows:
    \[
        \begin{bmatrix}
            - & v_1 & - \\
            & \vdots & \\
            - & v_k & -
        \end{bmatrix}.
    \]
    Performing any row operations on this matrix yields a matrix whose rows are still a basis of $V$.
    Put the matrix into reduced row echelon form:
    \[
        \begin{bmatrix}
            * & 1 & * \cdots * & 0 & * \cdots * & 0 \\
            && 0 \cdots 0 & 1 & * \cdots * & 0  \\
            &&&& 0 \cdots 0 & 1 \\
            &&&&&& \ddots
        \end{bmatrix}
    \]
    Without loss of generality, we can choose the $v_i$'s to be the rows of this matrix.
    Let $i_j$ be the index of the $j$th pivot column in the matrix.

    Then the natural projection $V \to \<e_{i_1}, \dots, e_{i_k}\>$ sends basis elements $v_j \mapsto e_{i_j}$.
    This is a linear surjection with the domain and codomain both of dimension $k$, so it must be an isomorphism of vector spaces.
\end{proof}

\begin{pbox}[(a)]
\end{pbox}

\begin{proof}
    By Lemma 1, suppose the natural projection $T_xX \to \<e_{i_1}, \dots, e_{i_k}\>$ is an isomorphism of vector spaces.
    Then the coordinate function $F = (x_{i_1}, \dots, x_{i_k}) : \R^N \to \R^k$ restricts to an isomorphism $T_xX \to \R^k$ and the restriction $f = F|_X : X \to \R^k$ has derivative
    \[
        \dd{f}_x = \dd{F}_x|_{T_xX} = F|_{T_xX}.
    \]
    So $\dd{f}_x$ is an isomorphism, so $f$ restricts to a diffeomorphism between an open neighborhood of $x$ in $X$ and an open subset of $\R^k$.
    In other words, $f$ induces a smooth chart at $x$.
\end{proof}

\begin{pbox}[(b)]
\end{pbox}

\begin{proof}
    Part (a) gives us a local diffeomorphism $f : V \to U \seq \R^k$, which has $f(a_1, \dots, a_N) = (a_1, \dots, a_k)$ for all $(a_1, \dots, a_N) \in V$.
    Then the smooth inverse $g = f^{-1} : U \to V$ has
    \[
        (a_1, \dots, a_N)
            = g(f(a_1, \dots, a_N))
            = g(a_1, \dots, a_k)
            = (g_1(a), \dots, g_N(a)),
    \]
    so $g_i(a) = a_i$ for $i = 1, \dots, k$, hence
    \[
        g(a_1, \dots, a_k) = (a_1, \dots, a_k, g_{k+1}(a), \dots, g_N(a)).
    \]
\end{proof}

\begin{pbox}[3 Exercise 1.3.10]
\end{pbox}


\newpage
\begin{pbox}[4 Exercise 1.4.1]
\end{pbox}

\begin{proof}
    Let $x \in U$ be any point.
    Since $f$ is a submersion, there are local parameterizations $\phi : V \to X$ and $\psi : W \to Y$ at $x$ and $y$, respectively, such that $F = \psi^{-1} \circ f \circ \phi$ is the standard submersion.
    The intersection $U' = U \cap \phi(V)$ is an open neighborhood of $x$ in $X$.
    The parameterizations are homeomorphisms and the standard submersion is an open map, so
    \[
        f(U')
            = f|_{\phi(V)}(U')
            = (\psi \circ F \circ \phi^{-1})(U')
    \]
    is an open neighborhood of $f(x)$ in $\psi(W)$.
    Since $\psi(W)$ is open in $Y$, the image of $U'$ is also open in $Y$.
    Hence, $f(U')$ is an open neighborhood of $f(x)$ contained in $f(U)$, so by definition $f(U)$ is open in $Y$.
\end{proof}


\begin{pbox}[5 Exercise 1.4.2]
\end{pbox}

\begin{pbox}[(a)]
\end{pbox}

\begin{proof}
    By Problem 4 Exercise 1.4.1, $f(X)$ is an open subset of $Y$.
    Since $X$ is compact, $f(X)$ is compact and therefore also closed.
    Since $Y$ is connected and $f(X) \seq Y$ clopen, we either have $f(X) = \emptyset$ or $f(X) = Y$.
    Since $X$ is nonempty, the image is nonempty.
    Hence $f(X) = Y$, i.e., $f$ is surjective.
\end{proof}

\begin{pbox}[(b)]
\end{pbox}

\begin{proof}
    Suppose $f : X \to \R^n$ is a smooth map from a compact manifold.
    Then $f(X) \seq \R^n$ is compact, but $\R^n$ is not.
    In particular, $f(X) \ne \R^n$, so $f$ is not surjective.
    The contrapositive of part (a) tells us that $f$ is not a submersion.
\end{proof}

\begin{pbox}[6 Exercise 1.4.7]
\end{pbox}

\begin{proof}
    By the preimage theorem, $Z = f^{-1}(y)$ is a submanifold of $X$ of dimension
    \[
        \dim Z = \dim X - \dim Y = 0.
    \]
    Each point $x \in Z$ has a neighborhood $U \seq Z$ diffeomorphic to $\R^0 = \{0\}$.
    So $Z \cap U = \{x\}$ is an open subset of $Z$ for all $x \in Z$, i.e., $Z$ has the discrete topology.
    Moreover, $Z$ is a closed subset of the compact set $X$, which implies $Z$ is compact.
    Therefore, $Z$ must contain only finitely many points since the collection of all singletons forms an open cover.

    Say $Z = f^{-1}(y) = \{x_1, \dots, x_N\}$.
    Each $x_i$ is a regular point of $f$, so there are neighborhoods $x \in U_i \seq X$ and $y \in V_i \seq Y$ (and suitable parameterizations) on which $f$ is equivalent to the standard submersion.
    Since
    \[
        \dim U_i = \dim X = \dim Y = \dim V_i,
    \]
    then $f$ is actually locally equivalent to the identity map on an open subset of Euclidean space.
    In particular, $f|_{U_i} : U_i \to V_i$ is a diffeomorphism.

    Since $Z$ is a finite discrete subset of Euclidean, there is a positive radius for which the open balls $B_r(x_i)$ are all disjoint.
    Since $X$ has the subspace topology, the intersection $B_r(x_i) X$ is an open neighborhood of $x_i$ in $X$.
    Take $W_i = U_i \cap B_r(x_i)$ and $U = \bigcap_{i=1}^{N} f(W_i)$.
    Then $f^{-1}(U)$ is the disjoint union of $W_i' = W_i \cap f^{-1}(U)$, and each $W_i \cap f^{-1}(U)$ is mapped diffeomorphically to $U$.
\end{proof}

\begin{pbox}[7 Exercise 1.4.10]
\end{pbox}

\begin{proof}
    Let $f : M(n) \to S(n)$ be the map $f(A) = AA^T$ so $O(n) = f^{-1}(I)$.
    Then
    \[
        \dd{f}_I(A) = AI^T + IA^T = A + A^T.
    \]
    Then
    \[
        T_I O(n) = \ker \dd{f}_I = \{A \in M(n) : A^T = -A\}.
    \]
\end{proof}

\begin{pbox}[8 Exercise 1.4.12]
\end{pbox}

\begin{proof}
    Consider the map $f = \det|_{M(n) \setminus \{0\}}$; we must check that $0$ is a regular value of $f$.
    Let $A \in f^{-1}(0)$ and write
    \[
        A = \mat{x_1 & x_2 \\ x_3 & x_4}.
    \]
    Then the Jacobian of $f$ at $A$ is
    \[
        J_f(A) = \mat{x_4 & -x_3 & -x_2 & x_1}.
    \]
    Since $A$ is nonzero, some $x_i$ is nonzero, so the Jacobian is full rank.
    Therefore, $\dd{f}_A$ is surjective so $A$ is a regular point of $f$.
    Hence $0$ is a regular value of $f$, so $f^{-1}(0)$ is a manifold and is precisely the set of $2 \times 2$ matrices of ranks $1$.
\end{proof}

\begin{pbox}[9 Exercise 1.6.1]
\end{pbox}

\begin{pbox}[(a)]
\end{pbox}

\begin{proof}
    By definition, $A$ and $V$ are transverse if $\im \dd{A}_x + T_{Ax}V = T_{Ax}\R^n$.
    But we have
    \[
        \dd{A}_x = A, \quad T_{Ax} V = V, \isp{and} T_{Ax}\R^n = \R^n.
    \]
    So indeed, $A$ and $V$ are transverse precisely when $A(\R^k) + V = \R^n$.
\end{proof}

\begin{pbox}[(b)]
\end{pbox}

\begin{proof}
    By definition, $V$ and $W$ transverse if $T_x V + T_x W = T_x \R^n$.
    But we have
    \[
        T_x V = V, \quad T_x W = W, \isp{and} T_x \R^n = \R^n.
    \]
    So indeed, $V$ and $W$ are transverse precisely when $V + W = \R^n$.
\end{proof}

\begin{pbox}[10 Exercise 1.6.2]
\end{pbox}

\begin{pbox}[(a)]
\end{pbox}

Transverse since the $xy$-plane is the span $\<e_1, e_2\> \leq \R^3$, the $z$-axis the the span $\<e_3\> \leq \R^3$ and
\[
    \<e_1, e_2\> + \<e_3\> = \<e_1, e_2, e_3\> = \R^3.
\]

\begin{pbox}[(b)]
\end{pbox}

Transverse since $xy$-plane is the span $\<e_1, e_2\> \leq \R^3$ and the other plane contains $4e_2 - e_3$, so their sum also contains $e_3$ and is therefore all of $\R^3$.

\begin{pbox}[(c)]
\end{pbox}

Not transverse since both are contained in the $xy$-plane, which does not span $\R^3$.


\begin{pbox}[(d)]
\end{pbox}

Transverse if and only if $k + \ell \geq n$.

\begin{pbox}[(e)]
\end{pbox}

Transverse if and only if $k = n$ or $\ell = n$

\begin{pbox}[(f)]
\end{pbox}

Transverse since $(v, 0) \in V \times 0$ and $(v, v) \in \Delta(V)$ so $(0, v) \in V + \Delta(V)$.
Then the vectors $(v, 0)$ and $(0, v)$ as for all $v \in V$ span $V \times V$.

\begin{pbox}[(g)]
\end{pbox}

Transverse since every matrix $A \in M(n)$ can be written as
\[
    A = \frac{1}{2}(A + A^T) + \frac{1}{2}(A - A^T),
\]
where $A + A^T$ is symmetric and $A - A^T$ is skew-symmetric, hence the two subspaces span $M(n)$.

\begin{pbox}[11 Exercise 2.2.4]
\end{pbox}

\begin{proof}
    Per the hint, Exercise 1.1.4 gives us a diffeomorphism $\phi : B_a \to \R^k$.
    Then there is a diffeomorphism $g : \R^k \to \R^k$ defined by $g(x) = x + e_1$.
    The composition $f = \phi^{-1} \circ g \circ \phi$ is thus a diffeomorphism $B_a \to B_a$.

    We claim that $f$ has no fixed points.
    Suppose $x \in B_a$ is a fixed point, so
    \[
        x = f(x) = \phi^{-1}(g(\phi(x))),
    \]
    which gives us
    \[
        \phi(x) = g(\phi(x)) = \phi(x) + e_1.
    \]
    But this implies $e_1 = 0$, which is a contradiction.
\end{proof}

\begin{pbox}[12 Exercise 2.2.6]
\end{pbox}

\begin{proof}
    Let $f : B \to B$ be a continuous map from the closed unit $n$-ball to itself.
    Let $\eps > 0$ be given and use the Weierstrass approximation theorem to choose a polynomial $p : \R^n \to \R^n$ such that $\|f - p\|_{B} < \eps$.
    The result is trivial if $f = 0$, so we assume $f \ne 0$ and $\eps$ is small enough so that $p \ne 0$.
    In particular, $\|f\|_B$ and $\|p\|_B$ are nonzero with
    \[
        |\|f\|_{B} - \|p\|_{B}| \leq \|f - p\|_{B} < \eps.
    \]
    Define a new polynomial
    \[
        q = \frac{\|f\|_{B}}{\|p\|_{B}}p,
    \]
    which has $\|q\|_B = \|f\|_B \leq 1$.
    A priori, we do not know whether $p$ maps the ball back into itself $B$, but we do know $q(B) \seq B$.
    Moreover,
    \[
        \|f - q\|_B
            \leq \|f - p\|_B + \|p - q\|_B
            < \eps + \|p - q\|_B.
    \]
    We now estimate
    \[
        \|p - q\|_B
            = \left|1 - \frac{\|f\|_B}{\|p\|_B}\right|\|p\|_B
            = |\|p\|_B - \|f\|_B|
            < \eps,
    \]
    hence $\|f - q\|_B < 2\eps$.
    In other words, $f$ is approximable by a polynomial $q : B \to B$.
    Since $q$ is smooth, it has a fixed point: $x \in B$ with $q(x) = x$.
    Then
    \[
        |f(x) - x|
            = |f(x) - q(x)|
            \leq \|f - q\|_B
            < 2\eps,
    \]
    which means
    \[
        0 \leq \inf\{|f(y) - y| : y \in B\} \leq |f(x) - x| < 2\eps.
    \]
    Since this bound holds for all $\eps > 0$, we conclude that
    \[
        \inf\{|f(y) - y| : y \in B\} = 0.
    \]
    Since this $B$ is compact and $g(y) = |f(y) - y|$ is continuous, the infimum is attained somewhere on $B$.
    Hence, there is some $x \in B$ such that $|f(x) - x| = 0$, so $f(x) = x$ is a fixed point of $f$.
\end{proof}


\newpage
\begin{pbox}[13 Exercise 2.2.7]
\end{pbox}

\begin{proof}
    Per the hint let $f : \R^n \to \R^n$ be the map $v \mapsto Av/|Av|$.
    Consider the set
    \[\textstyle
        Q = \{x = (x_1, \dots, x_n) \mid x_i \geq 0 \text{ and }\|x\|_2 = 1\}.
    \]
    Then for $x \in Q$, we have
    \[
        Ax
            = \sum_{i=1}^{n} x_iAe_i
            = \sum_{i=1}^{n} x_i\left(\sum_{j=1}^{n} a_{ji}e_j\right)
            = \sum_{j=1}^{n}\left(\sum_{i=1}^{n} x_ia_{ji}\right)e_j.
    \]
    Since $x_i \geq 0$ and $a_{ji} \geq 0$, we know that $\sum_{i=1}^{n} x_i a_{ji} \geq 0$.
    Since $f(x) \in S^{n-1}$, i.e., $\|f(x)\|_2 = 1$, so indeed $f(x) \in Q$.
    Let $\phi : B^{n-1} \to Q$ be a diffeomorphism, then $g = \phi^{-1} \circ f \circ \phi$ is a smooth map from $B^{n-1}$ to itself.
    By the fixed point theorem, there is some $x \in B^{n-1}$ such that $g(x) = x$, then setting $y = \phi(x) \in Q$ we get $f(y) = y$.
    Hence, $Ay = |Ay|y$, so $|Ay|$ is a positive eigenvalue of $A$.
\end{proof}



\end{document}
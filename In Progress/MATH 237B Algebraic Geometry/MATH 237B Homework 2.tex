\documentclass[12pt]{article}

% Packages
\usepackage[margin=1in]{geometry}
\usepackage{fancyhdr, parskip}
\usepackage{amsmath, amsthm, amssymb}
\usepackage{tikz, tikz-cd}
\usepackage[shortlabels]{enumitem}

% Page Style
\makeatletter
\fancypagestyle{title}{
    \renewcommand{\headrulewidth}{0.4pt}
    \setlength{\headheight}{15pt}
    \fancyhead[R]{\@author}
    \fancyhead[L]{\@title}
    \fancyhead[C]{\@date}
}
\makeatother
\renewcommand{\maketitle}{\thispagestyle{title}}
\fancypagestyle{plain}{
    \fancyhf{}
    \renewcommand{\headrulewidth}{0pt}
    \renewcommand{\footrulewidth}{0pt}
    \fancyfoot[R]{\thepage}
}
\pagestyle{plain}

% Problem Box
\setlength{\fboxsep}{4pt}
\newlength{\myparskip}
\setlength{\myparskip}{\parskip}
\newsavebox{\savefullbox}
\newenvironment{fullbox}{\begin{lrbox}{\savefullbox}\begin{minipage}{\dimexpr\textwidth-2\fboxsep\relax}\setlength{\parskip}{\myparskip}}{\end{minipage}\end{lrbox}\framebox[\textwidth]{\usebox{\savefullbox}}}
\newenvironment{pbox}[1][]{\begin{fullbox}\ifx#1\empty\else\paragraph{#1}\phantom{}\fi}{\end{fullbox}}

% Theorem Environments
\theoremstyle{definition}
\newtheorem{lemma}{Lemma}

% Tikz Environments
\newenvironment{drawing}{\begin{center}\begin{tikzpicture}}{\end{tikzpicture}\end{center}}
% \tikzcdset{row sep/normal=0pt}
\newenvironment{cd}{\begin{center}\begin{tikzcd}}{\end{tikzcd}\end{center}}

% Default Commands
\newcommand{\isp}[1]{\quad\text{#1}\quad}
\newcommand{\N}{\mathbb{N}} 
\newcommand{\Z}{\mathbb{Z}}
\newcommand{\Q}{\mathbb{Q}}
\newcommand{\R}{\mathbb{R}}
\newcommand{\C}{\mathbb{C}}
\newcommand{\A}{\mathbb{A}}
\renewcommand{\P}{\mathbb{P}}
\newcommand{\eps}{\varepsilon}
\renewcommand{\phi}{\varphi}
\renewcommand{\emptyset}{\varnothing}
\newcommand{\<}{\langle}
\renewcommand{\>}{\rangle}
\newcommand{\isom}{\cong}
\newcommand{\eqc}{\overline}
\newcommand{\clo}{\overline}
\newcommand{\teq}{\trianglelefteq}
\DeclareMathOperator{\id}{id}
\DeclareMathOperator{\im}{im}

% Extra Commands
\DeclareMathOperator{\Spec}{Spec}
\renewcommand{\O}{\mathcal{O}}
\newcommand{\mm}{\mathfrak{m}}

\newcommand{\tensor}{\otimes}


% Document
\begin{document}
\title{MATH 237B Homework 2}
\author{Harry Coleman}
\date{January 18, 2022}
\maketitle


\begin{pbox}[1 Exercise 5.6.8]
    Let $k$ be an algebraically closed field.
    An $n$-fold point (over $k$) is a scheme of the form $X = \Spec R$ such that $X$ has only one point and $R$ is a $k$-algebra of vector space dimension $n$ over $k$ (i.e., $X$ has length $n$).
    Show that every double point is isomorphic to $\Spec k[x]/\<x^2\>$.
\end{pbox}

\begin{proof}
    Suppose $X = \Spec R$ is a double point, i.e., $R$ is a $k$-algebra with a unique prime ideal and $\dim_k R = 2$.
    Choose an element $y \in R$ such that $R = ky \oplus k$.
    Then we must have $y^2 = ay + b$ for some $a, b \in k$.
    Write $x = y - a/2$ and $c = a^2/4 + b$ for a change of variables, so that $R = kx \oplus k$ with $x^2 = c$.
    This gives a surjective $k$-algebra homomorphism $k[x]/\<x^2 - c\> \to R$.
    In particular, this is a $k$-linear map and
    \[
        \dim_k k[x]/\<x^2 - c\> = 2 = \dim_k R,
    \]
    so in fact $k[x]/\<x^2 - c\> \isom R$ as $k$-algebras.
    Then the unique prime ideal of $R$ corresponds to a unique prime ideal of $k[x]$ containing $\<x^2 - c\>$.
    If $c \ne 0$, then because $k$ is algebraically closed, we would have a pair of distinct prime ideals $\<x \pm \sqrt{c}\> \teq k[x]$ containing $\<x^2 - c\>$.
    It follows that $c = 0$, and we obtain an isomorphism
    \[
        \Spec k[x]/\<x^2\> \isom \Spec R = X
    \]
    as schemes over $k$.
\end{proof}


\begin{pbox}
    On the other hand, find two non-isomorphic triple points over $k$, and describe them geometrically.
\end{pbox}

Use $R = k[x]/\<x^3\>$ and $S = k[x, y]/\<x^2, xy, y^2\>$.

Unique points are $\<x\>$ and $\<x, y\>$, respectively, but second isn't principal, so $R \not\isom S$.

If the double point is a point with a linear direction, we might consider $\Spec R$ to be a point with a quadratic direction.

And we might consider $\Spec S$ to be a point with two linear directions.



\newpage
\begin{pbox}[2 Exercise 5.6.9]
    Show that for a scheme $X$ the following are equivalent:
    \begin{enumerate}[(i)]
    \item $X$ is reduced, i.e., for every open subset $U \subseteq X$ the ring $\O_X(U)$ has no nilpotent elements.
    \item For any open subset $U_i$ of an open affine cover $\{U_i\}$ of $X$, the ring $\O_X(U_i)$ has no nilpotent elements.
    \item For every point $P \in X$ the local ring $\O_{X, P}$ has no nilpotent elements.
    \end{enumerate}
\end{pbox}

\begin{proof}
    Assuming (i) is true, (ii) follows trivially.

    Assume (ii) is true and let $P \in X$.
    Choose an affine open cover $\{U_i = \Spec R_i\}$ of $X$ and suppose $P \in U_i$.
    Then $\O_X(U_i) = R_i$ and
    \[
        \O_{X, P} = \O_{U_i, P} = (R_i)_P
    \]
    Assume for contradiction that $a/b \in (R_i)_P$ is nilpotent with $a^n/b^n = 0 \in (R_i)_P$.
    That is, $a, b \in R_i$ with $b \notin P$ and $ua^n = 0 \in R_i$ for some $u \in R \setminus P$.
    The fact that $a/b \ne 0 \in (R_i)_P$ implies $ua \ne 0 \in R_i$, however $(ua)^n = u^na^n = 0 \in R_i$.
    That is, $ua$ is a nilpotent element of $R_i$, which contradicts our assumption of (ii), hence (iii) is true.

    Assume (iii) is true and let $U \subseteq X$ be open.
    Choose an affine open cover $\{U_i = \Spec R_i\}$ of $U$.
    Then for each $i$, the ringed space structure of $X$ gives us a ring homomorphism $\O_X(U) \to \O_X(U_i) = R_i$.
    Moreover, there is a natural ring homomorphism $R_i \to (R_i)_P$ for each $P \in U_i$.
    Then an element $f \in \O_X(U)$ is nonzero only if its image in some $R_i$ is nonzero.
    Now choose $P \in U_i \cap X_f$, so then $f \notin P$, telling us that $f_P$ is a unit in $(R_i)_P$.
    In particular, $f$ maps to a nonzero element under the ring homomorphism $\O_X(U) \to (R_i)_P = \O_{X, P}$.
    By assumption of (iii), $\O_{X, P}$ is reduced, so $f_P$ is not nilpotent.
    It follows that $f$ is not nilpotent (otherwise if $f^n = 0 \in \O_X(U)$ then it would map to $(f_P)^n = 0 \in (R_i)_P$), hence (i) is true.
\end{proof}


\newpage
\begin{pbox}[3 Exercise 5.6.11]
    Let $X = Z(x_1^2x_2 + x_1x_2^2x_3) \subseteq \A_\C^3$, and denote by $\pi_i$ the projection to the $i$th coordinate.
    Compute the scheme-theoretic fibers $X_{x_i = a} = \pi_i^{-1}(a)$ for all $a \in \C$, and determine the set of isomorphism classes of these schemes.
\end{pbox}

Let $\A_\C^3 = \Spec \C[x, y, z]$.

Ler $R = \C[x, y, z]/\<x^2y + xy^2z\>$, so $X = \Spec R$.

For the fiber $X_{x=a}$, we have the following corresponding commutative diagram of rings:
\begin{cd}
    R \tensor_{\C[x]} \C \rar[hookleftarrow] \dar[hookleftarrow] & \C \\
    R & \C[x] \uar["\operatorname{eval}_a"'] \lar["\operatorname{eval}_x"]
\end{cd}
This means that in $R \tensor_{\C[x]} \C$, we have
\[
    x \tensor 1 = x(1 \tensor 1) = (1 \tensor a) = a \tensor 1.
\]
In other words,
\[
    R \tensor_{\C[x]} \C = R/\<x - a\> = \C[y, z]/\<a^2y + ay^2z\>,
\]
so
\[
    X_{x=a} = \begin{cases}
        \Spec \C[y, z] & \text{if } a = 0, \\
        \Spec \C[y, z]/\<y(a + yz)\> & \text{if } a \ne 0.
    \end{cases}
\]
Similarly, for the fiber $X_{y=a}$, we have the following commutative diagram of rings:
\begin{cd}
    R \tensor_{\C[y]} \C \rar[hookleftarrow] \dar[hookleftarrow] & \C \\
    R & \C[y] \uar["\operatorname{eval}_a"'] \lar["\operatorname{eval}_y"]
\end{cd}
This gives us
\[
    X_{y=a} = \begin{cases}
        \Spec \C[x, z] & \text{if } a = 0, \\
        \Spec \C[x, z]/\<x(x + az)\> & \text{if } a \ne 0.
    \end{cases}
\]
Lastly,
\[
    X_{z=a} = \begin{cases}
        \Spec \C[x, y]/\<x^2y\> & \text{if } a = 0, \\
        \Spec \C[x, y]/\<xy(x + ay)\> & \text{if } a \ne 0.
    \end{cases}
\]

We now have the following isomorphism classes for $a \ne 0$:
\begin{enumerate}[(i)]
    \item $X_{x=0} \isom X_{y=0} \isom \A^2$,
    \item $X_{x=a}$ is the union of a line and a hyperbola,
    \item $X_{y=a}$ is the union of two lines through the origin,
    \item $X_{z=0}$ is the union of a two lines through the origin, one having multiplicity $2$,
    \item $X_{z=a}$ is the union of three lines through the origin.
\end{enumerate}



\newpage
\begin{pbox}[4 Exercise 5.6.12]
    Let $X$ be a prevariety over an algebraically closed field $k$, and let $P \in X$ be a (closed) point of $X$.
    Let $D = \Spec k[x]/\<x^2\>$ be the ``double point.''
    Show that the tangent space $T_{X, P}$ to $X$ at $P$ can be canonically identified with the set of morphisms $D \to X$ that map the unique point of $D$ to $P$.
\end{pbox}

\begin{proof}
    Since the tangent space is a local construction, an affine neighborhood of $P$ would have the same tangent space.
    Without loss of generality, we may assume $X$ is an affine variety, i.e., $X = \Spec R$ where $R$ is a finitely generated reduced $k$-algebra.
    Then $\mm = P$ is a maximal ideal of $R$ and the tangent space $T_{X, P}$ is the dual of the $k$-vector space $\mm/\mm^2$.

    (Additionally, the unique point of $D$ is the maximal ideal $\<\eqc{x}\> = \<x\>/\<x^2\>$.)

    Consider a morphism $f : D \to X$ such that $f(\<x\>) = P$.
    Since $f$ is a morphism of locally ringed spaces, the induced morphism of rings
    \[
        f^* : R \longrightarrow k[x]/\<x^2\>
    \]
    has $(f^*)^{-1}(\<\eqc{x}\>) = \mm$, which implies $\mm^2$ is mapped into $\<\eqc{x}\>^2 = \<0\>$.
    By the universal property of quotients, $f^*$ uniquely factors to a morphism
    \[
        \eqc{f^*} : R/\mm^2 \longrightarrow k[x]/\<x^2\>,
    \]
    which restricts to a $k$-linear map
    \[
        \eqc{f^*}|_{\mm/\mm^2} : \mm/\mm^2 \longrightarrow \<\eqc{x}\> = \<x\>/\<x^2\>.
    \]
    Evaluation at $1$ is a $k$-linear map $\operatorname{eval}_1 : \<\eqc{x}\> \to k$ (defined by $\eqc{x} \mapsto 1$).
    Composition gives us a $k$-linear functional
    \[
        \operatorname{eval}_1 \circ \eqc{f^*}|_{\mm/\mm^2} : \mm/\mm^2 \longrightarrow k,
    \]
    which is an element of $T_{X, P}$.
    

    On the other hand, given $v \in T_{X, P}$, we can define a $k$-linear map
    \begin{align*}
        \tilde{v} : \mm/\mm^2 &\longrightarrow \<\eqc{x}\>, \\
            a &\longmapsto v(a)\eqc{x}.
    \end{align*}
    If $\pi : R \to R/\mm^2$ is the quotient map, we obtain a $k$-linear map
    \[
        \tilde{v} \circ \pi|_{\mm} : \mm \longrightarrow \<\eqc{x}\>.
    \]
    Note that this map also agrees with multiplication.
    (Though a proper ideal is not a ring, we sometimes consider it to be a \textit{rng}---a ring lacking identity. 
    In this sense, the map is a morphism of rngs.)

    Since $\mm \teq R$ is maximal and $k$ is algebraically closed, we have $R/\mm = k$, so $R = \mm + k$.
    By the same argument, $k[x]/\<x^2\> = \<\eqc{x}\> + k$.
    We can therefore uniquely extend the map to a $k$-algebra homomorphism
    \[
        \phi : R \longrightarrow k[x]/\<x^2\>,
    \]
    where $\phi|_\mm = \tilde{v} \circ \pi|_{\mm}$.
    This corresponds to a morphism of varieties $D \to X$ that maps the unique point of $D$ to $P$.

    These constructions are inverse to each other, and thus describe the desired identification.
\end{proof}




\newpage
\begin{pbox}[5 Exercise 6.7.1]
    Let $X$ be a collection of four distinct points in some $\P^n$.
    What are the possible Hilbert functions $h_X$?
\end{pbox}

Note that the four points must live inside a copy of $\P^3$.
A linear change of basis gives
\[
    X \subseteq \P^3 = Z(x_4, \dots, x_n) \subseteq \P^n.
\]
In which case, $I(X) \subseteq \<x_4, \dots, x_n\> \teq k[x_0, \dots, x_n]$, meaning that $S(X)$ has no monomials containing $x_4, \dots, x_n$, so we can consider it to simply be a quotient of $k[x_0, \dots, x_3]$.
In other words, for computing the Hilbert functions, we can assume $X \subseteq \P^3$.


\begin{enumerate}[(i)]
    \item If the four points of $X$ are noncoplanar, then we can perform a linear change of basis to write the points as
    \[
        [1 : 0 : 0 : 0], [0 : 1 : 0 : 0], [0 : 0 : 1 : 0], [0 : 0 : 0 : 1] \in \P^3,
    \]
    then
    \[
        I(X) = \<x_0^{d_0} \cdots x_3^{d_3} \mid 0 \leq d_i \leq 2, \textstyle\sum d_i = 3\> \teq S = k[x_0, \dots, x_3].
    \]
    As this is ideal is generated by degree $3$ monomials, no smaller monomials are eliminated in $S(X) = S/I(X)$.
    And for $d \geq 3$, this ideal eliminates all monomials except those of the form $x_i^d$.
    That is,
    \[
        S(X)^{(d)} = \begin{cases}
            S^{(d)} & \text{if } d \leq 2, \\
            k \cdot \{x_0^d, \dots, x_3^d\} & \text{if } d \geq 3.
        \end{cases}
    \]
    Hence,
    \[
        h_X(s) = \begin{cases}
            \binom{s + 3}{s} & \text{if } s \leq 2, \\
            4 & \text{if } s \geq 3.
        \end{cases}
    \]

    \item If the four points of $X$ are coplanar, but not colinear, we can perform a linear change of basis to write the points as
    \[
        [1 : 0 : 0], [0 : 1 : 0], [0 : 0 : 1], [a : b : c] \in \P^2,
    \]
    where $a$ and $b$ are nonzero.
    Then
    \[
        I(X) = \left\<\begin{matrix}
            axyz = cx^2y = bx^2z, & bx^2y = cxy^2, \\
            bxyz = cxy^2 = ay^2z, & cx^2z = axz^2, \\
            cxyz = bxz^2 = ayz^2, & cy^2z = byz^2
        \end{matrix}\right\>
        \teq S = k[x, y, z].
    \]
    Using these relations, we can rewrite a monomial in $S(X)$ of degree at least $3$ and containing at least two variables as a monomial (times a scalar) containing only the variables $x$ and $y$.
    Moreover, we can rewrite a monomial of degree at least $3$ containing both $x$ and $y$ to have the degree of $y$ be $1$.
    That is, 
    \[
        S(X)^{(d)} = \begin{cases}
            S^{(d)} & \text{if } d \leq 2, \\
            k \cdot \{x^d, y^d, z^d, x^{d-1}y\} & \text{if } d \geq 3.
        \end{cases}
    \]
    Hence,
    \[
        h_X(s) = \begin{cases}
            \binom{s + 2}{s} & \text{if } s \leq 2, \\
            4 & \text{if } s \geq 3.
        \end{cases}
    \]

    \item If the four points are colinear, we can perform a linear change of basis to write the points as
    \[
        [1 : 0], [0 : 1], [1 : 1], [a : b] \in \P^1,
    \]
    
\end{enumerate}



\end{document}
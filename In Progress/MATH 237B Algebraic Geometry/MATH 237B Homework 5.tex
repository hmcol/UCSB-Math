\documentclass[12pt]{article}

% Packages
\usepackage[margin=1in]{geometry}
\usepackage{fancyhdr, parskip}
\usepackage{amsmath, amsthm, amssymb}
\usepackage{tikz, tikz-cd}

% Page Style
\makeatletter
\fancypagestyle{title}{
    \renewcommand{\headrulewidth}{0.4pt}
    \setlength{\headheight}{15pt}
    \fancyhead[R]{\@author}
    \fancyhead[L]{\@title}
    \fancyhead[C]{\@date}
}
\makeatother
\renewcommand{\maketitle}{\thispagestyle{title}}
\fancypagestyle{plain}{
    \fancyhf{}
    \renewcommand{\headrulewidth}{0pt}
    \renewcommand{\footrulewidth}{0pt}
    \fancyfoot[R]{\thepage}
}
\pagestyle{plain}

% Problem Box
\setlength{\fboxsep}{4pt}
\newlength{\myparskip}
\setlength{\myparskip}{\parskip}
\newsavebox{\savefullbox}
\newenvironment{fullbox}{\begin{lrbox}{\savefullbox}\begin{minipage}{\dimexpr\textwidth-2\fboxsep\relax}\setlength{\parskip}{\myparskip}}{\end{minipage}\end{lrbox}\framebox[\textwidth]{\usebox{\savefullbox}}}
\newenvironment{pbox}[1][]{\begin{fullbox}\ifx#1\empty\else\paragraph{#1}\phantom{}\fi}{\end{fullbox}}

% Theorem Environments
\theoremstyle{definition}
\newtheorem{lemma}{Lemma}

% Tikz Environments
\newenvironment{drawing}{\begin{center}\begin{tikzpicture}}{\end{tikzpicture}\end{center}}
% \tikzcdset{row sep/normal=0pt}
\newenvironment{cd}{\begin{center}\begin{tikzcd}}{\end{tikzcd}\end{center}}

% Default Commands
\newcommand{\isp}[1]{\quad\text{#1}\quad}
\newcommand{\N}{\mathbb{N}} 
\newcommand{\Z}{\mathbb{Z}}
\newcommand{\Q}{\mathbb{Q}}
\newcommand{\R}{\mathbb{R}}
\newcommand{\C}{\mathbb{C}}
\newcommand{\A}{\mathbb{A}}
\renewcommand{\P}{\mathbb{P}}
\newcommand{\eps}{\varepsilon}
\renewcommand{\phi}{\varphi}
\renewcommand{\emptyset}{\varnothing}
\newcommand{\<}{\langle}
\renewcommand{\>}{\rangle}
\newcommand{\isom}{\cong}
\newcommand{\eqc}{\overline}
\newcommand{\clo}{\overline}
\newcommand{\seq}{\subseteq}
\newcommand{\teq}{\trianglelefteq}
\DeclareMathOperator{\id}{id}
\DeclareMathOperator{\im}{im}

% Extra Commands
\newcommand{\OO}{\mathcal{O}}
\newcommand{\FF}{\mathcal{F}}
\newcommand{\GG}{\mathcal{G}}
\newcommand{\LL}{\mathcal{L}}
\newcommand{\mm}{\mathfrak{m}}

\DeclareMathOperator{\Spec}{Spec}
\DeclareMathOperator{\Proj}{Proj}

\newcommand{\inc}{\hookrightarrow}
\newcommand{\tensor}{\otimes}

% Document
\begin{document}
\title{MATH 237B Homework 5}
\author{Harry Coleman}
\date{February 8, 2022}
\maketitle

\begin{pbox}[1 Exercise 7.8.3]
    Let $f : \FF \to \GG$ be a morphism of locally free sheaves on a scheme $X$ over a field $k$.
    Let $P \in X$ be a point, and denote by $\FF(P)$ (resp. $\GG(P)$) the fiber of the vector bundle $\FF$ (resp. $\GG$) over $P$, which is a $k$-vector space.
    Are the following statements true or false?
\end{pbox}

If $\iota : P \to X$ is the inclusion map, then the fiber of $\FF$ over $P$ is defined as the pullback
\[
    \FF(P)
        = (\iota^*\FF)(P) 
        = (\iota^{-1}\FF)(P) \tensor_{(\iota^{-1}\OO_{X})(P)} \OO_P(P)
        = \FF_P \tensor_{\OO_{X, P}} k(P).
\]
Given $f : \FF \to \GG$, there is an induced morphism of stalks $f_P : \FF_P \to \GG_P$.
We have seen that $f$ is injective/surjective if and only if $f_P$ is injective/surjective at all points $P \in X$.
There is an induced morphism of $\OO_{X, P}$-modules, given by
\[
    f_P \tensor \id_{k(P)} : \FF_P \tensor_{\OO_{X, P}} k(P) \longrightarrow \GG_P \tensor_{\OO_{X, P}} k(P).
\]
This is the induced map of fibers $\FF(P) \to \GG(P)$.

\begin{pbox}[(i)]
    If $f : \FF \to \GG$ is injective then the induced map $\FF(P) \to \GG(P)$ is injective for all $P \in X$.
\end{pbox}

No.

Consider $X = \P^n$, with locally free sheaves $\FF = \OO_{\P^n}(-1)$ and $\GG = \OO_{\P^n}$.
Then we have an exact sequence of $\OO_{\P^n}$-modules
\begin{cd}
    0 \rar & \OO_{\P^n}(-1) \rar["\cdot x_0"] & \OO_{\P^n}.
\end{cd}
In other words, $f = \cdot x_0$ is an injective morphism of sheaves.
Consider the induced map on fibers:
\begin{cd}
    (\OO_{\P^n}(-1))_P \tensor_{\OO_{\P^n, P}} k(P) \rar & \OO_{\P^n, P} \tensor_{\OO_{\P^n, P}} k(P) = k(P)
\end{cd}
Note that the right side (the residue field $k(P)$) has the structure of an $\OO_{\P^n, P}$ module by the evaluation map $\phi \mapsto \phi(P) \in k(P)$.
So if we take $P = [0 : 1 : \cdots ] \in \P^n$, then any $\frac{g}{h} \in (\OO_{\P^n}(-1))_P$ will have $\frac{x_0g}{h} \in \OO_{\P^n, P}$, which is then zero in $k(P)$.
That is, the induced map on fibers is trivial.
But since the left side is not trivial, the map is not injective.
    
\begin{pbox}[(ii)]
    If $f : \FF \to \GG$ is surjective then the induced map $\FF(P) \to \GG(P)$ is surjective for all $P \in X$.
\end{pbox}

Yes.

If $f$ is surjective, then the map on stalks $f_P$ is surjective, which implies that the map on fibers $f_P \tensor \id_{k(P)}$ is surjective.
This follows from the fact that tensoring is right-exact.


\newpage
\begin{pbox}[2 Exercise 7.8.4]
    Prove the following generalizations of example 7.1.16: If $X$ is a smooth curve over some field $k$, $\LL$ a line bundle on $X$, and $P \in X$ a point, then there is an exact sequence
    \begin{cd}
        0 \rar & \LL(-P) \rar & \LL \rar & k_p \rar & 0,
    \end{cd}
    where $k_P$ denotes the ``skyscraper sheaf''
    \[
        k_P(U) = \begin{cases}
            k &\text{if } P \in U, \\
            0 &\text{if } P \notin U.
        \end{cases}
    \]
\end{pbox}

Consider the case that $\LL = \OO_X$.
By definition, if $U \seq X$ is an open neighborhood of $P$, we have
\[
    \OO_X(-P)(U) = \{\phi \in K(X) \mid (\phi) - P \geq 0 \text{ on } U\}.
\]
This means $\phi \in \OO_X(-P)(U)$ if and only if $\phi \in \OO_X(U)$ with $\phi(P) = 0$.
And if $U \seq X$ is open not containing $P$, then $\OO_X(-P)(U) = \OO_X(U)$.
Suppose $s_0$ is a rational section on $\OO_X(-P)$, so that $\OO_X(-P) = \OO_X((s_0))$.
Then we can also write
\[
    \OO_X(-P)(U) = \OO_X((s_0))(U) = \{\phi \in K(X) \mid (\phi) + (s_0) \geq 0 \text{ on } U\}.
\]
Then we have a sequence of $\OO_X$-modules
\begin{cd}
    0 \rar & \OO_X(-P) \rar["\cdot s_0"] & \OO_X \rar["\operatorname{eval}_P"] & k_P \rar & 0.
\end{cd}
By construction, this sequence is exact on open sets $U \seq X$ containing $P$, since $\OO_X(-P)(U)$ is precisely the sections in $\OO_X(U)$ which evaluate to zero at $P$.
And when $U$ does not contain $P$, the skyscraper sheaf is zero, and the sequence describes an automorphism of $\OO_X(U)$.

For a general line bundle $\LL$, we choose a rational section $\ell$ on $\LL$, so that $\LL \isom \OO_X((\ell))$.
Then the desired short exact sequence is obtained via the first case.





\newpage
\begin{pbox}[3 Exercise 7.8.5]
    If $X$ is an \textit{affine} variety over a field $k$ and $\FF$ a locally free sheaf of rank $r$ on $X$, is then necessarily $\FF \isom \OO_X^{\oplus r}$?
\end{pbox}

\begin{pbox}[4 Exercise 7.8.6]
    Let $X$ be a scheme, and let $\FF$ be a locally free sheaf on $X$.
    Show that $(\FF^\vee)^\vee \isom \FF$.
    Show by example that this statement is in general false if $\FF$ is only quasi-coherent but not locally free.
\end{pbox}

\end{document}